\chapter{Mechanisation}\label{cha:mechanisation}

In the previous chapter, we defined and proved the soundness of our
type system. In this chapter, we will look at how this was
mechanically formalised --- proved within a proof assistant.

For this type system, I decided to mechanise this within
Agda~\cite{norell2009}. Agda is a dependently typed programming
language with an ML syntax, similar to that of Haskell's. It can be
used for general purpose programming, but because it is rooted in
Martin-Löf intuitionistic type theory~\cite{martin-lof1984}, it can
also be used as a proof assistant.

Like many other proof assistants, the way we prove properties within
Agda is by writing programs that satisfy types. It takes advantage of
the Curry-Howard correspondence, which says that propositions are
analogous to types, and proofs are analogous to programs that fulfil
that type. We first create types that represent our
propositions. Then we can then prove our propositions by constructing
a value for the program that satisfies the type, hence intuitionistic
logic is also known as constructive logic.

The framework within Agda for working with the type system is built
upon the one used for the simply typed lambda calculus by Wadler et
al.~\cite{wadler2018} The grammar maps directly onto
data types:
\begin{minted}{agda}
data Term : Set where
  `_        : Id → Term
  ƛ_⇒_      : Id → Term → Term
  _·_       : Term → Term → Term
  lt_⇐_in'_ : Id → Term → Term → Term
  ⟦_⟧       : Term → Term
  _>>=_     : Term → Term → Term
  □         : Term
  use       : Resource → Term → Term
  _×_       : Term → Term → Term
  π₁        : Term → Term
  π₂        : Term → Term
  _⋎_       : Term → Term → Term
\end{minted}
\begin{minted}{agda}
data Heap : Set where
  World : Heap
  `_    : Resource → Heap
  _∪_   : Heap → Heap → Heap
\end{minted}
\begin{minted}{agda}
data Type : Set where
  `_  : Id → Type
  _⇒_ : Type → Type → Type
  IO  : Heap → Type → Type
  □   : Type
  _×_ : Type → Type → Type
\end{minted}
\begin{minted}{agda}
data TypeScheme : Set where
  V_·_ : Id → TypeScheme → TypeScheme
  `_   : Type → TypeScheme
\end{minted}

For relations, such as the heap well-formed relation
$\textsf{ok} \ \rho$, we can also define a new data type. The well-formed
relation is parameterised over the heap:
\begin{minted}{agda}
data Ok : Heap → Set where
  OkZ : ∀ {r}
        --------
      → Ok (` r)
  OkS : ∀ {a b}
       → Ok a
       → Ok b
       → a ∩ b =∅
         ----------
       → Ok (a ∪ b)
  OkWorld : --------
            Ok World
\end{minted}
The proposition is now the type \mintinline{agda}{Ok \rho}, and the rules
of the relation are now the constructors.