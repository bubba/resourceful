\appendix
\chapter{} % needed to get section numbering?
\section{Definition of a Complete Partial Order}\label{sec:defin-compl-part}
A complete partial order (cpo) is a pair $(D, \sqsubseteq)$ consisting of a set
$D$ and a partial order $\sqsubseteq$ (a function that orders elements in
$D$, but not necessarily all of them, hence the term partial), such
that
\begin{enumerate}
\item there is a least element $\bot$
\item each directed subset $x_0 \sqsubseteq \ldots \sqsubseteq x_n \sqsubseteq \ldots$ has a least upper bound
  (\emph{lub})
\end{enumerate}

\section{Evaluation Function Notation}\label{sec:eval-funct-notat}
If $\mathbb{D} \subset \mathbb{V}$, and $d \in \mathbb{D}$, we will say $d \ \mathsf{in} \
\mathbb{V}$ to represent $d$ but treated as if its in $\mathbb{V}$. \\
We will then define the reverse

\[
  v | \mathbb{D} =
\begin{cases}
  d & \textsf{if} \ v = d \ \textsf{in} \ \mathbb{V} \ \textsf{for
    some} \ d \in \mathbb{D} \\
  \bot_{\mathbb{D}} & \textsf{otherwise}
\end{cases}
\]

\section{Transitivity of the subheap relation}\label{proof:subheaptransitive}
\begin{proof}
  We want to show $a \subtyp c$. Proceed with induction on $b \subtyp c$.
  \begin{description}
  \item[\rm\textsc{Top}]
    $c$ must be \textsf{World}, so from \textsc{Top} we have $a
    \subtyp c$.
  \item[\rm\textsc{Refl}] By definition of \textsc{Refl}, $b =
    c$, and so we get $a \subtyp c$ from $a \subtyp b$.
  \item[\rm\textsc{UnionL}] $c$ is of the form $\rho' \cup \rho''$, and
    from the premise we have $b \subtyp \rho'$. Use the induction
    hypothesis with $a \subtyp b$ and $b \subtyp \rho'$ to get $a \subtyp
    \rho'$, and then \textsc{UnionL} gives us $a \subtyp \rho' \cup \rho''$.
  \item[\rm\textsc{UnionR}]  $c$ is of the form $\rho'' \cup \rho'$, and
    from the premise we have $b \subtyp \rho'$. Use the induction
    hypothesis with $a \subtyp b$ and $b \subtyp \rho'$ to get $a \subtyp
    \rho'$, and then \textsc{UnionR} gives us $a \subtyp \rho'' \cup \rho'$.
  \end{description}
\end{proof}

\section{Helper lemmas}
\begin{lemma}\label{lem:unwrapLift}
  If $\Gamma \vdash \lift{e} : \IO_\rho \tau$, then $\Gamma \vdash e : \tau$.
\end{lemma}

\begin{proof}
  This might seem obvious, but because of subsumption we need to
  unwravel for the proof for it first. Begin with induction on $\Gamma \vdash
  \lift{e} : \IO_\rho \tau$.
  \begin{description}
  \item[\rm\textsc{Lift}] Straight from the premises, $\Gamma \vdash e : \tau$.
  \item[\rm\textsc{Sub}] We have the premise ${\Gamma \vdash \lift{e} : \IO_{\rho'}
    \tau}$. Just put this back into the induction hypothesis to get ${\Gamma \vdash e
    : \tau}$.
  \end{description}
\end{proof}

\begin{lemma}\label{lem:unwrapUse}
  If $\Gamma \vdash \use{r}{e} : \IO_\rho \tau$, then $\Gamma \vdash e : \tau$.
\end{lemma}
\begin{proof}
  Identical to that of Lemma~\ref{lem:unwrapLift}, except we handle
  the case \textsc{Use} instead of \textsc{Lift}.
\end{proof}