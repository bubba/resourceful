\appendix
\chapter{} % needed to get section numbering?

\section{Denotational Semantics of Hindley-Damas-Milner}\label{sec:denot-semant-hindl}

The \emph{semantic domain} of the expression language defines the
possible values an expression can have. It is a \emph{complete partial
  order}, defined as
\begin{align*}
  \mathbb{V} &= \mathbb{B}_0 + \cdots + \mathbb{B}_n + \mathbb{F} + \mathbb{W} \\
  \mathbb{F} &= \mathbb{V} \rightarrow \mathbb{V} \\
  \mathbb{W} &= \{ \cdot \}
\end{align*}
$\mathbb{F}$ is the set of functions, $\mathbb{W}$ is the set of error
values and $\mathbb{V}$ is the set of possible values.
$\mathbb{B}_0 , \ldots , \mathbb{B}_n$ represent the sets of
``basic'' values, such as booleans and naturals.

A complete partial order (cpo) is a pair $(D, \sqsubseteq)$ consisting of a set
$D$ and a partial order $\sqsubseteq$ (a function that orders elements in
$D$, but not necessarily all of them, hence the term partial), such
that
\begin{enumerate}
\item there is a least element $\bot$
\item each directed subset $x_0 \sqsubseteq \ldots \sqsubseteq x_n \sqsubseteq \ldots$ has a least upper bound
  (\emph{lub})
\end{enumerate}

Since $\mathbb{V}$ represents all possible data values, we
can extract an \emph{ideal} of it to model the values of certain
types. A subset $I$ of our cpo $\mathbb{V}$ is called an ideal, iff it
satisfies the following properties:
\begin{enumerate}
\item it is downwards closed: $\forall v_0 \in \mathbb{V}, v_1 \in \mathbb{V}, v_0 \sqsubseteq v_1 \rightarrow
  v_0 \in I \rightarrow v_1 \in I$.
\item it is closed under lubs of \omega-chains ($v_0 \sqsubseteq v_1 \sqsubseteq v_2 \sqsubseteq \ldots$)
\end{enumerate}

Bottom $\bot$ is useful for representing programs that do not terminate,
like below:
\[\letin{x = \lambda y. y}{x \ x}\]
What type does should this program have? It should assume the
type of whatever is needed. For instance, we would expect this program
to have a type of $\square$ as the argument is not used.
\[ (\lambda z. \square) \ (\letin{x = \lambda y. y}{x \ x}) : \square \]


An environment $\eta$ is defined as a map from identifiers to values
\[ \mathsf{Environment} : \mathsf{Id} \rightarrow \mathbb{V} \]
The semantic equation for the expression language describes how the
syntax (enclosed in $\llbracket\ \rrbracket$) is evaluated in a given environment.
\begin{align*}
  \mathcal{E} &: \mathsf{Expression} \rightarrow \mathsf{Environment} \rightarrow
  \mathbb{V} \\
  \mathcal{E} \llbracket x \rrbracket \eta
  &= \eta \llbracket x \rrbracket \\
  \mathcal{E} \llbracket e_1 e_2 \rrbracket \eta
  &=
    \begin{cases}
      \bot & \mathsf{if} \ v_1 = \bot \\
      (v_1 | \mathbb{F}) v_2 & \mathsf{if} \ v_1 \ \mathbb{F} \\
      \mathsf{wrong} & \mathsf{otherwise}
    \end{cases}
  \\
  & \quad \textsf{where} \ v_i = \mathcal{E} \llbracket e_i \rrbracket \eta , \ i = \{
    1, 2\} \\
  \mathcal{E} \llbracket \lambda x . \ e \rrbracket \eta
  &=
    (\lambda v . \ \mathcal{E} \llbracket e \rrbracket \eta [v / x ])
    \ \mathsf{in} \ \mathbb{V} \\
  \mathcal{E} \llbracket \textsf{let} \ x = e_1 \ \textsf{in} \ e_2 \rrbracket \eta
  &=
    \mathcal{E} \llbracket e_2 \rrbracket \ \eta [ \mathcal{E} \llbracket e_1 \rrbracket\rho / x ]
\end{align*}
The notation used here is described in~\cite{milner1978,damas1984}.
There is also a type evaluation function $\mathcal{T} : \mathsf{Type}
\rightarrow \mathsf{Valuation} \rightarrow \overline{\mathbb{V}}$
\begin{align*}
  \mathcal{T}\llbracket \square \rrbracket\psi &= \mathbb{B}_{ \square } \\
  \mathcal{T}\llbracket \mathsf{Bool} \rrbracket \psi &= \mathbb{B}_{\mathsf{Bool}} \\
  \mathcal{T}\llbracket \alpha \rrbracket \psi &= \psi \llbracket \alpha \rrbracket \\
  \mathcal{T} \llbracket \tau \rightarrow \tau' \rrbracket \psi &= \mathcal{T}\llbracket \tau \rrbracket \psi \ \rightarrow \
                             \mathcal{T} \llbracket \tau' \rrbracket \psi
\end{align*}
Where $\overline{\mathbb{V}}$ is the set of all ideals in $\mathbb{V}$
that do not contain $\mathsf{wrong}$. 
Given a valuation, we can define a relation between values and types.
\[ v :_\psi \tau \iff v \in \mathcal{T} \llbracket \tau \rrbracket\psi \]
With it, we can also define a relation between environments $\eta$ and
type environments $\Gamma$ (contexts), that says an environment respects a
type environment if each binding
in the context has the same type in the environment.
\[ \eta :_\psi \Gamma \iff \forall x : \tau \in \Gamma \ \eta\llbracket x \rrbracket :_\psi \tau\]
Finally, we can then define what semantic entailment means
given a type environment, expression and type.
\[
  \Gamma \vDash e : \tau \iff \forall \psi \forall \eta \ \eta:_\psi \Gamma \rightarrow \mathcal{E} \llbracket e \rrbracket \eta :_\psi \tau
\]

\section{Transitivity of the subheap relation}\label{proof:subheaptransitive}
\begin{proof}
  We want to show $c \subtyp a$. Proceed with induction on $c \subtyp b$.
  \begin{description}
  \item[\rm\textsc{Top}]
    $c$ must be \textsf{World}, so from \textsc{Top} we have $c
    \subtyp a$.
  \item[\rm\textsc{Refl}] By definition of \textsc{Refl}, $b =
    c$, and so we get $c \subtyp a$ from $b \subtyp a$.
  \item[\rm\textsc{UnionL}] $c$ is of the form $\rho' \cup \rho''$, and
    from the premise we have $\rho' \subtyp b$. Use the induction
    hypothesis with $b \subtyp a$ and $\rho' \subtyp b$ to get $\rho' \subtyp
    a$, and then \textsc{UnionL} gives us $\rho' \cup \rho'' \subtyp a$.
  \item[\rm\textsc{UnionR}]  $c$ is of the form $\rho'' \cup \rho'$, and
    from the premise we have $\rho'' \subtyp b$. Use the induction
    hypothesis with $b \subtyp a$ and $\rho'' \subtyp b$ to get $\rho''
    \subtyp a$, and then \textsc{UnionR} gives us $\rho'' \cup \rho' \subtyp a$.
  \end{description}
\end{proof}

\section{Helper lemmas}
\begin{lemma}\label{lem:unwrapLift}
  If $\Gamma \vdash \lift{e} : \IO_\rho \tau$, then $\Gamma \vdash e : \tau$.
\end{lemma}

\begin{proof}
  This might seem obvious, but because of subsumption we need to
  unwravel for the proof for it first. Begin with induction on $\Gamma \vdash
  \lift{e} : \IO_\rho \tau$.
  \begin{description}
  \item[\rm\textsc{Lift}] Straight from the premises, $\Gamma \vdash e : \tau$.
  \item[\rm\textsc{Sub}] We have the premise ${\Gamma \vdash \lift{e} : \IO_{\rho'}
    \tau}$. Just put this back into the induction hypothesis to get ${\Gamma \vdash e
    : \tau}$.
  \end{description}
\end{proof}

\begin{lemma}\label{lem:unwrapUse}
  If $\Gamma \vdash \use{r}{e} : \IO_\rho \tau$, then $\Gamma \vdash e : \tau$.
\end{lemma}
\begin{proof}
  Identical to that of Lemma~\ref{lem:unwrapLift}, except we handle
  the case \textsc{Use} instead of \textsc{Lift}.
\end{proof}
% LocalWords:  denotational Hindley Damas Milner cpo lub
