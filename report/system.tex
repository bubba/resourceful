\chapter{A Formal Definition}

To formalize the system of heaped monads, I introduce a toy language
and type system implementing the basic principles. The semantics are
mainly based off of the Hilney-Damas-Milner type system \cite{damasmilner}

The heap is very similar to the heap used in \cite{krishnaswami}

\section{Syntax}

\newcommand{\IO}{\mathsf{IO}}
\newcommand{\bind}{>\!\!>\!\!=} \newcommand{\concbind}{>\!\!>\!>\!\!=}
\newcommand{\subtyp}{\geq:}
\newcommand{\notsubtyp}{\ngeq:}
\newcommand{\lift}[1]{\ensuremath{\llbracket#1\rrbracket}}

\setlength{\grammarparsep}{20pt plus 1pt minus 1pt} % increase separation between rules
\setlength{\grammarindent}{12em} % increase separation between LHS/RHS
\renewcommand{\syntleft}{}
\renewcommand{\syntright}{}


\def\defaultHypSeparation{\hskip .05in}

We begin by giving the grammar for our monadic language.
It is an extension of the lambda calculus with let-polymorphism.
We use $x$, $\lambda x . e$ and $e \ e'$ to refer to variable lookup,
abstraction and application respectively, whilst we use the ML style
$\letin{x = e}{e'}$ for polymorphically binding $e$ to $x$ inside
$e'$.

More interestingly we have the monadic additions. Unlike
Krishnaswami~\cite{krishnaswami2006} we do not separate the language
into two expression and computation languages.
$\llbracket e \rrbracket$ lifts a regular expression into a monad (\texttt{return} in Haskell), and $e \bind e'$ is
the standard monadic sequencing or binding operation.

However what is new to our language is the $e \curlyvee e'$ operator, which
joins two monadic computations together into one that uses both
resources -- provided that their resources do not overlap.

\begin{grammar}

  <type $\tau$> ::= () | bool | $\tau \rightarrow \tau'$ | $\tau \times \tau'$ | $\textsf{IO}_\sigma \tau$
  <type scheme $\sigma$> ::= $\forall \alpha . \sigma$ | $\tau$

<expression $e$> ::= () | $e \times e'$
\alt $x$ | $\lambda x . e$ | $e \ e'$ | $\letin{x = e}{e'}$
%\alt $\textsf{if} \ e_1 \ \textsf{then} \ e_2 \ \textsf{else} \ e_3$
\alt $\lift{e}$ | $e \bind e'$ | $e \curlyvee e'$
\alt $\textsf{readFile}$
\alt $\textsf{readNetwork}$
% how do we specify what resources what functions use? add an explicit
% type annotation expression?
% \alt $e : \tau$

<resource $\rho$> ::= \textsf{File} | \textsf{Network} | \textsf{World}

<heap $\sigma$> ::= $\rho$ | $\sigma \cup \sigma'$

\end{grammar}

\subsection{Substitution}
Substitution on types is defined as usual, but extended to include the
new types in our system. That is, a substitution is a map from type
variables to types, written in the form $[\tau/\alpha]$. 
\begin{align*}
  () [\tau/\alpha] &= () \\
  \alpha' [\tau/\alpha] &=
             \begin{cases}
               \tau & \mathsf{if} \ \alpha' = \alpha \\
               \alpha' & \mathsf{otherwise}
             \end{cases} \\
  (\tau_1 \rightarrow \tau_2)[\tau/\alpha] &= (\tau_1[\tau/\alpha] \rightarrow \tau_2[\tau/\alpha]) \\
  (IO_\sigma \tau')[\tau/\alpha] &= \IO_\sigma \tau'[\tau/\alpha]
\end{align*}

\section{Static Semantics}

Our static semantics, or typing rules, begin with the syntax-directed
rules of the Hindley-Damas-Milner system.
\begin{mathpar}
    \inferrule*[Right=Var]{e : \tau' \in \Gamma \\ \tau' > \tau}{\Gamma \vdash e : \tau} \and
    \inferrule*[Right=App]{\Gamma \vdash e : \tau' \rightarrow \tau \\ \Gamma \vdash e' : \tau'}{\Gamma \vdash e \ e' : \tau} \and
    \inferrule*[Right=Abs]{\Gamma,x:\tau' \vdash e : \tau}{\Gamma \vdash \lambda x . e : \tau' \rightarrow \tau} \and
    \inferrule*[Right=Let]{\Gamma \vdash e : \tau' \\ \Gamma,x : \bar{\Gamma}(\tau') \vdash e' : \tau}
    {\Gamma \vdash \mathsf{let} \ x = e \ \mathsf{in} \ e' : \tau}
  \end{mathpar}
We also introduce the typing rule for $()$ expressions, to give us a
concrete type, as well as a rule for products which are constructed
through $e \times e'$.
\begin{mathpar}
  \inferrule*[Right=Unit]{ }{\Gamma \vdash () : ()} \and
  \inferrule*[Right=Product]{\Gamma \vdash e : \tau \\ \Gamma \vdash e' : \tau'}
    {\Gamma \vdash e \times e' : \tau \times \tau'}
\end{mathpar}

Now we introduce the monadic parts of the language. Our language only
has one monad, the $\IO$ monad, which is parameterised by both its
\textit{heap} and its encapsulated type. Monadic values are introduced
into the language with either $\lift{e}$
$\textsf{readFile}$ or $\textsf{readNet}$.
\begin{mathpar}
  \inferrule*[Right=Lift]{\Gamma \vdash e : \tau}{\Gamma \vdash \lift{e} : \IO_\sigma \tau} \and
    \inferrule*[Right=ReadFile]{ }{\Gamma \vdash \mathsf{readFile} : \IO_{\mathsf{File}} ()} \and
    \inferrule*[Right=ReadNet]{ }{\Gamma \vdash \mathsf{readNetwork} : \IO_{\mathsf{Network}} ()}
\end{mathpar}

$\textsf{readFile}$ and $\textsf{readNet}$ are examples of typical
operations that can consume a specific resource -- their heap consists
of just a single resource. The typing rule for $\lift{e}$, on the
other hand, allows a pure expression (or possibly another monad) to be
``lifted'' into a $\IO$ for any heap $\sigma$.

Once we have an $\IO$ value, we can sequence computation by binding
it with a function that returns another $\IO$.
\begin{mathpar}
  \inferrule*[Right=Bind]{\Gamma \vdash e : \IO_\sigma \tau \\ \Gamma \vdash e' : \tau \rightarrow \IO_\sigma
    \tau'}{\Gamma \vdash e \bind e' : \IO_\sigma \tau'}
\end{mathpar}
Note that the types of the two $\IO$s must use the same
resources. However, we want expressions such as
$$ readFile \bind readNet $$
to be well typed. In particular, we want the above expression to be of
type $\IO_{\textsf{File} \cup \textsf{Net}} ()$. But if \textsc{Bind}
requires the resources to be the same, we must first somehow ``cast''
$\textsf{readFile}$ and $\textsf{readNet}$ to this type.

Thus serves the purpose of the \textsc{Subsumption} rule.
\begin{mathpar}
  \inferrule*[Right=Subsumption]{\Gamma \vdash e : \IO_\sigma \tau \\ \sigma \subtyp \sigma'}
  {\Gamma \vdash e : \IO_{\sigma'} \tau}
\end{mathpar}
It allows something of type $\IO_\sigma \tau$ monad to be typed as $\IO_{\sigma'} \tau$, provided
that the heap $\sigma$ is a \textit{subtype} of $\sigma'$. The subtyping
rules for heaps, shown in figure~\ref{fig:subtyping}, define the
$\subtyp$ relation.

Intuitively, a heap $\sigma$ can thought of being a subtype of another heap
$\sigma'$, if $\sigma'$ subsumes $\sigma$. For example, $\textsf{Net} \subtyp
\textsf{Net} \cup \textsf{File}$, since $\textsf{Net} \cup \textsf{File}$
``overlaps'' with the heap $\textsf{Net}$.

If one views this relation as an ordering, then we have
$\textsf{World}$ defined as the least upper bound -- this represents
using all possible resources, and thus $\IO_{\textsf{World}}$ can be
thought of as the $IO$ monad in Haskell, where sequencing interacts
with the state of the entire world.

By constraining subtyping to only resources and not actual types, we
avoid the various issues associated with undecidability subtyping
gives us~\ref{????}.

The union of resources, for example,
$\textsf{Net} \cup \textsf{File}$, allows us to reason about using more
than one resource simultaneously. In fact, the purpose of this type
system is to help reason about using multiple resources in a safe
manner, by rejecting programs that use the same resource twice
simultaneously. In this light, the \textsc{Conc} rule is then the
heart and soul of this type system, by combining together heaps and
maintaining the invariant that the resources in the heap are all
unique.
\begin{mathpar}
  \inferrule*[Right=Conc]{\Gamma \vdash e_1 : \IO_\sigma \tau_1 \\ \Gamma \vdash e_2 : \IO_{\sigma'}
    \tau_2 \\
    \sigma \notsubtyp \sigma' \\ \sigma' \notsubtyp \sigma}
  {\Gamma \vdash e_1 \curlyvee e_2 : \IO_{\sigma \cup \sigma'} \ \tau_1 \times \tau_2}
\end{mathpar}
The combined premises $\sigma \notsubtyp \sigma'$ and
$\sigma' \notsubtyp \sigma$ ensure that the two heaps are distinct, as proved
in lemma~\ref{lem:distinctheap}. This restriction exists as we do not
want to allow programs that try to use the same resource concurrently,
such as
$$\textsf{readFile} \curlyvee \textsf{readFile}$$
We do however, want to allow programs that run two expressions that do
not share any resources, like
$$\textsf{readFile} \curlyvee \textsf{readNet}$$

The result of running two separate $\IO$ actions concurrently produces
an expression whose type is also an $\IO$ monad, but with the two
separate heaps joined. In the above example, we would be inferred the
type $\IO_{\textsf{File} \cup \textsf{Net}}$.

\begin{lemma}
  For any two heaps $\sigma$ and $\sigma'$, if $\sigma \notsubtyp \sigma'$ and $\sigma'
  \notsubtyp \sigma$ then the two heaps contain distinct resources.
\end{lemma}

\begin{figure}

\begin{mathpar}

\inferrule*[Right=Top]{ }{\sigma \subtyp \textsf{World}} \and
\inferrule*[Right=Refl]{ }{\sigma \subtyp \sigma} \and
\inferrule*[Right=Union1]{\sigma \subtyp \sigma''}{\sigma \cup \sigma' \subtyp \sigma''} \and
\hskip 1em
\inferrule*[Right=Union2]{\sigma' \subtyp \sigma''}{\sigma \cup \sigma' \subtyp \sigma''}

\end{mathpar}

\caption{Resource subtyping rules}
\label{fig:subtyping}
\end{figure}


\begin{figure}
  \begin{mathpar}
    \inferrule*[Right=Var]{x : \tau' \in \Gamma \\ \tau' > \tau}{\Gamma \vdash x : \tau} \and
    \inferrule*[Right=App]{\Gamma \vdash e : \tau \rightarrow \tau' \\ \Gamma \vdash e' : \tau}{\Gamma \vdash e \ e' : \tau'} \and
    \inferrule*[Right=Abs]{\Gamma,x : \tau \vdash e : \tau'} {\Gamma \vdash \lambda x . \ e : \tau \rightarrow
      \tau'} \and
    \inferrule*[Right=Let]{\Gamma \vdash e : \tau \\ \Gamma,x : \bar{\Gamma}(\tau) \vdash e' : \tau'}
    {\Gamma \vdash \mathsf{let} \ x = e \ \mathsf{in} \ e' : \tau'} \and
    % \inferrule*[Right=If]{\Gamma \vdash e_1 : \mathbf{Bool} \\ \Gamma \vdash e_2 : \tau \\ \Gamma \vdash e_3 : \tau}
    % {\Gamma \vdash \mathsf{if} \ e_1 \ \mathsf{then} \ e_2 \ \mathsf{else} \
    % e_3 : \tau} \and


    \inferrule*[Right=Product]{\Gamma \vdash e : \tau \\ \Gamma \vdash e' : \tau'}
    {\Gamma \vdash e \times e' : \tau \times \tau'} \and

    \inferrule*[Right=Lift]{\Gamma \vdash e : \tau}{\Gamma \vdash \llbracket e \rrbracket : \IO_\sigma \tau} \and
    \inferrule*[Right=Bind]{\Gamma \vdash e : \IO_\sigma \tau \\ \Gamma \vdash e' : \tau \rightarrow \IO_\sigma
      \tau'}{\Gamma \vdash e \bind e' : \IO_\sigma \tau'} \and
    \inferrule*[Right=Conc]{\Gamma \vdash e_1 : \IO_\sigma \tau_1 \\ \Gamma \vdash e_2 : \IO_{\sigma'}
      \tau_2 \\
      \sigma \notsubtyp \sigma' \\ \sigma' \notsubtyp \sigma}
    {\Gamma \vdash e_1 \curlyvee e_2 : \IO_{\sigma \cup \sigma'} \ \tau_1 \times \tau_2} \and
    \inferrule*[Right=ReadFile]{ }{\Gamma \vdash \mathsf{readFile} : \IO_{\mathsf{File}} ()} \and
    \inferrule*[Right=ReadNet]{ }{\Gamma \vdash \mathsf{readNet} : \IO_{\mathsf{Net}} ()} \and

    \inferrule*[Right=Subsumption]{\Gamma \vdash e : \IO_\sigma \tau \\ \sigma \subtyp \sigma'}
    {\Gamma \vdash e : \IO_{\sigma'} \tau} \and
    \inferrule*[Right=Unit]{ }{\Gamma \vdash () : ()}
  \end{mathpar}

  \caption{Typing rules}
\end{figure}

\section{Dynamic semantics}
Dynamic semantics model how the program actually executes at
runtime. It plays a crucial part in how we prove that the type system
is sound, as we can only guarantee that our type system ensures
error-free programs if we model what constitutes an error, and how it
propagates.

As shown in chapter~\ref{chapter:background}, the original work carried out in the Hindley-Damas-Milner systems were
based on denotational semantics.

\begin{figure}
  \begin{mathpar}
    % \inferrule*{x \in \eta}{\eta \vdash x \leadsto \Gamma(x)} \and
    % \inferrule*{x \notin \eta}{\eta \vdash x \leadsto \textsf{wrong}} \and
    % \inferrule*{ }{\eta \vdash \lambda x . e \leadsto [x,e,\eta]} \and
    % \inferrule*{\eta \vdash e_1 \leadsto [x,e_0,\eta'] \\
    %   \eta \vdash e_2 \leadsto v \\
    %   \eta', x \mapsto v \vdash e_0 \leadsto r}
    % {\eta \vdash e_1 e_2 \leadsto r } \and
    % \inferrule*{\eta \vdash e_1 \leadsto \llbracket v \rrbracket \\
    %   \eta \vdash e_2 \leadsto [x, e_0, \eta'] \\
    %   \eta, x \mapsto v \vdash e_0 \leadsto }{
    %   \eta \vdash e_1 \bind e_2 \leadsto \llbracket \rrbracket}
    \inferrule{e_1 \leadsto e_1'}{e_1 \ e_2 \leadsto e_1' \ e_2} \and
    \inferrule{e_2 \leadsto e_2'}{e_1 \ e_2 \leadsto e_1 \ e_2'} \and
    \inferrule{ }{ (\lambda x . e_1) e_2 \leadsto e_1 [ e_2 / x ] } \and
    \inferrule{ }{\letin{x = e_1}{e_2} \leadsto e_2 [ e_1 / x]} \and
    \inferrule{e_1 \leadsto \llbracket v \rrbracket}{e_1 \bind e_2 \leadsto e_2 v} \and
    \inferrule{e_1 \leadsto \llbracket v \rrbracket \\ e_2 \leadsto \llbracket w \rrbracket}{e_1 \curlyvee e_2 \leadsto \llbracket v \times w \rrbracket}
  \end{mathpar}
  \caption{Semantics}
\end{figure}

\begin{figure}
  \begin{mathpar}
    \inferrule{ }{ () \ \textsf{value}} \and
    \inferrule{ }{ e_1 \times e_2 \ \textsf{value} } \and
    \inferrule{ }{ \lift{e} \ \textsf{value} } \and
    \inferrule{ }{ \lambda x. e \ \textsf{value} }
  \end{mathpar}
  \caption{Terminal values}
\end{figure}

% Note that the subsumption is limited to only the heap of the $\IO$
% monad.

% The motivating example for the subtyping rules is that the program

% \begin{math}
%   \mathsf{if} \ x \ \mathsf{then} \ \mathsf{readFile} \ \mathsf{else} \
%   \mathsf{readNetwork}
% \end{math}

% Should have the type $\IO_{\mathsf{File} \cup \mathsf{Network}} ()$


% \section{Semantics}

% A denotational semantics is provided to prove certain properties of
% the language and its type inference.

% We want to be able to prove that the type system is sound -- that is
% that if our system infers the type for an expression $e : \tau$, then the
% expression does indeed semantically evaluate to a value in $\tau$.

% All values are modelled with an element inside the set of
% $\mathbb{V}$.
% It consists of values from various other sets, such as
% $\mathbb{B}_{\mathsf{Bool}}$, the set of values for a the
% $\mathsf{Bool}$ type.

% Semantic domain:

% \begin{align*}
%   \mathbb{V} &= \mathbb{B}_{()} + \mathbb{B}_{\textsf{Bool}} + \mathbb{F} +
%                \mathbb{W} \\
%   \mathbb{B}_{()} &= \{ \ () \ \} \\
%   \mathbb{B}_{\mathsf{Bool}} &=
%                                \{ \textsf{True}, \textsf{False},
%                                \bot_{\textsf{Bool}} \} \\
%   \mathbb{F} &= \mathbb{V} \rightarrow \mathbb{V} \\
%   \mathbb{W} &= \{ \ . \ \}
% \end{align*}


% We can use the syntax for semantic entailment, $\Vdash v : \tau$, which is
% defined as

% \begin{math}
%   v \in \mathbb{V}^\tau \iff \Vdash v : \tau
% \end{math}

% Whilst on the syntactic side, $\Gamma$ is the type environment, we also
% define a semantic environment $\rho$. It is is a map of variable
% identifiers to values.

% \begin{math}
%   \rho : \textsf{Id} \rightarrow \mathbb{V}
% \end{math}


% We use $\llbracket e \rrbracket$ notation to represent passing a
% syntactic argument $e$ to a semantic function.

% We say an environment respects a type environment $\rho : \Gamma$ if:
% \begin{math}
%   \rho : \Gamma \iff \forall x : \tau \in \Gamma \rho \llbracket x \rrbracket : \tau
% \end{math}


% $\mathbb{B}_{Bool} = \{ \textsf{true} \textsf{false} \bot_{bool} \}$

% Semantic function $\mathbb{E} : \mathsf{Expr} \rightarrow \mathsf{Env} \rightarrow \mathbb{V}$


% There is a conditional operator $\hookrightarrow : \mathbb{B}_{\textsf{Bool}} \rightarrow \mathbb{V} \rightarrow \mathbb{V} \rightarrow \mathbb{V}$, defined as follows.

% \begin{align*}
%   t \hookrightarrow v, v' =
%   \begin{cases}
%     v & \textsf{if} \ t = \textsf{true} \\
%     v' & \textsf{if} \ t = \textsf{false} \\
%     \bot_v & \textsf{if} \ t = \bot_t
%   \end{cases}
% \end{align*}

% \begin{align*}
%   \mathbf{E} \llbracket x \rrbracket \rho &= \rho(x) \\
%   \mathbf{E} \llbracket e \ e' \rrbracket \rho &= \textsf{isF} v_1 \hookrightarrow
%                                               {v_1}_{|F} v_2 \\
%   \mathbf{E} \llbracket \lambda x . \ e \rrbracket \rho &=
%   (\lambda v . \ \mathbf{E} \llbracket e \rrbracket \rho \{ v / x \} ) \
%                                                  \textsf{in} \ \mathbb{V}
% \end{align*}

% Semantic domain + semantic function = semantic algebra

\section{Type safety}

\begin{lemma}
  If $\sigma' > \sigma$ and $\Gamma,x : \sigma' \vdash e : \tau$, then $\Gamma, x : \sigma \vdash e : \tau$.
  \label{lem:generalInContext}
\end{lemma}
\begin{proof}
  Proved by Milner and Damas. Need to extend to this system?
\end{proof}

\begin{lemma}
  If $\Gamma \vdash e : \tau'$ and $\tau' > \tau$, then $\Gamma \vdash e : \tau$.
\end{lemma}
\begin{proof}
  We prove this by induction on the proof for $\Gamma \vdash e : \tau'$.
  \begin{description}
  \item[\textmd{\boxed{\textsc{Var}}}]
    From the premise we have some $\tau''$ where $x : \tau'' \in \Gamma$ and $\tau'' >
    \tau'$. By transitivity we then have $\tau'' > \tau$. And thus we apply
    \textsc{Var} again with $x : \tau'' \in \Gamma$ and $\tau'' > \tau$ to get $\Gamma
    \vdash e : \tau$. 
  \item[\textmd{\boxed{\textsc{App}}}]
    With $\Gamma \vdash e e' : \tau'$, the premises for this rule are $\Gamma \vdash e : \tau'' \rightarrow \tau'$ and $\Gamma \vdash e' :
    \tau''$.
    The definition of $\tau' > \tau$ is that there exists a
    substitution $S$ that replaces any bound type variables, such that
    $S(\tau') = \tau$.
    Therefore if $\tau' > \tau$, then $\tau'' \rightarrow \tau' > \tau'' \rightarrow \tau$ as we can use the
    same substitution $S$. 
    Now that we have $\tau'' \rightarrow \tau' > \tau'' \rightarrow \tau$ and $\Gamma \vdash e : \tau'' \rightarrow
    \tau'$, we can use the induction hypothesis to obtain $\Gamma \vdash e : \tau'' \rightarrow
    \tau$. From here we reapply \textsc{App} to get $\Gamma \vdash e e' : \tau$ as
    needed.
  \item[\textmd{\boxed{\textsc{Abs}}}]
    Our proof tree will look like
    \begin{mathpar}
      \inferrule{\Gamma,x : \tau_1' \vdash e : \tau_2'}{\Gamma \vdash \lambda x . e : \tau_1' \rightarrow \tau_2'}
    \end{mathpar}
    and we will have $\tau' = \tau_1' \rightarrow \tau_2'$. When we think of $\tau' > \tau$, we
    have $\tau_1' \rightarrow \tau_2' > \tau$. So there is some substitution that will
    bring us to $\tau$.

    But if you look at the structure of $\tau_1' \rightarrow \tau_2'$, the
    substitution must bring us to another type of
    $\tau_1 \rightarrow \tau_2$.  The substitution will either replace something in
    $\tau_1'$, something in $\tau_2'$, or both, but since $>$ is reflexive
    with an empty substitution, both $\tau_1' > \tau_1$ and
    $\tau_2' > \tau_2$ hold. Therefore it is enough to prove $\Gamma \vdash \lambda x . e :
    \tau_1 \rightarrow \tau_2$.

    We can proceed to use lemma~\ref{lem:generalInContext} with
    $\tau_1' > \tau_1$ and $\Gamma,x : \tau_1' \vdash e : \tau_2'$, to get
    $\Gamma, x : \tau_1 \vdash e : \tau_2'$.  And by applying the induction hypothesis
    with $\tau_2' > \tau_2$, we get $\Gamma, x : \tau_1 \vdash e : \tau_2$. From here we
    proceed by reapplying \textsc{Abs} to get $\Gamma \vdash \lambda x . e : \tau_1 \rightarrow
    \tau_2$.
  \item[\textmd{\boxed{\textsc{Let}}}]
    For $\tau' > \tau$, we have
    \begin{mathpar}
      \inferrule{\Gamma \vdash e_1 : \tau'' \\ \Gamma, x : \bar{\Gamma}(\tau'') \vdash e_2 :
        \tau'}{\Gamma \vdash \letin{x = e_1}{e_2} : \tau'}
    \end{mathpar}
    We just need to apply the induction hypothesis on the second
    premise, obtaining
    $\Gamma, x : \bar{\Gamma}(\tau'') \vdash e_2 : \tau$, and thus
    $\Gamma \vdash \letin{x = e_1}{e_2} : \tau$.
  \item[\textmd{\boxed{\textsc{Product}}}]
    Our expression will have some type $\Gamma \vdash e_1 \times e_2 : \tau_1' \times \tau_2'$, where $\tau_1' \times
    \tau_2' < \tau$. Similar to the case for \textsc{Abs}, due to the
    structure of $\tau_1' \times \tau_2'$, any substitution that will change it
    to $\tau$ will result in a type of $\tau_1 \times \tau_2$, for some $\tau_1$ and
    $\tau_2$ where $\tau_1' < \tau_1$ and $\tau_2' < \tau_2$. 
    Therefore if we can prove $\Gamma \vdash e_1 \times e_2 : \tau_1 \times \tau_2$, we will
    have proved $\Gamma \vdash e_1 \times e_2 : \tau$.
    
    Taking the premises of \textsc{Product}, we get $\Gamma \vdash e_1 : \tau_1'$
    and $\Gamma \vdash e_2 : \tau_2'$. Apply the induction hypothesis on both, giving
    $\Gamma \vdash e_1 : \tau_1$ and $\Gamma \vdash e_2 : \tau_2$, and reapply \textsc{Product} to
    arrive at $\Gamma \vdash e_1 \times e_2 : \tau_1 \times \tau_2$.
  \item[\textmd{\boxed{\textsc{Lift}}}]
    If we have $\Gamma \vdash \lift{e} : \IO_\sigma \tau'$ and $\tau' > \tau$, then we need to
    show $\Gamma \vdash \lift{e} : \IO_\sigma \tau$. Take the premise $\Gamma \vdash e : \tau'$,
    and use the induction hypothesis to get $\Gamma \vdash e : \tau$, and therefore
    the desired result.
  \item[\textmd{\boxed{\textsc{Bind}}}]
    Similarly to \textsc{Lift}, we need to show $\Gamma \vdash e \bind e' :
    \IO_\sigma \tau$, given $\Gamma \vdash e \bind e' : \IO_\sigma \tau'$ and $\tau' > \tau$.
    We have $\IO_\sigma \tau' > \IO_\sigma \tau$ as per the previous cases, and the
    premise $\Gamma \vdash e' : \tau'' \rightarrow \IO_\sigma \tau'$. Because there exists a
    substitution from $\IO_\sigma \tau'$ to $\IO_\sigma \tau$, there also exists a
    substitution from $\tau'' \rightarrow \IO_\sigma \tau'$ to $\tau'' \rightarrow \IO_\sigma \tau$, thus
    $\tau'' \rightarrow \IO_\sigma \tau' > \tau'' \rightarrow \IO_\sigma \tau$.

    Applying the induction hypothesis to the premise $\Gamma \vdash e' : \tau'' \rightarrow \IO_\sigma \tau'$,
    we get $\Gamma \vdash e' : \tau'' \rightarrow \IO_\sigma \tau$, and can then use \textsc{Bind}
    again to get $\Gamma \vdash e \bind e' : \IO_\sigma \tau$ as needed.
  \item[\textmd{\boxed{\textsc{Conc}}}]
    Our conclusion for this rule is $\Gamma \vdash e_1 \curlyvee e_2 : \IO_{\sigma \cup \sigma'} \tau_1'
    \times \tau_2'$. As per always, if $\IO_{\sigma \cup \sigma'} \tau_1' \times \tau_2' = \tau' > \tau$,
    then the only substitutions possible would have to fit into
    $\IO_{\sigma \cup \sigma'} \tau_1 \times \tau_2 = \tau$. And with this, $\tau_1' > \tau_1$ as well
    as $\tau_2' > \tau_2$. We then just need to apply the induction
    hypothesis to $\Gamma \vdash e_1 : \IO_\sigma \tau_1'$ and $\Gamma \vdash e_2 : \IO_\sigma \tau_2'$
    which will allow us to use \textsc{Conc} to arrive at $\Gamma \vdash e_1 \curlyvee
    e_2 : \IO_{\sigma \cup \sigma'} \tau_1 \times \tau_2$ as needed.
  \item[\textmd{\boxed{\textsc{Unit}}}]
    We have $\Gamma \vdash () : ()$. If $() = \tau'$ and $\tau' > \tau$, all
    substitutions on $()$ will result in $()$. Thus $\tau = ()$, and we
    have already shown $\Gamma \vdash () : ()$.
  \item[\textmd{\boxed{\textsc{ReadFile}, \textsc{ReadNet}}}]
    The same principle for \textsc{Unit} applies here to prove these
    cases.
  \item[\textmd{\boxed{\textsc{Subsumption}}}]
    For the conclusion $\Gamma \vdash e : \IO_{\sigma'} \tau'$, we again have that
    $\IO_{\sigma'} \tau' > \IO_{\sigma'} \tau$. Therefore there exists a substitution $S$
    from the former type to the latter, and by the definition of
    substitution on $\IO$, must be a substitution from $\tau'$ to $\tau$.

    Thus we can take the premise $\Gamma \vdash e : \IO_\sigma \tau'$, and knowing that
    $\tau' > \tau$ apply the induction hypothesis and \textsc{Subsumption}
    to arrive at $\Gamma \vdash e : \IO_{\sigma'} \tau$ as needed.
  \end{description}
\end{proof}

\begin{lemma}
  If $\Gamma \vdash e' : \tau'$ and $\Gamma, x : \tau' \vdash e : \tau$ then $\Gamma \vdash e [e' / x] : \tau$
\end{lemma}
\begin{proof}
  asdf
\end{proof}

\begin{theorem}[Preservation]
  If $\Gamma \vdash e : \tau$ and $e \leadsto e'$, then $\Gamma \vdash e' : \tau$
\end{theorem}

\begin{proof}
  We prove this via structural induction on $e \leadsto e'$:
  \begin{description}
  \item[\boxed{e_1 \ e_2 \leadsto e_1' \ e_2}] We know that
    $\Gamma \vdash e_1 \ e_2 : \tau$, now we need to prove
    $\Gamma \vdash e_1' \ e_2 : \tau$. For
    $e_1 \ e_2 \leadsto e_1' \ e_2$, we only have one small-step rule for it
    , which has the premise $e_1 \leadsto e_1'$. Also since
    $\Gamma \vdash e_1 \ e_2 : \tau$, the typing rule \textsc{App} gives us
    $\Gamma \vdash e_1 : \tau' \rightarrow \tau$ and $\Gamma \vdash e_2 : \tau'$.

    Now that we have $\Gamma \vdash e_1: \tau' \rightarrow \tau$ and $e_1 \leadsto e_1'$, we can use
    the induction hypothesis to show $\Gamma \vdash e_1' : \tau' \rightarrow \tau$. And once
    again by applying \textsc{App}, we end up with $\Gamma \vdash e_1' \ e_2 :
    \tau$ as needed.

  \item[\boxed{e_1 \ e_2 \leadsto e_1 \ e_2'}] This can be proved with the
    same method above.

  \item[\boxed{(\lambda x . e_1) e_2 \leadsto e_1 [e_2 / x]}] Our type is
    $\Gamma \vdash (\lambda x . e_1) e_2 : \tau$, and we need to show that
    $\Gamma \vdash e_1 [e_2 / x] : \tau$. 
    By applying \textsc{App} the premises
    are $\Gamma \vdash \lambda x . e_1 : \tau' \rightarrow \tau$ and $\Gamma \vdash e_2 : \tau'$. We can then apply
    \textsc{Abs} to get $\Gamma, x : \tau' \vdash e_1 : \tau$.
    Lemma~\ref{lemma:1} then gives us $\Gamma \vdash e_1 [e_2 / x] : \tau$ as
    needed.
    
  \end{description}
\end{proof}

\begin{theorem}[Progress]
  If $. \vdash e : \tau$ then either $e \textsf{val}$, or there exists an $e'$
  such that $e \leadsto e'$.
\end{theorem}


\begin{theorem}[Soundness]
  If $e$ is inferred type $\tau$, $e$ evaluates to a value in $\tau$. \\
  $\Gamma \vdash e : \tau \rightarrow \Gamma \Vdash e : \tau$
\end{theorem}

Theorem 2, completeness $\Gamma \Vdash e : \tau \rightarrow \Gamma \vdash e : \tau$

Theorem 3, principal types

Theorem 4, most specific heap for monad?
%%% Local Variables:
%%% TeX-master: "report"
%%% TeX-engine: luatex
%%% TeX-command-extra-options: "-shell-escape"
%%% End:
