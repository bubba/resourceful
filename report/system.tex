\chapter{A Formal Definition}\label{chapter:system}

In this chapter we will formalise our resource-tracking, monadic IO
language, based on the Hindley-Damas-Milner type system that we saw
in the previous chapter. Just like before we will define the syntax,
static semantics and dynamic semantics, but to avoid ambiguity we will
also redefine concepts from HDM in more detail.

\section{Syntax}

\setlength{\grammarparsep}{20pt plus 1pt minus 1pt} % increase separation between rules
\setlength{\grammarindent}{12em} % increase separation between LHS/RHS
\renewcommand{\syntleft}{}
\renewcommand{\syntright}{}

%\newcommand{\square}{\ensuremath{\raisebox{-0.35mm}{\square}}}

\def\defaultHypSeparation{\hskip .05in}

We begin by giving the grammar as shown in
figure~\ref{fig:grammar}. The expression language is an extension of
the lambda calculus with let polymorphism. As before we use $x$,
$\lambda x . e$ and $e \ e'$ to refer to variable lookup, abstraction and
application respectively, and $\letin{x = e}{e'}$ for polymorphically
binding $e$ to $x$ inside $e'$.

Our language also contains product types, $\tau \times \tau'$, which are the same
as tuples \texttt{(a, b)} in Haskell. New product types can be
introduced with $e \times e'$, and can similarly be eliminated with the
projection expressions $\pi_1 \ e$ and $\pi_2 \ e$. $\square$ is the unit type
(\mintinline{haskell}{()} in Haskell), and has the eponymous
constructor $\square$.

More interestingly we have the monadic additions. Unlike
Krishnaswami~\cite{krishnaswami2006} we do not separate the language
into expression and computation languages. $\lift{e}$, lifts a regular
expression into the $\IO$ monad (\mintinline{haskell}{return} in
Haskell), and $e \bind e'$ is the standard monadic sequencing or
binding operation.

Resources $r$ represent something that we want to keep track of and
prevent from being accessed concurrently, for example a file system or
database. The resources we chose for the grammar are such examples.

Heaps $\rho$ are then formed from resources. They differ from the notion
of heaps in separation logic, and are constructed from either single
resources or by merging other heaps.  They are used to tag the
resources that a computation in an $\IO$ monad might be
accessing. There is also the notion of a \textsf{World} heap which
encapsulates all possible resources and sub-heaps. It can be thought
of as the ordinary \mintinline{haskell}{IO} monad in Haskell, where
the outside world is one indivisible resource causing everything to be
executed sequentially.

$\use{r}{e}$, pronounced \textit{e using r}, lifts a value into the
$\IO$ monad similarly to $\lift{e}$, but creates a new heap with the
resource $r$ and tags the monad with it. Novel to our language is the
$e \curlyvee e'$ operator. It joins two monadic computations together into one
that uses both their resources, returning a new heap made by merging
$e$ and $e'$s heaps --- provided that they do not overlap and are well
formed. Heap well formedness will be defined later on.

\begin{figure}
\begin{grammar}

  <type $\tau$> ::= $\square$ | $\alpha$ | $\tau \rightarrow \tau'$ | $\tau \times \tau'$ | $\textsf{IO}_\rho \tau$
  
  <type scheme $\sigma$> ::= $\forall \alpha . \sigma$ | $\tau$

  <context $\Gamma$> ::= $\centerdot$ | $\Gamma , x : \sigma$

  <expression $e$> ::= $\square$ | $e \times e'$ | $\pi_1 \ e$ $\pi_2 \ e$
  \alt $x$ | $\lambda x . e$ | $e \ e'$ | $\letin{x = e}{e'}$
  % \alt $\textsf{if} \ e_1 \ \textsf{then} \ e_2 \ \textsf{else} \ e_3$
  \alt $\lift{e}$ | $\use{r}{e}$ | $e \bind e'$ | $e \curlyvee e'$

  <resource $r$> ::= \textsf{File} | \textsf{Network} |
  \textsf{Database} | \ldots

  <heap $\rho$> ::= $r$ | $\rho \cup \rho'$ | \textsf{World}

\end{grammar}
\caption{The grammar for our resourceful language.}\label{fig:grammar}
\end{figure}

\emph{Free type variables} are defined on types, type schemes and contexts as
follows:
\begin{align*}
  \ftv(\square) &= \emptyset \\
  \ftv(\alpha) &= \{ \alpha \} \\
  \ftv(\tau \rightarrow \tau') &= \ftv(\tau) \cup \ftv(\tau') \\
  \ftv(\IO_\rho \tau) &= \IO_\rho \ftv(\tau) \\
  \ftv(\forall \alpha_1 , \ldots , \alpha_n . \tau) &= \ftv(\tau) - \{ \alpha_1 , \ldots, \alpha_n \} \\
  \ftv(\Gamma) &= \bigcup_{x : \sigma \in \Gamma} \ftv(\sigma)
\end{align*}
\begin{samepage}
\emph{Free variables} are defined on terms --- be careful not to mix these up
with free type variables!
\[
\begin{aligned}
  \fv(x) &= \{ x \} & \fv(\lambda x . e) &= \fv(e) \setminus \{ x \} \\
  \fv(e \ e') &= \fv(e) \cup \fv(e') & \fv(\letin{x = e}{e'}) &= \fv(e) \cup
                                    (\fv(e') \setminus \{ x \}) \\
  \fv(\lift{e}) &= \fv(e) & \fv(e \bind e') &= \fv(e) \cup \fv(e') \\
  \fv(\square) &= \emptyset & \fv(\use{r}{e}) &= \fv(e) \\
  \fv(e_1 \times e_2) &= \fv(e_1) \cup \fv(e_2) & \fv(\pi_i e) &= \fv{(e)}_{i = 1,
                                          2} \\
  \fv(e_1 \curlyvee e_2) &= \fv(e_1) \cup \fv(e_2)
\end{aligned}
\]
\end{samepage}

\subsection{Substitution}
Substitution may seem self-explanatory, but it is important we define
it crystal clear. It plays a vital role in proving properties about
our system, and there exists subtle differences with how it is defined
on terms, types and type schemes. A substitution is a map from type
variables to types, written as $\tau'[\tau/\alpha]$ to replace the type the type
$\tau$ for the type variable $\alpha$ in the type $\tau'$.
\begin{align*}
  \square [\tau/\alpha] &= \square \\
  \alpha' [\tau/\alpha] &=
             \begin{cases}
               \tau & \mathsf{if} \ \alpha' = \alpha \\
               \alpha' & \mathsf{otherwise}
             \end{cases} \\
  (e \times e') [\tau/\alpha] &= e[\tau/\alpha] \times e'[\tau/\alpha] \\
  (\tau_1 \rightarrow \tau_2)[\tau/\alpha] &= (\tau_1[\tau/\alpha] \rightarrow \tau_2[\tau/\alpha]) \\
  (\IO_\rho \tau')[\tau/\alpha] &= \IO_\rho \tau'[\tau/\alpha]
\end{align*}
Substitution is associative, and a substitution that substitutes
multiple type variables at once can be written like
$[\tau_1/\alpha_1,\ldots,\tau_m/\alpha_m]$. It is also extended to type schemes, but note
that this only substitutes \emph{free} type variables.
\begin{align*}
  \forall \alpha' . \sigma[\tau/\alpha] &=
            \begin{cases}
              \forall \alpha' . \sigma & \mathsf{if} \ \alpha' = \alpha \\
              \forall \alpha' . \sigma[\tau/\alpha] & \mathsf{otherwise}
            \end{cases} \\
  \tau[\tau/\alpha] &= \tau[\tau/\alpha] \ \textsf{(Substituion on type)}
\end{align*}

\newcommand{\dom}{\operatorname{dom}}

It should not be confused with instantiation, where the \emph{bound}
type variables $\forall \alpha_1 , \ldots , \alpha_n$ are substituted inside the type of a
type scheme. Instantiation is not a function though, it is
a relation $\sigma > \tau$ in which we say $\sigma$ can be instantiated to $\tau$.
\begin{mathpar}
  \boxed{\sigma > \tau} \\
  \infer{\dom(s) = \{\alpha_1 , \ldots, \alpha_n\} \\ \tau'[s] = \tau}{\forall \alpha_1 , \ldots , \alpha_n
    . \tau' > \tau}
\end{mathpar}
The domain of a substitution, $\operatorname{dom}(s)$, is the set of
type variables that it will replace, for example:
\[ \dom([\tau_1/\alpha,\tau_2/\beta]) = \{\alpha, \beta\} \]
Instantiation can then be read as, if
there exists a substitution $s$ that substitutes exactly all the bound
type variables in the type scheme
$\forall \alpha_1, \ldots, \alpha_n . \tau'$ to give the type $\tau$, then
$\forall \alpha_1, \ldots, \alpha_n . \tau' > \tau$.

Furthermore we define the relation $\sigma > \sigma'$ on type schemes as well,
and say that $\sigma$ \emph{is more general than} $\sigma'$ if for all
$\tau$, $\sigma' > \tau \rightarrow \sigma > \tau$.

Substitution (not instantiation) is also defined on
contexts. Substitution on contexts only substitutes \emph{free} type
variables.
\begin{align*}
  \centerdot[\tau/\alpha] = \centerdot \\
  (\Gamma , x : \sigma) [\tau/\alpha] &= \Gamma[\tau/\alpha] , x : \sigma[\tau/\alpha]
\end{align*}

There also exist term-level substitutions which map variables to
other terms, and can be applied as follows:
\begin{align*}
  \square [v/\alpha] &= \square \\
  \alpha' [v/\alpha] &=
             \begin{cases}
               v & \mathsf{if} \ \alpha' = \alpha \\
               \alpha' & \mathsf{otherwise}
             \end{cases} \\
  (e \ e') [v/\alpha] &= e[v/\alpha] \ e'[v/\alpha] \\
  \lambda x . e [v/\alpha] &=
                  \begin{cases}
                    \lambda x . e & \mathsf{if} \ x = \alpha \\
                    \lambda x . (e [v/\alpha]) & \mathsf{otherwise}
                  \end{cases} \\
  \letin{x = e'}{e} [v/\alpha] &=
                               \begin{cases}
                                 \letin{x = (e' [v/\alpha])}{e} & \mathsf{if} \
                                 x = \alpha \\
                                 \letin{x = (e' [v/\alpha])}{(e [v/\alpha])}
                                 & \mathsf{otherwise}
                               \end{cases} \\  
  (e_1 \times e_2) [v/\alpha] &= e_1[v/\alpha] \times e_2[v/\alpha] \\
  (\pi_i \ e) [v/\alpha] &= {\pi_i (e [v/\alpha])}_{i = 1, 2} \\
  \lift{e} [v/\alpha] &= \lift{e[v/\alpha]} \\
  \use{r}{e} [v/\alpha] &= \use{r}{e[v/\alpha]} \\
  (e \bind e') [v/\alpha] &= e[v/\alpha] \bind e'[v/\alpha] \\
  (e_1 \curlyvee e_2) [v/\alpha] &= e_1[v/\alpha] \curlyvee e_2 [v/\alpha]
\end{align*}

\subsection{Barendregt's Variable Convention}

Sometimes, we will end up in a scenario with two separate expressions
such as $\lambda x . x$ and $(y \ x)$. We know that the $x$ inside the first
expression is distinct from the $x$ in the second expression, but when
working with proofs we will need to show this somehow. We will use the
Barendregt variable convention~\cite{barendregt1984} to deal with
this: If we want to show that the $x$ inside ${\lambda x . x}$ is
different from the $x$ in $(y \ x)$, we can use \emph{alpha
  conversion} to rename $x$ to $z$ and get ${\lambda z . z}$. This new
expression is in fact equivalent to ${\lambda x . x}$, and we can always
choose a new unique name to avoid collisions with any \emph{free
  variables}. The Barendregt variable convention says that whenever we
have a bound variable (an $x$ inside a $\lambda x . e$ or
$\letin{x=e'}{e}$), we can just assume that we have performed alpha
conversion on it so that the variable name is unique.

\section{Static Semantics}
Our static semantics begin with the syntax-directed
rules of the Hindley-Damas-Milner system for the typing relation $\Gamma \vdash e : \tau$.

\begin{mathpar}
  \inferrule*[Right=Var]{x : \sigma \in \Gamma \\ \sigma > \tau}{\Gamma \vdash x : \tau} \and
  \inferrule*[Right=App]{\Gamma \vdash e : \tau' \rightarrow \tau \\ \Gamma \vdash e' : \tau'}{\Gamma \vdash e \ e' : \tau} \\
  \inferrule*[Right=Abs]{\Gamma,x:\tau' \vdash e : \tau}{\Gamma \vdash \lambda x . e : \tau' \rightarrow \tau} \and
  \inferrule*[Right=Let]{\Gamma \vdash e' : \tau' \\ \Gamma,x : \overline{\Gamma}(\tau') \vdash e : \tau}
  {\Gamma \vdash \mathsf{let} \ x = e' \ \mathsf{in} \ e : \tau}
\end{mathpar}

Like before, $\overline{\Gamma}$ is defined as
\[ \overline{\Gamma}(\tau) = \forall \alpha_1 \ldots \alpha_n . \tau \
\textsf{where} \ \{ \alpha_1, \ldots, \alpha_n \} = \ftv(\tau) \setminus \ftv(\Gamma) \]
  
The typing rule for $\square$ expressions introduces the monomorphic type,
and the rule for products introduces $e \times e'$.
\begin{mathpar}
  \inferrule*[Right=Unit]{ }{\Gamma \vdash \square : \square} \and
  \inferrule*[Right=Product]{\Gamma \vdash e : \tau \\ \Gamma \vdash e' : \tau'}
    {\Gamma \vdash e \times e' : \tau \times \tau'}
\end{mathpar}
Product types also have two eliminators, which project out the inner
type.
\begin{mathpar}
  \infer*[Right=Proj1]{\Gamma \vdash e : \tau \times \tau'}{\Gamma \vdash \pi_1 e : \tau} \and
  \infer*[Right=Proj2]{\Gamma \vdash e : \tau \times \tau'}{\Gamma \vdash \pi_2 e : \tau'}
\end{mathpar}

Now we introduce the monadic parts of the language. Our language only
has one type of monad, the $\IO$ monad, which is parameterised by both
its \emph{heap} $\rho$ and its encapsulated type. Monadic values can be
introduced into the language with $\lift{e}$, which lifts a pure term into
\textbf{any} \emph{well formed} heap. 
\begin{mathpar}
  \infer*[Right=Lift]{\Gamma \vdash e : \tau \\ \textsf{ok} \ \rho}{\Gamma \vdash \lift{e} : \IO_\rho \tau}
\end{mathpar}
We need to be careful what heaps we allow terms to be lifted into, as
the entire point of this system is to avoid heaps containing duplicate
resources. It would be all too easy to introduce a nonsensical type
like $\IO_{\textsf{File} \cup \textsf{File}} \tau$ with \textsc{Lift}, if it
were not for the premise $\textsf{ok} \ \rho$.

$\textsf{ok} \ \rho$ is a new relation we define in
Figure~\ref{fig:heapwellformedness} to establish what heaps we
consider to be well formed, in a similar vein to
Krishnaswami~\cite{krishnaswami2006}. All heaps consisting of a single
resource are well formed, as well as heaps made by merging two other
well formed heaps \textit{that are distinct}, i.e.\ they do not share
any of the same resources.
\begin{figure}
  \centering
  \begin{mathpar}
    \boxed{\textsf{ok} \ \rho} \and
    \infer{ }{\textsf{ok} \ \textsf{World}} \and
    \infer{ }{\textsf{ok} \ r} \and
    \infer{
      \textsf{ok} \ \rho \\
      \textsf{ok} \ \rho' \\
      \rho \cap \rho' = \emptyset}
    {\textsf{ok} \ \rho \cup \rho'}    
  \end{mathpar}
  \caption{Rules for heap well formedness.}\label{fig:heapwellformedness}
\end{figure}

Whilst \textsc{Lift} lets us construct monads in any heap, we will
eventually want to have constructors for $\IO$ monads that use
specific resources. When first designing the system, we used
placeholder functions that just used fixed resources:
\begin{mathpar}
  \infer{ }{\Gamma \vdash \mathsf{readFile} : \IO_{\mathsf{File}} \square} \and
  \infer{ }{\Gamma \vdash \mathsf{readNetwork} : \IO_{\mathsf{Network}} \square}
\end{mathpar}
$\textsf{readFile}$ and $\textsf{readNet}$ are examples of typical
operations that can consume a specific resource --- their heap consists
of just a single resource. This was then generalised to
$\use{r}{e}$, which lifts any pure term into an $\IO$ monad with a heap
consisting of the resource $r$.
\begin{mathpar}
  \infer*[Right=Use]{\Gamma \vdash e : \tau}{\Gamma \vdash \use{r}{e} : \IO_r e}
\end{mathpar}
Note that it only uses a single resource, not any arbitrary heap. A
heap with just one resource is always well formed, so there is no need
for an $\textsf{ok} \ r$ premise.  \textsc{Use} is meant to be used to
annotate terms with that a computation needs for itself, whilst on the
other hand \textsc{Lift} is for bringing pure expressions into a
monadic computation.

Once we have an $\IO$ value, we can sequence computation by binding
it with a function that returns another $\IO$.
\begin{mathpar}
  \inferrule*[Right=Bind]{\Gamma \vdash e : \IO_\rho \tau' \\ \Gamma \vdash e' : \tau' \rightarrow \IO_\rho
    \tau}{\Gamma \vdash e \bind e' : \IO_\rho \tau}
\end{mathpar}
Note that the types of the two $\IO$s must use the same
resources. However, we want expressions such as
\[ \use{\textsf{File}}{\square} \bind \lambda x . \use{\textsf{Net}}{\square} \]
to also be well typed. In particular, we want the above expression to be of
type $\IO_{\textsf{File} \cup \textsf{Net}} \square$. But if \textsc{Bind}
requires the heaps to be the same, we must first somehow ``cast''
$\textsf{readFile}$ and $\textsf{readNet}$ to the same type.
This is the purpose of the \textsc{Sub} (subsumption) rule:
\begin{mathpar}
  \inferrule*[Right=Sub]{\Gamma \vdash e : \IO_{\rho'} \tau \\ \rho' \subtyp \rho \\
    \textsf{ok} \ \rho}
  {\Gamma \vdash e : \IO_\rho \tau}
\end{mathpar}
It lets something of type $\IO_{\rho'} \tau$ become a $\IO_\rho \tau$, provided
that the heap $\rho$ is a \textit{subheap} of $\rho'$. The subheap
rules for heaps, shown in figure~\ref{fig:subheap}, define the
$\subtyp$ relation.

\begin{figure}

\begin{mathpar}
\boxed{\rho' \subtyp \rho} \\
  
\inferrule*[Right=Top]{ }{\textsf{World} \subtyp \rho} \and
\inferrule*[Right=Refl]{ }{\rho \subtyp \rho} \and
\inferrule*[Right=UnionL]{\rho' \subtyp \rho}{\rho' \cup \rho'' \subtyp \rho} \and
\hskip 1em
\inferrule*[Right=UnionR]{\rho' \subtyp \rho}{\rho'' \cup \rho' \subtyp \rho}
\end{mathpar}

\caption{Subheap rules}\label{fig:subheap}
\end{figure}

Intuitively, a heap $\sigma$ can thought of being a subheap of another heap
$\sigma'$, if $\sigma'$ subsumes $\sigma$, similarly to how subtyping works. For
example, $\textsf{Net} \subtyp \textsf{Net} \cup \textsf{File}$, since
$\textsf{Net} \cup \textsf{File}$ ``overlaps'' with the heap
$\textsf{Net}$.

\begin{figure}
  \centering
  \tikz \graph[layered layout] {
    world/"\textsf{World}"; file/"\textsf{File}"; net/"\textsf{Net}";
    database/"\textsf{Database}"; databasenet/"$\textsf{Database} \cup \textsf{Net}$";
    filenet/"$\textsf{File} \cup \textsf{Net}$";
    world -> filenet;
    world -> databasenet;
    databasenet -> {database, net};
    filenet -> {file, net};
  };
  \caption{An example of some heaps and their subheap orderings.}
\end{figure}

If one views this relation as an ordering, then we have
$\textsf{World}$ defined as the least upper bound --- this represents
using all possible resources, and as mentioned earlier
$\IO_{\textsf{World}}$ can be thought of as the $IO$ monad in Haskell,
where sequencing interacts with the state of the entire world.  In a
sense this could be viewed as a form of subtyping, but by constraining
it to just resources and not actual types, we avoid the extra overhead
and complexity subtyping would normally give us in the presence of
type inference~\cite{dolan2017}.

The subheaping relation is both reflexive (by definition), and
transitive.
\begin{theorem}
  For all $a$, $b$, $c$, if $c \subtyp b$ and $b \subtyp a$ then $c
  \subtyp a$.
\end{theorem}
\begin{proof}
  By induction on $c \subtyp b$. See
  proof~\ref{proof:subheaptransitive} in the appendix.
\end{proof}

The purpose of this type system is to allow programs to be run
concurrently, but reject the ones that concurrently access the same
resource. In this regard, the \textsc{Conc} rule is the heart and soul
of the system. It takes two monadic expressions, merges their heaps
and returns a product of their two inner types, inside $\IO$.
\begin{mathpar}
  \infer*[Right=Conc]{
    \Gamma \vdash e_1 : \IO_{\rho_1} \tau_1 \\
    \Gamma \vdash e_2 : \IO_{\rho_2} \tau_2 \\
    \textsf{ok} \ \rho_1 \cup \rho_2}
  {\Gamma \vdash e_1 \curlyvee e_2 : \IO_{\rho_1 \cup \rho_2} \ \tau_1 \times \tau_2}
\end{mathpar}
The premises ensure that the new merged heap must be well formed, as we do not
want to allow programs that try to use the same resource concurrently,
such as
\[ \use{\textsf{File}}{\square} \curlyvee \use{\textsf{File}}{\square} \]
We do however, want to allow programs that run two expressions that do
not share any resources:
\[ \use{\textsf{File}}{\square} \curlyvee \use{\textsf{Net}}{\square} : \IO_{\textsf{File} \cup \textsf{Net}} \]

All together these rules define the typing relation, and are
displayed in full in Figure~\ref{fig:typingrules}. To give a better
idea of how they are used in practice, we will look at some examples
of various programs and their types.

\begin{figure}
  \begin{mathpar}
    \boxed{\Gamma \vdash e : \tau} \\
    
    \inferrule*[Right=Var]{x : \sigma \in \Gamma \\ \sigma > \tau}{\Gamma \vdash x : \tau} \and
    \inferrule*[Right=App]{\Gamma \vdash e : \tau' \rightarrow \tau \\ \Gamma \vdash e' : \tau'}{\Gamma \vdash e \ e' : \tau} \and
    \inferrule*[Right=Abs]{\Gamma,x : \tau \vdash e : \tau'} {\Gamma \vdash \lambda x . e : \tau \rightarrow
      \tau'} \and
    \inferrule*[Right=Let]{\Gamma \vdash e' : \tau' \\ \Gamma,x : \overline{\Gamma}(\tau') \vdash e : \tau}
    {\Gamma \vdash \mathsf{let} \ x = e' \ \mathsf{in} \ e : \tau} \and
    % \inferrule*[Right=If]{\Gamma \vdash e_1 : \mathbf{Bool} \\ \Gamma \vdash e_2 : \tau \\ \Gamma \vdash e_3 : \tau}
    % {\Gamma \vdash \mathsf{if} \ e_1 \ \mathsf{then} \ e_2 \ \mathsf{else} \
    % e_3 : \tau} \and

    \inferrule*[Right=Unit]{ }{\Gamma \vdash \square : \square} \\
    \inferrule*[Right=Product]{\Gamma \vdash e : \tau \\ \Gamma \vdash e' : \tau'}
    {\Gamma \vdash e \times e' : \tau \times \tau'} \and

    \inferrule*[Right=Proj1]{\Gamma \vdash e : \tau \times \tau'}{\Gamma \vdash \pi_1 : \tau} \and     
    \inferrule*[Right=Proj2]{\Gamma \vdash e : \tau \times \tau'}{\Gamma \vdash \pi_2 : \tau'} \\

    \infer*[Right=Lift]{\Gamma \vdash e : \tau \\ \textsf{ok} \ \rho}{\Gamma \vdash \lift{e} :
      \IO_\rho \tau} \and
    \infer*[Right=Use]{\Gamma \vdash e : \tau}{\Gamma \vdash \use{r}{e} : \IO_r \tau} \\
    \inferrule*[Right=Bind]{\Gamma \vdash e : \IO_\rho \tau' \\ \Gamma \vdash e' : \tau' \rightarrow \IO_\rho
      \tau}{\Gamma \vdash e \bind e' : \IO_\rho \tau} \\
    \infer*[Right=Conc]{
      \Gamma \vdash e_1 : \IO_{\rho_1} \tau_1 \\
      \Gamma \vdash e_2 : \IO_{\rho_2} \tau_2 \\
      \textsf{ok} \ \rho_1 \cup \rho_2}
    {\Gamma \vdash e_1 \curlyvee e_2 : \IO_{\rho_1 \cup \rho_2} \ \tau_1 \times \tau_2} \and

    \inferrule*[Right=Sub]{\Gamma \vdash e : \IO_{\rho'} \tau \\ \rho' \subtyp \rho \\
      \textsf{ok} \ \rho}
    {\Gamma \vdash e : \IO_\rho \tau}

  \end{mathpar}

  \caption{The typing rules for our resourceful language.}\label{fig:typingrules}
\end{figure}

\subsection{Examples}
\subsubsection{Monadic binding}
Here we show a proof tree for the expression
\[\letin{x = \lambda y . \lift{\square}}{(x \ \square) \bind \lambda z . \llbracket z
    \rrbracket_{\textsf{File}}}\]
The lift operator in $\lambda y . \lift{\square}$ can lift into any
heap. But the smallest heap we can use for this overall program is
\textsf{File}, which is needed by $\lambda z . \use{\textsf{File}}{z}$. So
when we choose a heap to lift $\square$ into, we lift it into \textsf{File}.
\begin{mathpar}
  \mprset {sep=1em}
  \infer{
    \infer{
      \infer{}{\centerdot, y : \alpha \vdash \square : \square}
    } {
      \centerdot , y : \alpha \vdash \llbracket \square \rrbracket : \IO_{\textsf{File}} \square
    } \\
    \textsf{ok} \ \textsf{File}
  }
  { \centerdot \vdash \lambda y . \lift{\square} : \alpha \rightarrow \IO_{\textsf{File}} \square \\ \mathbf{(1)} }
  \\
  \infer{
    \infer{
      \infer{}{\centerdot,x :\alpha \rightarrow \IO_{\textsf{File}} \square \vdash x : \alpha \rightarrow
        \IO_{\textsf{File}} \square}
      \\\\
      \infer{}{\centerdot, x : \alpha \rightarrow \IO_{\textsf{File}} \square \vdash \square : \square}
    }{\centerdot, x : \alpha \rightarrow \IO_{\textsf{File}} \square  \vdash x \ \square :
      \IO_{\textsf{File}} \square}
    \\
    \infer{
      \infer{
        \infer{z : \alpha \in \centerdot, x : \alpha \rightarrow \IO_{\textsf{File}} \square, z : \alpha \\ \alpha > \square}
        {\centerdot, x : \alpha \rightarrow \IO_{\textsf{File}} \square, z : \alpha \vdash z : \square}
      }
      {\centerdot, x : \alpha \rightarrow \IO_{\textsf{File}} \square , z : \alpha \vdash \llbracket z
        \rrbracket_{\textsf{File}} : \IO_{\textsf{File}} \square}
    }{
      \centerdot, x : \alpha \rightarrow \IO_{\textsf{File}} \square  \vdash \lambda z . \llbracket z
      \rrbracket_{\textsf{File}} : \IO_{\textsf{File}} \square
    }
  }{
    \centerdot, x : \alpha \rightarrow \IO_{\textsf{File}} \square \vdash (x \ \square) \bind \lambda z . \llbracket z
    \rrbracket_{\textsf{File}} : \IO_{\textsf{File}} \square \\ \mathbf{(2)}
  }
  
  \infer{ \mathbf{(1)} \\ \mathbf{(2)} } {\centerdot \vdash \letin{x = \lambda y
      . \lift{\square}}{(x \ \square) \bind \lambda z . \llbracket z \rrbracket_{\textsf{File}}} :
    \IO_{\textsf{File}} \square}
\end{mathpar}

\subsubsection{Let polymorphism}
This example doesn't contain any resourceful elements, but is just an
example of how let polymorphism allows the same variable to be given
different types at each call site. Note how the premises in
$\mathbf{(1)}$ and $\mathbf{(2)}$ assign $x$ different types.
\begin{mathpar}
  % app
  \infer{
    . , x : \alpha \rightarrow \alpha \vdash x : (\square \rightarrow \square) \rightarrow (\square \rightarrow \square) \\
    . , x : \alpha \rightarrow \alpha \vdash x : \square \rightarrow \square
  }
  {\centerdot, x : \alpha \rightarrow \alpha \vdash x \ x : \square \rightarrow \square \\ \mathbf{(1)}}
  \\
  \infer{
    . , x : \alpha \rightarrow \alpha \vdash x : \square \rightarrow \square \\
    . , x : \alpha \rightarrow \alpha \vdash \square : \square
  }
  {\centerdot, x : \alpha \rightarrow \alpha \vdash x \ \square : \square \\ \mathbf{(2)}}
  \\
  \infer{
    \infer{
      \infer{
        y : \alpha \in \centerdot , y : \alpha \\ \alpha > \alpha
      }{\centerdot , y : \alpha \vdash y : \alpha}
    }
    {\centerdot \vdash \lambda y . y : \alpha \rightarrow \alpha} \\
    % app
    \infer{
      \mathbf{(1)} \\ \mathbf{(2)}
    }{\centerdot, x : \alpha \rightarrow \alpha \vdash (x \ x) \ ( x \ \square) : \square}
  }
  { \centerdot \vdash \letin{x = \lambda y . y}{(x \ x) \ (x \ \square)} : \square}
\end{mathpar}

\subsubsection{Concurrency}
Here is our first example of accessing two resources concurrently ---
with a bit of imagination one can think of this as reading a file
from a disk whilst simultaneously fetching data over the network.
\begin{mathpar}
  \mprset {sep=1em}
  \infer{
    % app
    \infer{
      %abs
      \infer{
        % use
        \infer{
          \centerdot, x : \alpha \vdash x : \alpha \\
          \textsf{ok} \ \textsf{File}
        }{\centerdot, x : \alpha \vdash \llbracket x \rrbracket_{\textsf{File}} : \IO_{\textsf{File}} \alpha}
      }{\centerdot \vdash \lambda x . \llbracket x \rrbracket_{\textsf{File}} : \alpha \rightarrow \IO_{\textsf{File}} \alpha}
      \\
      \centerdot \vdash \square : \square
    }{\centerdot \vdash (\lambda x . \llbracket x \rrbracket_{\textsf{File}}) \ \square : \IO_{\textsf{File}} \square}
    \\
    \infer{ %use
      \centerdot \vdash \square : \square \\ \textsf{ok} \ \textsf{Net}
    }{\centerdot \vdash \llbracket \square \rrbracket_{\textsf{Net}} : \IO_{\textsf{Net}} \square}
    \\
    \textsf{File} \cap \textsf{Net} = \emptyset
  }
  {\centerdot \vdash (\lambda x . \llbracket x \rrbracket_{\textsf{File}}) \ \square) \curlyvee \llbracket \square \rrbracket_{\textsf{Net}} : \IO_{\textsf{File} \cup \textsf{Net}} \square \times \square}
\end{mathpar}

\subsubsection{Subsumption}
% \begin{minipage}{\textwidth}
Assume there is a function inside our context called
\textsf{writeFile} with the type $\square \rightarrow \IO_{\textsf{File}} \square$. We can
subsume its heap to be part of a larger heap, like
$\textsf{File} \cup \textsf{Net}$. We will show this with the expression
\[ \llbracket \square \rrbracket_{\textsf{File}} \curlyvee \llbracket \square \rrbracket_{\textsf{Net}}
\bind
\lambda x . \textsf{writeFile} \ (\pi_1 \ x)
\]
  \begin{mathpar}
    \Gamma = \centerdot, \textsf{writeFile} : \square \rightarrow \IO_{\textsf{File}} \square \\
    \infer{ %conc
      \infer{ %use
        \Gamma \vdash \square : \square
      }{\Gamma \vdash \llbracket \square \rrbracket_{\textsf{File}} : \IO_{\textsf{File}} \square} \\
      \infer{ %use
        \Gamma \vdash \square : \square
      }{\Gamma \vdash \llbracket \square \rrbracket_{\textsf{Net}} : \IO_{\textsf{Net}} \square} \\
      \textsf{File} \cap \textsf{Net} = \emptyset
    }
    {\Gamma \vdash \llbracket \square \rrbracket_{\textsf{File}} \curlyvee \llbracket \square \rrbracket_{\textsf{Net}} : \IO_{\textsf{File}
        \cup \textsf{Net}} \square \\ \mathbf{(1)}}
    \\
    \infer{ %app
      \Gamma , x : \square \times \square \vdash \textsf{writeFile} : \square \rightarrow \IO_{\textsf{File}} \square
      \\
      \infer{
        \Gamma , x : \square \times \square \vdash x : \square \times \square
      }{\Gamma , x : \square \times \square \vdash \pi_1 \ x : \square}
    }{\Gamma , x : \square \times \square \vdash \textsf{writeFile} \ (\pi_1 \ x) :
      \IO_{\textsf{File}} \square \\ \mathbf{(2)}}
    \\
    \infer{ %bind
      \mathbf{(1)} \\
      \infer{ %abs
        \infer*[Right=Sub]{ %sub
          \mathbf{(2)} \\
          \textsf{File} \geq: \textsf{File} \cup \textsf{Net} \\
          \textsf{ok} \ \textsf{File} \cup \textsf{Net}
        }{\Gamma, x : \square \times \square \vdash \textsf{writeFile} \ (\pi_1 \ x) : \IO_{\textsf{File} \cup \textsf{Net}} \square}
      }{\Gamma \vdash \lambda x . \textsf{writeFile} \ (\pi_1 \ x) : \IO_{\textsf{File} \cup \textsf{Net}} \square}
    }{\Gamma \vdash
      \llbracket \square \rrbracket_{\textsf{File}} \curlyvee \llbracket \square \rrbracket_{\textsf{Net}}
      \bind
      \lambda x . \textsf{writeFile} \ (\pi_1 \ x)
      : \IO_{\textsf{File} \cup \textsf{Net}} \square}
  \end{mathpar}

% \end{minipage}

\section{Dynamic semantics}
As shown in Chapter~\ref{chapter:background}, the dynamic semantics in
the original Hindley-Damas-Milner system was based on denotational
semantics. In our type system, we will use operational
semantics. Operational semantics are similar to what we have seen
before in the definition of the static semantics. We define a bunch of
inference rules, and from these build up
proofs. Tofte~\cite{tofte1988} had the idea of using operational
semantics for not just the typing rules, but also for the dynamic
semantics. We choose this approach over denotational semantics as it
unifies our approach to types and evaluation, and as Tofte said,
\textsl{``it seems a bit unfortunate that we should have to understand
  domain theory to be able to investigate whether a type inference
  system admits faulty programs''}.

\subsection{Values}
Before we can talk about how we evaluate a program, we need to define
what constitutes a fully evaluated program. That is, what terms are a
result of a completed computation. We define a unary relation called
\textsf{value}, and give rules describing what terms can be considered
values in Figure~\ref{fig:values}.

For example, we cannot evaluate the unit type any further, therefore
all $\square$s are considered values. The same goes for product types, but
only if both inner components are values themselves:
$\square \times \square$ is a finished value, but
$f \ e \times \square$ might still have evaluation left to do. A lambda on its
own is a value too. Without being applied to an argument it cannot be
evaluated any further: The computation inside of it is suspended. In a
similar fashion a lifted computation cannot be computed any further
\textit{on its own}. We will see later how binding can run this
computation, but by itself it will not evaluate to anything.

\begin{figure}
  \begin{mathpar}
    \boxed{\textsf{value} \ e} \\
    \inferrule{ }{ \textsf{value} \ \square } \and
    \inferrule{\textsf{value} \ e_1 \\ \textsf{value} \ e_2}{ \textsf{value} \ e_1 \times e_2 } \and
    \inferrule{ }{ \textsf{value} \ \lift{e} } \and
    \inferrule{ }{ \textsf{value} \ \use{r}{e} } \and
    \inferrule{ }{ \textsf{value} \ \lambda x. e }
  \end{mathpar}
  \caption{Terminal values}\label{fig:values}
\end{figure}

\subsection{Small-step semantics}
The approach to operational semantics we will be taking is
\emph{small-step operational semantics}. In small-step operational
semantics we define a step relation $a \leadsto b$ which says that in one
``step'', $a$ \textit{reduces to} $b$. $b$ might then go onto reduce
further if it is able to, or it could be a finished value. So
reduction can be thought of as evaluation of a program, bit by
bit. Small-step semantics differs from big-step semantics, where the
relation $a \Downarrow b$ says that at the end of the day, $a$ will reduce to
$b$, and $b$ will not reduce any further.

As an example, if an expression $e_1$ reduces to $e_1'$, i.e.\ $e_1 \leadsto
e_1'$, then we want the application $e_1 \ e_2$ to reduce
as well. We can write this as
\begin{mathpar}
  \inferrule{e_1 \leadsto e_1'}{e_1 \ e_2 \leadsto e_1' \ e_2}
\end{mathpar}

We call this type of reduction rule which takes smaller reductions and
updates it within a bigger structure, \xi-reduction. There are
\xi-reductions for other expressions with structure inside, namely
product types, the concurrent operator and the bind operator.

Another type of reduction rule is \beta-reduction, which comes from the
lambda calculus. When we apply an argument to an abstraction, we
substitute the bound variable inside the abstraction with the
argument. The \beta-reduction rule defines this.
\begin{mathpar}
  \inferrule{\textsf{value} \ e_2}{ (\lambda x . e_1) \ e_2 \leadsto e_1 [ e_2 / x ]}
\end{mathpar}

It is important to note the premise here that states the argument
being applied must be a value. This enforces a strict evaluation
order, since in order for the argument to be a value it must be
completely reduced. A lazily evaluated semantics might forgo this
extra requirement, so that the argument can be reduced after
substitution.

As mentioned earlier, lifted expressions $\lift{e}$ are suspended much
like lambdas, and so are values since they cannot reduce any
further \emph{on their own}. However, with the bind operator, the
value inside them can be extracted out and fed into an abstraction.
\begin{mathpar}
  \inferrule{ }{\lift{e} \bind e' \leadsto e' \ e}
\end{mathpar}
Unlike the semantic rule for \beta-reduction, there is no premise
enforcing that $\lift{e}$ is a value, since all lifted terms are
values anyway.

For an expression lifted into a resourceful $\IO$ monad with
$\use{\rho}{e}$, one might be tempted to just reduce this to a $\lift{e}$.
\begin{mathpar}
  \infer{ }{\use{r}{e} \leadsto \lift{e}}
\end{mathpar}
And we could then also define the reduction for concurrency like so:
\begin{mathpar}
\inferrule{ }{\lift{v} \curlyvee \lift{w} \leadsto \llbracket v \times w \rrbracket}
\end{mathpar}
However, the intermediate $\lift{e}$ can be confusing. The purpose of
the lift operator is to lift a pure value into any possible resource
bound monad. When we see a lift, we think of the typing judgement
\textsc{lift} that allows it to fit any heap, when in fact a use
should restrict what heaps it can go into. For these reasons, we
instead define its reduction identically to lift.
\begin{mathpar}
  \inferrule{ }{\use{r}{e} \bind e' \leadsto e' \ e}
\end{mathpar}
Concurrency is then defined as chaining together two binds, and
returning the lifted product of the two results.
\begin{mathpar}
  \inferrule{ }{v \curlyvee w \leadsto
    v \bind \lambda v . (w \bind \lambda w . \lift{v \times w})}
\end{mathpar}
This might seem like the opposite of concurrency --- executing the
computation in sequence --- but because our monad does not have any
state (see Section~\ref{section:modellingstate}),
\begin{samepage}
\[v \bind \lambda v . (w \bind (\lambda w . \lift{v \times w}))\]
is identical to
\[w \bind \lambda w . (v \bind (\lambda v . \lift{v \times w}))\]
\end{samepage}
In fact the previous
definition for concurrency as
$\lift{v} \curlyvee \lift{w} \leadsto \llbracket v \times w \rrbracket$ will have the same reduction steps at
the end of the day. In a real-world programming language
implementation, the evaluation would actually be implemented concurrently.

Although the $v \curlyvee w$ might evaluate to two separate $\bind$s, the
static semantics of concurrency and binding are different. For example, given
$\Gamma \vdash \lift{v} : \IO_\rho \tau$ we are allowed to bind $lift{v}$ with itself:
\[
\Gamma \vdash \lift{v} \bind \lambda v . (\lift{v} \bind (\lambda v . \lift{v \times v})) : \IO_\rho (\tau \times \tau)
\]
But with the concurrent operator, this is a type error, since the
premise $\textsf{ok} \ \rho \cup \rho$ does not hold: There is no type in any context
that can be given to $v \curlyvee v$.

\begin{figure}
  \begin{mathpar}
    \boxed{e \leadsto e'} \\

    % lambda calculus + HM
    \inferrule{e_1 \leadsto e_1'}{e_1 \ e_2 \leadsto e_1' \ e_2} \and
    \inferrule{e_2 \leadsto e_2'}{e_1 \ e_2 \leadsto e_1 \ e_2'} \and
    \inferrule{\textsf{value} \ e'}{ (\lambda x . e) \ e' \leadsto e [ e' / x ] } \and
    \inferrule{ }{\letin{x = e'}{e} \leadsto e [e' / x]} \\

    % product types
    \inferrule{e_1 \leadsto e_1'}{e_1 \times e_2 \leadsto e_1' \times e_2} \and
    \inferrule{e_2 \leadsto e_2'}{e_1 \times e_2 \leadsto e_1 \times e_2'} \and

    \inferrule{e \leadsto e'}{\pi_1 e \leadsto \pi_1 e'} \and
    \inferrule{e \leadsto e'}{\pi_2 e \leadsto \pi_2 e'} \and

    \inferrule{ }{\pi_1 (e_1 \times e_2) \leadsto e_1} \and
    \inferrule{ }{\pi_2 (e_1 \times e_2) \leadsto e_2} \and

    % monads
    \inferrule{e_1 \leadsto e_1'}{e_1 \bind e_2 \leadsto e_1' \bind e_2} \and
    \inferrule{ }{\lift{v} \bind e_2 \leadsto e_2 \ v} \and
    \inferrule{ }{\use{r}{e} \bind e' \leadsto e' \ e} \\

    % resource stuff
    \inferrule{ }{v \curlyvee w \leadsto v \bind \lambda v . (w \bind \lambda w . \lift{v \times
        w})} \
  \end{mathpar}
  \caption{Dynamic Semantics}\label{fig:dynamicsemantics}
\end{figure}

\subsection{The reflexive and transitive closure}

We can go a \textit{step} further and define $\twoheadrightarrow$ as the reflexive,
transitive closure of $\leadsto$. What does that mean? If $\leadsto$ is a binary
relation of two terms, then the transitive closure
$\twoheadrightarrow$ is a new relation that maintains all the relations of
$\leadsto$ and is transitive, i.e.\ if $a \twoheadrightarrow b$ and
$b \twoheadrightarrow c$, then $a \twoheadrightarrow c$.  A reflexive transitive closure extends this so
that for any $a$, $a \twoheadrightarrow a$.

Intuitively speaking, if the small-step inference rule $a \leadsto b$ says
that $a$ reduces to $b$ in exactly one step, then
$a \twoheadrightarrow b$ says $a$ reduces to $b$ in zero or more steps. Instead of
reducing one step at a time, it can go all the way.

\begin{samepage}
  For example, if we
have the expression ${(\lambda x . x \ \square) \ (\lambda y . y)}$ then have the
following relations:
\[ (\lambda x . x \ \square) \ (\lambda y . y) \leadsto (\lambda y . y) \ \square \leadsto \square \]
\[ (\lambda x . x \ \square) \ (\lambda y . y) \twoheadrightarrow \square\]
\end{samepage}
We do not use this relation to prove any properties about our system,
but it sheds light on how our expression language eventually reduces
to a value in the end.

\begin{figure}
  \begin{center}
    \tikz \graph[layered layout, grow=right, edges={
      decoration={snake,amplitude=0.5mm,segment length=2mm,post
        length=1mm},decorate}] {
      a -> b -> c
    };
    \qquad
    \tikz \graph[layered layout,grow=right,edges={->>}] {
      a ->[orient] b -> c;
      a ->[bend right] c;
      { [same layer] a, b, c };
      a ->[loop above] a;

      b ->[loop above] b;

      c ->[loop above] c;

      % a -> c;
      % b -> c;
      % a -> c;
    };
\end{center}
\caption{Left: the $\leadsto$ relation. Right: The reflexive, transitive
  closure of $\leadsto$, the reduction relation $\twoheadrightarrow$.}\label{fig:reduction}
\end{figure}

%%% Local Variables:
%%% TeX-master: "report"
%%% TeX-engine: luatex
%%% TeX-command-extra-options: "-shell-escape"
%%% End:

% LocalWords:  monadic HDM monomorphic Krishnaswami unary
