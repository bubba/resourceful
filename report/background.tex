\chapter{Background}\label{chapter:background}

\section{Separation logic}\label{sec:separationologic}
Separation logic~\cite{ohearn2019,reynolds2002} is an extension
to Hoare logic, a system for formalising properties about imperative
programs. Specifications can be formalised as code, preceded by a
precondition and succeeded by a postcondition.
\[
  {\color{gray} \{\textit{precondition}\}} \
  \textit{code} \
  {\color{gray} \{\textit{postcondition}\}}
\]
The preconditions and postconditions make assertions about what points
to what inside the \textit{heap}. For example, in the example that
O'Hearn gives a cyclic list is constructed from two pointers.
\begin{gather*}
  {\color{gray} \{ x \mapsto 0 * y \mapsto 0 \} }\\
  [x] = y; \\
  [y] = x; \\
  {\color{gray} \{ x \mapsto y * y \mapsto x \} }
\end{gather*}
The code sets the value of $x$ is set to the location of $y$, and
vice-versa. Separation logic extends Hoare logic with the notion of
splitting the heap. In the example above, instead of asserting
$x \mapsto y \wedge y \mapsto x$, the separation conjunction $*$ is used, which says
``$x$ points to $y$ \textit{and separately} $y$ points to $x$''. The
heap must be able to split into disjoint subheaps, such that
$x \mapsto y$ holds in one and $y \mapsto x$ in the other. This means that
$x$ and $y$ cannot point to the same location in the heap.
These preconditions and postconditions can be derived from so-called
small axioms, which for an imperative language might contain an axiom
for storing to the heap:
\[ {\color{gray} \{x \mapsto -\}} [x] = v {\color{gray} \{x \mapsto v\}} \]
The \textit{frame rule} allows a specification to be extended by
adding extra predicates that do not mention any variables used by the
code.
\[
  \infer{\{p\} \ \emph{code} \ \{q\}}
  {\{p * r\} \ \emph{code} \ \{ q * r \}} \
  \text{\parbox{2in}{where no variable
    occurring free in r is modified by \emph{code}}}
\]
This rule ends up being quite significant, as its the reason why
separation logic allows for programs to be reasoned about locally ---
the predicates involving the heap are shrunk so that they only contain
variables that the code modifies.

There is also the \emph{concurrency rule} for reasoning about
concurrency, which gives the precondition and postcondition needed for
two programs that are run in parallel.
\[
  \infer{ {\color{gray} \{p_1\}} \ \emph{code}_1 \ {\color{gray} \{q_1\}}
    \\
    {\color{gray} \{p_2\}} \ \emph{code}_2 \ {\color{gray} \{q_2\}}}
  { {\color{gray} \{p_1 * p_2\}} \ \emph{code}_1 \ || \ \emph{code}_2 \
    {\color{gray} \{q_1 * q_2\}}}
\]

Krishnaswami built upon separation logic with a higher-order ML like
language\cite{krishnaswami2006} with types. It utilised monadic
binding for sequencing, allowing imperative computation within an
otherwise pure language.

\section{Monadic effect systems}
As the language we will develop in Chapter~\ref{chapter:system} is
based around monads, it is a good idea to take a brief look at what
exactly they are, and dip our toes into the cool waters of category
theory.

A monad is often sardonically defined as
\begin{quote}
  a monoid in the category of endofunctors
\end{quote}
But that means nothing to me. Lets break this down:

A \textit{category} is a collection of objects (like a set), and
morphisms (like functions between sets). In the same way, Haskell has
\textsf{Hask} --- the category of types. Its objects are types such as
\mintinline{haskell}{Bool} and \mintinline{haskell}{Int}, and its
morphisms are functions taking a type and returning another.
\begin{minted}{haskell}
Bool, Int :: * -- Bool and Int are types
True :: Bool
42 :: Int
f : Bool -> Int -- a morphism between Bool and Int
\end{minted}
A \textit{functor} is a mapping between categories that
preserves the structure. That is, a functor maps objects and morphisms
from one category to objects and morphisms in another category.
In the same way, a \mintinline{haskell}{Functor} in Haskell maps a
type to another type, and its functions to other functions with
\mintinline{haskell}{fmap}. 
\begin{minted}{haskell}
Maybe :: * -> *
Maybe Bool :: * -- Maps Bool to Maybe Bool
fmap :: (a -> b) -> (f a -> f b) -- Maps (a -> b) to (f a -> f b)
\end{minted}

An \textit{endofunctor} is a \textit{functor} from one category to the
same category. That is, it maps objects and morphisms from one
category, to objects and morphisms in the same category. But what does
that mean in Haskell? Well, since \mintinline{haskell}{Functor} maps
types to types, we are mapping from \textsf{Hask} to \textsf{Hask} ---
all \mintinline{haskell}{Functor}s in Haskell are endofunctors!

A \textit{monad} is defined as:
\begin{enumerate}
\item An endofunctor $T : X \rightarrow X$
\item With a natural transformation to ``join'' $f : T(T(X)) \rightarrow T(X)$
\item And another natural transformation: $g : I \rightarrow T(X)$
\item That satisfy some properties (the monad laws)
\end{enumerate}

Or in Haskell,
\begin{minted}{haskell}
class Monad m a where
  join :: m (m a) -> m a
  return :: a -> m a
\end{minted}
However as you might be aware, the actual definition in Haskell uses
bind \mintinline{haskell}{>>=}, not the mathematical definition of
\mintinline{haskell}{join}. It can be implemented in terms of
\mintinline{haskell}{>>=} instead, and vice-versa using the fact that
a monad is also a functor.
\begin{minted}{haskell}
(>>=) :: m a -> (a -> m b) -> m b
join m = m >>= \x -> x
m >>= f = join (fmap f m)
\end{minted}

The parallel to a \textit{monoid} is drawn because a monoid is defined as:
\begin{enumerate}
\item A set
\item With an associative binary operation $f : X \times X \rightarrow X$
\item And an identity element: $g : \mathbf{1} \rightarrow X$
\item That satisfy some properties
\end{enumerate}

A \textit{monoid} is a set with an associative binary operation and an
identity element, that satisfies some laws.
Therefore a \textit{monoid in the category of
  endofunctors} is
\begin{enumerate}
\item An endofunctor $T : X \rightarrow X$
\item With an associative binary operation $f : T \times T \rightarrow T$
\item And an identity element $T$
\end{enumerate}
But an identity element can be considered a function like $g :
\mathbf{1} \rightarrow T$
And 


\section{Type-level programming within GHC}

\setminted[haskell]{breaklines}

Before creating the system in Chapter~\ref{chapter:system}, we
originally attempted to embed a resourceful system within GHC's type
system. Whilst the original Haskell 2010 language specification was
based on System F, there has been much
work~\cite{eisenberg2016}\cite{weirich2017} carried out to add
dependent types to GHC via a plethora of language extensions. Namely,
\texttt{TypeFamilies} and \texttt{GADT} provide a lot of the power
needed to carry out type-level programming, and we leveraged this
expressiveness to emulate a resource-tracking version of the
\mintinline{haskell}{IO} monad, dubbed \mintinline{haskell}{SubIO}. It
carried around a heap with a phantom type parameter:
\begin{minted}{haskell}
newtype SubIO (a :: [j]) b = SubIO (IO b)
  deriving (Functor, Applicative, Monad)
\end{minted}

We then defined a subclass of \mintinline{haskell}{Monad} that
extended it with a sequencing operator and concurrent operator, both
which merged the heap phantom type parameters together.
\begin{minted}{haskell}
class Monad (m j) => HeapMonad (m :: [x] -> * -> *) j where
  (>>>=) :: m j a -> (a -> m k b) -> m (j ** k) b
  (|||) :: m j a -> m k b -> m (j ** k) b
  (|||) x f = x >>>= const f
\end{minted}
The \texttt{**} operator was a type family that merged
together two heaps, and threw a type error if they overlapped.
\begin{minted}{haskell}
type family Overlap a b :: Bool where
  Overlap '[] b = 'False
  Overlap (a ': as) b = If (MemberP a b) 'True (Overlap as b)

type family a ** b :: [c] where
  (a ** b) = If (Overlap a b)
                (TypeError ('Text "Heaps overlap!"))
                (a :++ b)
\end{minted}
An instance for \mintinline{haskell}{SubIO} was given, and with it
some file operations from \mintinline{haskell}{System.IO}, transformed to accept
the file argument as a type argument, so it could be tracked in the heap.
\begin{minted}{haskell}
instance HeapMonad SubIO j where
  (SubIO x) >>>= f = SubIO (x >>= \z -> let (SubIO y) = f z in y)
  (SubIO x) ||| (SubIO y) = SubIO (forkIO (x >> pure ()) >> y)

readFile :: forall filePath. KnownSymbol filePath => SubIO '[filePath] String
readFile = fileOp Prelude.readFile

appendFile :: forall filePath. KnownSymbol filePath => String -> SubIO '[filePath] ()
appendFile x = fileOp (`Prelude.appendFile` x)

writeFile :: forall filePath. KnownSymbol filePath => String -> SubIO '[filePath] ()
writeFile x = fileOp (`Prelude.writeFile` x)

fileOp :: forall filePath a. KnownSymbol filePath => (FilePath -> IO a) -> SubIO '[filePath] a
fileOp f = let fp = symbolVal (Proxy :: Proxy filePath)
             in SubIO (f fp)
\end{minted}
All together, this meant that functions could be written with the
familiar do syntax, and ran concurrently with
\texttt{|||} --- but only if their heaps did not overlap.
\begin{minted}{haskell}
someIO = do
  writeFile @"foo.txt" "hello"
  s <- readFile
  return s

someMoreIO x = do
  appendFile @"bar.txt" x
  readFile >>= SubIO . putStrLn

concExample = runSubIO $ someIO ||| someMoreIO "blah"
--- this *should* type error
-- badConcExample = runSubIO $ someIO ||| someIO
\end{minted}
However we soon found out that overlapping instances don't always
throw a type error, as the phantom type parameter keeping track of the
heap was evaluated \textbf{lazily}. There was a workaround that
involved extracting the heap down to the term level and forcing
evaluation via \mintinline{haskell}{Data.Proxy} and
\mintinline{haskell}{seq}, but we think there are most likely better
ways to embed this within GHC's type system.

\section{Hindley-Damas-Milner}

In this section, we are going to look at the Hindley-Damas-Milner type
system~\cite{damas1982}. It is one of the first formalisations of a
polymorphic type system, originally made to formalise the type system
of the ML language. It was heavily influential at the time and still
continues to be so today, paving the way for the type systems System
F, and programming languages such Haskell, O-Caml and more.

\subsection{Syntax}

\def\defaultHypSeparation{\hskip .05in}
\newcommand{\letin}[2]{\mathsf{let} \ #1 \ \mathsf{in} \ #2}

\begin{figure}
  \begin{grammar}

    <type $\tau$> ::= $()$ | $\alpha$ | $\tau \rightarrow \tau'$
    
    <type scheme $\sigma$> ::= $\tau$ | $\forall \alpha . \sigma$

    <expression $e$> ::= $x$
    \alt $\lambda x . e$
    \alt $e \ e'$
    \alt $\mathsf{let} \ x = e \ \mathsf{in} \ e'$

    <context $\Gamma$> ::= $\centerdot$ | $\Gamma, x : \centerdot$

  \end{grammar}
  \caption{Grammar of the applicative language}\label{grm:applang}
\end{figure}

We define the grammar for a simple applicative language in
figure~\ref{grm:applang}.
It defines expressions, types and type schemes. The distinguishment
between type and type scheme is necessary so that quantifiers can only
appear at the top level. Quantifiers bind type variables.

Free type variables are type variables which have not be bound.
\begin{align*}
  \mathrm{fv}(\forall \alpha . \tau) &= \mathrm{fv}(\tau) \\ {\alpha} \\
  \mathrm{fv}(()) &= \{ \} \\
  \mathrm{fv}(\alpha) &= \{ \alpha \} \\
  \mathrm{fv}(\tau \rightarrow \tau') &= \mathrm{fv}(\tau) \cup \mathrm{fv}(\tau')
\end{align*}

A \textit{type environment} is a mapping from variables to type schemes. 
\[\Gamma : \mathsf{Variable} \ \rightarrow \sigma\]

A substitution $S$ maps type variables to types. A substitution of the
form $[\tau/\alpha]$ maps $\alpha$ to $\tau$.

It is extended to operate on types, such that every occurance of a
type variable in a type is substituted.
For instance, $(\alpha \rightarrow \alpha)[\tau/\alpha] = \tau \rightarrow \tau$ can be read as ``$(\alpha \rightarrow \alpha)$,
replacing every $\alpha$ with $\tau$.''

% \begin{minipage}{1.0\linewidth}
  Substitution is \textbf{associative}. It is defined on types as
  \begin{align*}
    () [\tau/\alpha] &= () \\
    \alpha' [\tau/\alpha] &=
                            \begin{cases}
                              \tau & \mathsf{if} \ \alpha' = \alpha \\
                              \alpha' & \mathsf{otherwise}
                            \end{cases} \\
    (\tau_1 \rightarrow \tau_2)[\tau/\alpha] &= (\tau_1[\tau/\alpha] \rightarrow \tau_2[\tau/\alpha])
  \end{align*}
% \end{minipage}

And on type schemes as
\[(\forall \alpha . \sigma)[\tau/\alpha] = \forall \tau . \sigma[\tau/\alpha]\]

We may write $S(\sigma)$ to apply an arbitrary substitution to a type
scheme.

$[\tau_1/\alpha_1, \ldots, \tau_n/\alpha_n]$ may be used to notate the composition of
substitutions $S_1 = [\tau_1/\alpha_1], \ldots, S_n = [\tau_n/\alpha_n]$, $S_1(\ldots(S_n(\sigma)))$.

We say a type $\tau$ is an \textit{instance} of a type $\tau'$,
written as $t > \tau'$ if there exists a substitution $S$ such that

\[ \tau > \tau' \rightarrow \exists S. S(\tau) = \tau' \]

Intuitively, this can be thought of as $\tau$ being more general than
type $\tau'$. For example,
\[ \alpha \rightarrow \beta > \alpha \rightarrow \alpha \]
with the subsitution $[\alpha/\beta]$, but there exists no substitution for
\[ \alpha \rightarrow \alpha \ngtr \alpha \rightarrow \beta \]

\section{Static semantics}

Static semantics define the typing rules.

Before we look at the Hindley-Damas-Milner system, designed for
type-inference, we should take a look at a simpler variant.

The simply-typed lambda caclulus is a variant of the lambda
calculus, and very much a pre-cursor to what we will be building up
to. It defined three typing rules:
\begin{mathpar}
  \inferrule{x : \tau \in \Gamma}{\Gamma \vdash x : \tau} \and
  \inferrule{\Gamma, x : \tau' \vdash e : \tau}{\Gamma \vdash \lambda x : \tau . e : \tau' \rightarrow \tau} \and
  \inferrule{\Gamma \vdash e_1 : \tau' \rightarrow \tau \\ \Gamma \vdash e_2 : \tau'}{\Gamma \vdash e_1 e_2 : \tau}
\end{mathpar}
These rules consist of some \textit{judgements} above and below a line. There
are some regular judgements you might have seen in regular set theory,
such as $x : \tau \in \Gamma$ --- $x : \tau$ is in $\Gamma$. But there are also typing
judgements, which tell us what type a term has in what context. For
example, $\Gamma \vdash e_1 : \tau' \rightarrow \tau$ can be read as ``given the context
$\Gamma$, the expression $e_1$ is of type $:\tau' \rightarrow \tau$''.

A typing rule tell us that from the judgements above the line,
called \textit{premises}, we are allowed to derive the judgement
below the line, known as the \textit{conclusion}. So to put it all
together, the rule
\begin{mathpar}
  \inferrule{\Gamma \vdash e_1 : \tau' \rightarrow \tau \\ \Gamma \vdash e_2 : \tau'}{\Gamma \vdash e_1 e_2 : \tau}
\end{mathpar}
can be read as: ``If $e_1$ has the type $\tau' \rightarrow \tau$ in the context $\Gamma$,
and if $e_2$ has the type $\tau'$ in the context $\Gamma$, then we can derive
that $e_1 e_2$ has the type $\tau$ also in $\Gamma$.

This particular rule defines how application of two terms should be
typed in the simply-typed lambda calculus. 

\subsubsection{Let polymorphism}

Consider the below expression with the context $\Gamma = \{ a :
\textsf{Int} \rightarrow \textsf{Bool}, b : \textsf{Int} \}$
\[ \lambda x . (x (a (x \ b)))\]

What type should be inferred for $x$? It is used as both an
$\textsf{Int} \rightarrow \textsf{Int}$ and a $\textsf{Bool} \rightarrow \alpha$, where $\alpha$ is
the type of the entire expression. One possibility is $x : \alpha \rightarrow \alpha$.

\[ \letin{x = \lambda y. y}{x (a (x \ b))} \]

In the original definition by Milner there exited separate rules for
instantiation and generalisation. In our system we have combined them
into \textsc{Var} and \textsc{Let} respectively, so that they are
\textit{syntax-directed} --- there is exactly one rule for each form of
expression. This will make it easier to prove soundness later on.

As a part of this, we define $\bar{\Gamma}$ to generalize a type $\tau$,
binding over any free type variables.
\[ \bar{\Gamma}(\tau) =^{\textsf{def}} \forall \alpha_1, \ldots, \alpha_n . \tau \
\textsf{where} \ \{ \alpha_1, \ldots, \alpha_n \} = \mathrm{fv}(\tau) \setminus \mathrm{fv}(\Gamma) \]

\begin{figure}
  \centering
  \begin{mathpar}
    \inferrule*[Right=Var]{e : \tau' \in \Gamma \\ \tau' > \tau}{\Gamma \vdash e : \tau} \and
    \inferrule*[Right=App]{\Gamma \vdash e : \tau' \rightarrow \tau \\ \Gamma \vdash e' : \tau'}{\Gamma \vdash e \ e' : \tau} \and
    \inferrule*[Right=Abs]{\Gamma,x:\tau' \vdash e : \tau}{\Gamma \vdash \lambda x . e : \tau' \rightarrow \tau} \and
    \inferrule*[Right=Let]{\Gamma \vdash e : \tau' \\ \Gamma,x : \bar{\Gamma}(\tau') \vdash e' : \tau}
    {\Gamma \vdash \mathsf{let} \ x = e \ \mathsf{in} \ e' : \tau}
  \end{mathpar}
  \caption{Type inference rules}\label{rules:hmtypeinference}
\end{figure}


\subsection{Dynamic Semantics}

Milner and Damas created a denotational semantics for the language.

The semantics is defined by a semantic algebra, which itself is
comprised of a \textit{semantic domain} and \textit{semantic
  equation}.

The semantic domain defines the possible values an expression in our
language can have. It is a \textbf{complete partial order} (often referred to
as a \textit{cpo}): A pair $(D, \sqsubseteq)$ of a set $D$ and a partial order
$\sqsubseteq$ (a function that orders elements in $D$, but not necessarily all
of them, hence the term \textit{partial}), such that:

\begin{enumerate}
\item there is a least element $\bot$
\item each directed subset $x_0 \sqsubseteq \ldots \sqsubseteq x_n \sqsubseteq \ldots$ has a least upper bound
  (lub)
\end{enumerate}

\begin{align*}
  \mathbb{V} &= \mathbb{B}_{()} + \mathbb{B}_{bool} + \mathbb{F} + \mathbb{W} \\
  \mathbb{F} &= \mathbb{V} \rightarrow \mathbb{V} \\
  \mathbb{W} &= \{ . \}
\end{align*}

Or visually,

\begin{center}
  \tikz \graph[layered layout] { "$\mathbb{V}$" ->
    { "$\mathbb{B}_{()}$" -> "$()$",
      "$\mathbb{B}_{\textrm{Bool}}$" -> {true, false},
      I} ->
    "$\bot$"; };
\end{center}


Function space $D \rightarrow E$

Coalesced sum $D + E$

Since this cpo $\mathbb{V}$ represents all possibly data values, we
can extract a subset of it to model the values of certain
types~\ref{shamirwadge77}.
A subset $I$ of our cpo $\mathbb{V}$ is called an ideal, iff it
satisfies the following properties:

\begin{enumerate}
\item it is downwards closed: $\forall v_0 \in V, v_1 \in V, v_0, v_0 \sqsubseteq v_1 \rightarrow
  v_0 \in I \rightarrow \v_1 \in I$.
  
\item it is closed under lubs of \omega-chains.
\end{enumerate}

Our function domain $\mathbb{F}$ is a map from $\mathbb{V}$ to
$\mathbb{V}$. Maps over ideals are defined as
$I \rightarrow I' \equiv \{ v \in V | v \in \mathbb{F} \ \mathsf{and} \ \forall v' \in I \
(v_{|\mathbb{F}})v' \in I' \}$.

With all the mechanisms in place, we can now define what it means for
a value to semantically be a type:

\[v \in \mathbb{V}^\tau \iff \vDash v : \tau\]

Note that $v : \tau$ is a relation, not a function -- a value can be a
member of multiple types, and this should be read as ``$v$ is a $\tau$''.

\subsubsection{Bottom}
$\bot$ ends up being very useful to represent values that don't exist.
Take for example the following program which doesn't terminate. 

\[\letin{x = \lambda y. y}{x x}\]

What type does should this program have? It should assume the
type of whatever is needed. For instance, we would expect this program
to have a type of $()$ as the argument is not used.

$$(\lambda z. ()) (\letin{x = \lambda y. y}{x x}) : ()$$

Since $\bot$ is a member of all ideals, this program is well typed.

\subsubsection{Some notation}

If $\mathbb{D} \subset \mathbb{V}$, and $d \in \mathbb{D}$, we will say $d \ \mathsf{in} \
\mathbb{V}$ to represent $d$ but treated as if its in $\mathbb{V}$. \\
We will then define the reverse

$$v | \mathbb{D} =
\begin{cases}
  d & \textsf{if} \ v = d \ \textsf{in} \ \mathbb{V} \ \textsf{for
    some} \ d \in \mathbb{D} \\
  \bot_{\mathbb{D}} & \textsf{otherwise}
\end{cases}
$$


\subsubsection{Evaluation function}
The semantic equation $\mathcal{E} : \mathsf{Expression} \rightarrow
\mathsf{Environment} \rightarrow \mathbb{V}$ lies at the heart of the semantics,
and defines how the syntax is evaluated.

\begin{align*}
  \mathcal{E} \llbracket x \rrbracket \eta
  &= \eta \llbracket x \rrbracket \\
  \mathcal{E} \llbracket e_1 e_2 \rrbracket \eta
  &=
    \begin{cases}
      \bot & \mathsf{if} \ v_1 = \bot \\
      (v_1 | \mathbb{F}) v_2 & \mathsf{if} \ v_1 \in \mathbb{F} \\
      \mathsf{wrong} & \mathsf{otherwise}
    \end{cases}
  \\
  & \quad \textsf{where} \ v_i = \mathcal{E} \llbracket e_i \rrbracket \eta , \ i = \{
    1, 2\} \\
  \mathcal{E} \llbracket \lambda x . \ e \rrbracket \eta
  &=
    (\lambda v . \ \mathcal{E} \llbracket e \rrbracket \eta [v / x ])
    \ \mathsf{in} \ \mathbb{V} \\
  \mathcal{E} \llbracket \textsf{let} \ x = e_1 \ \textsf{in} \ e_2 \rrbracket \eta
  &=
    \mathcal{E} \llbracket e_2 \rrbracket \ \eta [ \mathcal{E} \llbracket e_1 \rrbracket\rho / x ]
\end{align*}

Note that these evaluation rules are not as strict as the semantics
defined by Milner~\cite{milner1978} -- namely, that the second argument
of application and let binding is not checked if it is a \textsf{wrong}.

We introduce the notion of an environment $\eta : \mathsf{Variable} \rightarrow
\mathbb{V}$. It is a map of variables bound to values.

An environment $\eta$ can be said to \textit{respect} a type environment
$\Gamma$ if all bindings in $\Gamma$ can be found in $\eta$ with the same type.
$$\eta : \Gamma \iff \forall x : \tau \in \Gamma. \ \eta \llbracket x \rrbracket : \tau$$

$$\Gamma \vDash e : \tau \iff
\forall \eta. \ \eta : \Gamma \rightarrow \mathcal{E} \llbracket e \rrbracket \eta : \tau $$

An assertion of the form above is said to be \textit{closed} if there
are no free type variables in $\Gamma$ or $\tau$, and an assertion only holds
iff its closed instances hold.

Let $\overline{\mathbb{V}}$ be the set of all ideals in $\mathbb{V}$
that do not contain $\mathsf{wrong}$.

There is also a type evaluation function $\mathcal{T} : \mathsf{Type}
\rightarrow \mathsf{Valuation} \rightarrow \overline{\mathbb{V}}$

\begin{align*}
  \mathcal{T}\llbracket () \rrbracket\psi &= \mathbb{B}_{()} \\
  \mathcal{T}\llbracket \mathsf{Bool} \rrbracket \psi &= \mathbb{B}_{\mathsf{Bool}} \\
  \mathcal{T}\llbracket \alpha \rrbracket \psi &= \psi \llbracket \alpha \rrbracket \\
  \mathcal{T} \llbracket \tau \rightarrow \tau' \rrbracket \psi &= \mathcal{T}\llbracket \tau \rrbracket \psi \ \rightarrow \
                             \mathcal{T} \llbracket \tau' \rrbracket \psi
\end{align*}

\subsection{Correctness}

Both Milner~\cite{milner1978} and Damas~\cite{damas1982} proved semantic
soundness for the system.

TODO: Is this too trivial to be a lemma? The proof is pretty vacuous
\newtheorem{lemma}{Lemma}
\begin{lemma}[to be named]\label{lem:1}
  If $v : \tau$ and $\eta : \Gamma$, then $\eta[v/x] : \Gamma,x : \tau$
\end{lemma}
\begin{proof}
  If $\eta : \Gamma$, then $\forall x : \tau \in \Gamma \ \eta\llbracket x \rrbracket : \tau$.
  And if $v : \tau$ then $\eta[v/x] \llbracket x \rrbracket : \tau$.
  So $\eta[v/x] : \Gamma,x : \tau$, because for all $y : \tau' \in \Gamma,x : \tau$, either $y =
  x$ and the substitution evaluates to the right type, or $y \neq x$ but
  because $\eta : \Gamma$, we have $\eta \llbracket y \rrbracket : \tau'$.
\end{proof}

\newtheorem{theorem}{Theorem}
\begin{theorem}[Semantic Soundness]
  $\Gamma \vdash e : \tau \rightarrow \Gamma \vDash e : \tau$ \\
  If $e$ has type $\tau$, $e$ does actually evaluate to a value in $\tau$.
\end{theorem}
\begin{proof}
  We need to prove $\forall \eta. \eta : \Gamma \rightarrow \mathcal{E} \llbracket e \rrbracket \eta : \tau$. We can do
  this via induction on $e$:

  \begin{description}
  \item[\boxed{x}] This is the base case of the induction. Since
    $\Gamma \vdash x : \tau$ , the rule \textsc{Var} gives us
    $x : \tau \in \Gamma$. The evaluation function produces
    $\mathcal{E} \llbracket x \rrbracket \eta = \eta \llbracket x \rrbracket$, but because
    $\eta : \Gamma$, $x : \tau \in \eta$, so $\eta \llbracket x \rrbracket : \tau$.
  \item[\boxed{e_1 e_2}] The type given is
    $\Gamma \vdash e_1 e_2 : \tau$, and via \textsc{App} we have
    $\Gamma \vdash e_1 : \tau' \rightarrow \tau$ and
    $\Gamma \vdash e_2 : \tau'$.  By applying the induction hypothesis, we get
    $\Gamma \vDash e_1 : \tau' \rightarrow \tau$ and $\Gamma \vDash e_2 : \tau'$, so
    $v_1 \in \tau' \rightarrow \tau$ and therefore $v_1 \in \mathbb{F}$. \\
    This brings us to
    $\mathcal{E} \llbracket e_1 e_2 \rrbracket \eta = (v_1 | \mathbb{F}) v_2$. We can tell
    what the application of the $v_2$ will give us by looking at the
    definition for the ideal of $v_1$:
    ${\tau' \rightarrow \tau = \{ v \in \mathbb{V} | v \in \mathbb{F} \wedge \forall v' \in \tau' (v |
      \mathbb{F}) v' \in \tau \}}$.  So
    $(v_1 | \mathbb{F}) v_2 \in \tau$, as is required to show
    $\Gamma \vDash e_1 e_2 : \tau$.
  \item[\boxed{\lambda x . e}] Our type is
    $\Gamma \vdash \lambda x . e : \tau' \rightarrow \tau$ and our typing rule \textsc{Abs} tells us
    the antecedent is $\Gamma,x:\tau' \vdash e :
    \tau$. Evaluating our expression gives
    $\mathcal{E} \llbracket \lambda x . e \rrbracket \eta = (\lambda v. \mathcal{E} \llbracket e \rrbracket \eta[v / x]) \
    in \ \mathbb{V}$. \\
    If we can prove
    $\mathcal{E} \llbracket e \rrbracket \eta [v/x] : \tau$ where
    $v : \tau'$, then we can prove that
    ${\lambda v . \mathcal{E} \llbracket e \rrbracket \eta [v/x] : \tau' \rightarrow \tau}$. But note that for
    $\Gamma, x : \tau' \vdash e : \tau$, lemma~\ref{lem:1} tells us $\eta[v/x] : \Gamma,x : \tau'$.
    And with this fact, $\Gamma,x : \tau' \vDash e : \tau$ from the
    induction hypothesis gives us $\mathcal{E} \llbracket e \rrbracket \eta [v/x] : \tau$
    where $v : \tau'$, as needed.
  \item[\boxed{\letin{x = e_1}{e_2}}] $\Gamma \vdash \letin{x = e_1}{e_2} : \tau$, and
    \textsc{Let} tells us $\Gamma \vdash e_1 : \tau'$ and $\Gamma,x : \tau' \vdash e_2 : \tau$. The
    evaluation function for let expressions is
    ${\mathcal{E} \llbracket \letin{x = e_1}{e_2} \rrbracket \eta
    = \mathcal{E} \llbracket e_2 \rrbracket (\eta [\mathcal{E} \llbracket e_1 \rrbracket \eta / x ])}$. We know that
  $\mathcal{E} \llbracket e_1 \rrbracket \eta : \tau'$ because of the induction hypothesis on
  $\Gamma \vdash e_1 : \tau'$, and will refer to this value as $v_1 : \tau'$. \\
  From lemma~\ref{lem:1}, in $\Gamma,x : \tau' \vdash e_2 : \tau$ the substituted environment
  respects the type environment $\eta[v_1/x] : \Gamma,x : \tau'$.
  Therefore $\Gamma,x : \tau' \vDash e_2 : \tau$ tells us that
  $\mathcal{E} \llbracket e_2 \rrbracket (\eta [v_1/x]) : \tau$. This is
  identical to our hypothesis, and our proof is done.
  \end{description}
  
\end{proof}

%%% Local Variables:
%%% TeX-master: "report"
%%% TeX-engine: luatex
%%% TeX-command-extra-options: "-shell-escape"
%%% End:
