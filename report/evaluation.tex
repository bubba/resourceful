\chapter{Evaluation} \label{chapter:evaluation}

\section{Heap well formedness}
In the system defined in the previous chapter we have proved it is
sound, and moreover we can show that every concurrent expression has a
well typed heap -- that is a resource is not used more than once in the
heap, and thus resources are not accessed concurrently.

\begin{theorem}[Concurrenct heap is well formed]
  If $\Gamma \vdash e_1 \curlyvee e_2 : IO_\rho \tau$, then $\textsf{ok} \ \rho$
  \label{theorem:conc}
\end{theorem}
\begin{proof}
  By induction on the possible proofs for $\Gamma \vdash e_1 \curlyvee e_2 : IO_\rho \tau$.
  \begin{description}
  \item[\textmd{\boxed{\textsc{Conc}}}]
    We need to show $\textsf{ok} \ \rho_1 \cup \rho_2$, and from the premises we have $\textsf{ok} \ \rho_1 \cup \rho_2$.
  \item[\textmd{\boxed{\textsc{Sub}}}]
    We need to show $\textsf{ok} \ \rho'$ and from the premises, we also have $\textsf{ok} \ \rho'$. 
  \end{description}
\end{proof}
As you can see, this is straightforward to prove. Perhaps it is too
straightforward. What we would really like to prove is for all terms
$e$:
\newtheorem{conjecture}{Conjecture}
\begin{conjecture}
  If $\centerdot \vdash e : \IO_\rho \tau$, then $\textsf{ok} \ \rho$
  \label{conjecture:strongconc}
\end{conjecture}
This is a stronger version of theorem~\ref{theorem:conc}. However this
is not made easy in the system as it stands, for one simple yet
annoying reason: Types are not necessarily well formed, and
\textsc{Abs} allows any type to be introduced. This can be easily
shown with the identity function.

$$\inferrule*[Right=Abs]{
  \inferrule*[Right=Var]{x : \IO_{\textsf{File} \cup \textsf{File}} \tau  \in
    \centerdot , x : \IO_{\textsf{File} \cup \textsf{File}} \tau \\  \IO_{\textsf{File} \cup \textsf{File}} \tau >
    \IO_{\textsf{File} \cup \textsf{File}} \tau }{\centerdot , x : \IO_{\textsf{File} \cup
      \textsf{File}} \tau \vdash x : \IO_{\textsf{File} \cup \textsf{File}} \tau}
}
{\centerdot \vdash \lambda x . x : \IO_{\textsf{File} \cup \textsf{File}} \tau \rightarrow \IO_{\textsf{File} \cup \textsf{File}} \tau}
$$

It acts as a mechanism to generate types with malformed heaps, such as
an $\IO$ monad that contains two \textsf{File}s in its heap.  In
reality this is not an issue, as there is no way to construct a value
to pass into this lambda. Because of this I believe that conjecture~\ref{conjecture:strongconc}this is still
provable, but only when the context only contains well formed type
schemes, and hence why it only applies for the empty context.

We can extend this to include a relation for well-formed contexts and
well-formed types, and use this in our definition. 

\begin{mathpar}
  \boxed{\textsf{ok} \ \Gamma} \\
  \infer{ }{\textsf{ok} \ \centerdot} \and \infer{\textsf{ok} \ \Gamma \\ \textsf{ok}
    \ \sigma}{\textsf{ok} \ \Gamma , x : \sigma } \\
  \boxed{\textsf{ok} \ \sigma} \\
  \infer{\textsf{ok} \ \tau}{\textsf{ok} \ \forall \alpha_1 , \ldots , \alpha_n \cdot \tau} \\
  \boxed{\textsf{ok} \ \tau} \\
  \infer{ }{\textsf{ok} \ \square} \and
  \infer{ }{\textsf{ok} \ \alpha} \and
  \infer{\textsf{ok} \ \tau \\ \textsf{ok} \ \tau'}{\textsf{ok} \ \tau \rightarrow \tau'}
  \and
  \infer{\textsf{ok} \ \tau \\ \textsf{ok} \ \tau'}{\textsf{ok} \ \tau \times \tau'}
  \and
  \infer{\textsf{ok} \ \tau \\ \textsf{ok} \ \rho}{\textsf{ok} \ \IO_\rho \tau}
\end{mathpar}

\begin{conjecture}
  If $\textsf{ok} \ \Gamma$ and $\Gamma \vdash e : \IO_\rho \tau$, then $\textsf{ok} \ \rho$
\end{conjecture}
So far two lemmas have been proven to aid in the proof of the above
conjecture.
\begin{lemma}
  If $\textsf{ok} \ \tau$ then for any substitution $s$, $\textsf{ok} \ \tau[s]$
\end{lemma}
\begin{lemma}
  If $\textsf{ok} \ \sigma$ and $\sigma > \tau$, then $\textsf{ok} \ \tau$
\end{lemma}

This introduces a notion of ``well-typed types'' and adds quite a bit
of extra overhead to the system, so it is left as further work.

\section{Let-polymorphism}
All this theory was motivated by a very practical cause: to provide
some simple concurrency guarantees at the type level in a programming
language. Thus the intention is that some of this may eventually make
its way into a type system of an existing or novel language.

Because of this, rather than just using the simply typed lambda
calculus, the system was based off of the Hindley-Damas-Milner system
with its polymorphic type discipline. It cannot be understated the
amount of extra complexity and work this introduced into the proofs
for soundness, but polymorphism is a near-essential feature in todays
modern functional programming languages. The let polymorphism does not
interact much with the monadic and resourceful parts of the type
system, but it is desirable to show that the two are compatible with
each other regardless.

\section{Almost syntax directed}
As mentioned earlier, the system was based off a syntax directed
treatment of Hindley-Damas-Milner. That meant that there were only
four rules instead of the usual six, where instantiation and
generalisation are rolled into \textsc{Var} and \textsc{Let}
respectively. The main benefit of this was that it meant there was
exactly one typing judgement for each form of syntax, so constructing
proofs for a program was a breeze -- you just follow the corresponding rules
for each term. For example, in the expression $\letin{z = \lambda x . x}{z \
  z}$, a let consisting of an abstraction and an application, the proof begins with a
\textsc{Let}, then the premises \textsc{Abs} and \textsc{App} as shown
in figure~\ref{fig:syntaxdirected}.


\begin{figure}
  \centering
  \begin{mathpar}
    \infer*[Right=Let]{
      \infer*[Right=Abs]{
        \infer*[Right=Var]{x : \tau \in \Gamma \\\\ \tau > \tau}{\Gamma , x : \tau \vdash x : \tau}
      }{\Gamma \vdash \lambda x . x : \tau \rightarrow \tau} \\
      \infer*[Right=App]{
        \infer*[Right=Var]{z : \tau \rightarrow \tau \in \Gamma \\\\ \tau \rightarrow \tau > \tau \rightarrow \tau}{\Gamma , z : \tau \rightarrow \tau \vdash z : \tau \rightarrow \tau} \\\\
        \infer*[Right=Var]{z : \tau \rightarrow \tau \in \Gamma \\\\ \tau \rightarrow \tau > \tau \rightarrow \tau}{\Gamma , z : \tau \rightarrow \tau
          \vdash z : \tau \rightarrow \tau}
      }{\Gamma , z : \tau \vdash z \ z : \tau \rightarrow \tau}
    }
    {\Gamma \vdash \letin{z = \lambda x . x}{z \ z} : \tau \rightarrow \tau}
  \end{mathpar}
  \tikz \graph[tree layout] {
    let/"$\letin{z = \lambda x . x}{z \ z}$";
    abs/"$\lambda x . x$";
    app/"$z \ z$";
    var0/"$x$"; var1/"$z$"; var2/"$z$";
    let -> {abs, app};
    abs -> var0;
    app -> {var1, var2};
  };
  \qquad
  \tikz \graph[tree layout] {
    let/"\textsc{let}";
    abs/"\textsc{abs}";
    app/"\textsc{app}";
    var0/"\textsc{var}"; var1/"\textsc{var}"; var2/"\textsc{var}";
    let -> {abs, app};
    abs -> var0;
    app -> {var1, var2};
  };
  \caption{How synatx directed typing rules mean that the proof tree
    is isomorphic to the syntax tree.} \label{fig:syntaxdirected}
\end{figure}

In our system however, the subsumption rule \textsc{sub} prevents
this. For every typing judgement of the form $\Gamma \vdash e : \IO_\rho \tau$, there
are two possible proofs that can be constructed for it: the proof
corresponding to the syntax of $e$, and \textsc{sub}. Furthermore,
because the subheap relation is reflexive, its possible to have an
infinite proof tree with repeated applications of \textsc{sub} as
shown in figure~\ref{fig:infinitesub}. When
working with proofs this did not turn out to be too big of an
issue, but it is undoubtedly a little bit unsatisfying to lose such a
nice property of the system.

\begin{figure}
  \centering
  \begin{mathpar}
    \mprset {sep=1em}
    \infer*[Left=Sub]{
      \infer*[Left=Sub]{
        \infer*[Left=Sub]{
          \vdots \\
          \textsf{File} \subtyp \textsf{File} \\ \textsf{ok} \ \textsf{File}
        }{\Gamma \vdash \use{\textsf{File}}{\square} : \IO_{\textsf{File}}\square} \\
        \textsf{File} \subtyp \textsf{File} \\ \textsf{ok} \ \textsf{File}
      }{\Gamma \vdash \use{\textsf{File}}{\square} : \IO_{\textsf{File}}\square} \\
      \textsf{File} \subtyp \textsf{File} \\ \textsf{ok} \ \textsf{File}
    }{\Gamma \vdash \use{\textsf{File}}{\square} : \IO_{\textsf{File}} \square}
  \end{mathpar}
  \tikz \graph[tree layout] {
    "$\use{\textsf{File}}{\square}$" -> "$\square$";
  };
  \qquad
  \tikz \graph[tree layout] {
    sub/"\textsc{sub}" ->[loop above] sub;
  };
  \caption{How an infinite chain of \textsc{sub} can be produced,
    leading nowhere.} \label{fig:infinitesub}

\end{figure}


\section{Modelling state within the monad} \label{section:modellingstate}

\section{If statements}

\section{Heap polymorphism}
It would be convenient if we were able to have expressions like the
following well-typed:
\begin{align*}
\Gamma &= \centerdot , f : \square \rightarrow \IO_{\textsf{File}} \square, g : \square \rightarrow \IO_{\textsf{Net}} \square \\
\Gamma &\vdash \letin{x = \llbracket \square \rrbracket}{(x \bind f) \curlyvee (x \bind g)} : \IO_{\textsf{File} \cup
  \textsf{Net}} \square \times \square
\end{align*}
This is not possible in the current type system: $x$ needs to be both
$\IO_{\textsf{File}}$ and $\IO_{\textsf{Net}}$ simultaneously! This is
very much the same question that Hindley, Damas and Milner set out to
solve, except instead of polymorphic \textit{types} we want to have
polymorphic \textit{heaps}.
The basic idea for this would involve extending the definition of a
heap to allow for variable lookup:
\begin{grammar}
  <heap $\rho$> ::= $r$ | $\sigma \cup \sigma'$ | \textsf{World} | $\textbf{h}$
\end{grammar}
And then redefining substitutions so that it is a pair of a map from
type variables to types, and a map from heap variables to heaps.
$$ S : ( \alpha \mapsto \tau \times h \mapsto \rho ) $$
Then in the definition of substitution, whenever a mapping from a heap
variable to heap is encountered we would apply the substitution.
\begin{align*}
  (\IO_{\rho'} \tau)[h/\rho] &= \IO_{\rho'[h/\rho]} \tau \\
  r[h/\rho] &= r  && \textsf{Definining substitution on heaps now} \\
  (\rho_1 \cup \rho_2)[h/\rho] &= \rho_1[h/\rho] \cup \rho_2[h/\rho] \\
  \textsf{World}[h/\rho] &= \textsf{World} \\
  h'[h/\rho] &=
            \begin{cases}
              \rho & \mathsf{if} h' = h \\
              h' & \mathsf{otherwise}
            \end{cases}
\end{align*}

%%% Local Variables:
%%% TeX-master: "report"
%%% TeX-engine: luatex
%%% TeX-command-extra-options: "-shell-escape"
%%% End:
