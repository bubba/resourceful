% -*- fill-column: 80; TeX-command-extra-options: "-shell-escape" -*-
\documentclass{report}
\usepackage{minted}
\usepackage{syntax}
\usepackage{mathpartir}
\usepackage{amssymb}
\usepackage{amsmath}
\usepackage{amsthm}
\usepackage{tikz}
\usetikzlibrary{graphs,graphdrawing}
\usegdlibrary{layered}
\begin{document}

\newcommand{\llbracket}{[\![}
\newcommand{\rrbracket}{]\!]}
\newcommand{\IO}{\mathsf{IO}}
\newcommand{\bind}{>\!\!>\!\!=} \newcommand{\concbind}{>\!\!>\!>\!\!=}
\newcommand{\subtyp}{\geq:}
\newcommand{\notsubtyp}{\ngeq:}
\newcommand{\lift}[1]{\ensuremath{\llbracket#1\rrbracket}}


\chapter{Introduction}
The immutability and pureness of certain functional languages make them seem
like a perfect fit for parallelism and concurrency.
However, in real world applications code is very rarely completely pure -- State
and IO interactions.

The most popular solution today is to capture the entire state of the outside
universe and write functions that operate on state. That way any particle that
may have been perturbed by writing to \texttt{stdout} is appropriately reflected.

Concurrent Clean models this by threading the World in and out of
functions. Uniqueness types guarantee that the same world is only used once, so
that the programmer does create an alternative timeline by duplicating it.

\begin{minted}{clean}
readFile :: !String !*World -> (!MaybeError FileError String,!*World)
\end{minted}

Haskell, Idris (ML?) treat the World as state. Any function that interacts with
the World returns a function, which returns a new version of the World alongside
the function result.
\footnote{The actual definition in GHC is \\ \mintinline{haskell}{newtype IO a = IO (State# RealWorld -> (# State# RealWorld, a #))}}

\mint{haskell}|type IO a = World -> (World, a)|

This can then be easily fit into a monad, allowing for sequencing of multiple IO
actions.

\begin{minted}{haskell}
class Monad IO where
  return x = \w -> (w, x)
  x >>= f = \w -> let (w', y) = x w in f y w'
\end{minted}

Now IO actions can be easily chained together in a type-safe way that ensures
their ordering.

This ordering however, imposes limitations. One of the main benefits of pure
functional languages, is that since expressions do not have side effects, there
is no restriction on what order they need to be evaluated in.

Take for example the following snippet.
\begin{minted}{haskell}
f, g, h :: Int -> Int
f x = g x + h x
\end{minted}

\texttt{g} could be evaluted before \texttt{h}, or \texttt{h} could
be evaluted before \texttt{g}. It wouldn't make a difference because there
are no side effects. One might be tempted then to evaluate the two expressions
concurrently -- and that would be safe.

The same cannot be said for impure IO actions -- and the type system prevents that.
\begin{minted}{haskell}
f, g, h :: Int -> IO Int
f x = g x >>= \y -> y + h x
\end{minted}
We need to explicitly bind the actions and sequence evaluation.
Does this mean that concurrency is not possible for IO actions? No, as many
languages provide primitives to run these actions concurrently in a type safe
way. Haskell has \mintinline{haskell}{forkIO}, but for simpliclty we are going
to assume a higher level function for running two IO actions simultaneously and
collecting the results.

\begin{minted}{haskell}
concurrently :: IO a -> IO b -> IO (a, b)
\end{minted}

Now we can use it to run our two IO actions side by side safely. 
\begin{minted}{haskell}
f, g, h :: Int -> IO Int
f x = g x `concurrently` h x >>= \(a, b) -> return (a + b)
\end{minted}

But what if \texttt{g} and \texttt{h} actually looked like this?

\begin{minted}{haskell}
readFile :: FilePath -> IO String
writeFile :: FilePath -> String -> IO ()

g x = do
  txt <- readFile "foo.txt"
  return (x + (read txt))
h x = do
  writeFile "foo.txt" "hello"
  return (42 - x)
\end{minted}

Running these two functions concurrently could be disastrous as the order in
which they execute could affect the outcome of the program, and all of a sudden
some innocuous looking IO functions end up introducing non-determinism and
tricky race conditions into our program.

We know statically, that a program such as \mint{haskell}|g x `concurrently` h
x| should probably not be allowed. But then why did the type system allow it?
Has it failed us? The goal of the type system is to disallow as many incorrect
programs as possible while allowing as many correct programs as possible. It is
a fine line as to what programs are defined correct and what are defined as
incorrect -- a type system too lenient and buggy programs will creep through. A
type system too strict and the programmer will end up wasting time fighting the
type checker.

In this piece of research however, we are going to investigate the point in the
design space that rejects such programs. We do \textbf{not} want to allow
programs that have glaring race conditions, where we can see that there is a
contentious access of resources.

The genesis of the rest of this work is based around the idea of modelling the
resources in use at the type level. We begin by adding another type parameter to
our IO type to represent what resource an IO action uses:

\mint{haskell}|type IO r a = World -> (World, a)|

This is a phantom type parameter, as it only exists at the type level. Now our
type signatures could look like this, annotating the API with what resources it
might use.

\begin{minted}{haskell}
data Resource = FileSystem | Net | Database | OpenGL | ...
readFile :: FilePath -> IO FileSystem ()
writeFile :: FilePath -> IO FileSystem String
readSocket :: Socket -> IO Net ()
runQuery :: Query a -> IO Database a
swapBuffers :: IO OpenGL ()
\end{minted}

Keep in mind we are painting in broad strokes when we use the word
``resource''. In the running example the resource is a file,
\texttt{foo.txt}. But the notion of a resource can be as broad or as specific as
the author of a function needs it to be. It could represent a specific database
instance, or a single network socket. For simplicity in our example we will
consider the entire file system as a single resource.

Now that we know what resources each $\IO$ action is using, we would like to
change the type of our concurrent function to take advantage of this new
information. Perhaps we would like to reject any two functions that use the same
resource, i.e. it only accepts $\IO$ actions with distinct resources.

\mint{haskell}|concurrently :: r /~ s => IO r a -> IO s b -> IO ? (a, b)|

You can read $r /~ s$ as ``r is distinct from s'', or the opposite of a
\texttt{r ~ s} equality constraint that one might see in a type signature. This
of course however, does not exist in Haskell.  And what does it exactly mean for
two resources to ``be distinct''? And what resources would the returned $\IO$
use?

These are questions that are answered in chapter~\ref{chapter:system} with a
formal definition of a type system that tracks resource usage. We explore a
specific point in the design space, where the type system rejects programs like
$$
\textsf{readFile} \curlyvee \textsf{readFile}
$$
But accepts and assigns types to programs such as
$$
\textsf{readFile} \curlyvee \textsf{readNet} : \IO_{\textsf{File} \cup \textsf{Net}} \ () \times ()
$$


\chapter{Background} \label{chapter:background}

In this chapter, we are going to look at the Hindley-Damas-Milner type
system. It is one of the de-facto formalisations of a polymorphic type
system, based off of the ML programming language. It was heavily
influential at the time and still continues to be so today, inspiring
the design of Haskell, O-Caml and many programming languages.

\section{Hindley-Damas-Milner}
\subsection{Syntax}

\def\defaultHypSeparation{\hskip .05in}
\newcommand{\letin}[2]{\mathsf{let} \ #1 \ \mathsf{in} \ #2}

\begin{figure}
  \begin{grammar}

    <type $\tau$> ::= $()$ | $\alpha$ | $\tau \rightarrow \tau'$
    
    <type scheme $\sigma$> ::= $\tau$ | $\forall \alpha . \sigma$

    <expression $e$> ::= $x$
    \alt $\lambda x . e$
    \alt $e \ e'$
    \alt $\mathsf{let} \ x = e \ \mathsf{in} \ e'$

    <context $\Gamma$> ::= $\centerdot$ | $\Gamma, x : \centerdot$

  \end{grammar}
  \caption{Grammar of the applicative language}
  \label{grm:applang}
\end{figure}

We define the grammar for a simple applicative language in
figure~\ref{grm:applang}.
It defines expressions, types and type schemes. The distinguishment
between type and type scheme is necessary so that quantifiers can only
appear at the top level. Quantifiers bind type variables.

Free type variables are type variables which have not be bound.
\begin{align*}
  \mathrm{fv}(\forall \alpha . \tau) &= \mathrm{fv}(\tau) \\ {\alpha} \\
  \mathrm{fv}(()) &= \{ \} \\
  \mathrm{fv}(\alpha) &= \{ \alpha \} \\
  \mathrm{fv}(\tau \rightarrow \tau') &= \mathrm{fv}(\tau) \cup \mathrm{fv}(\tau')
\end{align*}

A \textit{type environment} is a mapping from variables to type schemes. 
$$\Gamma : \mathsf{Variable} \ \rightarrow \sigma$$

A substitution $S$ maps type variables to types. A substitution of the
form $[\tau/\alpha]$ maps $\alpha$ to $\tau$.

It is extended to operate on types, such that every occurance of a
type variable in a type is substituted.
For instance, $(\alpha \rightarrow \alpha)[\tau/\alpha] = \tau \rightarrow \tau$ can be read as ``$(\alpha \rightarrow \alpha)$,
replacing every $\alpha$ with $\tau$.''

% \begin{minipage}{1.0\linewidth}
  Substitution is \textbf{associative}. It is defined on types as
  \begin{align*}
    () [\tau/\alpha] &= () \\
    \alpha' [\tau/\alpha] &=
                            \begin{cases}
                              \tau & \mathsf{if} \ \alpha' = \alpha \\
                              \alpha' & \mathsf{otherwise}
                            \end{cases} \\
    (\tau_1 \rightarrow \tau_2)[\tau/\alpha] &= (\tau_1[\tau/\alpha] \rightarrow \tau_2[\tau/\alpha])
  \end{align*}
% \end{minipage}

And on type schemes as
$$(\forall \alpha . \sigma)[\tau/\alpha] = \forall \tau . \sigma[\tau/\alpha]$$

We may write $S(\sigma)$ to apply an arbitrary substitution to a type
scheme.

$[\tau_1/\alpha_1, \ldots, \tau_n/\alpha_n]$ may be used to notate the composition of
substitutions $S_1 = [\tau_1/\alpha_1], \ldots, S_n = [\tau_n/\alpha_n]$, $S_1(\ldots(S_n(\sigma)))$.

We say a type $\tau$ is an \textit{instance} of a type $\tau'$,
written as $t > \tau'$ if there exists a substitution $S$ such that

$$ \tau > \tau' \rightarrow \exists S. S(\tau) = \tau' $$

Intuitively, this can be thought of as $\tau$ being more general than
type $\tau'$. For example,
$$ \alpha \rightarrow \beta > \alpha \rightarrow \alpha $$
with the subsitution $[\alpha/\beta]$, but there exists no substitution for
$$ \alpha \rightarrow \alpha \ngtr \alpha \rightarrow \beta$$

\section{Static semantics}

Static semantics define the typing rules.

Before we look at the Hindley-Damas-Milner system, designed for
type-inference, we should take a look at a simpler variant.

The simply-typed lambda caclulus is a variant of the lambda
calculus, and very much a pre-cursor to what we will be building up
to. It defined three typing rules:
\begin{mathpar}
  \inferrule{x : \tau \in \Gamma}{\Gamma \vdash x : \tau} \and
  \inferrule{\Gamma, x : \tau' \vdash e : \tau}{\Gamma \vdash \lambda x : \tau . e : \tau' \rightarrow \tau} \and
  \inferrule{\Gamma \vdash e_1 : \tau' \rightarrow \tau \\ \Gamma \vdash e_2 : \tau'}{\Gamma \vdash e_1 e_2 : \tau}
\end{mathpar}
These rules consist of some \textit{judgements} above and below a line. There
are some regular judgements you might have seen in regular set theory,
such as $x : \tau \in \Gamma$ -- $x : \tau$ is in $\Gamma$. But there are also typing
judgements, which tell us what type a term has in what context. For
example, $\Gamma \vdash e_1 : \tau' \rightarrow \tau$ can be read as ``given the context
$\Gamma$, the expression $e_1$ is of type $:\tau' \rightarrow \tau$''.

A typing rule tell us that from the judgements above the line,
called \textit{premises}, we are allowed to derive the judgement
below the line, known as the \textit{conclusion}. So to put it all
together, the rule
\begin{mathpar}
  \inferrule{\Gamma \vdash e_1 : \tau' \rightarrow \tau \\ \Gamma \vdash e_2 : \tau'}{\Gamma \vdash e_1 e_2 : \tau}
\end{mathpar}
can be read as: ``If $e_1$ has the type $\tau' \rightarrow \tau$ in the context $\Gamma$,
and if $e_2$ has the type $\tau'$ in the context $\Gamma$, then we can derive
that $e_1 e_2$ has the type $\tau$ also in $\Gamma$.

This particular rule defines how application of two terms should be
typed in the simply-typed lambda calculus. 

\subsubsection{Let polymorphism}

Consider the below expression with the context $\Gamma = \{ a :
\textsf{Int} \rightarrow \textsf{Bool}, b : \textsf{Int} \}$
$$ \lambda x . (x (a (x \ b)))$$

What type should be inferred for $x$? It is used as both an
$\textsf{Int} \rightarrow \textsf{Int}$ and a $\textsf{Bool} \rightarrow \alpha$, where $\alpha$ is
the type of the entire expression. One possibility is $x : \alpha \rightarrow \alpha$.

$$ \letin{x = \lambda y. y}{x (a (x \ b))} $$

In the original definition by Milner there exited separate rules for
instantiation and generalisation. In our system we have combined them
into \textsc{Var} and \textsc{Let} respectively, so that they are
\textit{syntax-directed} -- there is exactly one rule for each form of
expression. This will make it easier to prove soundness later on.

As a part of this, we define $\bar{\Gamma}$ to generalize a type $\tau$,
binding over any free type variables.
$$ \bar{\Gamma}(\tau) =^{\textsf{def}} \forall \alpha_1, \ldots, \alpha_n . \tau \
\textsf{where} \ \{ \alpha_1, \ldots, \alpha_n \} = \mathrm{fv}(\tau) \setminus \mathrm{fv}(\Gamma)$$

\begin{figure}
  \centering
  \begin{mathpar}
    \inferrule*[Right=Var]{e : \tau' \in \Gamma \\ \tau' > \tau}{\Gamma \vdash e : \tau} \and
    \inferrule*[Right=App]{\Gamma \vdash e : \tau' \rightarrow \tau \\ \Gamma \vdash e' : \tau'}{\Gamma \vdash e \ e' : \tau} \and
    \inferrule*[Right=Abs]{\Gamma,x:\tau' \vdash e : \tau}{\Gamma \vdash \lambda x . e : \tau' \rightarrow \tau} \and
    \inferrule*[Right=Let]{\Gamma \vdash e : \tau' \\ \Gamma,x : \bar{\Gamma}(\tau') \vdash e' : \tau}
    {\Gamma \vdash \mathsf{let} \ x = e \ \mathsf{in} \ e' : \tau}
  \end{mathpar}
  \caption{Type inference rules}
  \label{rules:hmtypeinference}
\end{figure}


\subsection{Dynamic Semantics}

Milner and Damas created a denotational semantics for the language.

The semantics is defined by a semantic algebra, which itself is
comprised of a \textit{semantic domain} and \textit{semantic
  equation}.

The semantic domain defines the possible values an expression in our
language can have. It is a \textbf{complete partial order} (often referred to
as a \textit{cpo}): A pair $(D, \sqsubseteq)$ of a set $D$ and a partial order
$\sqsubseteq$ (a function that orders elements in $D$, but not necessarily all
of them, hence the term \textit{partial}), such that:

\begin{enumerate}
\item there is a least element $\bot$
\item each directed subset $x_0 \sqsubseteq \ldots \sqsubseteq x_n \sqsubseteq \ldots$ has a least upper bound
  (lub)
\end{enumerate}

\begin{align*}
  \mathbb{V} &= \mathbb{B}_{()} + \mathbb{B}_{bool} + \mathbb{F} + \mathbb{W} \\
  \mathbb{F} &= \mathbb{V} \rightarrow \mathbb{V} \\
  \mathbb{W} &= \{ . \}
\end{align*}

Or visually,

\begin{center}
  \tikz \graph[layered layout] { "$\mathbb{V}$" ->
    { "$\mathbb{B}_{()}$" -> "$()$",
      "$\mathbb{B}_{\textrm{Bool}}$" -> {true, false},
      I} ->
    "$\bot$"; };
\end{center}


Function space $D \rightarrow E$

Coalesced sum $D + E$

Since this cpo $\mathbb{V}$ represents all possibly data values, we
can extract a subset of it to model the values of certain
types~\ref{shamirwadge77}.
A subset $I$ of our cpo $\mathbb{V}$ is called an ideal, iff it
satisfies the following properties:

\begin{enumerate}
\item it is downwards closed: $\forall v_0 \in V, v_1 \in V, v_0, v_0 \sqsubseteq v_1 \rightarrow
  v_0 \in I \rightarrow \v_1 \in I$.
  
\item it is closed under lubs of \omega-chains.
\end{enumerate}

Our function domain $\mathbb{F}$ is a map from $\mathbb{V}$ to
$\mathbb{V}$. Maps over ideals are defined as
$I \rightarrow I' \equiv \{ v \in V | v \in \mathbb{F} \ \mathsf{and} \ \forall v' \in I \
(v_{|\mathbb{F}})v' \in I' \}$.

With all the mechanisms in place, we can now define what it means for
a value to semantically be a type:

$$v \in \mathbb{V}^\tau \iff \vDash v : \tau$$

Note that $v : \tau$ is a relation, not a function -- a value can be a
member of multiple types, and this should be read as ``$v$ is a $\tau$''.

\subsubsection{Bottom}
$\bot$ ends up being very useful to represent values that don't exist.
Take for example the following program which doesn't terminate. 

$$\letin{x = \lambda y. y}{x x}$$

What type does should this program have? It should assume the
type of whatever is needed. For instance, we would expect this program
to have a type of $()$ as the argument is not used.

$$(\lambda z. ()) (\letin{x = \lambda y. y}{x x}) : ()$$

Since $\bot$ is a member of all ideals, this program is well typed.

\subsubsection{Some notation}

If $\mathbb{D} \subset \mathbb{V}$, and $d \in \mathbb{D}$, we will say $d \ \mathsf{in} \
\mathbb{V}$ to represent $d$ but treated as if its in $\mathbb{V}$. \\
We will then define the reverse

$$v | \mathbb{D} =
\begin{cases}
  d & \textsf{if} \ v = d \ \textsf{in} \ \mathbb{V} \ \textsf{for
    some} \ d \in \mathbb{D} \\
  \bot_{\mathbb{D}} & \textsf{otherwise}
\end{cases}
$$


\subsubsection{Evaluation function}
The semantic equation $\mathcal{E} : \mathsf{Expression} \rightarrow
\mathsf{Environment} \rightarrow \mathbb{V}$ lies at the heart of the semantics,
and defines how the syntax is evaluated.

\begin{align*}
  \mathcal{E} \llbracket x \rrbracket \eta
  &= \eta \llbracket x \rrbracket \\
  \mathcal{E} \llbracket e_1 e_2 \rrbracket \eta
  &=
    \begin{cases}
      \bot & \mathsf{if} \ v_1 = \bot \\
      (v_1 | \mathbb{F}) v_2 & \mathsf{if} \ v_1 \in \mathbb{F} \\
      \mathsf{wrong} & \mathsf{otherwise}
    \end{cases}
  \\
  & \quad \textsf{where} \ v_i = \mathcal{E} \llbracket e_i \rrbracket \eta , \ i = \{
    1, 2\} \\
  \mathcal{E} \llbracket \lambda x . \ e \rrbracket \eta
  &=
    (\lambda v . \ \mathcal{E} \llbracket e \rrbracket \eta [v / x ])
    \ \mathsf{in} \ \mathbb{V} \\
  \mathcal{E} \llbracket \textsf{let} \ x = e_1 \ \textsf{in} \ e_2 \rrbracket \eta
  &=
    \mathcal{E} \llbracket e_2 \rrbracket \ \eta [ \mathcal{E} \llbracket e_1 \rrbracket\rho / x ]
\end{align*}

Note that these evaluation rules are not as strict as the semantics
defined by Milner~\cite{milner1978} -- namely, that the second argument
of application and let binding is not checked if it is a \textsf{wrong}.

We introduce the notion of an environment $\eta : \mathsf{Variable} \rightarrow
\mathbb{V}$. It is a map of variables bound to values.

An environment $\eta$ can be said to \textit{respect} a type environment
$\Gamma$ if all bindings in $\Gamma$ can be found in $\eta$ with the same type.
$$\eta : \Gamma \iff \forall x : \tau \in \Gamma. \ \eta \llbracket x \rrbracket : \tau$$

$$\Gamma \vDash e : \tau \iff
\forall \eta. \ \eta : \Gamma \rightarrow \mathcal{E} \llbracket e \rrbracket \eta : \tau $$

An assertion of the form above is said to be \textit{closed} if there
are no free type variables in $\Gamma$ or $\tau$, and an assertion only holds
iff its closed instances hold.

Let $\overline{\mathbb{V}}$ be the set of all ideals in $\mathbb{V}$
that do not contain $\mathsf{wrong}$.

There is also a type evaluation function $\mathcal{T} : \mathsf{Type}
\rightarrow \mathsf{Valuation} \rightarrow \overline{\mathbb{V}}$

\begin{align*}
  \mathcal{T}\llbracket () \rrbracket\psi &= \mathbb{B}_{()} \\
  \mathcal{T}\llbracket \mathsf{Bool} \rrbracket \psi &= \mathbb{B}_{\mathsf{Bool}} \\
  \mathcal{T}\llbracket \alpha \rrbracket \psi &= \psi \llbracket \alpha \rrbracket \\
  \mathcal{T} \llbracket \tau \rightarrow \tau' \rrbracket \psi &= \mathcal{T}\llbracket \tau \rrbracket \psi \ \rightarrow \
                             \mathcal{T} \llbracket \tau' \rrbracket \psi
\end{align*}

\subsection{Correctness}

Both Milner~\cite{milner1978} and Damas~\cite{damas1982} proved semantic
soundness for the system.

TODO: Is this too trivial to be a lemma? The proof is pretty vacuous
\newtheorem{lemma}{Lemma}
\begin{lemma}[to be named]
  If $v : \tau$ and $\eta : \Gamma$, then $\eta[v/x] : \Gamma,x : \tau$
  \label{lem:1}
\end{lemma}
\begin{proof}
  If $\eta : \Gamma$, then $\forall x : \tau \in \Gamma \ \eta\llbracket x \rrbracket : \tau$.
  And if $v : \tau$ then $\eta[v/x] \llbracket x \rrbracket : \tau$.
  So $\eta[v/x] : \Gamma,x : \tau$, because for all $y : \tau' \in \Gamma,x : \tau$, either $y =
  x$ and the substitution evaluates to the right type, or $y \neq x$ but
  because $\eta : \Gamma$, we have $\eta \llbracket y \rrbracket : \tau'$.
\end{proof}

\newtheorem{theorem}{Theorem}
\begin{theorem}[Semantic Soundness]
  $\Gamma \vdash e : \tau \rightarrow \Gamma \vDash e : \tau$ \\
  If $e$ has type $\tau$, $e$ does actually evaluate to a value in $\tau$.
\end{theorem}
\begin{proof}
  We need to prove $\forall \eta. \eta : \Gamma \rightarrow \mathcal{E} \llbracket e \rrbracket \eta : \tau$. We can do
  this via induction on $e$:

  \begin{description}
  \item[\boxed{x}] This is the base case of the induction. Since
    $\Gamma \vdash x : \tau$ , the rule \textsc{Var} gives us
    $x : \tau \in \Gamma$. The evaluation function produces
    $\mathcal{E} \llbracket x \rrbracket \eta = \eta \llbracket x \rrbracket$, but because
    $\eta : \Gamma$, $x : \tau \in \eta$, so $\eta \llbracket x \rrbracket : \tau$.
  \item[\boxed{e_1 e_2}] The type given is
    $\Gamma \vdash e_1 e_2 : \tau$, and via \textsc{App} we have
    $\Gamma \vdash e_1 : \tau' \rightarrow \tau$ and
    $\Gamma \vdash e_2 : \tau'$.  By applying the induction hypothesis, we get
    $\Gamma \vDash e_1 : \tau' \rightarrow \tau$ and $\Gamma \vDash e_2 : \tau'$, so
    $v_1 \in \tau' \rightarrow \tau$ and therefore $v_1 \in \mathbb{F}$. \\
    This brings us to
    $\mathcal{E} \llbracket e_1 e_2 \rrbracket \eta = (v_1 | \mathbb{F}) v_2$. We can tell
    what the application of the $v_2$ will give us by looking at the
    definition for the ideal of $v_1$:
    ${\tau' \rightarrow \tau = \{ v \in \mathbb{V} | v \in \mathbb{F} \wedge \forall v' \in \tau' (v |
      \mathbb{F}) v' \in \tau \}}$.  So
    $(v_1 | \mathbb{F}) v_2 \in \tau$, as is required to show
    $\Gamma \vDash e_1 e_2 : \tau$.
  \item[\boxed{\lambda x . e}] Our type is
    $\Gamma \vdash \lambda x . e : \tau' \rightarrow \tau$ and our typing rule \textsc{Abs} tells us
    the antecedent is $\Gamma,x:\tau' \vdash e :
    \tau$. Evaluating our expression gives
    $\mathcal{E} \llbracket \lambda x . e \rrbracket \eta = (\lambda v. \mathcal{E} \llbracket e \rrbracket \eta[v / x]) \
    in \ \mathbb{V}$. \\
    If we can prove
    $\mathcal{E} \llbracket e \rrbracket \eta [v/x] : \tau$ where
    $v : \tau'$, then we can prove that
    ${\lambda v . \mathcal{E} \llbracket e \rrbracket \eta [v/x] : \tau' \rightarrow \tau}$. But note that for
    $\Gamma, x : \tau' \vdash e : \tau$, lemma~\ref{lem:1} tells us $\eta[v/x] : \Gamma,x : \tau'$.
    And with this fact, $\Gamma,x : \tau' \vDash e : \tau$ from the
    induction hypothesis gives us $\mathcal{E} \llbracket e \rrbracket \eta [v/x] : \tau$
    where $v : \tau'$, as needed.
  \item[\boxed{\letin{x = e_1}{e_2}}] $\Gamma \vdash \letin{x = e_1}{e_2} : \tau$, and
    \textsc{Let} tells us $\Gamma \vdash e_1 : \tau'$ and $\Gamma,x : \tau' \vdash e_2 : \tau$. The
    evaluation function for let expressions is
    ${\mathcal{E} \llbracket \letin{x = e_1}{e_2} \rrbracket \eta
    = \mathcal{E} \llbracket e_2 \rrbracket (\eta [\mathcal{E} \llbracket e_1 \rrbracket \eta / x ])}$. We know that
  $\mathcal{E} \llbracket e_1 \rrbracket \eta : \tau'$ because of the induction hypothesis on
  $\Gamma \vdash e_1 : \tau'$, and will refer to this value as $v_1 : \tau'$. \\
  From lemma~\ref{lem:1}, in $\Gamma,x : \tau' \vdash e_2 : \tau$ the substituted environment
  respects the type environment $\eta[v_1/x] : \Gamma,x : \tau'$.
  Therefore $\Gamma,x : \tau' \vDash e_2 : \tau$ tells us that
  $\mathcal{E} \llbracket e_2 \rrbracket (\eta [v_1/x]) : \tau$. This is
  identical to our hypothesis, and our proof is done.
  \end{description}
  
\end{proof}

%%% Local Variables:
%%% TeX-master: "report"
%%% TeX-engine: luatex
%%% TeX-command-extra-options: "-shell-escape"
%%% End:


\include{System}

\include{system2}

Extensions: Dependently typed

\bibliographystyle{plain}
\bibliography{report}

\end{document}

% LocalWords: stdout LocalWords Idris haskell
%%% Local Variables:
%%% TeX-engine: luatex
%%% End: