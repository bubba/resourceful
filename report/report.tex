% -*- fill-column: 80; TeX-command-extra-options: "-shell-escape" -*-
\documentclass{report}
\usepackage{minted}
\usepackage{syntax}
\usepackage{mathpartir}
\usepackage{amssymb}
\usepackage{amsmath}
\usepackage{amsthm}
\usepackage{tikz}
\usetikzlibrary{graphs,graphdrawing,decorations.pathreplacing,decorations.pathmorphing,arrows.meta}
\usegdlibrary{layered,trees}
\usepackage{multicol}
\usepackage{hyperref}

\usepackage{fontspec}
\setmonofont{Menlo}[Scale=0.8]

\usepackage{newunicodechar}
\newfontface\mathsymbolfont{STIXGeneral}
\newunicodechar{⟦}{{\mathsymbolfont{\llbracket}}}
\newunicodechar{⟧}{{\mathsymbolfont{\rrbracket}}}
\newunicodechar{⊢}{{\mathsymbolfont{\vdash}}}
\newunicodechar{⦂}{{\mathsymbolfont{⦂}}}
\newunicodechar{⋎}{{\mathsymbolfont{\curlyvee}}}
\newunicodechar{∷}{{\mathsymbolfont{∷}}}

\newif\ifdeclaration
\declarationtrue

\begin{document}

\newcommand{\llbracket}{[\![}
\newcommand{\rrbracket}{]\!]}
\newcommand{\IO}{\mathsf{IO}}
\newcommand{\bind}{>\!\!>\!\!=} \newcommand{\concbind}{>\!\!>\!>\!\!=}
\newcommand{\subtyp}{\geq:}
\newcommand{\notsubtyp}{\ngeq:}
\newcommand{\lift}[1]{\ensuremath{\llbracket#1\rrbracket}}
\newcommand{\use}[2]{\ensuremath{\llbracket#2\rrbracket_{#1}}}
\newcommand{\ftv}{\operatorname{FTV}}
\newcommand{\fv}{\operatorname{FV}}
\newtheorem{theorem}{Theorem}
\newtheorem{lemma}{Lemma}

\author{Luke Lau}
\title{A Resourceful Monad for IO}
\begin{titlepage}
  \begin{center}
    {\Huge A Resourceful  Monad for IO}
    \vskip 0.25in
    \centerline{\Large Luke Lau}
    \vfill
    \large
    {\hsize5in
      {\uppercase{\Large \bf A Dissertation}}
      \vskip 0.2in
      Presented to the University of Dublin, Trinity College \\
      In Partial Fulfilment of the Requirements for the Degree of \\
      \vskip 0.2in
      {\bf Masters in Computer Science}
    }

    \vfill
  \end{center}

  \noindent {\bf Supervisor of Dissertation} \\
  Glenn Strong

\end{titlepage}

\ifdeclaration
% {\Large \bf \uppercase{Declaration}}
\chapter*{Declaration}

I hereby declare that this project is entirely my own work and that it has not
been submitted as an exercise for a degree at this or any other university.

\begin{flushleft}
  \vskip 2cm
  Luke Lau \\
  \smallskip
  April 2020
\end{flushleft}


\pagebreak
\fi

\tableofcontents

\chapter{Introduction}
The immutability and purity of some functional languages make them seem like a
perfect fit for parallelism and concurrency. The lack of side effects mean we
are free to compute expressions in whatever order we please, and there is no
shared mutable state to plague us with race conditions. However in the
real world, code is never completely pure. Useful programs need to interact with
the real world at some point, whether that be by reading from a keyboard or
sending packets over a network. That is, they need to carry out side
effects: And the order in which they are carried out is important.

So when it comes to side effects in a language with referential transparency,
the language must model the sequence in which effects are carried out. One of
the most successful approaches has been to capture the state of the outside
world, and have functions with side effects that write to that state. This
way, if writing to \texttt{stdout} causes some particles to be perturbed on a
display monitor, then we can track that by saying the state of the world has
changed, and that it matters what order it changed. We don't actually track
what particles are there, where they might be or what their current charge
is. All we need to know is that the world is slightly different than what it
was before.

Concurrent Clean models this by threading the world in and out of
functions. Uniqueness types guarantee that the same world is only used once, so
that the programmer doesn't create an alternative timeline by duplicating it.
\begin{minted}[breaklines]{clean}
readFile :: !String !*World -> (!MaybeError FileError String, !*World)
\end{minted}

Haskell also treats the world as a state, but without the uniquness guarantee.
Any function that interacts with the world returns a function, which returns a
new version of the World alongside the function result.\footnote{The actual
  definition in GHC is \\ \mintinline{haskell}{newtype IO a = IO (State#
    RealWorld -> (# State# RealWorld, a #))}}
\mint{haskell}|type IO a = World -> (World, a)|
How Haskell ensures that an old world isn't erroneously used is by hiding the
updating of the state from the user, tucking it away into a \textit{monad}. This
idea of using monads to sequence together stateful computations was first
introduced by Peyton Jones and
Wadler~\cite{peytonjones1993}\cite{wadler1995}. The programmer no longer needs
to keep track of the state of the world. They can keep their imperative code
imperative, and their pure code pure. This marriage of monads and IO is one of
the crown jewels of functional language research to come out of Haskell.
\begin{minted}{haskell}
instance Monad IO where
  return x = \w -> (w, x)
  x >>= f = \w -> let (w', y) = x w in f y w'
\end{minted}
Now IO actions can be easily chained together in a type-safe way that ensures
their ordering.  Unfortunately, this ordering imposes limitations. One of the
main benefits of pure functional languages is that since expressions do not
have side effects, there is no restriction on what order they need to be
evaluated in.  Take for example the following program:
\begin{minted}{haskell}
f, g, h :: Int -> Int
f x = g x + h x
\end{minted}
\texttt{g} could be evaluated before \texttt{h}, or \texttt{h} could be
evaluated before \texttt{g}. It wouldn't make a difference because there are no
side effects. One might be tempted then to evaluate the two expressions
concurrently, and indeed that would be safe to do so.  The same cannot be said
for impure IO actions however, and Haskell's type system is well aware of that.
\begin{minted}{haskell}
f, g, h :: Int -> IO Int
f x = g x >>= \y -> y + h x
\end{minted}
We need to explicitly bind the actions and sequence evaluation.  Does this mean
that concurrency is impossible for IO actions? Not at all, many languages
provide primitives to run these actions concurrently in a type safe
way. Haskell's \texttt{base} library has \mintinline{haskell}{forkIO}, but for
simpliclty we are going to assume the existence of a higher level function that
runs two IO actions simultaneously and collects the results.
\begin{minted}{haskell}
concurrently :: IO a -> IO b -> IO (a, b)
\end{minted}
Now we can use it to run our two IO actions side by side safely. 
\begin{minted}{haskell}
f, g, h :: Int -> IO Int
f x = g x `concurrently` h x >>= \(a, b) -> return (a + b)
\end{minted}
But what if \texttt{g} and \texttt{h} actually looked like this?
\begin{minted}[samepage]{haskell}
g x = do
  txt <- readFile "foo.txt"
  return (x + (read txt))
h x = do
  writeFile "foo.txt" "hello"
  return (42 - x)

readFile :: FilePath -> IO String
writeFile :: FilePath -> String -> IO ()
\end{minted}

Running these two functions concurrently could be disastrous, as the order in
which they execute could affect the outcome of the program. These innocuous
looking IO actions then end up introducing non-determinism and race conditions.
\begin{samepage}
We know statically that a program such as
\mint{haskell}|writeFile "foo.txt" "a" `concurrently` writeFile "foo.txt" "b"|
\noindent should probably not be allowed, because it is blatantly accessing the
same resource simultaneously.
\end{samepage}
But then why did the type system allow it?  Has it failed us? The goal of the
type system is to disallow as many incorrect programs as possible whilst
allowing all correct programs. It is a fine line as to what programs are
deemed correct and what are deemed incorrect. A type system too lenient and
buggy programs will creep through: but a type system too strict and the
programmer will waste time fighting the type checker.

In this dissertation we aim to find a point in the design space that rejects
such programs as the one above. We do \textbf{not} want to allow programs that
have glaring race conditions, where we can see that there is a contentious
access of resources. Our work is based around the idea of keeping track of what
resources are in use, at the type level. We first imagine what this might look
like in Haskell by adding another type parameter to
our IO type, representing what resource an IO action uses:
\mint{haskell}|type IO r a = World -> (World, a)|
$r$ is a phantom type parameter, which only exists at the type level. Now our
type signatures could look like this, annotating the API with what resources it
might use.
\begin{minted}{haskell}
data Resource = FileSystem | Net | Database | OpenGL | ...
readFile :: FilePath -> IO FileSystem ()
writeFile :: FilePath -> IO FileSystem String
readSocket :: Socket -> IO Net ()
runQuery :: Query a -> IO Database a
swapBuffers :: IO OpenGL ()
\end{minted}

Keep in mind we are painting in broad strokes when we use the word
``resource''. In the running example the resource has been a file,
\texttt{foo.txt}, but the notion of a resource can be as broad or as specific as
a function needs it to be. It could represent a specific database
instance, or a network socket. For simplicity in our example we will
consider the entire file system as a single resource, the entire network as
a single resource, and so on.

Now that we know what resources each IO action is using, we would like to change
the type of our concurrent function to take advantage of this new
information. Perhaps we would like to reject any two functions that use the same
resource, i.e.\ it only accepts IO actions with distinct resources.
\mint{haskell}|concurrently :: r \~ s => IO r a -> IO s b -> IO ? (a, b)|
You can read \verb$r \~ s$ as ``r is distinct from s'', or the opposite of a
\verb$r ~ s$ equality constraint that one might see in a type signature. This,
however, does not exist in Haskell.  And what does it exactly mean for two
resources to ``be distinct''? And what resources would the returned $\IO$ use?

These questions are answered with a
formal definition of a type system that tracks resource usage. We explore a
specific point in the design space, where the type system rejects programs like
\[ \textsf{readFile} \curlyvee \textsf{readFile} \]
but accepts and assigns types to programs such as
\[ \textsf{readFile} \curlyvee \textsf{readNet} : \IO_{\textsf{File} \cup \textsf{Net}} \
  \square \times \square \]
In short, we aim to create a type system that keeps tracks of the resources
being used, so that the programmer doesn't have to. It \textbf{does not} aim to
solve concurrency --- there will still be programs that have concurrency errors
that the type system will still allow. We just aim to narrow down the scope of
valid programs, by ruling out those with blatant, concurrent resource access
errors. 

\section{Overview}
This dissertation has been written in a style that doesn't assume prior
knowledge of type theory, and so it should double as a tutorial along our
journey to create our resourceful type system.

In Chapter~\ref{chapter:background} we will talk about the inspirations of the
type system, namely separation logic, and how it parallels with our monadic
language. We will also briefly go over what we mean by a monad formally, and
then look at the original design of Hindley-Damas-Milner which we will build
upon.
Chapter~\ref{chapter:system} introduces the language, its type system and its
semantics.  It gives a complete definition of all the parts needed to prove
properties.  We prove these properties in Chapter~\ref{cha:properties}, in which
we eventually build up and present a proof of its soundness. This proof, and the
others that accompany it, are then mechanised within the dependently typed
language and proof assistant Agda. The methodology of this is discussed in
Chapter~\ref{cha:mechanisation}. Finally, Chapter~\ref{cha:evaluation}
evaluates our type system. We look at how it might be refined, and find many
areas of further work that would be interesting to explore.

\chapter{Background} \label{chapter:background}

In this chapter, we are going to look at the Hindley-Damas-Milner type
system. It is one of the de-facto formalisations of a polymorphic type
system, based off of the ML programming language. It was heavily
influential at the time and still continues to be so today, inspiring
the design of Haskell, O-Caml and many programming languages.

\section{Hindley-Damas-Milner}
\subsection{Syntax}

\def\defaultHypSeparation{\hskip .05in}
\newcommand{\letin}[2]{\mathsf{let} \ #1 \ \mathsf{in} \ #2}

\begin{figure}
  \begin{grammar}

    <type $\tau$> ::= $()$ | $\alpha$ | $\tau \rightarrow \tau'$
    
    <type scheme $\sigma$> ::= $\tau$ | $\forall \alpha . \sigma$

    <expression $e$> ::= $x$
    \alt $\lambda x . e$
    \alt $e \ e'$
    \alt $\mathsf{let} \ x = e \ \mathsf{in} \ e'$

    <context $\Gamma$> ::= $\centerdot$ | $\Gamma, x : \centerdot$

  \end{grammar}
  \caption{Grammar of the applicative language}
  \label{grm:applang}
\end{figure}

We define the grammar for a simple applicative language in
figure~\ref{grm:applang}.
It defines expressions, types and type schemes. The distinguishment
between type and type scheme is necessary so that quantifiers can only
appear at the top level. Quantifiers bind type variables.

Free type variables are type variables which have not be bound.
\begin{align*}
  \mathrm{fv}(\forall \alpha . \tau) &= \mathrm{fv}(\tau) \\ {\alpha} \\
  \mathrm{fv}(()) &= \{ \} \\
  \mathrm{fv}(\alpha) &= \{ \alpha \} \\
  \mathrm{fv}(\tau \rightarrow \tau') &= \mathrm{fv}(\tau) \cup \mathrm{fv}(\tau')
\end{align*}

A \textit{type environment} is a mapping from variables to type schemes. 
$$\Gamma : \mathsf{Variable} \ \rightarrow \sigma$$

A substitution $S$ maps type variables to types. A substitution of the
form $[\tau/\alpha]$ maps $\alpha$ to $\tau$.

It is extended to operate on types, such that every occurance of a
type variable in a type is substituted.
For instance, $(\alpha \rightarrow \alpha)[\tau/\alpha] = \tau \rightarrow \tau$ can be read as ``$(\alpha \rightarrow \alpha)$,
replacing every $\alpha$ with $\tau$.''

% \begin{minipage}{1.0\linewidth}
  Substitution is \textbf{associative}. It is defined on types as
  \begin{align*}
    () [\tau/\alpha] &= () \\
    \alpha' [\tau/\alpha] &=
                            \begin{cases}
                              \tau & \mathsf{if} \ \alpha' = \alpha \\
                              \alpha' & \mathsf{otherwise}
                            \end{cases} \\
    (\tau_1 \rightarrow \tau_2)[\tau/\alpha] &= (\tau_1[\tau/\alpha] \rightarrow \tau_2[\tau/\alpha])
  \end{align*}
% \end{minipage}

And on type schemes as
$$(\forall \alpha . \sigma)[\tau/\alpha] = \forall \tau . \sigma[\tau/\alpha]$$

We may write $S(\sigma)$ to apply an arbitrary substitution to a type
scheme.

$[\tau_1/\alpha_1, \ldots, \tau_n/\alpha_n]$ may be used to notate the composition of
substitutions $S_1 = [\tau_1/\alpha_1], \ldots, S_n = [\tau_n/\alpha_n]$, $S_1(\ldots(S_n(\sigma)))$.

We say a type $\tau$ is an \textit{instance} of a type $\tau'$,
written as $t > \tau'$ if there exists a substitution $S$ such that

$$ \tau > \tau' \rightarrow \exists S. S(\tau) = \tau' $$

Intuitively, this can be thought of as $\tau$ being more general than
type $\tau'$. For example,
$$ \alpha \rightarrow \beta > \alpha \rightarrow \alpha $$
with the subsitution $[\alpha/\beta]$, but there exists no substitution for
$$ \alpha \rightarrow \alpha \ngtr \alpha \rightarrow \beta$$

\section{Static semantics}

Static semantics define the typing rules.

Before we look at the Hindley-Damas-Milner system, designed for
type-inference, we should take a look at a simpler variant.

The simply-typed lambda caclulus is a variant of the lambda
calculus, and very much a pre-cursor to what we will be building up
to. It defined three typing rules:
\begin{mathpar}
  \inferrule{x : \tau \in \Gamma}{\Gamma \vdash x : \tau} \and
  \inferrule{\Gamma, x : \tau' \vdash e : \tau}{\Gamma \vdash \lambda x : \tau . e : \tau' \rightarrow \tau} \and
  \inferrule{\Gamma \vdash e_1 : \tau' \rightarrow \tau \\ \Gamma \vdash e_2 : \tau'}{\Gamma \vdash e_1 e_2 : \tau}
\end{mathpar}
These rules consist of some \textit{judgements} above and below a line. There
are some regular judgements you might have seen in regular set theory,
such as $x : \tau \in \Gamma$ -- $x : \tau$ is in $\Gamma$. But there are also typing
judgements, which tell us what type a term has in what context. For
example, $\Gamma \vdash e_1 : \tau' \rightarrow \tau$ can be read as ``given the context
$\Gamma$, the expression $e_1$ is of type $:\tau' \rightarrow \tau$''.

A typing rule tell us that from the judgements above the line,
called \textit{premises}, we are allowed to derive the judgement
below the line, known as the \textit{conclusion}. So to put it all
together, the rule
\begin{mathpar}
  \inferrule{\Gamma \vdash e_1 : \tau' \rightarrow \tau \\ \Gamma \vdash e_2 : \tau'}{\Gamma \vdash e_1 e_2 : \tau}
\end{mathpar}
can be read as: ``If $e_1$ has the type $\tau' \rightarrow \tau$ in the context $\Gamma$,
and if $e_2$ has the type $\tau'$ in the context $\Gamma$, then we can derive
that $e_1 e_2$ has the type $\tau$ also in $\Gamma$.

This particular rule defines how application of two terms should be
typed in the simply-typed lambda calculus. 

\subsubsection{Let polymorphism}

Consider the below expression with the context $\Gamma = \{ a :
\textsf{Int} \rightarrow \textsf{Bool}, b : \textsf{Int} \}$
$$ \lambda x . (x (a (x \ b)))$$

What type should be inferred for $x$? It is used as both an
$\textsf{Int} \rightarrow \textsf{Int}$ and a $\textsf{Bool} \rightarrow \alpha$, where $\alpha$ is
the type of the entire expression. One possibility is $x : \alpha \rightarrow \alpha$.

$$ \letin{x = \lambda y. y}{x (a (x \ b))} $$

In the original definition by Milner there exited separate rules for
instantiation and generalisation. In our system we have combined them
into \textsc{Var} and \textsc{Let} respectively, so that they are
\textit{syntax-directed} -- there is exactly one rule for each form of
expression. This will make it easier to prove soundness later on.

As a part of this, we define $\bar{\Gamma}$ to generalize a type $\tau$,
binding over any free type variables.
$$ \bar{\Gamma}(\tau) =^{\textsf{def}} \forall \alpha_1, \ldots, \alpha_n . \tau \
\textsf{where} \ \{ \alpha_1, \ldots, \alpha_n \} = \mathrm{fv}(\tau) \setminus \mathrm{fv}(\Gamma)$$

\begin{figure}
  \centering
  \begin{mathpar}
    \inferrule*[Right=Var]{e : \tau' \in \Gamma \\ \tau' > \tau}{\Gamma \vdash e : \tau} \and
    \inferrule*[Right=App]{\Gamma \vdash e : \tau' \rightarrow \tau \\ \Gamma \vdash e' : \tau'}{\Gamma \vdash e \ e' : \tau} \and
    \inferrule*[Right=Abs]{\Gamma,x:\tau' \vdash e : \tau}{\Gamma \vdash \lambda x . e : \tau' \rightarrow \tau} \and
    \inferrule*[Right=Let]{\Gamma \vdash e : \tau' \\ \Gamma,x : \bar{\Gamma}(\tau') \vdash e' : \tau}
    {\Gamma \vdash \mathsf{let} \ x = e \ \mathsf{in} \ e' : \tau}
  \end{mathpar}
  \caption{Type inference rules}
  \label{rules:hmtypeinference}
\end{figure}


\subsection{Dynamic Semantics}

Milner and Damas created a denotational semantics for the language.

The semantics is defined by a semantic algebra, which itself is
comprised of a \textit{semantic domain} and \textit{semantic
  equation}.

The semantic domain defines the possible values an expression in our
language can have. It is a \textbf{complete partial order} (often referred to
as a \textit{cpo}): A pair $(D, \sqsubseteq)$ of a set $D$ and a partial order
$\sqsubseteq$ (a function that orders elements in $D$, but not necessarily all
of them, hence the term \textit{partial}), such that:

\begin{enumerate}
\item there is a least element $\bot$
\item each directed subset $x_0 \sqsubseteq \ldots \sqsubseteq x_n \sqsubseteq \ldots$ has a least upper bound
  (lub)
\end{enumerate}

\begin{align*}
  \mathbb{V} &= \mathbb{B}_{()} + \mathbb{B}_{bool} + \mathbb{F} + \mathbb{W} \\
  \mathbb{F} &= \mathbb{V} \rightarrow \mathbb{V} \\
  \mathbb{W} &= \{ . \}
\end{align*}

Or visually,

\begin{center}
  \tikz \graph[layered layout] { "$\mathbb{V}$" ->
    { "$\mathbb{B}_{()}$" -> "$()$",
      "$\mathbb{B}_{\textrm{Bool}}$" -> {true, false},
      I} ->
    "$\bot$"; };
\end{center}


Function space $D \rightarrow E$

Coalesced sum $D + E$

Since this cpo $\mathbb{V}$ represents all possibly data values, we
can extract a subset of it to model the values of certain
types~\ref{shamirwadge77}.
A subset $I$ of our cpo $\mathbb{V}$ is called an ideal, iff it
satisfies the following properties:

\begin{enumerate}
\item it is downwards closed: $\forall v_0 \in V, v_1 \in V, v_0, v_0 \sqsubseteq v_1 \rightarrow
  v_0 \in I \rightarrow \v_1 \in I$.
  
\item it is closed under lubs of \omega-chains.
\end{enumerate}

Our function domain $\mathbb{F}$ is a map from $\mathbb{V}$ to
$\mathbb{V}$. Maps over ideals are defined as
$I \rightarrow I' \equiv \{ v \in V | v \in \mathbb{F} \ \mathsf{and} \ \forall v' \in I \
(v_{|\mathbb{F}})v' \in I' \}$.

With all the mechanisms in place, we can now define what it means for
a value to semantically be a type:

$$v \in \mathbb{V}^\tau \iff \vDash v : \tau$$

Note that $v : \tau$ is a relation, not a function -- a value can be a
member of multiple types, and this should be read as ``$v$ is a $\tau$''.

\subsubsection{Bottom}
$\bot$ ends up being very useful to represent values that don't exist.
Take for example the following program which doesn't terminate. 

$$\letin{x = \lambda y. y}{x x}$$

What type does should this program have? It should assume the
type of whatever is needed. For instance, we would expect this program
to have a type of $()$ as the argument is not used.

$$(\lambda z. ()) (\letin{x = \lambda y. y}{x x}) : ()$$

Since $\bot$ is a member of all ideals, this program is well typed.

\subsubsection{Some notation}

If $\mathbb{D} \subset \mathbb{V}$, and $d \in \mathbb{D}$, we will say $d \ \mathsf{in} \
\mathbb{V}$ to represent $d$ but treated as if its in $\mathbb{V}$. \\
We will then define the reverse

$$v | \mathbb{D} =
\begin{cases}
  d & \textsf{if} \ v = d \ \textsf{in} \ \mathbb{V} \ \textsf{for
    some} \ d \in \mathbb{D} \\
  \bot_{\mathbb{D}} & \textsf{otherwise}
\end{cases}
$$


\subsubsection{Evaluation function}
The semantic equation $\mathcal{E} : \mathsf{Expression} \rightarrow
\mathsf{Environment} \rightarrow \mathbb{V}$ lies at the heart of the semantics,
and defines how the syntax is evaluated.

\begin{align*}
  \mathcal{E} \llbracket x \rrbracket \eta
  &= \eta \llbracket x \rrbracket \\
  \mathcal{E} \llbracket e_1 e_2 \rrbracket \eta
  &=
    \begin{cases}
      \bot & \mathsf{if} \ v_1 = \bot \\
      (v_1 | \mathbb{F}) v_2 & \mathsf{if} \ v_1 \in \mathbb{F} \\
      \mathsf{wrong} & \mathsf{otherwise}
    \end{cases}
  \\
  & \quad \textsf{where} \ v_i = \mathcal{E} \llbracket e_i \rrbracket \eta , \ i = \{
    1, 2\} \\
  \mathcal{E} \llbracket \lambda x . \ e \rrbracket \eta
  &=
    (\lambda v . \ \mathcal{E} \llbracket e \rrbracket \eta [v / x ])
    \ \mathsf{in} \ \mathbb{V} \\
  \mathcal{E} \llbracket \textsf{let} \ x = e_1 \ \textsf{in} \ e_2 \rrbracket \eta
  &=
    \mathcal{E} \llbracket e_2 \rrbracket \ \eta [ \mathcal{E} \llbracket e_1 \rrbracket\rho / x ]
\end{align*}

Note that these evaluation rules are not as strict as the semantics
defined by Milner~\cite{milner1978} -- namely, that the second argument
of application and let binding is not checked if it is a \textsf{wrong}.

We introduce the notion of an environment $\eta : \mathsf{Variable} \rightarrow
\mathbb{V}$. It is a map of variables bound to values.

An environment $\eta$ can be said to \textit{respect} a type environment
$\Gamma$ if all bindings in $\Gamma$ can be found in $\eta$ with the same type.
$$\eta : \Gamma \iff \forall x : \tau \in \Gamma. \ \eta \llbracket x \rrbracket : \tau$$

$$\Gamma \vDash e : \tau \iff
\forall \eta. \ \eta : \Gamma \rightarrow \mathcal{E} \llbracket e \rrbracket \eta : \tau $$

An assertion of the form above is said to be \textit{closed} if there
are no free type variables in $\Gamma$ or $\tau$, and an assertion only holds
iff its closed instances hold.

Let $\overline{\mathbb{V}}$ be the set of all ideals in $\mathbb{V}$
that do not contain $\mathsf{wrong}$.

There is also a type evaluation function $\mathcal{T} : \mathsf{Type}
\rightarrow \mathsf{Valuation} \rightarrow \overline{\mathbb{V}}$

\begin{align*}
  \mathcal{T}\llbracket () \rrbracket\psi &= \mathbb{B}_{()} \\
  \mathcal{T}\llbracket \mathsf{Bool} \rrbracket \psi &= \mathbb{B}_{\mathsf{Bool}} \\
  \mathcal{T}\llbracket \alpha \rrbracket \psi &= \psi \llbracket \alpha \rrbracket \\
  \mathcal{T} \llbracket \tau \rightarrow \tau' \rrbracket \psi &= \mathcal{T}\llbracket \tau \rrbracket \psi \ \rightarrow \
                             \mathcal{T} \llbracket \tau' \rrbracket \psi
\end{align*}

\subsection{Correctness}

Both Milner~\cite{milner1978} and Damas~\cite{damas1982} proved semantic
soundness for the system.

TODO: Is this too trivial to be a lemma? The proof is pretty vacuous
\newtheorem{lemma}{Lemma}
\begin{lemma}[to be named]
  If $v : \tau$ and $\eta : \Gamma$, then $\eta[v/x] : \Gamma,x : \tau$
  \label{lem:1}
\end{lemma}
\begin{proof}
  If $\eta : \Gamma$, then $\forall x : \tau \in \Gamma \ \eta\llbracket x \rrbracket : \tau$.
  And if $v : \tau$ then $\eta[v/x] \llbracket x \rrbracket : \tau$.
  So $\eta[v/x] : \Gamma,x : \tau$, because for all $y : \tau' \in \Gamma,x : \tau$, either $y =
  x$ and the substitution evaluates to the right type, or $y \neq x$ but
  because $\eta : \Gamma$, we have $\eta \llbracket y \rrbracket : \tau'$.
\end{proof}

\newtheorem{theorem}{Theorem}
\begin{theorem}[Semantic Soundness]
  $\Gamma \vdash e : \tau \rightarrow \Gamma \vDash e : \tau$ \\
  If $e$ has type $\tau$, $e$ does actually evaluate to a value in $\tau$.
\end{theorem}
\begin{proof}
  We need to prove $\forall \eta. \eta : \Gamma \rightarrow \mathcal{E} \llbracket e \rrbracket \eta : \tau$. We can do
  this via induction on $e$:

  \begin{description}
  \item[\boxed{x}] This is the base case of the induction. Since
    $\Gamma \vdash x : \tau$ , the rule \textsc{Var} gives us
    $x : \tau \in \Gamma$. The evaluation function produces
    $\mathcal{E} \llbracket x \rrbracket \eta = \eta \llbracket x \rrbracket$, but because
    $\eta : \Gamma$, $x : \tau \in \eta$, so $\eta \llbracket x \rrbracket : \tau$.
  \item[\boxed{e_1 e_2}] The type given is
    $\Gamma \vdash e_1 e_2 : \tau$, and via \textsc{App} we have
    $\Gamma \vdash e_1 : \tau' \rightarrow \tau$ and
    $\Gamma \vdash e_2 : \tau'$.  By applying the induction hypothesis, we get
    $\Gamma \vDash e_1 : \tau' \rightarrow \tau$ and $\Gamma \vDash e_2 : \tau'$, so
    $v_1 \in \tau' \rightarrow \tau$ and therefore $v_1 \in \mathbb{F}$. \\
    This brings us to
    $\mathcal{E} \llbracket e_1 e_2 \rrbracket \eta = (v_1 | \mathbb{F}) v_2$. We can tell
    what the application of the $v_2$ will give us by looking at the
    definition for the ideal of $v_1$:
    ${\tau' \rightarrow \tau = \{ v \in \mathbb{V} | v \in \mathbb{F} \wedge \forall v' \in \tau' (v |
      \mathbb{F}) v' \in \tau \}}$.  So
    $(v_1 | \mathbb{F}) v_2 \in \tau$, as is required to show
    $\Gamma \vDash e_1 e_2 : \tau$.
  \item[\boxed{\lambda x . e}] Our type is
    $\Gamma \vdash \lambda x . e : \tau' \rightarrow \tau$ and our typing rule \textsc{Abs} tells us
    the antecedent is $\Gamma,x:\tau' \vdash e :
    \tau$. Evaluating our expression gives
    $\mathcal{E} \llbracket \lambda x . e \rrbracket \eta = (\lambda v. \mathcal{E} \llbracket e \rrbracket \eta[v / x]) \
    in \ \mathbb{V}$. \\
    If we can prove
    $\mathcal{E} \llbracket e \rrbracket \eta [v/x] : \tau$ where
    $v : \tau'$, then we can prove that
    ${\lambda v . \mathcal{E} \llbracket e \rrbracket \eta [v/x] : \tau' \rightarrow \tau}$. But note that for
    $\Gamma, x : \tau' \vdash e : \tau$, lemma~\ref{lem:1} tells us $\eta[v/x] : \Gamma,x : \tau'$.
    And with this fact, $\Gamma,x : \tau' \vDash e : \tau$ from the
    induction hypothesis gives us $\mathcal{E} \llbracket e \rrbracket \eta [v/x] : \tau$
    where $v : \tau'$, as needed.
  \item[\boxed{\letin{x = e_1}{e_2}}] $\Gamma \vdash \letin{x = e_1}{e_2} : \tau$, and
    \textsc{Let} tells us $\Gamma \vdash e_1 : \tau'$ and $\Gamma,x : \tau' \vdash e_2 : \tau$. The
    evaluation function for let expressions is
    ${\mathcal{E} \llbracket \letin{x = e_1}{e_2} \rrbracket \eta
    = \mathcal{E} \llbracket e_2 \rrbracket (\eta [\mathcal{E} \llbracket e_1 \rrbracket \eta / x ])}$. We know that
  $\mathcal{E} \llbracket e_1 \rrbracket \eta : \tau'$ because of the induction hypothesis on
  $\Gamma \vdash e_1 : \tau'$, and will refer to this value as $v_1 : \tau'$. \\
  From lemma~\ref{lem:1}, in $\Gamma,x : \tau' \vdash e_2 : \tau$ the substituted environment
  respects the type environment $\eta[v_1/x] : \Gamma,x : \tau'$.
  Therefore $\Gamma,x : \tau' \vDash e_2 : \tau$ tells us that
  $\mathcal{E} \llbracket e_2 \rrbracket (\eta [v_1/x]) : \tau$. This is
  identical to our hypothesis, and our proof is done.
  \end{description}
  
\end{proof}

%%% Local Variables:
%%% TeX-master: "report"
%%% TeX-engine: luatex
%%% TeX-command-extra-options: "-shell-escape"
%%% End:


\chapter{A Formal Definition} \label{chapter:system}

To formalize the system of heaped monads, I introduce a toy language
and type system implementing the basic principles. The semantics are
mainly based off of the Hilney-Damas-Milner type system \cite{damasmilner}

\section{Syntax}

\setlength{\grammarparsep}{20pt plus 1pt minus 1pt} % increase separation between rules
\setlength{\grammarindent}{12em} % increase separation between LHS/RHS
\renewcommand{\syntleft}{}
\renewcommand{\syntright}{}

%\newcommand{\square}{\ensuremath{\raisebox{-0.35mm}{\square}}}


\def\defaultHypSeparation{\hskip .05in}

We begin by giving the grammar for our monadic language as shown in
figure~\ref{fig:grammar}.  It is an extension of the lambda calculus
with let-polymorphism.  We use $x$, $\lambda x . e$ and $e \ e'$ to refer to
variable lookup, abstraction and application respectively, whilst we
use the ML style $\letin{x = e}{e'}$ for polymorphically binding $e$
to $x$ inside $e'$.

There is also a product type, $\tau \times \tau'$, which is the same as the tuple
\texttt{(a, b)} in Haskell. New product types can be introduced with
$e \times e'$, and can similarly be eliminated with the projection
expressions $\pi_1 e$ and $\pi_2 e$.

More interestingly we have the monadic additions. Unlike
Krishnaswami~\cite{krishnaswami2006} we do not separate the language
into two expression and computation languages.  $\lift{e}$, lifts a
regular expression into the $\IO$ monad (\texttt{return} in Haskell), and
$e \bind e'$ is the standard monadic sequencing or binding operation.

Our language also has heaps $\rho$ and resources $r$, which are different from the
notion of heaps in separation logic. Resources represent something
that we want to keep track of in the type and prevent from being
accessed concurrently, for example a file system or database. We
provide a couple of placeholder resources to start off with.

These can then be used to build up a heap, which is constructed from
resources and merging other heaps. It is used to tag the resources
that a computation in an $\IO$ monad might be accessing. There is also
the notion of a \textsf{World} heap which encapsulates all possible
resources and sub-heaps. It can be thought of as the default
\texttt{IO} monad in Haskell, where everything acting on the real
world is treated sequentially.

$\use{r}{e}$, pronounced \textit{e using r}, which similarly to
$\lift{e}$ lifts a value in the $\IO$ monad, but it uses the resource
$r$ in the process, starting a new heap with that resource. Unique to
our language is the $e \curlyvee e'$ operator. It joins two monadic
computations together into one that uses both resources, returning a
new heap made by merging $e$ and $e'$s heaps -- provided that they do
not overlap and are well formed (Well formedness will be defined later
on).

Note that for our type schemes, we may omit the bindings and write
$\forall \alpha . \alpha$ as $\alpha$ in places where it is unambiguous, such as in the context.

\begin{figure}
\begin{grammar}

  <type $\tau$> ::= $\square$ | $\alpha$ | $\tau \rightarrow \tau'$ | $\tau \times \tau'$ | $\textsf{IO}_\rho \tau$
  
  <type scheme $\sigma$> ::= $\forall \alpha . \sigma$ | $\tau$

  <context $\Gamma$> ::= $\centerdot$ | $\Gamma , x : \sigma$

  <expression $e$> ::= $\square$ | $e \times e'$ | $\pi_1 \ e$ $\pi_2 \ e$
  \alt $x$ | $\lambda x . e$ | $e \ e'$ | $\letin{x = e}{e'}$
  % \alt $\textsf{if} \ e_1 \ \textsf{then} \ e_2 \ \textsf{else} \ e_3$
  \alt $\lift{e}$ | $\use{r}{e}$ | $e \bind e'$ | $e \curlyvee e'$

  <resource $r$> ::= \textsf{File} | \textsf{Network} |
  \textsf{Database} | \ldots

  <heap $\rho$> ::= $r$ | $\rho \cup \rho'$ | \textsf{World}

\end{grammar}
\caption{The grammar for our resourceful language.} \label{fig:grammar}
\end{figure}

\section{Substitution}
Substitution may seem self-explanatory, but it is important to define
it correctly as it will play a big part in proving essential
properties of our system, and there are subtle differences with how it
is defined on terms, types and type schemes.

Substitution on types is defined as usual, but extended to include the
new types in our system. That is, a substitution is a map from type
variables to types, written in the form $[\tau/\alpha]$ for a single type.
\begin{align*}
  \square [\tau/\alpha] &= \square \\
  \alpha' [\tau/\alpha] &=
             \begin{cases}
               \tau & \mathsf{if} \ \alpha' = \alpha \\
               \alpha' & \mathsf{otherwise}
             \end{cases} \\
  (e \times e') [\tau/\alpha] &= e[\tau/\alpha] \times e'[\tau/\alpha] \\
  (\tau_1 \rightarrow \tau_2)[\tau/\alpha] &= (\tau_1[\tau/\alpha] \rightarrow \tau_2[\tau/\alpha]) \\
  (\IO_\rho \tau')[\tau/\alpha] &= \IO_\rho \tau'[\tau/\alpha]
\end{align*}

A substitution that substitutes multiple type variables at once can be
written like $[\tau_1/\alpha_1,\ldots,\tau_m/\alpha_m]$.

Substitution is also extended to type schemes, but note that his only
substitutes \textbf{free} type variables.
\begin{align*}
  \forall \alpha' . \sigma[\tau/\alpha] &=
            \begin{cases}
              \forall \alpha' . \sigma & \mathsf{if} \ \alpha' = \alpha \\
              \forall \alpha' . \sigma[\tau/\alpha] & \mathsf{otherwise}
            \end{cases} \\
  \tau[\tau/\alpha] &= \tau[\tau/\alpha] \ \textsf{(Substituion on type)}
\end{align*}

It should not be confused with instantiation, where the \textbf{bound}
type variables $\forall \alpha_1 , \ldots , \alpha_n$ are substituted inside the type of a
type scheme. Instantiation is not a function though, instead it is
a relation $\sigma > \tau$ in which we say $\sigma$ can be instantiated to $\tau$.
\begin{mathpar}
  \boxed{\sigma > \tau} \\
  \infer{\operatorname{dom}(s) = \{\alpha_1 , \ldots, \alpha_n\} \\ \tau'[s] = \tau}{\forall \alpha_1 , \ldots , \alpha_n
    . \tau' > \tau}
\end{mathpar}
The domain of a substitution $\operatorname{dom}(s)$ is the set of
type variables that it will replace,
i.e.
$\operatorname{dom}([\tau_1/\alpha,\tau_2/\beta]) = \{\alpha, \beta\}$. This relation can be
read as, if there exists a substitution $s$ that substitutes exactly
all the bound type variables in the type scheme
$\forall \alpha_1, \ldots, \alpha_n . \tau'$ to give the type $\tau$, then $\forall \alpha_1, \ldots, \alpha_n . \tau' >
\tau$.

Furthermore we define the relation $\sigma > \sigma'$ on type schemes as well,
where for any two type schemes $\sigma$ and $\sigma'$,
$\sigma > \sigma'$ if for all $\tau$, $\sigma' > \tau \rightarrow \sigma > \tau$.

We then extend this to define substitution on contexts, not
instantiation. Substitution on contexts only substitutes \textbf{free}
type variables.
\begin{align*}
  \centerdot[\tau/\alpha] = \centerdot \\
  (\Gamma , x : \sigma) [\tau/\alpha] &= \Gamma[\tau/\alpha] , x : \sigma[\tau/\alpha]
\end{align*}
Substitution is also defined on terms, replacing term variables
instead of type variables.

\begin{align*}
  \square [v/\alpha] &= \square \\
  \alpha' [v/\alpha] &=
             \begin{cases}
               v & \mathsf{if} \ \alpha' = \alpha \\
               \alpha' & \mathsf{otherwise}
             \end{cases} \\
  (e \ e') [v/\alpha] &= e[v/\alpha] \ e'[v/\alpha] \\
  \lambda x . e [v/\alpha] &=
                  \begin{cases}
                    \lambda x . e & \mathsf{if} \ x = \alpha \\
                    \lambda x . (e [v/\alpha]) & \mathsf{otherwise}
                  \end{cases} \\
  \letin{x = e'}{e} [v/\alpha] &=
                               \begin{cases}
                                 \letin{x = (e' [v/\alpha])}{e} & \mathsf{if} \
                                 x = \alpha \\
                                 \letin{x = (e' [v/\alpha])}{(e [v/\alpha])}
                                 & \mathsf{otherwise}
                               \end{cases} \\  
  (e_1 \times e_2) [v/\alpha] &= e_1[v/\alpha] \times e_2[v/\alpha] \\
  (\pi_i \ e) [v/\alpha] &= {\pi_i (e [v/\alpha])}_{i = 1, 2} \\
  \lift{e} [v/\alpha] &= \lift{e[v/\alpha]} \\
  \use{r}{e} [v/\alpha] &= \use{r}{e[v/\alpha]} \\
  (e \bind e') [v/\alpha] &= e[v/\alpha] \bind e'[v/\alpha] \\
  (e_1 \curlyvee e_2) [v/\alpha] &= e_1[v/\alpha] \curlyvee e_2 [v/\alpha]
\end{align*}

\section{Static Semantics}

Our static semantics, or typing rules, begin with the syntax-directed
rules of the Hindley-Damas-Milner system of the form $\Gamma \vdash e : \tau$.

\begin{mathpar}
    \inferrule*[Right=Var]{e : \tau' \in \Gamma \\ \tau' > \tau}{\Gamma \vdash e : \tau} \and
    \inferrule*[Right=App]{\Gamma \vdash e : \tau' \rightarrow \tau \\ \Gamma \vdash e' : \tau'}{\Gamma \vdash e \ e' : \tau} \and
    \inferrule*[Right=Abs]{\Gamma,x:\tau' \vdash e : \tau}{\Gamma \vdash \lambda x . e : \tau' \rightarrow \tau} \and
    \inferrule*[Right=Let]{\Gamma \vdash e : \tau' \\ \Gamma,x : \overline{\Gamma}(\tau') \vdash e' : \tau}
    {\Gamma \vdash \mathsf{let} \ x = e \ \mathsf{in} \ e' : \tau}
  \end{mathpar}

Note that in the syntax-directed rules, the typing judgement assigns
types to terms, not type schemes~\ref{tofte}. Instantiation occurs
directly within the \textsc{Var} rule, and generalisation occurs
implicitly in $\overline{\Gamma}$.

\newcommand{\fv}{\operatorname{FV}}

Like before, $\overline{\Gamma}$ is defined as
$$ \overline{\Gamma}(\tau) =^{\textsf{def}} \forall \alpha_1, \ldots, \alpha_n . \tau \
\textsf{where} \ \{ \alpha_1, \ldots, \alpha_n \} = \fv(\tau) \setminus \fv(\Gamma)$$

Free type variables are defined on types, type schemes and contexts as
follows:
\begin{align*}
  \fv(\square) &= \{ \} \\
  \fv(\alpha) &= \{ \alpha \} \\
  \fv(\tau \rightarrow \tau') &= \fv(\tau) \cup \fv(\tau') \\
  \fv(\IO_\rho \tau) &= \IO_\rho \fv(\tau) \\
  \fv(\forall \alpha_1 , \ldots , \alpha_n . \tau) &= \fv(\tau) - \{ \alpha_1 , \ldots, \alpha_n \} \\
  \fv(\Gamma) &= \bigcup_{x : \sigma \in \Gamma} \fv(\sigma)
\end{align*}
  
We also introduce the typing rule for $\square$ expressions, to give us a
concrete type, as well as a rule for products which are introduced
through $e \times e'$.
\begin{mathpar}
  \inferrule*[Right=Unit]{ }{\Gamma \vdash \square : \square} \and
  \inferrule*[Right=Product]{\Gamma \vdash e : \tau \\ \Gamma \vdash e' : \tau'}
    {\Gamma \vdash e \times e' : \tau \times \tau'}
\end{mathpar}
Product types also have two eliminators available for them, projecting
out the inner type.
\begin{mathpar}
  \infer*[Right=Proj1]{\Gamma \vdash e : \tau \times \tau'}{\Gamma \vdash \pi_1 e : \tau} \and
  \infer*[Right=Proj2]{\Gamma \vdash e : \tau \times \tau'}{\Gamma \vdash \pi_2 e : \tau'}
\end{mathpar}

Now we introduce the monadic parts of the language. Our language only
has one type of monad, the $\IO$ monad, which is parameterised by both
its \textit{heap} $\rho$ and its encapsulated type. Monadic values are
introduced into the language with $\lift{e}$, which lifts a pure term into
\textbf{any} \textit{well formed} heap. 
\begin{mathpar}
  \infer*[Right=Lift]{\Gamma \vdash e : \tau \\ \textsf{ok} \ \rho}{\Gamma \vdash \lift{e} : \IO_\rho \tau}
\end{mathpar}
We need to be careful what heaps we allow terms to be lifted into, as
the entire point of this system is to avoid heaps containing duplicate
resources. For example, it would be all to easy to just introduce an
expression of $\IO_{File \cup File} \tau$ with \textsc{Lift}, if it was not
for the premise $\textsf{ok} \ \rho$.

$\textsf{ok} \ \rho$ is a new relation we define in
figure~\ref{fig:heapwellformedness} to establish what heaps we
consider to be \textit{well formed}, in a similar vein to
Krishnaswami~\ref{krishnaswami2006}. All heaps consisting of a single
resource are well formed, and all heaps made by unioning two other well
formed heaps \textit{that are distinct} -- i.e. do not share any of the
same resources -- are well formed.
\begin{figure}
  \centering
  \begin{mathpar}
    \boxed{\textsf{ok} \ \rho} \and
    \infer{ }{\textsf{ok} \ \textsf{World}} \and
    \infer{ }{\textsf{ok} \ r} \and
    \infer{
      \textsf{ok} \ \rho \\
      \textsf{ok} \ \rho' \\
      \rho \cap \rho' = \emptyset}
    {\textsf{ok} \ \rho \cup \rho'}    
  \end{mathpar}
  \caption{Heap well formedness}
  \label{fig:heapwellformedness}
\end{figure}

Whilst \textsc{Lift} lets us construct monads in any heap, we will
eventually want to have some constructors for $\IO$monads using
specific resources. When first designing the system, we used
placeholder expressions for fixed resources:
\begin{mathpar}
  \infer{ }{\Gamma \vdash \mathsf{readFile} : \IO_{\mathsf{File}} \square} \and
  \infer{ }{\Gamma \vdash \mathsf{readNetwork} : \IO_{\mathsf{Network}} \square}
\end{mathpar}
$\textsf{readFile}$ and $\textsf{readNet}$ are examples of typical
operations that can consume a specific resource -- their heap consists
of just a single resource. This was then generalised to
$\use{r}{e}$, which lifts any pure term into an $\IO$ monad with a heap
consisting of the resource $r$.
\begin{mathpar}
  \infer*[Right=Use]{\Gamma \vdash e : \tau}{\Gamma \vdash \use{r}{e} : \IO_r e}
\end{mathpar}
Note that it is defined with a single resource, not a heap. A heap
with just one resource is always well formed, so there is no need
for a $\textsf{ok} \ r$ premise in this judgement.
\textsc{Use} is meant to be used to annotate terms that use resources that
should not be accessed concurrently, whilst on the other hand
\textsc{Lift} is for bringing pure expressions into a monadic
computation.

Once we have an $\IO$ value, we can sequence computation by binding
it with a function that returns another $\IO$.
\begin{mathpar}
  \inferrule*[Right=Bind]{\Gamma \vdash e : \IO_\rho \tau \\ \Gamma \vdash e' : \tau \rightarrow \IO_\rho
    \tau'}{\Gamma \vdash e \bind e' : \IO_\rho \tau'}
\end{mathpar}
Note that the types of the two $\IO$s must use the same
resources. However, we want expressions such as
$$ \use{\textsf{File}}{\square} \bind \use{\textsf{Net}}{\square} $$
to be well typed. In particular, we want the above expression to be of
type $\IO_{\textsf{File} \cup \textsf{Net}} \square$. But if \textsc{Bind}
requires the resources to be the same, we must first somehow ``cast''
$\textsf{readFile}$ and $\textsf{readNet}$ to this type.

Thus serves the purpose of the \textsc{Sub} (subsumption) rule.
\begin{mathpar}
  \inferrule*[Right=Sub]{\Gamma \vdash e : \IO_\rho \tau \\ \rho \subtyp \rho' \\
    \textsf{ok} \ \rho}
  {\Gamma \vdash e : \IO_{\rho'} \tau}
\end{mathpar}
It allows something of type $\IO_\rho \tau$ monad to be typed as $\IO_{\rho'} \tau$, provided
that the heap $\rho$ is a \textit{subheap} of $\rho'$. The subheap
rules for heaps, shown in figure~\ref{fig:subheap}, define the
$\subtyp$ relation.

\begin{figure}

\begin{mathpar}
\boxed{\rho \subtyp \rho'} \\
  
\inferrule*[Right=Top]{ }{\rho \subtyp \textsf{World}} \and
\inferrule*[Right=Refl]{ }{\rho \subtyp \rho} \and
\inferrule*[Right=UnionL]{\rho \subtyp \rho'}{\rho \subtyp \rho' \cup \rho''} \and
\hskip 1em
\inferrule*[Right=UnionR]{\rho \subtyp \rho'}{\rho \subtyp \rho'' \cup \rho'}

\end{mathpar}

\caption{Subheap rules}
\label{fig:subheap}
\end{figure}

Intuitively, a heap $\sigma$ can thought of being a subheap of another heap
$\sigma'$, if $\sigma'$ subsumes $\sigma$, similarly to how subtyping works. For
example, $\textsf{Net} \subtyp \textsf{Net} \cup \textsf{File}$, since
$\textsf{Net} \cup \textsf{File}$ ``overlaps'' with the heap
$\textsf{Net}$.

\begin{figure}
  \centering
  \tikz \graph[layered layout] {
    world/"\textsf{World}"; file/"\textsf{File}"; net/"\textsf{Net}";
    database/"\textsf{Database}"; databasenet/"$\textsf{Database} \cup \textsf{Net}$";
    filenet/"$\textsf{File} \cup \textsf{Net}$";
    world -> filenet;
    world -> databasenet;
    databasenet -> {database, net};
    filenet -> {file, net};
  };
  \caption{An example of some heaps and their subheap orderings.}
\end{figure}

If one views this relation as an ordering, then we have
$\textsf{World}$ defined as the least upper bound -- this represents
using all possible resources, and as mentioned earlier
$\IO_{\textsf{World}}$ can be thought of as the $IO$ monad in Haskell,
where sequencing interacts with the state of the entire world.

The subheaping relation is both reflexive (by definition), and transitive.
\begin{theorem}
  For all $a$, $b$, $c$, if $a \subtyp b$ and $b \subtyp c$ then $a
  \subtyp c$.
\end{theorem}
\begin{proof}
  By induction on $b \subtyp c$. See
  proof~\ref{proof:subheaptransitive} in the appendix.
\end{proof}

By constraining subtyping to only resources and not actual types, we
avoid the various issues associated with undecidability subtyping
gives us~\ref{????}.

The union of resources, for example,
$\textsf{Net} \cup \textsf{File}$, allows us to reason about using more
than one resource simultaneously. In fact, the purpose of this type
system is to help reason about using multiple resources in a safe
manner, by rejecting programs that use the same resource twice
simultaneously. In this light, the \textsc{Conc} rule is then the
heart and soul of this type system, by combining together heaps and
maintaining the invariant that the resources in the heap are all
unique.
\begin{mathpar}
  \infer*[Right=Conc]{
    \Gamma \vdash e_1 : \IO_{\rho_1} \tau_1 \\
    \Gamma \vdash e_2 : \IO_{\rho_2} \tau_2 \\
    \textsf{ok} \ \rho_1 \cup \rho_2}
  {\Gamma \vdash e_1 \curlyvee e_2 : \IO_{\rho_1 \cup \rho_2} \ \tau_1 \times \tau_2}
\end{mathpar}
The premises include that the new merged heap must be well formed, as we do not
want to allow programs that try to use the same resource concurrently,
such as
$$\use{\textsf{File}}{\square} \curlyvee \use{\textsf{File}}{\square}$$
We do however, want to allow programs that run two expressions that do
not share any resources, like
$$\use{\textsf{File}}{\square} \curlyvee \use{\textsf{Net}}{\square}$$

The result of running two separate $\IO$actions concurrently produces
an expression whose type is also an $\IO$ monad, but with the two
separate heaps joined. In the above example, it would be infered the
type $\IO_{\textsf{File} \cup \textsf{Net}}$.

All together these rules define the typing judgement, and are
displayed in full in figure~\ref{fig:typingrules}. Let's look at some
examples at how they can be used to derive types for various
programs.

\subsection{Monadic binding}
Here we show a proof tree for the expression
$$\letin{x = \lambda y . \lift{\square}}{(x \ \square) \bind \lambda z . \llbracket z
  \rrbracket_{\textsf{File}}}$$
We have a use of the lift operator, from which we can then infer
must be a heap of \textsf{File}, because of the use operator
restricting it later down the line.
\begin{mathpar}
  \mprset {sep=1em}
  \infer{
    \infer{
      \infer{}{\centerdot, y : \alpha \vdash \square : \square}
    } {
      \centerdot , y : \alpha \vdash \llbracket \square \rrbracket : \IO_{\textsf{File}} \square
    } \\
    \textsf{ok} \ \textsf{File}
  }
  { \centerdot \vdash \lambda y . \lift{\square} : \alpha \rightarrow \IO_{\textsf{File}} \square \\ \mathbf{(1)} }
  \\
  \infer{
    \infer{
      \infer{}{\centerdot,x :\alpha \rightarrow \IO_{\textsf{File}} \square \vdash x : \alpha \rightarrow
        \IO_{\textsf{File}} \square}
      \\\\
      \infer{}{\centerdot, x : \alpha \rightarrow \IO_{\textsf{File}} \square \vdash \square : \square}
    }{\centerdot, x : \alpha \rightarrow \IO_{\textsf{File}} \square  \vdash x \ \square :
      \IO_{\textsf{File}} \square}
    \\
    \infer{
      \infer{
        \infer{z : \alpha \in \centerdot, x : \alpha \rightarrow \IO_{\textsf{File}} \square, z : \alpha \\ \alpha > \square}
        {\centerdot, x : \alpha \rightarrow \IO_{\textsf{File}} \square, z : \alpha \vdash z : \square}
      }
      {\centerdot, x : \alpha \rightarrow \IO_{\textsf{File}} \square , z : \alpha \vdash \llbracket z
        \rrbracket_{\textsf{File}} : \IO_{\textsf{File}} \square}
    }{
      \centerdot, x : \alpha \rightarrow \IO_{\textsf{File}} \square  \vdash \lambda z . \llbracket z
      \rrbracket_{\textsf{File}} : \IO_{\textsf{File}} \square
    }
  }{
    \centerdot, x : \alpha \rightarrow \IO_{\textsf{File}} \square \vdash (x \ \square) \bind \lambda z . \llbracket z
    \rrbracket_{\textsf{File}} : \IO_{\textsf{File}} \square \\ \mathbf{(2)}
  }
  
  \infer{ \mathbf{(1)} \\ \mathbf{(2)} } {\centerdot \vdash \letin{x = \lambda y
      . \lift{\square}}{(x \ \square) \bind \lambda z . \llbracket z \rrbracket_{\textsf{File}}} :
    \IO_{\textsf{File}} \square}
\end{mathpar}

\subsection{Let polymorphism}
This example doesn't contain any resourceful elements, but is just an
example of how let polymorphism allows the same variable lookup to be
inferred different types on each call site. Note how the premises in
$\mathbf{(1)}$ and $\mathbf{(2)}$ all infer $x$ to have different types.
\begin{mathpar}
  % app
  \infer{
    . , x : \alpha \rightarrow \alpha \vdash x : (\square \rightarrow \square) \rightarrow (\square \rightarrow \square) \\
    . , x : \alpha \rightarrow \alpha \vdash x : \square \rightarrow \square
  }
  {\centerdot, x : \alpha \rightarrow \alpha \vdash x \ x : \square \rightarrow \square \\ \mathbf{(1)}}
  \\
  \infer{
    . , x : \alpha \rightarrow \alpha \vdash x : \square \rightarrow \square \\
    . , x : \alpha \rightarrow \alpha \vdash \square : \square
  }
  {\centerdot, x : \alpha \rightarrow \alpha \vdash x \ \square : \square \\ \mathbf{(2)}}
  \\
  \infer{
    \infer{
      \infer{
        y : \alpha \in \centerdot , y : \alpha \\ \alpha > \alpha
      }{\centerdot , y : \alpha \vdash y : \alpha}
    }
    {\centerdot \vdash \lambda y . y : \alpha \rightarrow \alpha} \\
    % app
    \infer{
      \mathbf{(1)} \\ \mathbf{(2)}
    }{\centerdot, x : \alpha \rightarrow \alpha \vdash (x \ x) \ ( x \ \square) : \square}
  }
  { \centerdot \vdash \letin{x = \lambda y . y}{(x \ x) \ (x \ \square)} : \square}
\end{mathpar}

\subsection{Concurrency}
Here is our first example of accessing two resources concurrently --
with a bit of imagination one can think of this as reading a file
from a disk whilst simultaneously fetching data over the network.
\begin{mathpar}
  \mprset {sep=1em}
  \infer{
    % app
    \infer{
      %abs
      \infer{
        % use
        \infer{
          \centerdot, x : \alpha \vdash x : \alpha \\
          \textsf{ok} \ \textsf{File}
        }{\centerdot, x : \alpha \vdash \llbracket x \rrbracket_{\textsf{File}} : \IO_{\textsf{File}} \alpha}
      }{\centerdot \vdash \lambda x . \llbracket x \rrbracket_{\textsf{File}} : \alpha \rightarrow \IO_{\textsf{File}} \alpha}
      \\
      \centerdot \vdash \square : \square
    }{\centerdot \vdash (\lambda x . \llbracket x \rrbracket_{\textsf{File}}) \ \square : \IO_{\textsf{File}} \square}
    \\
    \infer{ %use
      \centerdot \vdash \square : \square \\ \textsf{ok} \ \textsf{Net}
    }{\centerdot \vdash \llbracket \square \rrbracket_{\textsf{Net}} : \IO_{\textsf{Net}} \square}
    \\
    \textsf{File} \cap \textsf{Net} = \emptyset
  }
  {\centerdot \vdash (\lambda x . \llbracket x \rrbracket_{\textsf{File}}) \ \square) \curlyvee \llbracket \square \rrbracket_{\textsf{Net}} : \IO_{\textsf{File} \cup \textsf{Net}} \square \times \square}
\end{mathpar}

\subsection{Subsumption}
Assuming there is a function inside our context called
\textsf{writeFile} with the type, $\square \rightarrow \IO_{\textsf{File}} \square$, we can
subsume its heap to be part of a larger heap, namely
$\textsf{File} \cup \textsf{Net}$. We will show this with the expression
$$ \llbracket \square \rrbracket_{\textsf{File}} \curlyvee \llbracket \square \rrbracket_{\textsf{Net}}
\bind
\lambda x . \textsf{writeFile} \ (\pi_1 \ x)
$$
{
  \begin{mathpar}
    \Gamma = \centerdot, \textsf{writeFile} : \square \rightarrow \IO_{\textsf{File}} \square \\
    \infer{ %conc
      \infer{ %use
        \Gamma \vdash \square : \square
      }{\Gamma \vdash \llbracket \square \rrbracket_{\textsf{File}} : \IO_{\textsf{File}} \square} \\
      \infer{ %use
        \Gamma \vdash \square : \square
      }{\Gamma \vdash \llbracket \square \rrbracket_{\textsf{Net}} : \IO_{\textsf{Net}} \square} \\
      \textsf{File} \cap \textsf{Net} = \emptyset
    }
    {\Gamma \vdash \llbracket \square \rrbracket_{\textsf{File}} \curlyvee \llbracket \square \rrbracket_{\textsf{Net}} : \IO_{\textsf{File}
        \cup \textsf{Net}} \square \\ \mathbf{(1)}}
    \\
    \infer{ %app
      \Gamma , x : \square \times \square \vdash \textsf{writeFile} : \square \rightarrow \IO_{\textsf{File}} \square
      \\
      \infer{
        \Gamma , x : \square \times \square \vdash x : \square \times \square
      }{\Gamma , x : \square \times \square \vdash \pi_1 \ x : \square}
    }{\Gamma , x : \square \times \square \vdash \textsf{writeFile} \ (\pi_1 \ x) :
      \IO_{\textsf{File}} \square \\ \mathbf{(2)}}
    \\
    \infer{ %bind
      \mathbf{(1)} \\
      \infer{ %abs
        \infer*[Right=Sub]{ %sub
          \mathbf{(2)} \\
          \textsf{File} \geq: \textsf{File} \cup \textsf{Net} \\
          \textsf{ok} \ \textsf{File} \cup \textsf{Net}
        }{\Gamma, x : \square \times \square \vdash \textsf{writeFile} \ (\pi_1 \ x) : \IO_{\textsf{File} \cup \textsf{Net}} \square}
      }{\Gamma \vdash \lambda x . \textsf{writeFile} \ (\pi_1 \ x) : \IO_{\textsf{File} \cup \textsf{Net}} \square}
    }{\Gamma \vdash
      \llbracket \square \rrbracket_{\textsf{File}} \curlyvee \llbracket \square \rrbracket_{\textsf{Net}}
      \bind
      \lambda x . \textsf{writeFile} \ (\pi_1 \ x)
      : \IO_{\textsf{File} \cup \textsf{Net}} \square}
  \end{mathpar}
}

\begin{figure}
  \begin{mathpar}
    \boxed{\Gamma \vdash e : \tau} \\
    
    \inferrule*[Right=Var]{x : \tau' \in \Gamma \\ \tau' > \tau}{\Gamma \vdash x : \tau} \and
    \inferrule*[Right=App]{\Gamma \vdash e : \tau' \rightarrow \tau \\ \Gamma \vdash e' : \tau'}{\Gamma \vdash e \ e' : \tau} \and
    \inferrule*[Right=Abs]{\Gamma,x : \tau \vdash e : \tau'} {\Gamma \vdash \lambda x . \ e : \tau \rightarrow
      \tau'} \and
    \inferrule*[Right=Let]{\Gamma \vdash e' : \tau' \\ \Gamma,x : \overline{\Gamma}(\tau') \vdash e : \tau}
    {\Gamma \vdash \mathsf{let} \ x = e' \ \mathsf{in} \ e : \tau} \and
    % \inferrule*[Right=If]{\Gamma \vdash e_1 : \mathbf{Bool} \\ \Gamma \vdash e_2 : \tau \\ \Gamma \vdash e_3 : \tau}
    % {\Gamma \vdash \mathsf{if} \ e_1 \ \mathsf{then} \ e_2 \ \mathsf{else} \
    % e_3 : \tau} \and

    \inferrule*[Right=Unit]{ }{\Gamma \vdash \square : \square} \\
    \inferrule*[Right=Product]{\Gamma \vdash e : \tau \\ \Gamma \vdash e' : \tau'}
    {\Gamma \vdash e \times e' : \tau \times \tau'} \and

    \inferrule*[Right=Proj1]{\Gamma \vdash e : \tau \times \tau'}{\Gamma \vdash \pi_1 : \tau} \and     
    \inferrule*[Right=Proj2]{\Gamma \vdash e : \tau \times \tau'}{\Gamma \vdash \pi_2 : \tau'} \\

    \infer*[Right=Lift]{\Gamma \vdash e : \tau \\ \textsf{ok} \ \rho}{\Gamma \vdash \lift{e} :
      \IO_\rho \tau} \and
    \infer*[Right=Use]{\Gamma \vdash e : \tau}{\Gamma \vdash \use{r}{e} : \IO_r e} \\
    \inferrule*[Right=Bind]{\Gamma \vdash e : \IO_\rho \tau' \\ \Gamma \vdash e' : \tau' \rightarrow \IO_\rho
      \tau}{\Gamma \vdash e \bind e' : \IO_\rho \tau} \\
    \infer*[Right=Conc]{
      \Gamma \vdash e_1 : \IO_{\rho_1} \tau_1 \\
      \Gamma \vdash e_2 : \IO_{\rho_2} \tau_2 \\
      \textsf{ok} \ \rho_1 \cup \rho_2}
    {\Gamma \vdash e_1 \curlyvee e_2 : \IO_{\rho_1 \cup \rho_2} \ \tau_1 \times \tau_2} \and
    
    \inferrule*[Right=Sub]{\Gamma \vdash e : \IO_\rho \tau \\ \rho \subtyp \rho' \\
      \textsf{ok} \ \rho}
    {\Gamma \vdash e : \IO_{\rho'} \tau}

  \end{mathpar}

  \caption{Typing rules} \label{fig:typingrules}
\end{figure}

\section{Dynamic semantics}
Dynamic semantics model how the program actually executes at
runtime. It plays a crucial part in how we prove that the type system
is sound, as we can only guarantee that our type system ensures
error-free programs if we model how the program runs.

As shown in chapter~\ref{chapter:background}, the original work
carried out in the Hindley-Damas-Milner systems were based on
denotational semantics. In our type system, we will use operational
semantics. Operational semantics are similar to what we have seen
before in the definition of the static semantics. We define a bunch of
inference rules, and from these build up proofs. Tofte~\cite{tofte1988}
had the idea of using operational semantics for not for the
typing rules, but also for the dynamic semantics. We choose this
approach over denotational semantics as it unifies our approach to
modelling and proving, and to quote Mads, ``it seems a bit
unfortunate that we should have to understand domain theory to be able
to investigate whether a type inference system admits faulty
programs''.


\subsection{Values}
Before we talk about how we evaluate a program, we need to
define what constitutes a finished evaluation -- that is, what
terms are a result of a completed computation.

We define a value property, and give rules describing what terms can
be considered values in figure~\ref{fig:values}. For example, we
cannot evaluate the $\square$ type any further, therefore all $\square$s
are considered values. The same goes for product types, but only if
both inner components are values themselves. $\square \times \square$ is a
finished value, but $f \ e \times \square$ might still have evaluation left to
do on the right hand side.
A lambda on its own is a value too --
without being applied to an argument it cannot be evaluated any
further. The computation inside of it is suspended.
In a similar fashion a lifted computation cannot be computed any
further \textit{on its own}. We will see later how binding can run
this computation, but by itself it will not evaluate to anything.

\begin{figure}
  \begin{mathpar}
    \boxed{\textsf{value} \ e} \\
    \inferrule{ }{ \textsf{value} \ \square } \and
    \inferrule{\textsf{value} \ e_1 \\ \textsf{value} \ e_2}{ \textsf{value} \ e_1 \times e_2 } \and
    \inferrule{ }{ \textsf{value} \ \lift{e} } \and
    \inferrule{ }{ \textsf{value} \ \lambda x. e }
  \end{mathpar}
  \caption{Terminal values} \label{fig:values}
\end{figure}

\subsection{Small-step semantics}

Specifically, we will be using small-step operational semantics. In
small-step operational semantics we define a step relation
$a \leadsto b$ which says that in one ``step'', $a$ \textit{reduces to}
$b$. $b$ might then go onto reduce further if it is able to, or it
could be its final value. Reduction can be thought of as
evaluation. Small-step semantics differs from big-step semantics,
where a relation $a \Downarrow b$ says that at the end of the day, $a$ will
reduce to $b$, and $b$ will not reduce any further.


As an example, if an expression $e_1$ reduces to $e_1'$, i.e. $e_1 \leadsto
e_1'$, then we want the $e_1$ in the application $e_1 \ e_2$ to reduces
as well. We can write this as
\begin{mathpar}
  \inferrule{e_1 \leadsto e_1'}{e_1 \ e_2 \leadsto e_1' \ e_2}
\end{mathpar}

We call this type of reduction which takes smaller reductions and
updates it within a bigger structure \xi-reduction. There are
\xi-reduction for other expressions with structure inside, namely
product types, the concurrent operator and the bind operator.

Another
type of reduction rule is \beta-reduction, which comes from the lambda
calculus. When we apply a term to a function, we substitute the bound
variable inside the lambda with the argument. A \beta-reduction rule
defines this.
\begin{mathpar}
  \inferrule{\textsf{value} \ e_2}{ (\lambda x . e_1) e_2 \leadsto e_1 [ e_2 / x ]}
\end{mathpar}

It is important to note the premise that the argument being applied to
the lambda must be a value. This enforces a strict evaluation order,
since in order for the term to be a value it must be completely
reduced. A lazily evaluated semantics might forgo this extra
requirement, so that the argument can be reduced after substitution.

As mentioned earlier, lifted expressions $\lift{e}$ are suspended much
like lambdas, and as such are values since they cannot reduce any
further \textit{on their own}. However, with the bind operator, the
value inside them can be extracted out and fed into a function.
\begin{mathpar}
  \inferrule{ }{\lift{e} \bind e' \leadsto e' \ e}
\end{mathpar}

Unlike the semantic rule for \beta-reduction, there is no premise
enforcing that $\lift{e}$ is a value, since all lifted terms are
values anyway.

For an expression lifted into a resourceful $\IO$ monad with
$\use{\rho}{e}$, one might be tempted to just reduce this to a $lift{e}$.
\begin{mathpar}
  \infer{ }{\use{r}{e} \leadsto \lift{e}}
\end{mathpar}

And we could then also define the reduction for concurrency like so:
\begin{mathpar}
\inferrule{ }{\lift{v} \curlyvee \lift{w} \leadsto \llbracket v \times w \rrbracket}
\end{mathpar}

However the intermediate $\lift{e}$ can be confusing -- the purpose of
the lift operator is to lift a pure value into any possible resource
bound monad. When we see a lift, we think of the typing judgement that
allows it to fit any heap, when in fact a use should restrict what
heaps it can go into. For these reasons, we instead define its
reduction identically to lift.
\begin{mathpar}
  \inferrule{ }{\use{r}{e} \bind e' \leadsto e' \ e}
\end{mathpar}

Concurrency is then define as chaining together two binds, and
returning the lifted product of the two results.
\begin{mathpar}
  \inferrule{ }{v \curlyvee w \leadsto
    v \bind \lambda v . (w \bind \lambda w . \lift{v \times w})}
\end{mathpar}

This might seem like the opposite of concurrency -- executing the
computation in sequence -- but because our monad does not have any
state (see section~\ref{section:modellingstate},
${v \bind \lambda v . (w \bind (\lambda w . \lift{v \times w}))}$ is identical to
$w \bind \lambda w . (v \bind (\lambda v . \lift{v \times w}))$. In fact the previous
definition for concurrency as
$\lift{v} \curlyvee \lift{w} \leadsto \llbracket v \times w\rrbracket$ will have the same reduction steps at
the end of the day. In an actual implementation in a programming
language, the concurrent operator should be replaced with something
actually concurrent.

Although the $v \curlyvee w$ might evaluate to two separate $\bind$s, their
static semantics are different. Namely, given
$\Gamma \vdash \lift{v} : \IO_\rho \tau$,
$$
\Gamma \vdash \lift{v} \bind \lambda v . (\lift{v} \bind (\lambda v . \lift{v \times v})) : \IO_\rho (\tau \times \tau)
$$

But for the concurrent operator, this is a type error, since the
premise $\textsf{ok} \ \rho \cup \rho$ does not hold: There is no type in any context
that can be given to $v \curlyvee v$

\begin{figure}
  \begin{mathpar}
    \boxed{e \leadsto e'} \\
    % \inferrule*{x \in \eta}{\eta \vdash x \leadsto \Gamma(x)} \and
    % \inferrule*{x \notin \eta}{\eta \vdash x \leadsto \textsf{wrong}} \and
    % \inferrule*{ }{\eta \vdash \lambda x . e \leadsto [x,e,\eta]} \and
    % \inferrule*{\eta \vdash e_1 \leadsto [x,e_0,\eta'] \\
    %   \eta \vdash e_2 \leadsto v \\
    %   \eta', x \mapsto v \vdash e_0 \leadsto r}
    % {\eta \vdash e_1 e_2 \leadsto r } \and
    % \inferrule*{\eta \vdash e_1 \leadsto \llbracket v \rrbracket \\
    %   \eta \vdash e_2 \leadsto [x, e_0, \eta'] \\
    %   \eta, x \mapsto v \vdash e_0 \leadsto }{
    %   \eta \vdash e_1 \bind e_2 \leadsto \llbracket \rrbracket}


    % lambda calculus + HM
    \inferrule{e_1 \leadsto e_1'}{e_1 \ e_2 \leadsto e_1' \ e_2} \and
    \inferrule{e_2 \leadsto e_2'}{e_1 \ e_2 \leadsto e_1 \ e_2'} \and
    \inferrule{\textsf{value} \ e'}{ (\lambda x . e) e' \leadsto e [ e' / x ] } \and
    \inferrule{ }{\letin{x = e}{e'} \leadsto e' [e / x]} \\

    % product types
    \inferrule{e_1 \leadsto e_1'}{e_1 \times e_2 \leadsto e_1' \times e_2} \and
    \inferrule{e_2 \leadsto e_2'}{e_1 \times e_2 \leadsto e_1 \times e_2'} \and

    \inferrule{e \leadsto e'}{\pi_1 e \leadsto \pi_1 e'} \and
    \inferrule{e \leadsto e'}{\pi_2 e \leadsto \pi_2 e'} \and

    \inferrule{ }{\pi_1 (e_1 \times e_2) \leadsto e_1} \and
    \inferrule{ }{\pi_2 (e_1 \times e_2) \leadsto e_2} \and

    % monads
    \inferrule{e_1 \leadsto e_1'}{e_1 \bind e_2 \leadsto e_1' \bind e_2} \and
    \inferrule{ }{\lift{v} \bind e_2 \leadsto e_2 \ v} \and
    \inferrule{ }{\use{r}{e} \bind e' \leadsto e' \ e} \\

    % resource stuff
    \inferrule{ }{v \curlyvee w \leadsto v \bind \lambda v . (w \bind \lambda w . \lift{v \times
        w})} \
  \end{mathpar}
  \caption{Dynamic Semantics} \label{fig:reduction}
\end{figure}

\subsection{Reduction relation}

We can define $\twoheadrightarrow$ as the reflexive, transitive closure of
$\leadsto$. What does that mean? If
$\leadsto$ is a relation on two terms, then the transitive closure
$\twoheadrightarrow$ is a new relation that contains $\leadsto$ and is transitive, i.e. if
$a \twoheadrightarrow b$ and $b \twoheadrightarrow c$, then
$a \twoheadrightarrow c$.  A reflexive transitive closure extends this so that for any
$a$, $a \twoheadrightarrow a$.

Intuitively speaking, if the small-step inference rule $a \leadsto b$ says
that $a$ reduces to $b$ in exactly one step, then
$a \twoheadrightarrow b$ says $a$ reduces to $b$ in zero or more steps -- instead of
reducing one step at a time, it goes all the way.

\begin{figure}
  \begin{center}
    \tikz \graph[layered layout, grow=right, edges={
      decoration={snake,amplitude=0.5mm,segment length=2mm,post
        length=1mm},decorate}] {
      a -> b -> c
    };
    \qquad
    \tikz \graph[layered layout,grow=right,edges={->>}] {
      a ->[orient] b -> c;
      a ->[bend right] c;
      { [same layer] a, b, c };
      a ->[loop above] a;

      b ->[loop above] b;

      c ->[loop above] c;

      % a -> c;
      % b -> c;
      % a -> c;
    };
\end{center}
\end{figure}


% Note that the subsumption is limited to only the heap of the $\IO$
% monad.

% The motivating example for the subtyping rules is that the program

% \begin{math}
%   \mathsf{if} \ x \ \mathsf{then} \ \mathsf{readFile} \ \mathsf{else} \
%   \mathsf{readNetwork}
% \end{math}

% Should have the type $\IO_{\mathsf{File} \cup \mathsf{Network}} \square$


% \section{Semantics}

% A denotational semantics is provided to prove certain properties of
% the language and its type inference.

% We want to be able to prove that the type system is sound -- that is
% that if our system infers the type for an expression $e : \tau$, then the
% expression does indeed semantically evaluate to a value in $\tau$.

% All values are modelled with an element inside the set of
% $\mathbb{V}$.
% It consists of values from various other sets, such as
% $\mathbb{B}_{\mathsf{Bool}}$, the set of values for a the
% $\mathsf{Bool}$ type.

% Semantic domain:

% \begin{align*}
%   \mathbb{V} &= \mathbb{B}_{\square} + \mathbb{B}_{\textsf{Bool}} + \mathbb{F} +
%                \mathbb{W} \\
%   \mathbb{B}_{\square} &= \{ \ \square \ \} \\
%   \mathbb{B}_{\mathsf{Bool}} &=
%                                \{ \textsf{True}, \textsf{False},
%                                \bot_{\textsf{Bool}} \} \\
%   \mathbb{F} &= \mathbb{V} \rightarrow \mathbb{V} \\
%   \mathbb{W} &= \{ \ . \ \}
% \end{align*}


% We can use the syntax for semantic entailment, $\Vdash v : \tau$, which is
% defined as

% \begin{math}
%   v \in \mathbb{V}^\tau \iff \Vdash v : \tau
% \end{math}

% Whilst on the syntactic side, $\Gamma$ is the type environment, we also
% define a semantic environment $\rho$. It is is a map of variable
% identifiers to values.

% \begin{math}
%   \rho : \textsf{Id} \rightarrow \mathbb{V}
% \end{math}


% We use $\llbracket e \rrbracket$ notation to represent passing a
% syntactic argument $e$ to a semantic function.

% We say an environment respects a type environment $\rho : \Gamma$ if:
% \begin{math}
%   \rho : \Gamma \iff \forall x : \tau \in \Gamma \rho \llbracket x \rrbracket : \tau
% \end{math}


% $\mathbb{B}_{Bool} = \{ \textsf{true} \textsf{false} \bot_{bool} \}$

% Semantic function $\mathbb{E} : \mathsf{Expr} \rightarrow \mathsf{Env} \rightarrow \mathbb{V}$


% There is a conditional operator $\hookrightarrow : \mathbb{B}_{\textsf{Bool}} \rightarrow \mathbb{V} \rightarrow \mathbb{V} \rightarrow \mathbb{V}$, defined as follows.

% \begin{align*}
%   t \hookrightarrow v, v' =
%   \begin{cases}
%     v & \textsf{if} \ t = \textsf{true} \\
%     v' & \textsf{if} \ t = \textsf{false} \\
%     \bot_v & \textsf{if} \ t = \bot_t
%   \end{cases}
% \end{align*}

% \begin{align*}
%   \mathbf{E} \llbracket x \rrbracket \rho &= \rho(x) \\
%   \mathbf{E} \llbracket e \ e' \rrbracket \rho &= \textsf{isF} v_1 \hookrightarrow
%                                               {v_1}_{|F} v_2 \\
%   \mathbf{E} \llbracket \lambda x . \ e \rrbracket \rho &=
%   (\lambda v . \ \mathbf{E} \llbracket e \rrbracket \rho \{ v / x \} ) \
%                                                  \textsf{in} \ \mathbb{V}
% \end{align*}

% Semantic domain + semantic function = semantic algebra

\section{Type Safety}

A type system is no good unless we can prove it's worth its salt. In
this section we will prove a number of lemmas and theorems which will
eventually show that the type system is sound. We will begin with a
couple of properties about contexts that we will later use on in
various other proofs. The first states that if we have two variables
in the context with the same names but different type schemes, then we
can ignore the first one as it is overshadowed by the second -- any
reference to $x$ will result in $x : \sigma'$.
\begin{lemma} \label{lem:drop}
  If $\Gamma , x : \sigma, x : \sigma' \vdash e : \tau$, then $\Gamma , x : \sigma' \vdash e : \tau$
\end{lemma}
The second states that we can sneak in another variable before another
variable of the same name, for the exact same reason.
\begin{lemma} \label{lem:sneak}
  If $\Gamma , x : \sigma' \vdash e : \tau$, then for any $\sigma$, $\Gamma , x : \sigma, x : \sigma' \vdash e : \tau$
\end{lemma}
We can then also say that if two variables $x$ and $y$ are indeed
different, we are free to permute them and swap them about.
\begin{lemma} \label{lem:swap}
  If $x \neq y$ and $\Gamma , x : \sigma, y : \sigma' \vdash e : \tau$, then $\Gamma , y : \sigma' , x : \sigma \vdash e
  : \tau$
\end{lemma}

We will also need to show this property about instantiation and the
close function.
\begin{lemma} \label{lem:close>}
  If $\sigma > \sigma'$ then $\overline{\Gamma , x : \sigma}(\tau) > \overline{\Gamma, x : \sigma'}(\tau)$.
\end{lemma}
\begin{proof}
  We provide an informal proof as follows.
  \begin{enumerate}
  \item If $\sigma > \sigma'$, then for every $\tau$ that $\sigma > \tau$, $\sigma' > \tau$.
  \item So $\sigma$ must parameterise over at least the same number of type
    variables if not \textit{more} than $\sigma'$.
  \item So $\sigma$ has at least the same number of free type variables if
    not \textit{less} than $\sigma'$.
  \item By definition of the close function,
    $\overline{\Gamma, x : \sigma}(\tau)$ will then have at least the same number
    of bound type variables if not \textit{more} than
    $\overline{\Gamma, x : \sigma'}(\tau)$.
  \item So if $\overline{\Gamma, x : \sigma'}(\tau)$ can be instantiated to some
    $\tau$, then $\overline{\Gamma, x : \sigma}(\tau)$ can also be instantiated to
    that $\tau$, as it has enough bound type variables to handle
    everything the former can -- if it had less type variables than the
    former, then it would not be able to instantiate these types, but
    for the converse excess type variables can be mapped to whatever.
  \end{enumerate}
\end{proof}

Now we prove the generalisation theorem, which states that if a type
scheme inside the context is an instantiation of another type scheme,
then we can use the more general type scheme and preserve how things
are typed. This is an adaptation of a lemma from Wright and
Felleisen~\cite[Lemma 4.6]{wright1994}, which is in turn an
adaptation of a lemma from Damas and Milner.

\begin{theorem}[Generalisation]
  If $\Gamma, x : \sigma' \vdash e : \tau$ and $\sigma > \sigma'$, then ${\Gamma, x : \sigma \vdash e : \tau}$.
  \label{lem:generalInContext}
\end{theorem}
\begin{proof}
  Begin with induction on the proof for $\Gamma , x : \sigma' \vdash e : \tau$.
  \begin{description}
  \item[\rm\textsc{Var}]
    From the premises we have $x : \sigma' \in \Gamma$ and $\sigma' > \tau$. We also have
    $\sigma > \sigma'$ but by definition of $\sigma > \sigma'$, if $\sigma' > \tau$ then $\sigma > \tau$.
    And we also have by definition $x : \sigma \in \Gamma , x : \sigma$. So putting the
    pieces together, we can use \textsc{Var} to get
    \begin{mathpar}
      \infer{x : \sigma \in \Gamma , x : \sigma \\ \sigma > \tau}
      {\Gamma , x : \sigma \vdash x : \tau}
    \end{mathpar}
  \item[\rm\textsc{Abs}]
    We have $\Gamma , x : \sigma' \vdash \lambda y . e : \tau$. If $x = y$, then we can safely
    say the premise $\Gamma , x : \sigma' , y : \sigma'' \vdash e : \tau$ becomes $\Gamma , y :
    \sigma'' \vdash e : \tau$  due to lemma~\ref{lem:drop}. And from this we can
    then use lemma~\ref{lem:sneak} to get $\Gamma, x : \sigma, y : \sigma'' \vdash e : \tau$,
    and ultimately $\Gamma , x : \sigma \vdash \lambda y . e : \tau$.

    If $x \neq y$, then we can get $\Gamma , y : \sigma'' , x : \sigma' \vdash e : \tau$ via
    lemma~\ref{lem:swap}. The induction hypothesis then results in $\Gamma
    , y : \sigma'' , x : \sigma \vdash e : \tau$, which we can then swap back again to
    get $\Gamma , x : \sigma , y : \sigma'' \vdash e : \tau$ and so $\Gamma , x : \sigma \vdash \lambda y . e :
    \tau$.
  \item[\rm\textsc{Let}]
    For $\letin{y = e'}{e}$, we have $\Gamma , x : \sigma' \vdash e' : \tau'$ and $\Gamma , x :
    \sigma' , y : \overline{\Gamma , x : \sigma'}(\tau') \vdash e : \tau$, and we aim to show
    $\Gamma , x : \sigma \vdash \letin{y=e'}{e} : \tau$.
    
    First off, we need to convert the $y : \overline{\Gamma , x : \sigma'}(\tau')$ to a
    $y : \overline{\Gamma, x : \sigma}(\tau')$ somehow. But it can be shown that if $\sigma >
    \sigma'$, then $\overline{\Gamma , x : \sigma}(\tau) > \overline{\Gamma , x : \sigma'}(\tau)$ due
    to lemma~\ref{lem:close>}. So by the inductive hypothesis,
    we are able to get $\Gamma , x : \sigma' , y : \overline{\Gamma, x : \sigma}(\tau') \vdash e : \tau$.

    If $x = y$ then we can drop $x :
    \sigma'$ from the context and sneak it back in as $\Gamma , x : \sigma, y :
    \overline{\Gamma, x : \sigma}(\tau') \vdash e : \tau$ with lemma~\ref{lem:drop} and
    lemma~\ref{lem:sneak}. The induction hypothesis then gives us $\Gamma ,
    x : \sigma \vdash e' : \tau'$ and we construct the proof back together with
    \textsc{Let} to give $\Gamma , x : \sigma \vdash \letin{y = e'}{e} : \tau$.

    If $x \neq y$, then the proof is a bit more complicated. We take the
    following steps:
    \begin{align*}
      \Gamma , x : \sigma' , y : \overline{\Gamma, x : \sigma}(\tau') \vdash e : \tau \\
      \Gamma , y : \overline{\Gamma, x : \sigma}(\tau') , x : \sigma' \vdash e : \tau &&  \text{by
                                                            swapping,
                                                            lemma~\ref{lem:swap}}
      \\
      \Gamma , y : \overline{\Gamma, x : \sigma}(\tau') , x : \sigma \vdash e : \tau && \text{by
                                                           inductive
                                                           hypothesis} \\
      \Gamma , x : \sigma , y : \overline{\Gamma, x : \sigma}(\tau') \vdash e : \tau && \text{by swapping again}
    \end{align*}
    And proceed to construct the proof for \textsc{Let} as previously.
  \item[\rm\textsc{App}]
    From the premises we have $\Gamma , x : \sigma' \vdash e : \tau' \rightarrow \tau$ and $\Gamma , x : \sigma' \vdash
    e' : \tau'$. We wish to show $\Gamma , x : \sigma \vdash e \ e' : \tau$.
    We apply the induction hypothesis to get $\Gamma , x : \sigma \vdash e : \tau' \rightarrow \tau$
    and  $\Gamma , x : \sigma \vdash e' : \tau$. Then use \textsc{App} to build up a
    proof of $\Gamma , x : \sigma \vdash e \ e' : \tau$.
  \item[The remaining cases] The rest of the possible proofs for
    $\Gamma , x : \sigma' \vdash e : \tau$ can all be proved by applying the
    induction hypothesis on their structure, much like the case for
    \textsc{App}, and so are omitted for brevity.
  \end{description}
\end{proof}

\begin{lemma}
  If $\Gamma \vdash e : \tau'$ and $\tau' > \tau$, then $\Gamma \vdash e : \tau$.
\end{lemma}
\begin{proof}
  We prove this by induction on the proof for $\Gamma \vdash e : \tau'$.
  \begin{description}
  \item[\textmd{\boxed{\textsc{Var}}}]
    From the premise we have some $\tau''$ where $x : \tau'' \in \Gamma$ and $\tau'' >
    \tau'$. By transitivity we then have $\tau'' > \tau$. And thus we apply
    \textsc{Var} again with $x : \tau'' \in \Gamma$ and $\tau'' > \tau$ to get $\Gamma
    \vdash e : \tau$. 
  \item[\textmd{\boxed{\textsc{App}}}]
    With $\Gamma \vdash e e' : \tau'$, the premises for this rule are $\Gamma \vdash e : \tau'' \rightarrow \tau'$ and $\Gamma \vdash e' :
    \tau''$.
    The definition of $\tau' > \tau$ is that there exists a
    substitution $S$ that replaces any bound type variables, such that
    $S(\tau') = \tau$.
    Therefore if $\tau' > \tau$, then $\tau'' \rightarrow \tau' > \tau'' \rightarrow \tau$ as we can use the
    same substitution $S$. 
    Now that we have $\tau'' \rightarrow \tau' > \tau'' \rightarrow \tau$ and $\Gamma \vdash e : \tau'' \rightarrow
    \tau'$, we can use the induction hypothesis to obtain $\Gamma \vdash e : \tau'' \rightarrow
    \tau$. From here we reapply \textsc{App} to get $\Gamma \vdash e e' : \tau$ as
    needed.
  \item[\textmd{\boxed{\textsc{Abs}}}]
    Our proof tree will look like
    \begin{mathpar}
      \inferrule{\Gamma,x : \tau_1' \vdash e : \tau_2'}{\Gamma \vdash \lambda x . e : \tau_1' \rightarrow \tau_2'}
    \end{mathpar}
    and we will have $\tau' = \tau_1' \rightarrow \tau_2'$. When we think of $\tau' > \tau$, we
    have $\tau_1' \rightarrow \tau_2' > \tau$. So there is some substitution that will
    bring us to $\tau$.

    But if you look at the structure of $\tau_1' \rightarrow \tau_2'$, the
    substitution must bring us to another type of
    $\tau_1 \rightarrow \tau_2$.  The substitution will either replace something in
    $\tau_1'$, something in $\tau_2'$, or both, but since $>$ is reflexive
    with an empty substitution, both $\tau_1' > \tau_1$ and
    $\tau_2' > \tau_2$ hold. Therefore it is enough to prove $\Gamma \vdash \lambda x . e :
    \tau_1 \rightarrow \tau_2$.

    We can proceed to use lemma~\ref{lem:generalInContext} with
    $\tau_1' > \tau_1$ and $\Gamma,x : \tau_1' \vdash e : \tau_2'$, to get
    $\Gamma, x : \tau_1 \vdash e : \tau_2'$.  And by applying the induction hypothesis
    with $\tau_2' > \tau_2$, we get $\Gamma, x : \tau_1 \vdash e : \tau_2$. From here we
    proceed by reapplying \textsc{Abs} to get $\Gamma \vdash \lambda x . e : \tau_1 \rightarrow
    \tau_2$.
  \item[\textmd{\boxed{\textsc{Let}}}]
    For $\tau' > \tau$, we have
    \begin{mathpar}
      \inferrule{\Gamma \vdash e_1 : \tau'' \\ \Gamma, x : \overline{\Gamma}(\tau'') \vdash e_2 :
        \tau'}{\Gamma \vdash \letin{x = e_1}{e_2} : \tau'}
    \end{mathpar}
    We just need to apply the induction hypothesis on the second
    premise, obtaining
    $\Gamma, x : \overline{\Gamma}(\tau'') \vdash e_2 : \tau$, and thus
    $\Gamma \vdash \letin{x = e_1}{e_2} : \tau$.
  \item[\textmd{\boxed{\textsc{Product}}}]
    Our expression will have some type $\Gamma \vdash e_1 \times e_2 : \tau_1' \times \tau_2'$, where $\tau_1' \times
    \tau_2' < \tau$. Similar to the case for \textsc{Abs}, due to the
    structure of $\tau_1' \times \tau_2'$, any substitution that will change it
    to $\tau$ will result in a type of $\tau_1 \times \tau_2$, for some $\tau_1$ and
    $\tau_2$ where $\tau_1' < \tau_1$ and $\tau_2' < \tau_2$. 
    Therefore if we can prove $\Gamma \vdash e_1 \times e_2 : \tau_1 \times \tau_2$, we will
    have proved $\Gamma \vdash e_1 \times e_2 : \tau$.
    
    Taking the premises of \textsc{Product}, we get $\Gamma \vdash e_1 : \tau_1'$
    and $\Gamma \vdash e_2 : \tau_2'$. Apply the induction hypothesis on both, giving
    $\Gamma \vdash e_1 : \tau_1$ and $\Gamma \vdash e_2 : \tau_2$, and reapply \textsc{Product} to
    arrive at $\Gamma \vdash e_1 \times e_2 : \tau_1 \times \tau_2$.
  \item[\textmd{\boxed{\textsc{Lift}}}]
    If we have $\Gamma \vdash \lift{e} : \IO_\sigma \tau'$ and $\tau' > \tau$, then we need to
    show $\Gamma \vdash \lift{e} : \IO_\sigma \tau$. Take the premise $\Gamma \vdash e : \tau'$,
    and use the induction hypothesis to get $\Gamma \vdash e : \tau$, and therefore
    the desired result.
  \item[\textmd{\boxed{\textsc{Bind}}}]
    Similarly to \textsc{Lift}, we need to show $\Gamma \vdash e \bind e' :
    \IO_\sigma \tau$, given $\Gamma \vdash e \bind e' : \IO_\sigma \tau'$ and $\tau' > \tau$.
    We have $\IO_\sigma \tau' > \IO_\sigma \tau$ as per the previous cases, and the
    premise $\Gamma \vdash e' : \tau'' \rightarrow \IO_\sigma \tau'$. Because there exists a
    substitution from $\IO_\sigma \tau'$ to $\IO_\sigma \tau$, there also exists a
    substitution from $\tau'' \rightarrow \IO_\sigma \tau'$ to $\tau'' \rightarrow \IO_\sigma \tau$, thus
    $\tau'' \rightarrow \IO_\sigma \tau' > \tau'' \rightarrow \IO_\sigma \tau$.

    Applying the induction hypothesis to the premise $\Gamma \vdash e' : \tau'' \rightarrow \IO_\sigma \tau'$,
    we get $\Gamma \vdash e' : \tau'' \rightarrow \IO_\sigma \tau$, and can then use \textsc{Bind}
    again to get $\Gamma \vdash e \bind e' : \IO_\sigma \tau$ as needed.
  \item[\textmd{\boxed{\textsc{Conc}}}]
    Our conclusion for this rule is $\Gamma \vdash e_1 \curlyvee e_2 : \IO_{\sigma \cup \sigma'} \tau_1'
    \times \tau_2'$. As per always, if $\IO_{\sigma \cup \sigma'} \tau_1' \times \tau_2' = \tau' > \tau$,
    then the only substitutions possible would have to fit into
    $\IO_{\sigma \cup \sigma'} \tau_1 \times \tau_2 = \tau$. And with this, $\tau_1' > \tau_1$ as well
    as $\tau_2' > \tau_2$. We then just need to apply the induction
    hypothesis to $\Gamma \vdash e_1 : \IO_\sigma \tau_1'$ and $\Gamma \vdash e_2 : \IO_\sigma \tau_2'$
    which will allow us to use \textsc{Conc} to arrive at $\Gamma \vdash e_1 \curlyvee
    e_2 : \IO_{\sigma \cup \sigma'} \tau_1 \times \tau_2$ as needed.
  \item[\textmd{\boxed{\textsc{Unit}}}]
    We have $\Gamma \vdash \square : \square$. If $\square = \tau'$ and $\tau' > \tau$, all
    substitutions on $\square$ will result in $\square$. Thus $\tau = \square$, and we
    have already shown $\Gamma \vdash \square : \square$.
  \item[\textmd{\boxed{\textsc{ReadFile}, \textsc{ReadNet}}}]
    The same principle for \textsc{Unit} applies here to prove these
    cases.
  \item[\textmd{\boxed{\textsc{Sub}}}]
    For the conclusion $\Gamma \vdash e : \IO_{\sigma'} \tau'$, we again have that
    $\IO_{\sigma'} \tau' > \IO_{\sigma'} \tau$. Therefore there exists a substitution $S$
    from the former type to the latter, and by the definition of
    substitution on $\IO$, must be a substitution from $\tau'$ to $\tau$.

    Thus we can take the premise $\Gamma \vdash e : \IO_\sigma \tau'$, and knowing that
    $\tau' > \tau$ apply the induction hypothesis and \textsc{Sub}
    to arrive at $\Gamma \vdash e : \IO_{\sigma'} \tau$ as needed.
  \end{description}
\end{proof}

\begin{lemma}
  If $\Gamma \vdash e' : \tau'$ and $\Gamma, x : \tau' \vdash e : \tau$ then $\Gamma \vdash e [e' / x] : \tau$
  \label{lemma:substitution}
\end{lemma}
\begin{proof}
  asdf
\end{proof}

\begin{theorem}[Preservation]
  If $\Gamma \vdash e : \tau$ and $e \leadsto e'$, then $\Gamma \vdash e' : \tau$
\end{theorem}

\begin{proof}
  We prove this via structural induction on the proof for $e \leadsto e'$:
  \begin{description}
  \item[\boxed{e_1 \ e_2 \leadsto e_1' \ e_2}] We know that
    $\Gamma \vdash e_1 \ e_2 : \tau$, now we need to prove
    $\Gamma \vdash e_1' \ e_2 : \tau$. For
    $e_1 \ e_2 \leadsto e_1' \ e_2$, we only have one small-step rule for it,
    , which has the premise $e_1 \leadsto e_1'$. Also since
    $\Gamma \vdash e_1 \ e_2 : \tau$, the typing rule \textsc{App} gives us
    $\Gamma \vdash e_1 : \tau' \rightarrow \tau$ and $\Gamma \vdash e_2 : \tau'$.

    Now that we have $\Gamma \vdash e_1 : \tau' \rightarrow \tau$ and $e_1 \leadsto e_1'$, we can use
    the induction hypothesis to show $\Gamma \vdash e_1' : \tau' \rightarrow \tau$. And once
    again by applying \textsc{App}, we end up with $\Gamma \vdash e_1' \ e_2 :
    \tau$ as needed.

  \item[\boxed{e_1 \ e_2 \leadsto e_1 \ e_2'}] This can be proved with the
    same method above.

  \item[\boxed{(\lambda x . e_1) e_2 \leadsto e_1 [e_2 / x]}] Our type is
    $\Gamma \vdash (\lambda x . e_1) e_2 : \tau$, and we need to show that
    $\Gamma \vdash e_1 [e_2 / x] : \tau$.
    Unfortunately, the small-step rule for this proof does not contain any
    premises.
    However the only typing rule that can give us $\Gamma \vdash (\lambda x . e_1) e_2
    : \tau$ is \textsc{App}. The premises for it are
    are $\Gamma \vdash \lambda x . e_1 : \tau' \rightarrow \tau$ and $\Gamma \vdash e_2 : \tau'$. We can then apply
    \textsc{Abs} to get $\Gamma, x : \tau' \vdash e_1 : \tau$. The proof tree like this:

    $$\inferrule*[Right=App]{
      \inferrule*[Right=Abs]{\Gamma, x : \tau' \vdash e_1 : \tau}{\Gamma \vdash (\lambda x . e_1) : \tau' \rightarrow \tau} \\ \Gamma \vdash e_2 : \tau
    }
      {\Gamma \vdash (\lambda x . e_1) e_2 : \tau}
    $$
    
    With $\Gamma, x : \tau' \vdash e_1 : \tau$ and $e_2 : \tau'$,
    lemma~\ref{lemma:substitution} then gives us
    $\Gamma \vdash e_1 [e_2 / x] : \tau$ as needed.

  \item[\boxed{\letin{x = e_1}{e_2} \leadsto e_2[e_1/x]}]
    $$
    \inferrule*[Right=Let]{\Gamma \vdash e_1 : \tau' \\ \Gamma, x : \overline{\Gamma}(\tau') \vdash e_2 : \tau}{\Gamma \vdash \letin{x = e_1}{e_2} : \tau}
    $$
    As $\overline{\Gamma}(\tau') = \forall \alpha_1 \ldots \alpha_n \tau'$ where $\{\alpha_1,\ldots,\alpha_n\} =
    \textsf{FTV}(\tau') \backslash \textsf{FTV}(\Gamma)$, we know that $\{\alpha_1,\ldots,\alpha_n\} \cap
    \textsf{FTV}(\Gamma) = \emptyset$.
    So by lemma~\ref{lemma:substitution} we have $\Gamma \vdash e_2 [e_1/x] : \tau$.

  \item[\boxed{\lift{e_1} \bind e_2 \leadsto e_1' \bind e_2}]
    The premise for this small-step rule is $e_1 \leadsto e_1'$. We have $\Gamma
    \vdash e_1 \bind e_2 : \tau$, but because we know the type more
    specifically because of the rule \textsc{Bind}, it must be of
    the form $\Gamma \vdash e_1 \bind e_2 : \IO_\sigma \tau$. Thus we need to show $\Gamma
    \vdash e_1' \bind e_2 : \IO_\sigma \tau$.

    We have the typing premises $\Gamma \vdash e_1 : \IO_\sigma \tau'$. We can just
    apply the induction hypothesis on $e_1 \leadsto e_1'$ to get $\Gamma \vdash e_1'
    : \IO_\sigma \tau'$ and reapply \textsc{Bind} to obtain $\Gamma \vdash e_1' \bind
    e_2 : \IO_\sigma \tau$.

  \item[\boxed{\lift{v} \bind e \leadsto e \ v}]
    Using the same observation as the previous case, we can prove this
    by showing $\Gamma \vdash e \ v : \IO_\sigma \tau$.
    We can get $\Gamma \vdash v : \tau'$ via the following proof tree:
    $$\inferrule*[Right=Bind]{\inferrule*[Right=Lift]{\Gamma \vdash v : \tau'}
      {\Gamma \vdash \lift{v} : \IO_\sigma \tau'} \\ \Gamma \vdash e : \tau' \rightarrow \IO_\sigma \tau}
    {\Gamma \vdash \lift{v} \bind e : \IO_\sigma \tau}$$
    And then by applying \textsc{App}, we arrive at $\Gamma \vdash e \ v : \IO_\sigma
    \tau$.

  \item[\boxed{e_1 \curlyvee e_2 \leadsto e_1' \curlyvee e_2}]
    We need to show $\Gamma \vdash e_1' \curlyvee e_2 : \IO_{\sigma \cup \sigma'} \tau \times
    \tau'$. \textsc{Conc} provides us with the premise $\Gamma \vdash e_1 : \IO_\sigma
    \tau$, which by applying the induction hypothesis with $e_1 \leadsto e_2$
    gives $\Gamma \vdash e_1' : \IO_\sigma \tau$. Reapply \textsc{Conc} to get the
    result needed.
  \item[\boxed{e_1 \curlyvee e_2 \leadsto e_1 \curlyvee e_2'}]
    This proof is the same as the case above.
    
  \item[\boxed{\lift{v} \curlyvee \lift{w} \leadsto \lift{v \times w}}]
    The aim is to show $\Gamma \vdash \lift{v \times w} : \IO_{\sigma \cup \sigma'} \tau \times \tau'$.
    We have this proof tree:
    $$\inferrule*[Right=Conc]{
      \inferrule*[Right=Lift]{\Gamma \vdash v : \tau}{\Gamma \vdash \lift{v} : \IO_\sigma \tau} \\
      \inferrule*[Right=Lift]{\Gamma \vdash w : \tau'}{\Gamma \vdash \lift{w} : \IO_\sigma \tau'}
      \\
      \sigma \notsubtyp \sigma' \\ \sigma' \notsubtyp \sigma}
    {\Gamma \vdash \lift{v} \curlyvee \lift{w} : \IO_{\sigma \cup \sigma'} \tau \times \tau'}$$
    Use \textsc{Product} to get $\Gamma \vdash v \times w : \tau \times \tau'$, and then
    \textsc{Lift} to get $\Gamma \vdash \lift{v \times w} : \IO_{\sigma \cup \sigma'} \tau \times \tau'$
    (Note that \textsc{Lift} can lift into $\IO_\sigma$ for any $\sigma$)

  \item[\textmd{\boxed{\textsf{readFile} \leadsto \lift{\square}}}]
    We know $\Gamma \vdash
    \textsf{readFile} : \IO_{\textsf{File}} \square$, and we can just use
    \textsc{Lift} to show that $\Gamma \vdash \lift{\square} : \IO_{\textsf{File}}
    \square$.
  \item[\textmd{\boxed{\textsf{readNet} \leadsto \lift{\square}}}]
    This is the same for above, except the heap is replaced with the
    singular resource $\textsf{Net}$.
  \end{description}
\end{proof}

\begin{theorem}[Progress]
  If $. \vdash e : \tau$ then either $e \ \textsf{val}$, or there exists an $e'$
  such that $e \leadsto e'$.
\end{theorem}

\begin{proof}
  We prove this via induction on the proof for $. \vdash e : \tau$
\end{proof}


\begin{theorem}[Soundness]
  If $e$ is inferred type $\tau$, $e$ evaluates to a value in $\tau$. \\
  $\Gamma \vdash e : \tau \rightarrow \Gamma \Vdash e : \tau$
\end{theorem}


Theorem 2, completeness $\Gamma \Vdash e : \tau \rightarrow \Gamma \vdash e : \tau$

Theorem 3, principal types

Theorem 4, most specific heap for monad?
%%% Local Variables:
%%% TeX-master: "report"
%%% TeX-engine: luatex
%%% TeX-command-extra-options: "-shell-escape"
%%% End:


A type system is no good unless we can prove it is worth its salt. In
this section we will prove a number of lemmas and theorems which will
eventually show that the type system is sound. We will begin with a
couple of properties about contexts that we will later use on in
various other proofs. The first states that if we have two variables
in the context with the same names but different type schemes, then we
can ignore the first one as it is overshadowed by the second -- any
reference to $x$ will result in $x : \sigma'$.
\begin{lemma} \label{lem:drop}
  If $\Gamma , x : \sigma, x : \sigma' \vdash e : \tau$, then $\Gamma , x : \sigma' \vdash e : \tau$
\end{lemma}
The second states that we can sneak in another variable before another
variable of the same name, for the exact same reason.
\begin{lemma} \label{lem:sneak}
  If $\Gamma , x : \sigma' \vdash e : \tau$, then for any $\sigma$, $\Gamma , x : \sigma, x : \sigma' \vdash e : \tau$
\end{lemma}
We can then also say that if two variables $x$ and $y$ are indeed
different, we are free to permute them and swap them about.
\begin{lemma} \label{lem:swap}
  If $x \neq y$ and $\Gamma , x : \sigma, y : \sigma' \vdash e : \tau$, then $\Gamma , y : \sigma' , x : \sigma \vdash e
  : \tau$
\end{lemma}

We will also need to show this property about instantiation and the
close function.
\begin{lemma} \label{lem:close>}
  If $\sigma > \sigma'$ then $\overline{\Gamma , x : \sigma}(\tau) > \overline{\Gamma, x : \sigma'}(\tau)$.
\end{lemma}
\begin{proof}
  We provide an informal proof as follows.
  \begin{enumerate}
  \item If $\sigma > \sigma'$, then for every $\tau$ that $\sigma > \tau$, $\sigma' > \tau$.
  \item So $\sigma$ must parameterise over at least the same number of type
    variables if not \textit{more} than $\sigma'$.
  \item So $\sigma$ has at least the same number of free type variables if
    not \textit{less} than $\sigma'$.
  \item By definition of the close function,
    $\overline{\Gamma, x : \sigma}(\tau)$ will then have at least the same number
    of bound type variables if not \textit{more} than
    $\overline{\Gamma, x : \sigma'}(\tau)$.
  \item So if $\overline{\Gamma, x : \sigma'}(\tau)$ can be instantiated to some
    $\tau$, then $\overline{\Gamma, x : \sigma}(\tau)$ can also be instantiated to
    that $\tau$, as it has enough bound type variables to handle
    everything the former can -- if it had less type variables than the
    former, then it would not be able to instantiate these types, but
    for the converse excess type variables can be mapped to whatever.
  \end{enumerate}
\end{proof}

Now we prove the generalisation theorem, which states that if a type
scheme inside the context is an instantiation of another type scheme,
then we can use the more general type scheme and preserve how things
are typed. This is an adaptation of a lemma from Wright and
Felleisen~\cite[Lemma 4.6]{wright1994}, which is in turn an
adaptation of a lemma from Damas and Milner.

\begin{theorem}[Generalisation]
  If $\Gamma, x : \sigma' \vdash e : \tau$ and $\sigma > \sigma'$, then ${\Gamma, x : \sigma \vdash e : \tau}$.
  \label{lem:generalInContext}
\end{theorem}
\begin{proof}
  Begin with induction on the proof for $\Gamma , x : \sigma' \vdash e : \tau$.
  \begin{description}
  \item[\rm\textsc{Var}]
    From the premises we have $x : \sigma' \in \Gamma$ and $\sigma' > \tau$. We also have
    $\sigma > \sigma'$ but by definition of $\sigma > \sigma'$, if $\sigma' > \tau$ then $\sigma > \tau$.
    And we also have by definition $x : \sigma \in \Gamma , x : \sigma$. So putting the
    pieces together, we can use \textsc{Var} to get
    \begin{mathpar}
      \infer{x : \sigma \in \Gamma , x : \sigma \\ \sigma > \tau}
      {\Gamma , x : \sigma \vdash x : \tau}
    \end{mathpar}
  \item[\rm\textsc{Abs}]
    We have $\Gamma , x : \sigma' \vdash \lambda y . e : \tau$. If $x = y$, then we can safely
    say the premise $\Gamma , x : \sigma' , y : \sigma'' \vdash e : \tau$ becomes $\Gamma , y :
    \sigma'' \vdash e : \tau$  due to lemma~\ref{lem:drop}. And from this we can
    then use lemma~\ref{lem:sneak} to get $\Gamma, x : \sigma, y : \sigma'' \vdash e : \tau$,
    and ultimately $\Gamma , x : \sigma \vdash \lambda y . e : \tau$.

    If $x \neq y$, then we can get $\Gamma , y : \sigma'' , x : \sigma' \vdash e : \tau$ via
    lemma~\ref{lem:swap}. The induction hypothesis then results in $\Gamma
    , y : \sigma'' , x : \sigma \vdash e : \tau$, which we can then swap back again to
    get $\Gamma , x : \sigma , y : \sigma'' \vdash e : \tau$ and so $\Gamma , x : \sigma \vdash \lambda y . e :
    \tau$.
  \item[\rm\textsc{Let}]
    For $\letin{y = e'}{e}$, we have $\Gamma , x : \sigma' \vdash e' : \tau'$ and $\Gamma , x :
    \sigma' , y : \overline{\Gamma , x : \sigma'}(\tau') \vdash e : \tau$, and we aim to show
    $\Gamma , x : \sigma \vdash \letin{y=e'}{e} : \tau$.
    
    First off, we need to convert the $y : \overline{\Gamma , x : \sigma'}(\tau')$ to a
    $y : \overline{\Gamma, x : \sigma}(\tau')$ somehow. But it can be shown that if $\sigma >
    \sigma'$, then $\overline{\Gamma , x : \sigma}(\tau) > \overline{\Gamma , x : \sigma'}(\tau)$ due
    to lemma~\ref{lem:close>}. So by the inductive hypothesis,
    we are able to get $\Gamma , x : \sigma' , y : \overline{\Gamma, x : \sigma}(\tau') \vdash e : \tau$.

    If $x = y$ then we can drop $x :
    \sigma'$ from the context and sneak it back in as $\Gamma , x : \sigma, y :
    \overline{\Gamma, x : \sigma}(\tau') \vdash e : \tau$ with lemma~\ref{lem:drop} and
    lemma~\ref{lem:sneak}. The induction hypothesis then gives us $\Gamma ,
    x : \sigma \vdash e' : \tau'$ and we construct the proof back together with
    \textsc{Let} to give $\Gamma , x : \sigma \vdash \letin{y = e'}{e} : \tau$.

    If $x \neq y$, then the proof is a bit more complicated. We take the
    following steps:
    \begin{align*}
      \Gamma , x : \sigma' , y : \overline{\Gamma, x : \sigma}(\tau') \vdash e : \tau \\
      \Gamma , y : \overline{\Gamma, x : \sigma}(\tau') , x : \sigma' \vdash e : \tau &&  \text{by
                                                            swapping,
                                                            lemma~\ref{lem:swap}}
      \\
      \Gamma , y : \overline{\Gamma, x : \sigma}(\tau') , x : \sigma \vdash e : \tau && \text{by
                                                           inductive
                                                           hypothesis} \\
      \Gamma , x : \sigma , y : \overline{\Gamma, x : \sigma}(\tau') \vdash e : \tau && \text{by swapping again}
    \end{align*}
    And proceed to construct the proof for \textsc{Let} as previously.
  \item[\rm\textsc{App}]
    From the premises we have $\Gamma , x : \sigma' \vdash e : \tau' \rightarrow \tau$ and $\Gamma , x : \sigma' \vdash
    e' : \tau'$. We wish to show $\Gamma , x : \sigma \vdash e \ e' : \tau$.
    We apply the induction hypothesis to get $\Gamma , x : \sigma \vdash e : \tau' \rightarrow \tau$
    and  $\Gamma , x : \sigma \vdash e' : \tau$. Then use \textsc{App} to build up a
    proof of $\Gamma , x : \sigma \vdash e \ e' : \tau$.
  \item[The remaining cases] The rest of the possible proofs for
    $\Gamma , x : \sigma' \vdash e : \tau$ can all be proved by applying the
    induction hypothesis on their structure, much like the case for
    \textsc{App}, and so are omitted for brevity.
  \end{description}
\end{proof}


\begin{lemma} \label{lem:ignore}
  TODO
\end{lemma}

\begin{lemma} \label{lem:subContextTyping}
  If $\Gamma \vdash e : \tau$, then for any substitution $s$, $\Gamma[s] \vdash e : \tau[s]$.
\end{lemma}


Another important lemma that we need to show is the substitution
lemma, which relates to substituting a variable for an expression,
whose type we can prove. The following proof is adapted from Wright
and Felleisen~\cite{wright1994}.

\begin{lemma}[Substitution] \label{lem:substitution}
  If $\Gamma \vdash e : \tau$ and ${\Gamma , x : \forall \alpha_1, \ldots, \alpha_n . \tau \vdash e' : \tau'}$, and ${x \notin \dom(\Gamma)}$ and
  ${\alpha_1 , \ldots, \alpha_n \cap \ftv(\Gamma) = \emptyset}$, then ${\Gamma \vdash e' [e/x] : \tau'}$.
\end{lemma}
Note that the domain of a context $\dom(\centerdot , x_1 : \sigma_1 , \ldots , x_n :
\sigma_n)$ is defined as $\{ x_1 , \ldots , x_n \}$.
\begin{proof}
  Begin with induction on the proof of $\Gamma , x : \forall \alpha_1, \ldots, \alpha_n . \tau \vdash e'
  : \tau'$. For brevity we will sometimes refer to $\forall \alpha_1,\ldots,\alpha_n . \tau$ as $\sigma$.
  \begin{description}
  \item[\rm\textsc{Var}] We have
    $\Gamma , x : \sigma \vdash y : \tau'$ and want to show $\Gamma \vdash y[e/x] : \tau'$.

    If $y \neq x$, then by definition $y [e/x] = y$, so we just need to
    show $\Gamma \vdash y : \tau'$. From the premises we also have
    $y : \tau' \in \Gamma , x : \sigma$, but since $y \ne x$ we have $y : \tau' \in \Gamma$. And
    from here we can use \textsc{Var} to get $\Gamma \vdash y : \tau'$.

    If $y = x$, then we need to show $\Gamma \vdash x : \tau'$. From the premise of
    ${\Gamma , x : \forall \alpha_1\ldots\alpha_n . \tau \vdash x : \tau'}$ we have
    ${\forall \alpha_1\ldots\alpha_n . \tau > \tau'}$, so there exists a substitution
    $s$ which replaces exactly $\alpha_1,\ldots,\alpha_n$, such that
    $\tau[s] = \tau'$.

    We can use lemma~\ref{lem:subContextTyping} to get
    $\Gamma[s] \vdash e : \tau[s]$, or $\Gamma[s] \vdash e : \tau'$. And furthermore, because
    ${\alpha_1 , \ldots, \alpha_n \cap \ftv(\Gamma) = \emptyset}$, applying $s$ to $\Gamma$ will do
    nothing since $\Gamma$'s free type variables are distinct, i.e. $\Gamma =
    \Gamma[s]$. Leaving us with $\Gamma \vdash e : \tau'$ as required.

  \item[\rm\textsc{Abs}]
    We have $\Gamma , x : \sigma \vdash \lambda y . e' : \tau_1 \rightarrow \tau_2$ and want to show $\Gamma \vdash (\lambda y
    . e')[e/x] : \tau_1 \rightarrow \tau_2$. Similarly to the case for $\textsc{var}$, we
    first check to see if $x = y$.

    If $y = x$, then the inner binding of $y$ will shadow $x$. That
    is, we have the premise $\Gamma , x : \sigma, x : \tau_1 \vdash e' : \tau_2$, and we can
    drop the previous $x$ from the context with
    lemma~\ref{lem:drop} to get $\Gamma , x : \tau_1 \vdash e' : \tau_2$.
    By definition, if $x = y$ then $(\lambda y . e')[e/x] = \lambda y . e'$, so we
    just need to show $\Gamma \vdash \lambda y . e' : \tau_1 \rightarrow \tau_2$, which can construct
    with \textsc{Abs} and $\Gamma , x : \tau_1 \vdash e' : \tau_2$.

    If $y \ne x$, then we the proof is more involved. From the premises
    we have
    $$\Gamma , x : \alpha_1,\ldots,\alpha_n . \tau , y : \tau_1 \vdash e' : \tau_2$$.
    We start by choosing a substitution $s$, which maps $\alpha_1 , \ldots, \alpha_n$
    to fresh type variables $\alpha'_1, \ldots,\alpha'_n$. Additionally, they are
    distinct as follows.
    $$\ftv(\Gamma) \cap \alpha_1,\ldots,\alpha_n \cap \alpha'_1,\ldots,\alpha'_n = \emptyset$$

    Now we manipulate the premise in the following order.
    \begin{align*}
      \Gamma , y : \tau_1 , x : \forall\alpha_1\ldots\alpha_n . \tau &\vdash e' : \tau_2 && \text{by swapping,
                                                   lemma~\ref{lem:swap}}
      \\
      (\Gamma , y : \tau_1 , x : \forall\alpha_1\ldots\alpha_n . \tau)[s] &\vdash e' : \tau_2 && \text{by
                                                     lemma~\ref{lem:subContextTyping}}
      \\
      \Gamma , y : \tau_1[s] , x : (\forall\alpha_1\ldots\alpha_n . \tau)[s] &\vdash e' : \tau_2 &&
                                                            \text{since
                                                            } \Gamma[s] = \Gamma
    \end{align*}
    And since the range of $s$ is exactly $\alpha_1,\ldots,\alpha_n$, we have
    $(\forall \alpha_1\ldots\alpha_n . \tau)[s] = \forall \alpha_1\ldots\alpha_n . \tau$ -- substitution only
    substitutes free type variables, not bound type variables, and
    here any possible free type variables are being shadowed.
    \begin{equation}
      \Gamma , y : \tau_1[s] , x : \forall \alpha_1\ldots\alpha_n . \tau \vdash e' : \tau_2
      \label{eq:substAbs2}
    \end{equation}

    % $\Gamma(z) = (\Gamma , y : \tau_1[s])(z)$ for all $z \in \fv(e)$
    We can apply lemma~\ref{lem:ignore} to $\Gamma \vdash e : \tau$, giving us
    \begin{equation}
      \Gamma , y : \tau_1[s] \vdash e : \tau\label{eq:substAbs3}
    \end{equation}
    And when we combine $x \notin \dom(\Gamma)$ with the fact that $x \ne y$, we get
    \begin{equation}
      x \notin \dom(\Gamma , y : \tau_1[s])
      \label{eq:substAbsNotIn}
    \end{equation}
    Because of how we chose $s$, we also have
    \begin{equation}
      \ftv(\Gamma, y : \tau_1[s]) \cap {\alpha_1,\ldots,\alpha_n} = \emptyset\label{eq:substAbs4}
    \end{equation}
    Eventually, we use (\ref{eq:substAbs3}), (\ref{eq:substAbs2}),
    (\ref{eq:substAbsNotIn}) and (\ref{eq:substAbs4}) with the
    induction hypothesis to arrive at
    \[ \Gamma , y : \tau_1[s] \vdash e' [e/x] : \tau_2[s] \]
    But $s$ is a bijection due to how we chose it: That means an
    inverse, $s^{-1}$ exists. We ``apply it to both sides'' with
    lemma~\ref{lem:subContextTyping}
    \begin{align*}
      (\Gamma , y : \tau_1[s])[s^{-1}] &\vdash e' [e/x] : (\tau_2[s])[s^{-1}] \\
      (\Gamma , y : \tau_1[s])[s^{-1}] &\vdash e' [e/x] : \tau_2 \\
      \Gamma[s^{-1}] , y : \tau_1[s][s^{-1}] &\vdash e' [e/x] : \tau_2 \\
      \Gamma[s^{-1}], y : \tau_1 &\vdash e' [e/x] : \tau_2
    \end{align*}
    But because $\alpha'_1 , \ldots, \alpha'_n \cap \ftv(\Gamma) = \emptyset$, we can get rid of that
    last substitution and arrive at ${\Gamma, y : \tau_1 \vdash e' [e/x] : \tau_2}$.
    From here, we build our way back up with \textsc{Abs}.
    \begin{align*}
      \Gamma \vdash \lambda y . (e' [e/x]) : \tau_1 \rightarrow \tau_2 \\
      \Gamma \vdash (\lambda y . e')[e/x] : \tau_1 \rightarrow \tau_2 && \text{because } x \ne y
    \end{align*}
    
  \item{\rm\textsc{Let}} We have ${\Gamma , x : \sigma \vdash \letin{y = e_1}{e_2} :
      \tau'}$
    and want to show ${\Gamma \vdash (\letin{y = e_1}{e_2})[e/x] : \tau'}$.
    
    The first premise can be fed directly into the induction hypothesis
    \begin{align}
      \Gamma , x : \forall \alpha_1\ldots\alpha_n . \tau \vdash e_1 : \tau_1 \nonumber \\
      \Gamma \vdash e_1 [e / x] : \tau_1 \label{eq:substLet7}
    \end{align}

    Now if $x = y$, we take the second premise as follows
    \begin{align*}
      \Gamma , x : \forall \alpha_1\ldots\alpha_n . \tau , x : \overline{\Gamma, x : \forall\alpha_1\ldots\alpha_n . \tau}(\tau_1)
      \vdash e_2 : \tau' \\
      \Gamma , x : \overline{\Gamma, x : \forall\alpha_1\ldots\alpha_n . \tau}(\tau_1) \vdash e_2 : \tau' &&
                                                                \text{by
                                                                lemma~\ref{lem:drop}}
      \\
      \Gamma \vdash \letin{x = e_1[e / x]}{e_2} : \tau' && \text{by \textsc{Let}}
    \end{align*}
    Since $x = y$, by the definition of substitution
    $\letin{y=e_1[e/x]}{e_2}$ is equivalent to $\letin{y=e_1}{e_2})[e/x]$.
    And so we reach our goal
    \[\Gamma \vdash (\letin{y = e_1}{e_2})[e/x] : \tau'\]

  If we are unfortuante enough that $y \ne x$, then from the second
  premise we use the swap lemma to get
  \begin{equation} \label{eq:substLet8}
    \Gamma , y : \overline{\Gamma, x : \forall\alpha_1\ldots\alpha_n . \tau}(\tau_1), x : \forall \alpha_1\ldots\alpha_n . \tau  \vdash
    e_2 : \tau
  \end{equation}
  And since we have $\Gamma \vdash e : \tau$, from lemma~\ref{lem:subContextTyping}
  there is
  \begin{equation} \label{eq:substLet9}
  \Gamma , y : \overline{\Gamma , x :\forall\alpha_1\ldots\alpha_n . \tau}(\tau_1) \vdash e : \tau    
  \end{equation}
  Now we want to call the inductive hypothesis on (\ref{eq:substLet8})
  and (\ref{eq:substLet9}), but that means we first need to show
  \[\{ \alpha_1,\ldots,\alpha_n\} \cap \ftv(\Gamma , y : \overline{\Gamma, x : \forall\alpha_1\ldots\alpha_n . \tau}(\tau_1))
    = \emptyset\]
  We do this as follows:
  \begin{align*}
    & \{ \alpha_1,\ldots,\alpha_n\} \cap \ftv(\Gamma , y : \overline{\Gamma, x : \forall\alpha_1\ldots\alpha_n . \tau}(\tau_1)) \\
    &\subseteq \{ \alpha_1,\ldots,\alpha_n\} \cap (\ftv(\Gamma) \cup \ftv(\overline{\Gamma, x : \forall\alpha_1\ldots\alpha_n
      . \tau}(\tau_1))) \\
    &= \{ \alpha_1,\ldots,\alpha_n\} \cap (\ftv(\overline{\Gamma, x : \forall\alpha_1\ldots\alpha_n . \tau}(\tau_1))
    \\
    &= \{ \alpha_1,\ldots,\alpha_n\} \cap (\ftv(\tau_1) \setminus (\ftv(\tau_1) \setminus \ftv(\Gamma,x:\forall
      \alpha_1\ldots\alpha_n.\tau))) \\
    &= \{ \alpha_1,\ldots,\alpha_n\} \cap \ftv(\tau_1) \cap \ftv(\Gamma, x : \forall \alpha_1\ldots\alpha_n . \tau) \\
    &\subseteq \{ \alpha_1,\ldots,\alpha_n\} \cap \ftv(\Gamma, x : \forall \alpha_1\ldots\alpha_n . \tau) \\
    &= \{ \alpha_1,\ldots,\alpha_n\} \cap (\ftv(\Gamma) \cup \ftv(\forall \alpha_1\ldots\alpha_n . \tau)) \\
    &= \{ \alpha_1,\ldots,\alpha_n\} \cap \ftv(\forall \alpha_1\ldots\alpha_n . \tau) \\
    &= \{ \alpha_1,\ldots,\alpha_n\} \cap \ftv(\tau) \setminus \{\alpha_1,\ldots,\alpha_n\} \\
    &= \emptyset
  \end{align*}
  And then we can use it to arrive at
  \[ \Gamma , y : \overline{\Gamma, x : \forall\alpha_1\ldots\alpha_n.\tau}(\tau) \vdash e_2 [e/x] : \tau' \]
  The next part involves an observation, that for any $x$ and $\sigma$ in
  any $\Gamma$, ${\overline{\Gamma}(y) > \overline{\Gamma, x : \sigma}(y)}$, as closing
  over a context with a \textit{smaller} domain will result in the
  type scheme having \textit{more} bound type variables -- i.e. it is
  more general. Therefore we can use the generalisation lemma,
  lemma~\ref{lem:generalisation}, to get
  \begin{align*}
    \Gamma , y : \overline{\Gamma}(\tau) \vdash e_2[e/x] : \tau' \\
    \Gamma \vdash \letin{y = e_1[e/x]}{e_2[e/x]} : \tau' && \text{with \textsc{Let}
                                               and
                                               (\ref{eq:substLet7})}
    \\
    \Gamma \vdash (\letin{y = e_1}{e_2})[e/x] : \tau' && \text{by definition of substitution}
  \end{align*}
  
  \item[The remaining cases] The rest of the proofs for $\Gamma \vdash e'[e/x] :
    \tau'$ are proved by applying the induction hypothesis on their structure.
  \end{description}
\end{proof}


\chapter{Mechanisation}\label{cha:mechanisation}
In this chapter, we will look at how the proofs in
Chapter~\ref{cha:properties} were mechanically formalised --- in other
words, proved within a proof assistant.  I decided to mechanise the
proofs within Agda~\cite{norell2009}, a dependently typed programming
language with an ML syntax, similar to that of Haskell's. It can be
used for general purpose programming, but because it is rooted in
Martin-Löf intuitionistic type theory~\cite{martin-lof1984}, it can
also be used as a proof assistant.

Like many other proof assistants, the way we prove properties within
Agda is by writing programs that satisfy types. It takes advantage of
the Curry-Howard correspondence, which says that propositions are
analogous to types, and proofs are analogous to programs that fulfil
that type. We first create types that represent our
propositions. Then we can then prove our propositions by constructing
a value for the program that satisfies the type, hence intuitionistic
logic is also known as constructive logic.

As a brief example, a proposition of the form
$\forall a. a \rightarrow b$, translates into a program of type \mintinline{agda}{∀ a → b}. If you read the proposition as ``\textit{if a, then b}'', then
you can read the type as ``\textit{given any proof of a, then I can
  construct a proof of b}''. And that is indeed what the program
should do: it is a function that takes an argument of $a$, and returns
something of type $b$.

The vast majority of the work in mechanising was in writing the proofs
related to polymorphism in the Hindley-Damas-Milner system, not the
resourceful parts. The contemporary syntactic version of
Hindley-Damas-Milner (what we are using here) has been successfully
formalised in Coq~\cite{dubois2000}, and a formalisation for System F
was recently created in Agda~\cite{chapman2019}. However as it stands,
we are not aware of any formalizations of Hindley-Damas-Milner within
Agda. Regardless, the techniques and approaches here are similar, and
we base much of the proofs off of the proofs in~\cite{wright1994}.

\section{Definitions}
The framework within Agda for working with the type system is built
upon the formalisation of the simply typed lambda calculus by Kokke et
al.~\cite{kokke2020} It begins with the grammar, shown in
Listing~\ref{lst:grammar}.

\setminted{fontsize=\small,samepage}

\begin{listing}
\begin{multicols}{2}
\begin{minted}[fontsize=\small,samepage]{agda}
data Term : Set where
  `_        : Id → Term
  ƛ_⇒_      : Id → Term → Term
  _·_       : Term → Term → Term
  lt_⇐_in'_ : Id → Term → Term → Term
  ⟦_⟧       : Term → Term
  _>>=_     : Term → Term → Term
  □         : Term
  use       : Resource → Term → Term
  _×_       : Term → Term → Term
  π₁        : Term → Term
  π₂        : Term → Term
  _⋎_       : Term → Term → Term
\end{minted}
\begin{minted}[fontsize=\small,samepage]{agda}
data Heap : Set where
  World : Heap
  `_    : Resource → Heap
  _∪_   : Heap → Heap → Heap
\end{minted}
\begin{minted}[fontsize=\small]{agda}
data Type : Set where
  `_  : Id → Type
  _⇒_ : Type → Type → Type
  IO  : Heap → Type → Type
  □   : Type
  _×_ : Type → Type → Type
\end{minted}
\begin{minted}[fontsize=\small]{agda}
data TypeScheme : Set where
  V_·_ : Id → TypeScheme → TypeScheme
  `_   : Type → TypeScheme
\end{minted}
\end{multicols}
\caption{Grammar definitions. Note that some of the notation (e.g.~$\forall$) had to be substituted due to
being reserved within Agda.}\label{lst:grammar}
\end{listing}

For relations, such as the heap well-formed relation
$\textsf{ok} \ \rho$, we define a new data type. The well-formed relation
is a relation on a heap, and so the data type is parameterised over
it. Furthermore, the rules directly correspond to data constructors ---
constructing one of these means we have constructed a proof that the
heap is well typed.
\begin{minted}{agda}
data Ok : Heap → Set where
  OkZ : ∀ {r}
        --------
      → Ok (` r)
  OkS : ∀ {a b}
       → Ok a
       → Ok b
       → a ∩ b =∅
         ----------
       → Ok (a ∪ b)
  OkWorld : --------
            Ok World
\end{minted}
As another example, we also define a data type to represent proof that
a heap is a subheap of another heap. This data type is parameterised
over two heaps, which appear in the type.
\begin{minted}[samepage]{agda}
data _≥:_ : Heap → Heap → Set where
  ≥:World : ∀ {ρ} → ρ ≥: World
  ≥:Refl : ∀ {ρ} → ρ ≥: ρ
  ≥:∪ˡ : ∀ {ρ ρ' ρ''}
       → ρ ≥: ρ'
         ------------
       → ρ ≥: ρ' ∪ ρ''
  ≥:∪ʳ : ∀ {ρ ρ' ρ''}
       → ρ ≥: ρ'
         ------------
       → ρ ≥: ρ'' ∪ ρ'
\end{minted}
If we wanted to show that $\textsf{Net} \subtyp \textsf{Net} \cup
\textsf{File}$, we follow the same steps we would carry out in a proof
tree in order to construct a value that inhabits the type \mintinline{agda}{` Net ≥: ` Net ∪ ` File}.
\begin{listing}[H]
  \centering
  \begin{minipage}{0.5\linewidth}
    \begin{minted}{agda}
_ : ` Net ≥: ` Net ∪ ` File
_ = ≥:∪ʳ ≥:Refl
    \end{minted}
  \end{minipage}%
  \begin{minipage}{0.5\linewidth}
    \[
      \infer*[Left=UnionL]{
        \infer*[Left=Refl]{ }{\textsf{\textsf{Net} \subtyp \textsf{Net}}}
      }
      {\textsf{Net} \subtyp \textsf{Net} \cup \textsf{File}}
    \]
  \end{minipage}
\end{listing}
The most important relation however, is the typing relation. With
Agda's Unicode support we are able to define the rules, shown in
Listing~\ref{lst:typingrules}, with a notation similar to what we used
in Chapter~\ref{chapter:system}. Typing judgements can be can
constructed as follows:
\begin{listing}[H]
  \centering
  \begin{minipage}{0.6\linewidth}
    \begin{minted}{agda}
_ : ∅ ⊢ ƛ "x" ⇒ ` "x" ⦂ (` "α" ⇒ ` "α")
_ = ⊢ƛ (⊢` Z (Inst SZ refl refl))
    \end{minted}
  \end{minipage}%
  \begin{minipage}{0.4\linewidth}
    \[
      \infer*[Left=Abs]{
        \infer*[Left=Var]{
          x : \alpha \in \centerdot, x : \alpha \\
          \alpha > \alpha
        }{
          \centerdot , x : \alpha \vdash x : \alpha
        }
      }
      {\centerdot \vdash \lambda x . x : \alpha \rightarrow \alpha}
    \]
  \end{minipage}
\end{listing}
The small-step relation is defined in a similar way:
\begin{minted}{agda}
data _↝_ : Term → Term → Set where
  ξ-·₁ : ∀ {e₁ e₂ e₁'}
       → e₁ ↝ e₁'
         ------------------
       → e₁ · e₂ ↝ e₁' · e₂
       
  ξ-·₂ : ∀ {e₁ e₂ e₂'}
       → e₂ ↝ e₂'
         ------------------
       → e₁ · e₂ ↝ e₁ · e₂'

  β-ƛ : ∀ {x e e'}
      → Value e'
        ------------------
      → (ƛ x ⇒ e) · e' ↝ e [ x := e' ]
  -- and so on ...
\end{minted}


\begin{listing}
  \begin{multicols}{2}
\begin{minted}[breaklines,samepage]{agda}
data _⊢_⦂_ : Context → Term → Type → Set where

  ⊢` : ∀ {Γ x σ τ}
     → x ⦂ σ ∈ Γ
     → σ > τ
       ---------
     → Γ ⊢ ` x ⦂ τ

  ⊢ƛ : ∀ {Γ x τ' τ e}
     → Γ , x ⦂ ` τ' ⊢ e ⦂ τ
       ------------------
     → Γ ⊢ ƛ x ⇒ e ⦂ (τ' ⇒ τ)

  ⊢· : ∀ {Γ e e' τ τ'}
     → Γ ⊢ e ⦂ τ' ⇒ τ
     → Γ ⊢ e' ⦂ τ'
       --------------
     → Γ ⊢ e · e' ⦂ τ

  ⊢lt : ∀ {Γ e e' τ τ' x}
      → Γ ⊢ e' ⦂ τ'
      → Γ , x ⦂ close Γ τ' ⊢ e ⦂ τ
        ---------------------
      → Γ ⊢ lt x ⇐ e' in' e ⦂ τ

  ⊢× : ∀ {Γ e e' τ τ'}
     → Γ ⊢ e ⦂ τ
     → Γ ⊢ e' ⦂ τ'
       --------------------
     → Γ ⊢ e × e' ⦂ τ × τ'

  ⊢π₁ : ∀ {Γ e τ τ'}
      → Γ ⊢ e ⦂ τ × τ'
        --------------
      → Γ ⊢ π₁ e ⦂ τ

  ⊢π₂ : ∀ {Γ e τ τ'}
      → Γ ⊢ e ⦂ τ × τ'
        --------------
      → Γ ⊢ π₂ e ⦂ τ'
\end{minted}
\begin{minted}[breaklines,samepage]{agda}
  ⊢□ : ∀ {Γ}
       
       ---------
     → Γ ⊢ □ ⦂ □

  -- Monadic rules

  ⊢⟦⟧ : ∀ {Γ e τ ρ}
      → Γ ⊢ e ⦂ τ
      → Ok ρ
        ----------------
      → Γ ⊢ ⟦ e ⟧ ⦂ IO ρ τ
      
  ⊢use : ∀ {Γ e τ r}
       → Γ ⊢ e ⦂ τ
         ---------------
       → Γ ⊢ use r e ⦂ IO (` r) τ
  
  ⊢>>= : ∀ {Γ e e' τ τ' ρ}
       → Γ ⊢ e ⦂ (IO ρ τ')
       → Γ ⊢ e' ⦂ (τ' ⇒ IO ρ τ)
         -------------------
       → Γ ⊢ e >>= e' ⦂ IO ρ τ

  ⊢⋎ : ∀ {Γ e₁ e₂ τ₁ τ₂ ρ₁ ρ₂}
     → Γ ⊢ e₁ ⦂ IO ρ₁ τ₁
     → Γ ⊢ e₂ ⦂ IO ρ₂ τ₂
     → Ok (ρ₁ ∪ ρ₂)
       -----------------------
     → Γ ⊢ e₁ ⋎ e₂ ⦂ IO (ρ₁ ∪ ρ₂) (τ₁ × τ₂)

  ⊢sub : ∀ {Γ e τ ρ ρ'}
         → Γ ⊢ e ⦂ IO ρ τ
         → ρ ≥: ρ'
         → Ok ρ'
           --------------
         → Γ ⊢ e ⦂ IO ρ' τ
\end{minted}
\end{multicols}
\caption{The typing rules as they are defined in Agda.}\label{lst:typingrules}
\end{listing}
\section{Type schemes and type variables}
One of the main design decisions made early on was how to represent
type schemes within Agda. Quantified type variables in type schemes are often
represented as a sequence
$\forall \alpha_1,\ldots,\alpha_n \cdot \tau$. This expands out to
$\forall \alpha_1 \cdot \cdots \cdot \forall \alpha_n \cdot \tau$ in the end, and is how type schemes are
ultimately defined in Agda, but we need to be able to reason about it
in sequence format for some of the proofs. A helper function
\texttt{VV} was created for this reason. It allows for type schemes
to be created and manipulated in terms of lists.
\begin{minted}{agda}
VV : List Id → Type → TypeScheme
VV (α ∷ αs) τ = V α · (VV αs τ)
VV [] τ = ` τ
\end{minted}
Now we can rewrite propositions, such as the substitution lemma
(Lemma~\ref{lem:substitution}), in a way that lets us bind the list
of quantified type variables and use it elsewhere.
\begin{minted}{agda}
subst : ∀ {Γ x e e' αs τ τ'}
      → Γ ⊢ e ⦂ τ
      → Γ , x ⦂ VV αs τ ⊢ e' ⦂ τ'
      → Disjoint αs (FTVC Γ)
        ----------------------
      → Γ ⊢ e' [ x := e ] ⦂ τ'
\end{minted}
There are also functions to extract the quantified type variables and
type from a type scheme, and equivalence proofs that can be used to
convince the type checker that they are equivalent to their
\texttt{VV} form.
\begin{minted}{agda}
TStype : TypeScheme → Type
TStype (V _ · σ) = TStype σ
TStype (` τ) = τ

TSvars : TypeScheme → List Id
TSvars (V α · σ) = α ∷ TSvars σ
TSvars (` τ) = []

TStype≡ : ∀ {αs τ} → TStype (VV αs τ) ≡ τ
TStype≡ {[]} {τ} = refl
TStype≡ {α ∷ αs} {τ} = TStype≡ {αs}

TSvars≡ : ∀ {αs τ} → TSvars (VV αs τ) ≡ αs
TSvars≡ {[]} = refl
TSvars≡ {α ∷ αs} = cong (_∷_ α) TSvars≡
\end{minted}

\section{Properties}
The type preservation theorem (Theorem~\ref{thm:preservation}) is proven
with the following function --- given proof that $e$ has type $\tau$ in the
empty context, and proof that $e$ reduces to $e'$ in one step, then we
can provide proof that $e'$ also has the type $\tau$ in the empty context.
\begin{minted}{agda}
preservation : ∀ {e e' τ}
             → ∅ ⊢ e ⦂ τ
             → e ↝ e'
               ----------
             → ∅ ⊢ e' ⦂ τ
\end{minted}
For progress (Theorem~\ref{thm:progress}), we need to be able to say
either the expression is a value or it reduces to something else. We
create a new data type to represent the two cases where this can
happen.
\begin{minted}{agda}
data Progress (e : Term) : Set where
  step : ∀ {e'}
       → e ↝ e'
         ------
       → Progress e
  done : Value e
         ----------
       → Progress e

progress : ∀ {e τ}
         → ∅ ⊢ e ⦂ τ
           ----------
         → Progress e
\end{minted}

\chapter{Evaluation and Further Work}\label{cha:evaluation}

\section{Separation logic}
In Section~\ref{sec:separationologic} we looked at separation logic,
and now that we have defined our system we can look at some parallels
between them. Take the frame rule again, which allows additional
predicates to be inferred in a specification, under the condition that
the code does not use any of the variables in the predicate.
\[
  \infer{{\color{gray} \{p\}} \ \emph{code} \ {\color{gray} \{q\}}}
  {{\color{gray} \{p * r\}} \ \emph{code} \ {\color{gray} \{ q * r \}}} \
  \text{\parbox{2in}{where no variable
    occurring free in r is modified by \emph{code}}}
\]
Our well-formed heap judgement has a rule that allows heaps to be
merged, under the condition that they do not overlap:
\[
  \infer{\textsf{ok} \ \rho \\ \textsf{ok} \ \rho' \\ \rho\cap\rho'=\emptyset}
  {\textsf{ok} \ \rho\cup\rho'}
\]
This mirrors the idea of separation logic, that we only need to worry
about the variables relevant to our code, or in our terminology, the
resources relevant to our type. So our subsumption rule, \textsc{Sub}, that lets
heaps get promoted to a larger heap, ends up being our version of the
frame rule. And the concurrency rule from separation logic:
\[
  \infer{ {\color{gray} \{p_1\}} \ \emph{code}_1 \ {\color{gray} \{q_1\}}
    \\
    {\color{gray} \{p_2\}} \ \emph{code}_2 \ {\color{gray} \{q_2\}}}
  { {\color{gray} \{p_1 * p_2\}} \ \emph{code}_1 \ || \ \emph{code}_2 \
    {\color{gray} \{q_1 * q_2\}}}
\]
Ends up becoming our concurrency typing rule, \textsc{conc}. The
preconditions and postconditions are the heaps in the $\IO$ type, and the
separating conjunction $*$ is our union $\cup$.
\[
  \infer{
    \Gamma \vdash e_1 : \IO_{\rho_1} \tau_1 \\
    \Gamma \vdash e_2 : \IO_{\rho_2} \tau_2 \\
    \textsf{ok} \ \rho_1 \cup \rho_2}
  {\Gamma \vdash e_1 \curlyvee e_2 : \IO_{\rho_1 \cup \rho_2} \ \tau_1 \times \tau_2}
\]

\section{Heap well formedness}
We have proved the system defined in Chapter~\ref{chapter:system} is
sound, and moreover we can show that every concurrent expression has a
well typed heap --- that is a resource is not used more than once in the
heap, and thus not accessed concurrently.

\begin{theorem}[Concurrent heap is well formed]\label{theorem:conc}
  If $\Gamma \vdash e_1 \curlyvee e_2 : IO_\rho \tau$, then $\textsf{ok} \ \rho$
\end{theorem}
\begin{proof}
  By induction on the possible proofs for $\Gamma \vdash e_1 \curlyvee e_2 : IO_\rho \tau$.
  \begin{description}
  \item[\rm\textsc{Conc}]
    We need to show $\textsf{ok} \ \rho_1 \cup \rho_2$, and from the premises we have $\textsf{ok} \ \rho_1 \cup \rho_2$.
  \item[\rm\textsc{Sub}]
    We need to show $\textsf{ok} \ \rho'$ and from the premises, we also have $\textsf{ok} \ \rho'$. 
  \end{description}
\end{proof}
As you can see, this is straightforward to prove. Perhaps it is too
straightforward. What we would really like to prove is for all terms
$e$:
\newtheorem{conjecture}{Conjecture}
\begin{conjecture}\label{conjecture:strongconc}
  If $\centerdot \vdash e : \IO_\rho \tau$, then $\textsf{ok} \ \rho$
\end{conjecture}
This is a stronger version of Theorem~\ref{theorem:conc}. However this
is not made easy in the system as it stands, for one simple yet
annoying reason: Types are not necessarily well formed, and
\textsc{Abs} allows any type to be introduced. This can be easily
shown with the identity function.

\[
  \inferrule*[Right=Abs]{
  \inferrule*[Right=Var]{x : \IO_{\textsf{File} \cup \textsf{File}} \tau  \in
    \centerdot , x : \IO_{\textsf{File} \cup \textsf{File}} \tau \\  \IO_{\textsf{File} \cup \textsf{File}} \tau >
    \IO_{\textsf{File} \cup \textsf{File}} \tau }{\centerdot , x : \IO_{\textsf{File} \cup
      \textsf{File}} \tau \vdash x : \IO_{\textsf{File} \cup \textsf{File}} \tau}
}
{\centerdot \vdash \lambda x . x : \IO_{\textsf{File} \cup \textsf{File}} \tau \rightarrow \IO_{\textsf{File} \cup \textsf{File}} \tau}
\]

It acts as a mechanism to generate types with malformed heaps, such as
an $\IO$ monad that contains two \textsf{File}s in its heap.  In
reality this is not an issue, as there is no way to construct a value
to pass into this lambda. Because of this I believe that Conjecture~\ref{conjecture:strongconc} is still
provable, but only when the context only contains well formed type
schemes, and hence why it only applies for the empty context.

We can extend this to include a relation for well-formed contexts and
well-formed types, and use this in our definition. 

\begin{mathpar}
  \boxed{\textsf{ok} \ \Gamma} \\
  \infer{ }{\textsf{ok} \ \centerdot} \and \infer{\textsf{ok} \ \Gamma \\ \textsf{ok}
    \ \sigma}{\textsf{ok} \ \Gamma , x : \sigma } \\
  \boxed{\textsf{ok} \ \sigma} \\
  \infer{\textsf{ok} \ \tau}{\textsf{ok} \ \forall \alpha_1 , \ldots , \alpha_n \cdot \tau} \\
  \boxed{\textsf{ok} \ \tau} \\
  \infer{ }{\textsf{ok} \ \square} \and
  \infer{ }{\textsf{ok} \ \alpha} \and
  \infer{\textsf{ok} \ \tau \\ \textsf{ok} \ \tau'}{\textsf{ok} \ \tau \rightarrow \tau'}
  \and
  \infer{\textsf{ok} \ \tau \\ \textsf{ok} \ \tau'}{\textsf{ok} \ \tau \times \tau'}
  \and
  \infer{\textsf{ok} \ \tau \\ \textsf{ok} \ \rho}{\textsf{ok} \ \IO_\rho \tau}
\end{mathpar}

\begin{conjecture}
  If $\textsf{ok} \ \Gamma$ and $\Gamma \vdash e : \IO_\rho \tau$, then $\textsf{ok} \ \rho$
\end{conjecture}
So far two lemmas have been proven to aid in the proof of the above
conjecture.
\begin{lemma}
  If $\textsf{ok} \ \tau$ then for any substitution $s$, $\textsf{ok} \ \tau[s]$
\end{lemma}
\begin{lemma}
  If $\textsf{ok} \ \sigma$ and $\sigma > \tau$, then $\textsf{ok} \ \tau$
\end{lemma}

This introduces a notion of ``well-typed types'' and adds quite a bit
of extra overhead to the system, so it is left as further work.

\section{Let-polymorphism}
All this theory was motivated by a very practical cause: to provide
some simple concurrency guarantees at the type level in a programming
language. Thus the intention is that some of this may eventually make
its way into a type system of an existing or novel language.

So instead of starting with the simply typed lambda calculus, the
system was based off of the Hindley-Damas-Milner system with its
polymorphic type discipline. It cannot be understated the amount of
extra complexity and work this introduced into the proofs for
soundness, but polymorphism is an essential feature in today's
modern functional programming languages. The let polymorphism does not
interact much with the monadic and resourceful parts of the type
system, but it is desirable to show that the two are compatible with
each other regardless.

\section{Almost syntax directed}\label{sec:almost-synt-direct}
As mentioned earlier, the system was based off a syntax directed
treatment of Hindley-Damas-Milner. That meant that there were only
four rules instead of the usual six, where instantiation and
generalisation are rolled into \textsc{Var} and \textsc{Let}
respectively. The main benefit of this was that it meant there was
exactly one typing judgement for each form of syntax, so constructing
proofs for a program was a breeze --- you just follow the corresponding rules
for each term. For example, in the expression $\letin{z = \lambda x . x}{z \
  z}$, a let consisting of an abstraction and an application, the proof begins with a
\textsc{Let}, then the premises \textsc{Abs} and \textsc{App} as shown
in figure~\ref{fig:syntaxdirected}.

\begin{figure}
  \centering
  \begin{mathpar}
    \infer*[Right=Let]{
      \infer*[Right=Abs]{
        \infer*[Right=Var]{x : \tau \in \Gamma \\\\ \tau > \tau}{\Gamma , x : \tau \vdash x : \tau}
      }{\Gamma \vdash \lambda x . x : \tau \rightarrow \tau} \\
      \infer*[Right=App]{
        \infer*[Right=Var]{z : \tau \rightarrow \tau \in \Gamma \\\\ \tau \rightarrow \tau > \tau \rightarrow \tau}{\Gamma , z : \tau \rightarrow \tau \vdash z : \tau \rightarrow \tau} \\\\
        \infer*[Right=Var]{z : \tau \rightarrow \tau \in \Gamma \\\\ \tau \rightarrow \tau > \tau \rightarrow \tau}{\Gamma , z : \tau \rightarrow \tau
          \vdash z : \tau \rightarrow \tau}
      }{\Gamma , z : \tau \vdash z \ z : \tau \rightarrow \tau}
    }
    {\Gamma \vdash \letin{z = \lambda x . x}{z \ z} : \tau \rightarrow \tau}
  \end{mathpar}
  \tikz \graph[tree layout] {
    let/"$\letin{z = \lambda x . x}{z \ z}$";
    abs/"$\lambda x . x$";
    app/"$z \ z$";
    var0/"$x$"; var1/"$z$"; var2/"$z$";
    let -> {abs, app};
    abs -> var0;
    app -> {var1, var2};
  };
  \qquad
  \tikz \graph[tree layout] {
    let/"\textsc{let}";
    abs/"\textsc{abs}";
    app/"\textsc{app}";
    var0/"\textsc{var}"; var1/"\textsc{var}"; var2/"\textsc{var}";
    let -> {abs, app};
    abs -> var0;
    app -> {var1, var2};
  };
  \caption{How syntax directed typing rules mean that the proof tree
    is isomorphic to the syntax tree.}\label{fig:syntaxdirected}
\end{figure}

In our system however, the subsumption rule \textsc{sub} prevents
this. For every typing judgement of the form $\Gamma \vdash e : \IO_\rho \tau$, there
are two possible proofs that can be constructed for it: the proof
corresponding to the syntax of $e$, and \textsc{sub}. Furthermore,
because the subheap relation is reflexive, its possible to have an
infinite proof tree with repeated applications of \textsc{sub} as
shown in figure~\ref{fig:infinitesub}. When
working with proofs this did not turn out to be too big of an
issue, but it is undoubtedly a little bit unsatisfying to lose such a
nice property of the system.

\begin{figure}
  \centering
  \begin{mathpar}
    \mprset {sep=1em}
    \infer*[Left=Sub]{
      \infer*[Left=Sub]{
        \infer*[Left=Sub]{
          \vdots \\
          \textsf{File} \subtyp \textsf{File} \\ \textsf{ok} \ \textsf{File}
        }{\Gamma \vdash \use{\textsf{File}}{\square} : \IO_{\textsf{File}}\square} \\
        \textsf{File} \subtyp \textsf{File} \\ \textsf{ok} \ \textsf{File}
      }{\Gamma \vdash \use{\textsf{File}}{\square} : \IO_{\textsf{File}}\square} \\
      \textsf{File} \subtyp \textsf{File} \\ \textsf{ok} \ \textsf{File}
    }{\Gamma \vdash \use{\textsf{File}}{\square} : \IO_{\textsf{File}} \square}
  \end{mathpar}
  \tikz \graph[tree layout] {
    "$\use{\textsf{File}}{\square}$" -> "$\square$";
  };
  \qquad
  \tikz \graph[tree layout] {
    sub/"\textsc{sub}" ->[loop above] sub;
  };
  \caption{How an infinite chain of \textsc{sub} can be produced,
    leading nowhere.}\label{fig:infinitesub}

\end{figure}


\section{Modelling state within the
  monad}\label{section:modellingstate}
The modelling of the IO monad is extremely simplified, and does
nothing more than sequence computation. To illustrate this point,
the reduction rule for concurrency is completely sequential.
\[ \infer{ }{v \curlyvee w \leadsto v \bind \lambda v . (w \bind \lambda w . \lift{v \times w})} \] In
the real world, this would be pointless. We want to be able to compute
these expressions concurrently, so rewriting everything in terms of
$\bind$ won't cut it. By modelling the state of the world, like what
actually happens in the \mintinline{haskell}{IO} monad of Haskell, we
can perform more advanced reasoning about things such as concurrency.
One such possible method would be to create a new rule for reduction
of monadic computations, that passes about state. Then the concurrency
rule could be written something like this:
\begin{mathpar}
  \boxed{\langle e, s \rangle \leadsto \langle e', s' \rangle} \and
  \infer{\langle e_1, s \rangle \leadsto \langle e_1', s_1' \rangle \\ \langle e_2, s \rangle \leadsto \langle e_2', s_2'
    \rangle}
  {\langle e_1 \curlyvee e_2, s \rangle \leadsto \langle e_1' \times e_2' , s_1' \cup s_2' \rangle}
\end{mathpar}
We could then reason further about how the two states $s_1' \cup s_2'$
are merged --- something that separation logic could be used for. 

\section{Extending the inference algorithm}
One of the main contributions of the Hindley-Damas-Milner system was
the inclusion of a type inference algorithm,
Algorithm~W~\cite{milner1978}. A type system on its own only lets us
ascertain that a program is well-typed, given that we supply it a
program and a type. But the design of Hindley-Damas-milner made it
possible for the type to be inferred algorithmically, from the program
alone. More impressively, this algorithm also always inferred the most
\emph{principal type}: the most general, polymorphic type a program
could have.  Algorithm~W freed ML programmers from having to
explicitly write type annotations, and is one of its most iconic
features. Naturally, we should explore extending Algorithm~W to handle
our new resourceful types.

\section{Heap polymorphism}
It would be convenient if we were able to have expressions like the
following well-typed:
\begin{align*}
\Gamma &= \centerdot , f : \square \rightarrow \IO_{\textsf{File}} \square, g : \square \rightarrow \IO_{\textsf{Net}} \square \\
\Gamma &\vdash \letin{x = \llbracket \square \rrbracket}{(x \bind f) \curlyvee (x \bind g)} : \IO_{\textsf{File} \cup
  \textsf{Net}} \square \times \square
\end{align*}
This is not possible in the current type system: $x$ needs to be both
$\IO_{\textsf{File}}$ and $\IO_{\textsf{Net}}$ simultaneously! This is
very much the same question that Hindley, Damas and Milner set out to
solve, except instead of polymorphic \textit{types} we want to have
polymorphic \textit{heaps}.
The basic idea for this would involve extending the definition of a
heap to allow for variable lookup, $\textbf{h}$:
\begin{grammar}
  <heap $\rho$> ::= $r$ | $\sigma \cup \sigma'$ | \textsf{World} | $\mathbf{h}$
\end{grammar}
And then redefining substitutions to be pairs of a map from
type variables to types, and a map from heap variables to heaps.
\[ S : ( \alpha \mapsto \tau \times h \mapsto \rho ) \]
Then in the definition of substitution, whenever a mapping from a heap
variable to heap is encountered we would apply the substitution.
\begin{align*}
  (\IO_{\rho'} \tau)[h/\rho] &= \IO_{\rho'[h/\rho]} \tau \\
  r[h/\rho] &= r  && \textsf{Definining substitution on heaps now} \\
  (\rho_1 \cup \rho_2)[h/\rho] &= \rho_1[h/\rho] \cup \rho_2[h/\rho] \\
  \textsf{World}[h/\rho] &= \textsf{World} \\
  h'[h/\rho] &=
            \begin{cases}
              \rho & \mathsf{if} h' = h \\
              h' & \mathsf{otherwise}
            \end{cases}
\end{align*}

\section{Dependently typed resources}
So far our resources have just been simple static placeholders that
represented system resources, for example the filesystem and the
network. This is just for the sake of argument: these resources can be
as general or as specific as is needed for the programmer's use
case.

Suppose a programmer wanted to perform multiple concurrent
operations on various files. A resource representing the entire
filesystem will be too coarse-grained for this purpose. Instead they
might want to have a separate resource for each file.
\setlength{\grammarindent}{2em}
\begin{grammar} \centering
  <resource $r$> ::= \textsf{foo.txt} | \textsf{bar.txt} | \textsf{baz.txt} \ldots
\end{grammar}
But this is very static. If the programmer writes some code that
operates on a new file, then these predefined resources need to have
it appended. And what if they don't know what files they will be
touching at typecheck time? Instead, it would be much more natural if
the programmer could write programs like this
\begin{align*}
  \textsf{readFile} \ \textsf{``foo.txt''} &: \IO_{\textsf{foo.txt}} \textsf{String} \\
  \textsf{writeFile} \ \textsf{``bar.txt''} &: \textsf{String} \rightarrow
  \IO_{\textsf{bar.txt}} \square
\end{align*}
And have the file automatically \emph{lifted} from the expression and
into the heap of the type. This is a job for \textbf{dependent
  types}: having the type depend on the term. However in our system, our resources and types are not unified. In a
dependently typed lambda calculus, the $\Pi$ rule substitutes values into
types like so\footnote{$\Gamma \vdash \tau : \star$ is a judgement saying that $\tau$ is a
  type in some universe.}
\begin{mathpar}
  \infer{\Gamma \vdash (\Pi x : \tau' . \tau) : \star \\ \Gamma , x : \tau' \vdash e : \tau}
  {\Gamma \vdash \lambda x . e : (\Pi x : \tau' . \tau)} \\
  \infer{\Gamma \vdash e : (\Pi x : \tau' . \tau) \\ e' : \tau'}
  {\Gamma \vdash e \ e' : \tau[e'/\tau']}
\end{mathpar}
We would need to combine the syntaxes for resources and types together
for this rule, or fleshing out types to include type constructors
through which resources and heaps could be embedded in.
Alternatively, we could be lazy and avoid this by making it dependent
only in the heap. We could introduce a type of lambda for
$\IO$ monads that tags it with a placeholder resource, which gets
substituted when an argument is applied.
\begin{mathpar}
  \infer{\Gamma, x : \tau' \vdash e : \tau }
  {\Gamma \vdash (\lambda_\IO x . e) : \tau' \rightarrow_\IO \tau} \\
  \infer{\Gamma \vdash e : \tau' \rightarrow_\IO \tau \\ \textsf{Value} \ e' \\ \Gamma \vdash e' : \tau'}
  {\Gamma \vdash e \ e' : \IO_{e'} \tau}
\end{mathpar}
It would be wise to restrict this to values, so we aren't going about
tagging resources with half-evaluated computations. We would also need
a notion of equality for values so that we can safely tell what values
are disjoint within a heap. This will pose some interesting challenges
in practice, like knowing statically what variable expressions might
represent the same resource. Does the typechecker here know that $x=y$?
\begin{flalign*}
  &x \gets \textsf{readLine} \\
  &\textsf{readFile} \ x \\
  &\letin{y = x}{\textsf{writeFile} \ y \ \textsf{``hello''}}
\end{flalign*}

\section{Casting Heaps}
One practical question to consider is what will this look like in a
real-world programming language. We have only been passing about the
primitive unit type, but realistically we are going to want to have
functions that actually have functionality, like the aforementioned
\[\textsf{readFile} : \textsf{String} \rightarrow \IO_\rho \textsf{String}\]
A function like this is not going to be built-in, instead it will
probably be part of a standard library. So how will it be
implemented? The easiest solution would be to
wrap around an existing function that reads files in a plain $\IO$
monad. Remember that plain $\IO$ is equivalent to
$\IO_{\textsf{World}}$:
\[\textsf{readFile'} : \textsf{String} \rightarrow \IO_{\textsf{World}}
  \textsf{String} \]

How do we bring this into a $\textsf{File}$ heap? If it were a pure
computation, then we could have just used the use operator:
\[\lambda s \rightarrow \use{\textsf{File}}{\textsf{readFile'} \ s }\]

But it's a monadic computation, so we would end up with a doubly
nested monad like
${\IO_{\textsf{File}} (\IO_{\textsf{World}} \textsf{String})}$. Can
we subsume it? No! Because \textsf{World} is at the top of the
sub-heap chain. It already uses all possible resources, and we can't
claim that it uses less than what it actually does! The type is
overly cautious in this case.

In an ideal world, standard library functions would built up from a
handful of built-ins for extremely low level primitives, that keep
track of things like file descriptors. Then when they are put
together, we could subsume them to keep track of higher-level
resources. For example, after opening several file descriptors, we
could say we are using them to access a database. But practically, we
are going to have to port functions that use the full
$\IO_{\textsf{World}}$ monad to a monad that uses only the resources
that we \emph{know} the function is accessing.

Haskell's \texttt{base} library contains the
\mintinline{haskell}{unsafePerformIO} function, which has the
blasphemous type signature \mintinline{haskell}{IO a -> a}. But it
provides an escape mechanism for library authors to tell the type
checker that something really doesn't have any side effects, and is
safe to use as such. If we are to allow library authors to annotate
their heaps correctly, one such solution would be to provide an
``unsafe'' function like this, except instead of discarding the $\IO$
part, it would discard heap annotations.
\[ \textsf{unsafeCastHeap} : \IO_\rho \alpha \rightarrow \IO_{\rho'} \alpha \]

There is an
implicit element of trust here that this will be used selectively by
library authors, only when they know that a function accesses a
specific set of resources, and only those resources.

Another more radical approach to this would be to flip \textsf{World}
on its head: Introduce a notion of an ``empty'' heap which uses no
resources, and as such is distinct from all other heaps. Existing code
in a plain $\IO$ monad would have this empty heap by default, and when a
library author wants to tag resources a function might use, they would
subsume it upwards into the heap they want.

%%% Local Variables:
%%% TeX-master: "report"
%%% TeX-engine: luatex
%%% TeX-command-extra-options: "-shell-escape"
%%% End:

% LocalWords:  monadic


\chapter{Conclusion}
We have presented a pragmatic approach to reasoning about concurrent access of
resources within a type system. We build upon the Hindley-Damas-Milner type
system, incorporating the idea of modelling effects with monads, and extend it
with our resourceful constructs. We can prove it retains type soundness, and in
the process of doing so we have also created a framework for mechanising the
proof of HDM's soundness within Agda.
Furthermore, what we have built here serves an excellent basis for future
work. There are many different approaches yet to explore that could extend the
system in interesting directions.

By allowing heaps to be merged and subsumed into using more resources than
necessary, we draw parallels to the local reasoning of separation logic's frame
rule. The frame rule is what allows separation logic to scale well, and so we also
have good reason to believe that our system will scale for the same reasons.

And although it is simple, the system occupies a point in the design space that
fits well into existing functional programming languages, providing a lot of
utility for very little complexity. Concurrent programming is a notoriously
difficult task, so a type system that can help whittle down cases of concurrent
resource access means its one less thing the programmer needs to worry about.

\bibliographystyle{acm}
\bibliography{report}

\appendix
\chapter{} % needed to get section numbering?
\section{Definition of a Complete Partial Order}\label{sec:defin-compl-part}
A complete partial order (cpo) is a pair $(D, \sqsubseteq)$ consisting of a set
$D$ and a partial order $\sqsubseteq$ (a function that orders elements in
$D$, but not necessarily all of them, hence the term partial), such
that
\begin{enumerate}
\item there is a least element $\bot$
\item each directed subset $x_0 \sqsubseteq \ldots \sqsubseteq x_n \sqsubseteq \ldots$ has a least upper bound
  (\emph{lub})
\end{enumerate}

\section{Evaluation Function Notation}\label{sec:eval-funct-notat}
If $\mathbb{D} \subset \mathbb{V}$, and $d \in \mathbb{D}$, we will say $d \ \mathsf{in} \
\mathbb{V}$ to represent $d$ but treated as if its in $\mathbb{V}$. \\
We will then define the reverse

\[
  v | \mathbb{D} =
\begin{cases}
  d & \textsf{if} \ v = d \ \textsf{in} \ \mathbb{V} \ \textsf{for
    some} \ d \in \mathbb{D} \\
  \bot_{\mathbb{D}} & \textsf{otherwise}
\end{cases}
\]

\section{Transitivity of the subheap relation}\label{proof:subheaptransitive}
\begin{proof}
  We want to show $a \subtyp c$. Proceed with induction on $b \subtyp c$.
  \begin{description}
  \item[\rm\textsc{Top}]
    $c$ must be \textsf{World}, so from \textsc{Top} we have $a
    \subtyp c$.
  \item[\rm\textsc{Refl}] By definition of \textsc{Refl}, $b =
    c$, and so we get $a \subtyp c$ from $a \subtyp b$.
  \item[\rm\textsc{UnionL}] $c$ is of the form $\rho' \cup \rho''$, and
    from the premise we have $b \subtyp \rho'$. Use the induction
    hypothesis with $a \subtyp b$ and $b \subtyp \rho'$ to get $a \subtyp
    \rho'$, and then \textsc{UnionL} gives us $a \subtyp \rho' \cup \rho''$.
  \item[\rm\textsc{UnionR}]  $c$ is of the form $\rho'' \cup \rho'$, and
    from the premise we have $b \subtyp \rho'$. Use the induction
    hypothesis with $a \subtyp b$ and $b \subtyp \rho'$ to get $a \subtyp
    \rho'$, and then \textsc{UnionR} gives us $a \subtyp \rho'' \cup \rho'$.
  \end{description}
\end{proof}

\section{Helper lemmas}
\begin{lemma}\label{lem:unwrapLift}
  If $\Gamma \vdash \lift{e} : \IO_\rho \tau$, then $\Gamma \vdash e : \tau$.
\end{lemma}

\begin{proof}
  This might seem obvious, but because of subsumption we need to
  unwravel for the proof for it first. Begin with induction on $\Gamma \vdash
  \lift{e} : \IO_\rho \tau$.
  \begin{description}
  \item[\rm\textsc{Lift}] Straight from the premises, $\Gamma \vdash e : \tau$.
  \item[\rm\textsc{Sub}] We have the premise ${\Gamma \vdash \lift{e} : \IO_{\rho'}
    \tau}$. Just put this back into the induction hypothesis to get ${\Gamma \vdash e
    : \tau}$.
  \end{description}
\end{proof}

\begin{lemma}\label{lem:unwrapUse}
  If $\Gamma \vdash \use{r}{e} : \IO_\rho \tau$, then $\Gamma \vdash e : \tau$.
\end{lemma}
\begin{proof}
  Identical to that of Lemma~\ref{lem:unwrapLift}, except we handle
  the case \textsc{Use} instead of \textsc{Lift}.
\end{proof}

\end{document}

% LocalWords: stdout LocalWords Idris haskell Agda
%%% Local Variables:
%%% TeX-engine: luatex
%%% End: