% -*- fill-column: 80; TeX-command-extra-options: "-shell-escape" -*-
\documentclass{report}
\usepackage{minted}
\usepackage{syntax}
\usepackage{mathpartir}
\usepackage{amssymb}
\usepackage{amsmath}
\usepackage{amsthm}
\usepackage{tikz}
\usetikzlibrary{graphs,graphdrawing,decorations.pathreplacing,decorations.pathmorphing,arrows.meta}
\usegdlibrary{layered,trees}
\usepackage{multicol}

\usepackage{fontspec}
\setmonofont{Menlo}[Scale=0.8]

\usepackage{newunicodechar}
\newfontface\mathsymbolfont{STIXGeneral}
\newunicodechar{⟦}{{\mathsymbolfont{\llbracket}}}
\newunicodechar{⟧}{{\mathsymbolfont{\rrbracket}}}
\newunicodechar{⊢}{{\mathsymbolfont{\vdash}}}
\newunicodechar{⦂}{{\mathsymbolfont{⦂}}}
\newunicodechar{⋎}{{\mathsymbolfont{\curlyvee}}}
\newunicodechar{∷}{{\mathsymbolfont{∷}}}

\begin{document}

\newcommand{\llbracket}{[\![}
\newcommand{\rrbracket}{]\!]}
\newcommand{\IO}{\mathsf{IO}}
\newcommand{\bind}{>\!\!>\!\!=} \newcommand{\concbind}{>\!\!>\!>\!\!=}
\newcommand{\subtyp}{\geq:}
\newcommand{\notsubtyp}{\ngeq:}
\newcommand{\lift}[1]{\ensuremath{\llbracket#1\rrbracket}}
\newcommand{\use}[2]{\ensuremath{\llbracket#2\rrbracket_{#1}}}

\author{Luke Lau}
\title{A Resourceful Monad for IO}
\begin{titlepage}
  \maketitle
\end{titlepage}

\chapter{Introduction}
The immutability and pureness of certain functional languages make them seem
like a perfect fit for parallelism and concurrency. The lack of side effects
mean we are free to compute in any whatever order we please, without any shared
mutable state, free from the fear of race conditions and deadlocks. However, in
the real world, code is never completely pure. Useful programs need to interact
with the real world at some point, whether that be by reading from standard
input or connecting to a network. 

Effectfullness and order

The most popular solution today is to capture the entire state of the outside
universe and write functions that operate on state. That way any particle that
may have been perturbed by writing to \texttt{stdout} is appropriately reflected.

Concurrent Clean models this by threading the \texttt{World} in and out of
functions. Uniqueness types guarantee that the same world is only used once, so
that the programmer does create an alternative timeline by duplicating it.

\begin{minted}[breaklines]{clean}
readFile :: !String !*World -> (!MaybeError FileError String,!*World)
\end{minted}

Haskell also treats the world as a state, but without the uniquness guarantee.
Any function that interacts with the World returns a function, which returns a
new version of the World alongside the function result.\footnote{The actual
  definition in GHC is \\ \mintinline{haskell}{newtype IO a = IO (State#
    RealWorld -> (# State# RealWorld, a #))}}

\mint{haskell}|type IO a = World -> (World, a)|

How Haskell ensures that an old World isn't erroneously used instead of the
fresh new one, is by hiding the implementation from the user, tucking it away
into a \textit{monad}. This notion of using monads to sequence together stateful
computations was first introduced by Peyton Jones and
Wadler~\cite{peytonjones1993}\cite{wadler1995}. The programmer no longer needs
to keep track of the world, and they can keep their imperative code imperative,
and their pure code pure. This marriage of monads and IO is one of the crown
jewels of functional language research to come out of Haskell.
\begin{minted}{haskell}
class Monad IO where
  return x = \w -> (w, x)
  x >>= f = \w -> let (w', y) = x w in f y w'
\end{minted}

Now IO actions can be easily chained together in a type-safe way that ensures
their ordering.

This ordering however, imposes limitations. One of the main benefits of pure
functional languages, is that since expressions do not have side effects, there
is no restriction on what order they need to be evaluated in.

Take for example the following snippet.
\begin{minted}{haskell}
f, g, h :: Int -> Int
f x = g x + h x
\end{minted}

\texttt{g} could be evaluted before \texttt{h}, or \texttt{h} could
be evaluted before \texttt{g}. It wouldn't make a difference because there
are no side effects. One might be tempted then to evaluate the two expressions
concurrently -- and that would be safe.

The same cannot be said for impure IO actions -- and the type system prevents that.
\begin{minted}{haskell}
f, g, h :: Int -> IO Int
f x = g x >>= \y -> y + h x
\end{minted}
We need to explicitly bind the actions and sequence evaluation.
Does this mean that concurrency is not possible for IO actions? No, as many
languages provide primitives to run these actions concurrently in a type safe
way. Haskell has \mintinline{haskell}{forkIO}, but for simpliclty we are going
to assume a higher level function for running two IO actions simultaneously and
collecting the results.

\begin{minted}{haskell}
concurrently :: IO a -> IO b -> IO (a, b)
\end{minted}

Now we can use it to run our two IO actions side by side safely. 
\begin{minted}{haskell}
f, g, h :: Int -> IO Int
f x = g x `concurrently` h x >>= \(a, b) -> return (a + b)
\end{minted}

But what if \texttt{g} and \texttt{h} actually looked like this?

\begin{minted}{haskell}
readFile :: FilePath -> IO String
writeFile :: FilePath -> String -> IO ()

g x = do
  txt <- readFile "foo.txt"
  return (x + (read txt))
h x = do
  writeFile "foo.txt" "hello"
  return (42 - x)
\end{minted}

Running these two functions concurrently could be disastrous as the order in
which they execute could affect the outcome of the program, and all of a sudden
some innocuous looking IO functions end up introducing non-determinism and
tricky race conditions into our program.

We know statically, that a program such as \mint{haskell}|g x `concurrently` h
x| should probably not be allowed. But then why did the type system allow it?
Has it failed us? The goal of the type system is to disallow as many incorrect
programs as possible while allowing as many correct programs as possible. It is
a fine line as to what programs are defined correct and what are defined as
incorrect --- a type system too lenient and buggy programs will creep through. A
type system too strict and the programmer will end up wasting time fighting the
type checker.

In this piece of research however, we are going to investigate the point in the
design space that rejects such programs. We do \textbf{not} want to allow
programs that have glaring race conditions, where we can see that there is a
contentious access of resources.

The genesis of the rest of this work is based around the idea of modelling the
resources in use at the type level. We begin by adding another type parameter to
our IO type to represent what resource an IO action uses:

\mint{haskell}|type IO r a = World -> (World, a)|

This is a phantom type parameter, as it only exists at the type level. Now our
type signatures could look like this, annotating the API with what resources it
might use.

\begin{minted}{haskell}
data Resource = FileSystem | Net | Database | OpenGL | ...
readFile :: FilePath -> IO FileSystem ()
writeFile :: FilePath -> IO FileSystem String
readSocket :: Socket -> IO Net ()
runQuery :: Query a -> IO Database a
swapBuffers :: IO OpenGL ()
\end{minted}

Keep in mind we are painting in broad strokes when we use the word
``resource''. In the running example the resource is a file,
\texttt{foo.txt}. But the notion of a resource can be as broad or as specific as
the author of a function needs it to be. It could represent a specific database
instance, or a single network socket. For simplicity in our example we will
consider the entire file system as a single resource, the entire network as
a single resource, and so on.

Now that we know what resources each $\IO$ action is using, we would like to
change the type of our concurrent function to take advantage of this new
information. Perhaps we would like to reject any two functions that use the same
resource, i.e. it only accepts $\IO$ actions with distinct resources.

\mint{haskell}|concurrently :: r /~ s => IO r a -> IO s b -> IO ? (a, b)|

You can read \verb$r \~ s$ as ``r is distinct from s'', or the opposite of a
\verb$r \~ s$ equality constraint that one might see in a type signature. This
of course however, does not exist in Haskell.  And what does it exactly mean for
two resources to ``be distinct''? And what resources would the returned $\IO$
use?

These are questions that are answered in Chapter~\ref{chapter:system} with a
formal definition of a type system that tracks resource usage. We explore a
specific point in the design space, where the type system rejects programs like
\[
\textsf{readFile} \curlyvee \textsf{readFile}
\]
But accepts and assigns types to programs such as
\[
\textsf{readFile} \curlyvee \textsf{readNet} : \IO_{\textsf{File} \cup \textsf{Net}} \ () \times ()
\]
In short, we aim to create a type system that keeps tracks of the resources
being used, so that the programmer doesn't have to. It \textbf{does not} aim to
solve concurrency --- there will still be programs that have concurrency errors
that the type system will still allow. We just aim to narrow down the scope of
valid programs, by ruling out those with blatant, concurrent resource access
errors. 

\section{Overview}

In Chapter~\ref{chapter:background} we will talk about the inspirations of the
type system, namely separation logic, and how it parallels with our monadic
language. We will also briefly go over what we mean by a monad formally, and
then look at the original design of Hindley-Damas-Milner which we will build upon.

Chapter~\ref{chapter:system} introduces the language, its type system and its
semantics.  It gives a complete definition of all the parts needed to prove
properties.  We prove these properties in Chapter~\ref{cha:properties}, in
which we eventually build up and present a proof of its soundness. This proof,
and the others that accompany it, are then mechanised within the dependently
typed language and proof assistant Agda. The technique and design used to do so
are discussed in Chapter~\ref{cha:mechanisation}.

Finally, Chapter~\ref{cha:evaluation} evaluates our type system, how it might be
refined, and areas of further work that would be interesting to explore.

\chapter{Background}\label{chapter:background}

\section{Separation logic}\label{sec:separationologic}
Much of the original inspiration for this work came from separation
logic, and how it can be used to formalise properties of concurrent
programs. Separation logic~\cite{ohearn2019,reynolds2002} is an
extension to Hoare logic, a system for formalising properties about
imperative programs, and has been used to prove properties about
resource access in concurrent programs~\cite{ohearn2007}. It is
formalised through \emph{specifications}, which consist of some code
that alongside a \emph{precondition} a \emph{postcondition}.
\[
  {\color{gray} \{\textit{precondition}\}} \
  \textit{code} \
  {\color{gray} \{\textit{postcondition}\}}
\]
The preconditions and postconditions make assertions about what points
to what inside the \textit{heap}. For example, in the example that
O'Hearn gives, a cyclic list is constructed from two pointers.
\begin{gather*}
  {\color{gray} \{ x \mapsto 0 * y \mapsto 0 \} }\\
  [x] = y; \\
  [y] = x; \\
  {\color{gray} \{ x \mapsto y * y \mapsto x \} }
\end{gather*}
The code sets the value of $x$ is set to the location of $y$, and
vice-versa. Separation logic extends Hoare logic with the notion of
splitting the heap. In the example above, instead of asserting
$x \mapsto y \wedge y \mapsto x$, the separation conjunction $*$ is used, which says
``$x$ points to $y$ \textit{and separately} $y$ points to $x$''. The
heap must be able to split into disjoint subheaps, such that
$x \mapsto y$ holds in one and $y \mapsto x$ in the other. This means that
$x$ and $y$ cannot point to the same location in the heap.
These preconditions and postconditions can be derived from so-called
small axioms, which for an imperative language might contain an axiom
for storing to the heap:
\[ {\color{gray} \{x \mapsto -\}} [x] = v {\color{gray} \{x \mapsto v\}} \]
The \textit{frame rule} allows a specification to be extended by
adding extra predicates that do not mention any variables used by the
code.
\[
  \infer{{\color{gray} \{p\}} \ \emph{code} \ {\color{gray} \{q\}}}
  {{\color{gray} \{p * r\}} \ \emph{code} \ {\color{gray} \{ q * r \}}} \
  \text{\parbox{2in}{where no variable
    occurring free in r is modified by \emph{code}}}
\]
This rule ends up being quite significant, as its the reason why
separation logic allows for programs to be reasoned about locally ---
the predicates involving the heap are shrunk so that they only contain
variables that the code modifies.

There is also the \emph{concurrency rule} for reasoning about
concurrency, which gives the precondition and postcondition needed for
two programs that are run in parallel.
\[
  \infer{ {\color{gray} \{p_1\}} \ \emph{code}_1 \ {\color{gray} \{q_1\}}
    \\
    {\color{gray} \{p_2\}} \ \emph{code}_2 \ {\color{gray} \{q_2\}}}
  { {\color{gray} \{p_1 * p_2\}} \ \emph{code}_1 \ || \ \emph{code}_2 \
    {\color{gray} \{q_1 * q_2\}}}
\]
We will see later on in Chapter~\ref{cha:evaluation} how the system we
end up developing parallels with this. 

Krishnaswami built upon separation logic with a higher-order ML like
language\cite{krishnaswami2006} with types. It utilised monadic
binding for sequencing, allowing imperative computation within an
otherwise pure language. Unlike our system, this included separation
logic predicates directly within the language. We will not work with
predicates, and instead push all our guarantees into the type level.

\section{Monads categorically}
As the language we will develop in Chapter~\ref{chapter:system} is
based around monads, it is a good idea to take a brief look at what
exactly they are, and dip our toes into the cool waters of category
theory. There is an infamous definition\footnote{\textsc{James Iry} A Brief, Incomplete, and Mostly Wrong History of Programming Languages} of a monad that goes like:
\begin{quote}
  \textsl{A monad is just a monoid in the category of endofunctors,
    what's the problem?}
\end{quote}
Lets break this down: A \textit{category} is a collection of objects
(like a set), and morphisms (like functions between sets). In the same
way, Haskell has \textsf{Hask} --- the category of types. Its objects
are types such as \mintinline{haskell}{Bool} and
\mintinline{haskell}{Int}, and its morphisms are functions taking a
type and returning another.
\begin{minted}{haskell}
Bool, Int :: * -- Bool and Int are types
True :: Bool
42 :: Int
f : Bool -> Int -- a morphism between Bool and Int
\end{minted}
A \textit{functor} is a mapping between categories that
preserves the structure. That is, a functor maps objects and morphisms
from one category to objects and morphisms in another category.
In the same way, a \mintinline{haskell}{Functor} in Haskell maps a
type to another type, and its functions to other functions with
\mintinline{haskell}{fmap}. 
\begin{minted}{haskell}
Maybe :: * -> *
Maybe Bool :: * -- Maps Bool to Maybe Bool
fmap :: (a -> b) -> (f a -> f b) -- Maps (a -> b) to (f a -> f b)
\end{minted}
Functors can have mappings to other functors, and these are known as
\emph{natural transformations}.
An \textit{endofunctor} is a \textit{functor} from one category to the
same category. That is, it maps objects and morphisms from one
category, to objects and morphisms in the same category. But what does
that mean in Haskell? Well, since \mintinline{haskell}{Functor} maps
types to types, we are mapping from \textsf{Hask} to \textsf{Hask} ---
all \mintinline{haskell}{Functor}s in Haskell are endofunctors! They
exist as type constructors with kind \mintinline{haskell}{* -> *}.

A \textit{monad} is then defined as:
\begin{enumerate}
\item An endofunctor $T : X \rightarrow X$
\item With a natural transformation to flatten two monads together

${\mu : T(T(X)) \rightarrow T(X)}$
\item And another natural transformation $\eta : X \rightarrow T(X)$
\item That satisfy some properties (the monad laws)
\end{enumerate}

Or in Haskell,
\begin{minted}{haskell}
class Functor m => Monad m a where
  join :: m (m a) -> m a
  return :: a -> m a
\end{minted}
However as you might be aware, the actual definition in Haskell uses
bind \mintinline{haskell}{>>=}, not the mathematical definition of
\mintinline{haskell}{join}. It can be implemented in terms of
\mintinline{haskell}{>>=} instead, and vice-versa using the fact that
a monad is also a functor.
\begin{minted}{haskell}
(>>=) :: m a -> (a -> m b) -> m b
join m = m >>= \x -> x
m >>= f = join (fmap f m)
\end{minted}

The \textit{monoid in the category of endofunctors} bit then comes
about because a monoid is defined as:
\begin{enumerate}
\item A set $X$
\item With an associative binary operation $\circ : X \times X \rightarrow X$
\item And an identity element $e : X$
\item That satisfy some properties (the monoid laws)
\end{enumerate}

A monoid is the same deal as a monad, except functor composition is
replaced with the Cartesian product and the unit function is replaced
with the identity element.  And if monoids allow chaining together
elements, then monads in programming allow sequencing together
computation. In our system we will be using the more pragmatic Haskell
definition of a monad that uses bind rather than join. But the
fundamental idea of sequencing remains the same.

\section{Type-level programming within GHC}

\setminted[haskell]{breaklines}

Before creating the system in Chapter~\ref{chapter:system}, we
originally attempted to embed a resourceful system within GHC's type
system. Whilst the original Haskell 2010 language specification was
based on System F, there has been much
work~\cite{eisenberg2016}\cite{weirich2017} carried out to add
dependent types to GHC via a plethora of language extensions. Namely,
\texttt{TypeFamilies} and \texttt{GADT} provide a lot of the power
needed to carry out type-level programming, and we leveraged this
expressiveness to emulate a resource-tracking version of the
\mintinline{haskell}{IO} monad, dubbed \mintinline{haskell}{SubIO}. It
carried around a heap with a phantom type parameter:
\begin{minted}{haskell}
newtype SubIO (a :: [j]) b = SubIO (IO b)
  deriving (Functor, Applicative, Monad)
\end{minted}

We then defined a subclass of \mintinline{haskell}{Monad} that
extended it with a sequencing operator and concurrent operator, both
which merged the heap phantom type parameters together.
\begin{minted}{haskell}
class Monad (m j) => HeapMonad (m :: [x] -> * -> *) j where
  (>>>=) :: m j a -> (a -> m k b) -> m (j ** k) b
  (|||) :: m j a -> m k b -> m (j ** k) b
  (|||) x f = x >>>= const f
\end{minted}
The \texttt{**} operator was a type family that merged
together two heaps, and threw a type error if they overlapped.
\begin{minted}{haskell}
type family Overlap a b :: Bool where
  Overlap '[] b = 'False
  Overlap (a ': as) b = If (MemberP a b) 'True (Overlap as b)

type family a ** b :: [c] where
  (a ** b) = If (Overlap a b)
                (TypeError ('Text "Heaps overlap!"))
                (a :++ b)
\end{minted}
An instance for \mintinline{haskell}{SubIO} was given, and with it
some file operations from \mintinline{haskell}{System.IO}, transformed to accept
the file argument as a type argument, so it could be tracked in the heap.
\begin{minted}{haskell}
instance HeapMonad SubIO j where
  (SubIO x) >>>= f = SubIO (x >>= \z -> let (SubIO y) = f z in y)
  (SubIO x) ||| (SubIO y) = SubIO (forkIO (x >> pure ()) >> y)

readFile :: forall filePath. KnownSymbol filePath => SubIO '[filePath] String
readFile = fileOp Prelude.readFile

appendFile :: forall filePath. KnownSymbol filePath => String -> SubIO '[filePath] ()
appendFile x = fileOp (`Prelude.appendFile` x)

writeFile :: forall filePath. KnownSymbol filePath => String -> SubIO '[filePath] ()
writeFile x = fileOp (`Prelude.writeFile` x)

fileOp :: forall filePath a. KnownSymbol filePath => (FilePath -> IO a) -> SubIO '[filePath] a
fileOp f = let fp = symbolVal (Proxy :: Proxy filePath)
             in SubIO (f fp)
\end{minted}
All together, this meant that functions could be written with the
familiar do syntax, and ran concurrently with
\texttt{|||} --- but only if their heaps did not overlap.
\begin{minted}{haskell}
someIO = do
  writeFile @"foo.txt" "hello"
  s <- readFile
  return s

someMoreIO x = do
  appendFile @"bar.txt" x
  readFile >>= SubIO . putStrLn

concExample = runSubIO $ someIO ||| someMoreIO "blah"
--- this *should* type error
-- badConcExample = runSubIO $ someIO ||| someIO
\end{minted}
However we soon found out that overlapping instances don't always
throw a type error, as the phantom type parameter keeping track of the
heap was evaluated \textbf{lazily}. There was a workaround that
involved extracting the heap down to the term level and forcing
evaluation via \mintinline{haskell}{Data.Proxy} and
\mintinline{haskell}{seq}, but we think there are most likely better
ways to embed this within GHC's type system.

\section{Hindley-Damas-Milner}

One of our aims when developing this resource-tracking monadic type
system is to see how well it would integrate into existing functional
programming languages, specifically those of the ML family. So for
this reason we will basing our system off of the Hindley-Damas-Milner
(HDM) type system~\cite{damas1982}. Originally designed to formalise
the type system of ML, it is one of the first formalisations of a
polymorphic type system. It was heavily influential at the time and
still continues to be so today, paving the way for newer type systems
such as System F. 

\subsection{Syntax}

Before we can talk about a type system, we need to talk about the
language that it operates on. HDM uses an applicative language: A
language in which you can apply abstractions.
\def\defaultHypSeparation{\hskip .05in}
\newcommand{\letin}[2]{\mathsf{let} \ #1 \ \mathsf{in} \ #2}
\begin{grammar}

  <expression $e$> ::= $x$ | $\lambda x . e$ | $e \ e'$ | $\letin{x=e'}{e}$ % | $\square$

  <type $\tau$> ::= $\square$ | $\alpha$ | $\tau' \rightarrow \tau$
  
  <type scheme $\sigma$> ::= $\tau$ | $\forall \alpha . \sigma$

  <context $\Gamma$> ::= $\centerdot$ | $\Gamma, x : \sigma$

\end{grammar}
Expressions are an extension of the venerable lambda calculus, with the
addition of a new let expression that binds an expression to a
variable. As we will see later, this is notably different from
abstraction as it provides the gateway to polymorphism.

Types are either unit types, type variables or function types. However
expressions are not assigned types directly, instead they are given
type schemes which quantify over type variables. The distinction
between the two is necessary so that quantifiers can only
appear at the top level. 

A context is a linked list for looking up the type schemes of
variables. Whenever we need to work out the type of a variable
expression like $x$, we traverse the context to find its type scheme.

There is also a notion of \emph{free type variables}, which are type
variables \emph{inside a type scheme} which have not been bound
(quantified over).
\begin{align*}
  \ftv(\forall \alpha . \tau) &= \ftv(\tau) \setminus \{\alpha\} \\
  % \ftv(\square) &= \{ \} \\
  \ftv(\alpha) &= \{ \alpha \} \\
  \ftv(\tau \rightarrow \tau') &= \ftv(\tau') \cup \ftv(\tau)
\end{align*}

% A substitution $s$ maps type variables to types. A substitution of the
% form $[\tau/\alpha]$ maps $\alpha$ to $\tau$.

% It is extended to operate on types, such that every occurance of a
% type variable in a type is substituted.
% For instance, $(\alpha \rightarrow \alpha)[\tau/\alpha] = \tau \rightarrow \tau$ can be read as ``$(\alpha \rightarrow \alpha)$,
% replacing every $\alpha$ with $\tau$.''

% % \begin{minipage}{1.0\linewidth}
%   Substitution is \textbf{associative}. It is defined on types as
%   \begin{align*}
%     % \square [\tau/\alpha] &= \square \\
%     \alpha' [\tau/\alpha] &=
%                             \begin{cases}
%                               \tau & \mathsf{if} \ \alpha' = \alpha \\
%                               \alpha' & \mathsf{otherwise}
%                             \end{cases} \\
%     (\tau_1 \rightarrow \tau_2)[\tau/\alpha] &= (\tau_1[\tau/\alpha] \rightarrow \tau_2[\tau/\alpha])
%   \end{align*}
% % \end{minipage}

% And on type schemes as
% \[(\forall \alpha . \sigma)[\tau/\alpha] = \forall \tau . \sigma[\tau/\alpha]\]

% We may write $S(\sigma)$ to apply an arbitrary substitution to a type
% scheme.

% $[\tau_1/\alpha_1, \ldots, \tau_n/\alpha_n]$ may be used to notate the composition of
% substitutions $S_1 = [\tau_1/\alpha_1], \ldots, S_n = [\tau_n/\alpha_n]$, $S_1(\ldots(S_n(\sigma)))$.

% We say a type $\tau$ is an \textit{instance} of a type $\tau'$,
% written as $t > \tau'$ if there exists a substitution $S$ such that

% \[ \tau > \tau' \rightarrow \exists S. S(\tau) = \tau' \]

% Intuitively, this can be thought of as $\tau$ being more general than
% type $\tau'$. For example,
% \[ \alpha \rightarrow \beta > \alpha \rightarrow \alpha \]
% with the substitution $[\alpha/\beta]$, but there exists no substitution for
% \[ \alpha \rightarrow \alpha \ngtr \alpha \rightarrow \beta \]

\subsection{Static semantics}

A type system describes how we assign \emph{valid} types to our
expressions. Most of the time, it's through a typing relation like this
\[ \Gamma \vdash e : \tau \]
You can read this as ``\textit{In the context
  $\Gamma$, $e$ has the type $\tau$}''. This is a relation between a context,
an expression and a type, in the same way that $a \leq b$ is a relation
between two numbers. With relations we can form \textit{judgements},
such as ${1 \leq 2}$, or
${\Gamma \vdash \lambda x . (y \ x) : \alpha \rightarrow \tau}$.  These are just statements that we can
make --- they might be true, they might be false. Is $42 \leq 19$? Does
${\Gamma \vdash z : \beta}$? I don't know, you tell me. We need something else to be
able to tell whether or not they make sense: whether or not they are
\textit{well typed}. So there are \emph{typing rules} that allow us to
prove that these typing judgements are indeed well typed, and that the
expressions can have the type that they claim to have.
\begin{mathpar}
  \inferrule*[Right=Var]{x : \sigma \in \Gamma \\ \sigma > \tau}{\Gamma \vdash x : \tau} \and
  \inferrule*[Right=App]{\Gamma \vdash e : \tau' \rightarrow \tau \\ \Gamma \vdash e' : \tau'}{\Gamma \vdash e \ e' : \tau} \\
  \inferrule*[Right=Abs]{\Gamma,x:\tau' \vdash e : \tau}{\Gamma \vdash \lambda x . e : \tau' \rightarrow \tau} \and
  \inferrule*[Right=Let]{\Gamma \vdash e' : \tau' \\ \Gamma,x : \bar{\Gamma}(\tau') \vdash e : \tau}
  {\Gamma \vdash \mathsf{let} \ x = e' \ \mathsf{in} \ e : \tau}
\end{mathpar}
These rules consist of some \emph{premises} above a line, and a
\emph{conclusion} below it. Premises and conclusions are just other
judgements, that we have been able to prove.
The gist of these rules is that if you have proof of all the premises
above, then you can infer the conclusion at the bottom. So for example, you
can read \textsc{App} as \textsl{``If $e$ has the function type $\tau' \rightarrow \tau$
  in $\Gamma$ and $e'$ has the type $\tau'$ also in $\Gamma$, then $e'$ applied to
  $e$ has the type $\tau$ in $\Gamma$ too.''}.

\textsc{App} is one of the four typing rules in the syntax-directed
HDM type system, and tells us what happens to the types when we apply
an argument to a function. Let's take a look at what the others mean.
\textsc{Abs} relates to functions, sometimes called abstractions. It
says if $e$ has the type $\tau$ in the context $\Gamma$, \textit{extended}
with $x$ having the type scheme $\tau'$, then $\lambda x.e$ has the type ${\tau' \rightarrow
\tau}$ in $\Gamma$. In other words, if $e$ can have type $\tau$ provided it has
access to $x : \tau'$, then we can make a lambda out of it.

If we have a variable expression $x$, then \textsc{Var} tells us how
we can get the type for it. First, we need to make sure $x$ exists in
the context $\Gamma$. It will have some type scheme $\sigma$, but we can't
directly assign that to an expression --- the typing relation assigns
types to terms, not type schemes. Instead, we need to instantiate it
to a type with $\sigma > \tau$. This instantiation relation says that if we
have a type scheme $\forall \alpha_1\ldots\alpha_n . \tau'$, then there exists a mapping of
type variables to types
${\alpha_1\mapsto\tau_1,\ldots,\alpha_n\mapsto\tau_n}$ which we can apply to those bound type variables
in $\tau'$ to give us $\tau$.

As it turns out, these mappings from type
variables to types are really common, and they are called
\emph{substitutions}.  The whole rule then means, if we can look up a
type scheme $\sigma$ for $x$, and then instantiate to some type
$\tau$, then we can infer that $x$ has the type $\tau$ in $\Gamma$.

The \textsc{Let} rule introduces polymorphism to the language. That
is, it allows one expression to be used for multiple types. Consider
the identity function
\[ \lambda x . x \]
What type should it have? It could be used both as a
$\mathsf{Int} \rightarrow \mathsf{Int}$. But the reality is that we don't care
about the underlying type, and want it to work on \emph{anything}.
\[ \Gamma \vdash \lambda x . x : \alpha \rightarrow \alpha \] Now with type variables instead, we have
nicely abstracted over the concrete types. Assuming we have a context
like $\Gamma = \centerdot , a : \mathsf{Int}$, we might try to use it multiple
times, each with different types, by passing it in as an argument to
abstraction.
\[ \Gamma \vdash (\lambda y . (y \ y) \ (y \ a)) \ (\lambda x . x) : \mathsf{Int} \]
But this will not work, and it is in fact ill-typed. Our identity
function will get passed around as ${y : \alpha \rightarrow \alpha}$ through \textsc{Abs}. When
either $(y \ y)$ or $(y \ a)$ try to look up $y$ in \textsc{Var}, they
will find that they can't instantiate it to the type they want,
because $\alpha$ is not quantified over!

Instead, we need to pass around the identity function as ${y : \forall \alpha . \alpha
\rightarrow \alpha}$. With a let expression, we can instead say
\[ \Gamma \vdash \letin{y = \lambda x . x}{(y \ y) \ (y \ a)} : \mathsf{Int} \]
If we look at the first premise for \textsc{Let}, we can show that
${\Gamma \vdash \lambda x . x : \alpha \rightarrow \alpha}$. In the second premise however, we put it into
the context as $y : \overline{\Gamma}(\alpha \rightarrow \alpha)$. $\overline{\Gamma}(\alpha \rightarrow \alpha)$ is the
close function, which is defined as
\[ \overline{\Gamma}(\tau) = \forall \alpha_1 \ldots \alpha_n . \tau \
\textsf{where} \ \{ \alpha_1, \ldots, \alpha_n \} = \ftv(\tau) \setminus \ftv(\Gamma) \]
This will take the dangling free type variables in our type and create a
new type scheme that binds them by quantifying over them.
\[ \overline{\Gamma}(\alpha \rightarrow \alpha) = \forall \alpha . \alpha \rightarrow \alpha \]
And now that we have it quantified, \textsc{Var} is able to instantiate $y$
to \emph{both} an
${(\alpha \rightarrow \alpha) \rightarrow (\alpha \rightarrow \alpha)}$ and an ${\mathsf{Int} \rightarrow \mathsf{Int}}$.

\subsubsection{Syntax directed}
Note that we have been using the syntax-directed rules: a more modern
treatment of the original system. In the syntax-directed rules, the
typing relation assigns types to terms, not type
schemes~\cite[p.15]{tofte1988} as in~\cite{damas1982}. The original
system had six rules:
\begin{mathpar}
  \inferrule*[Right=Taut]{x : \sigma \in \Gamma}{\Gamma \vdash x : \sigma} \and
  \inferrule*[Right=Inst]{\Gamma \vdash e : \sigma \\ \sigma > \sigma'}{\Gamma \vdash e : \sigma'} \and
  \inferrule*[Right=Gen]{\Gamma \vdash e : \sigma \\ \alpha \notin \ftv(\Gamma)}{\Gamma \vdash e : \forall \alpha . \sigma}  \\
  \inferrule*[Right=Comb]{\Gamma \vdash e : \tau'\rightarrow\tau \\ \Gamma \vdash e' : \tau'}{\Gamma \vdash e \ e' : \tau}
  \and
  \inferrule*[Right=Abs]{\Gamma, x : \tau' \vdash e : \tau}{\Gamma \vdash \lambda x.e : \tau'\rightarrow\tau} \\
  \inferrule*[Right=Let]{\Gamma \vdash e : \sigma \\ \Gamma,x :\sigma \vdash e' : \tau}{\Gamma \vdash
    \letin{x=e}{e'} : \tau}
\end{mathpar}
As we can see, the syntax-directed rules have merged \textsc{Taut} and
\textsc{Inst} into \textsc{Var}, and \textsc{Gen} into \textsc{Let}
via $\overline{\Gamma}$. It is possible to prove that these two systems are
equivalent, but the benefit of having just four rules is that we now
have exactly one rule for each syntactic form of expression. This
means that \emph{the shape of the proof is identical to the shape of
  the syntax}. This is discussed in more detail in Section~\ref{sec:almost-synt-direct}.

\subsection{Dynamic Semantics}

Milner and Damas created a denotational semantics for the language.

The semantics is defined by a semantic algebra, which itself is
comprised of a \textit{semantic domain} and \textit{semantic
  equation}.

The semantic domain defines the possible values an expression in our
language can have. It is a \textbf{complete partial order} (often referred to
as a \textit{cpo}): A pair $(D, \sqsubseteq)$ of a set $D$ and a partial order
$\sqsubseteq$ (a function that orders elements in $D$, but not necessarily all
of them, hence the term \textit{partial}), such that:

\begin{enumerate}
\item there is a least element $\bot$
\item each directed subset $x_0 \sqsubseteq \ldots \sqsubseteq x_n \sqsubseteq \ldots$ has a least upper bound
  (lub)
\end{enumerate}

\begin{align*}
  \mathbb{V} &= \mathbb{B}_{\square} + \mathbb{B}_{bool} + \mathbb{F} + \mathbb{W} \\
  \mathbb{F} &= \mathbb{V} \rightarrow \mathbb{V} \\
  \mathbb{W} &= \{ . \}
\end{align*}

Or visually,

\begin{center}
  \tikz \graph[layered layout] { "$\mathbb{V}$" ->
    { "$\mathbb{B}_{\square}$" -> "$\square$",
      "$\mathbb{B}_{\textrm{Bool}}$" -> {true, false},
      I} ->
    "$\bot$"; };
\end{center}


Function space $D \rightarrow E$

Coalesced sum $D + E$

Since this cpo $\mathbb{V}$ represents all possibly data values, we
can extract a subset of it to model the values of certain
types %~\ref{shamirwadge77}.
A subset $I$ of our cpo $\mathbb{V}$ is called an ideal, iff it
satisfies the following properties:

\begin{enumerate}
\item it is downwards closed: $\forall v_0 \in V, v_1 \in V, v_0, v_0 \sqsubseteq v_1 \rightarrow
  v_0 \in I \rightarrow v_1 \in I$.
  
\item it is closed under lubs of \omega-chains.
\end{enumerate}

Our function domain $\mathbb{F}$ is a map from $\mathbb{V}$ to
$\mathbb{V}$. Maps over ideals are defined as
$I \rightarrow I' \equiv \{ v \in V | v \in \mathbb{F} \ \mathsf{and} \ \forall v' \in I \
(v_{|\mathbb{F}})v' \in I' \}$.

With all the mechanisms in place, we can now define what it means for
a value to semantically be a type:

\[v \in \mathbb{V}^\tau \iff \vDash v : \tau\]

Note that $v : \tau$ is a relation, not a function -- a value can be a
member of multiple types, and this should be read as ``$v$ is a $\tau$''.

\subsubsection{Bottom}
$\bot$ ends up being very useful to represent values that don't exist.
Take for example the following program which doesn't terminate. 

\[\letin{x = \lambda y. y}{x x}\]

What type does should this program have? It should assume the
type of whatever is needed. For instance, we would expect this program
to have a type of $\square$ as the argument is not used.

\[ (\lambda z. \square) (\letin{x = \lambda y. y}{x x}) : \square \]

Since $\bot$ is a member of all ideals, this program is well typed.

\subsubsection{Some notation}

If $\mathbb{D} \subset \mathbb{V}$, and $d \in \mathbb{D}$, we will say $d \ \mathsf{in} \
\mathbb{V}$ to represent $d$ but treated as if its in $\mathbb{V}$. \\
We will then define the reverse

$$v | \mathbb{D} =
\begin{cases}
  d & \textsf{if} \ v = d \ \textsf{in} \ \mathbb{V} \ \textsf{for
    some} \ d \in \mathbb{D} \\
  \bot_{\mathbb{D}} & \textsf{otherwise}
\end{cases}
$$


\subsubsection{Evaluation function}
The semantic equation $\mathcal{E} : \mathsf{Expression} \rightarrow
\mathsf{Environment} \rightarrow \mathbb{V}$ lies at the heart of the semantics,
and defines how the syntax is evaluated.

\begin{align*}
  \mathcal{E} \llbracket x \rrbracket \eta
  &= \eta \llbracket x \rrbracket \\
  \mathcal{E} \llbracket e_1 e_2 \rrbracket \eta
  &=
    \begin{cases}
      \bot & \mathsf{if} \ v_1 = \bot \\
      (v_1 | \mathbb{F}) v_2 & \mathsf{if} \ v_1 \in \mathbb{F} \\
      \mathsf{wrong} & \mathsf{otherwise}
    \end{cases}
  \\
  & \quad \textsf{where} \ v_i = \mathcal{E} \llbracket e_i \rrbracket \eta , \ i = \{
    1, 2\} \\
  \mathcal{E} \llbracket \lambda x . \ e \rrbracket \eta
  &=
    (\lambda v . \ \mathcal{E} \llbracket e \rrbracket \eta [v / x ])
    \ \mathsf{in} \ \mathbb{V} \\
  \mathcal{E} \llbracket \textsf{let} \ x = e_1 \ \textsf{in} \ e_2 \rrbracket \eta
  &=
    \mathcal{E} \llbracket e_2 \rrbracket \ \eta [ \mathcal{E} \llbracket e_1 \rrbracket\rho / x ]
\end{align*}

Note that these evaluation rules are not as strict as the semantics
defined by Milner~\cite{milner1978} -- namely, that the second argument
of application and let binding is not checked if it is a \textsf{wrong}.

We introduce the notion of an environment $\eta : \mathsf{Variable} \rightarrow
\mathbb{V}$. It is a map of variables bound to values.

An environment $\eta$ can be said to \textit{respect} a type environment
$\Gamma$ if all bindings in $\Gamma$ can be found in $\eta$ with the same type.
$$\eta : \Gamma \iff \forall x : \tau \in \Gamma. \ \eta \llbracket x \rrbracket : \tau$$

$$\Gamma \vDash e : \tau \iff
\forall \eta. \ \eta : \Gamma \rightarrow \mathcal{E} \llbracket e \rrbracket \eta : \tau $$

An assertion of the form above is said to be \textit{closed} if there
are no free type variables in $\Gamma$ or $\tau$, and an assertion only holds
iff its closed instances hold.

Let $\overline{\mathbb{V}}$ be the set of all ideals in $\mathbb{V}$
that do not contain $\mathsf{wrong}$.

There is also a type evaluation function $\mathcal{T} : \mathsf{Type}
\rightarrow \mathsf{Valuation} \rightarrow \overline{\mathbb{V}}$

\begin{align*}
  \mathcal{T}\llbracket \square \rrbracket\psi &= \mathbb{B}_{ \square } \\
  \mathcal{T}\llbracket \mathsf{Bool} \rrbracket \psi &= \mathbb{B}_{\mathsf{Bool}} \\
  \mathcal{T}\llbracket \alpha \rrbracket \psi &= \psi \llbracket \alpha \rrbracket \\
  \mathcal{T} \llbracket \tau \rightarrow \tau' \rrbracket \psi &= \mathcal{T}\llbracket \tau \rrbracket \psi \ \rightarrow \
                             \mathcal{T} \llbracket \tau' \rrbracket \psi
\end{align*}

\subsection{Correctness}

Both Milner~\cite{milner1978} and Damas~\cite{damas1982} proved semantic
soundness for the system.

TODO: Is this too trivial to be a lemma? The proof is pretty vacuous
\newtheorem{lemma}{Lemma}
\begin{lemma}[to be named]\label{lem:1}
  If $v : \tau$ and $\eta : \Gamma$, then $\eta[v/x] : \Gamma,x : \tau$
\end{lemma}
\begin{proof}
  If $\eta : \Gamma$, then $\forall x : \tau \in \Gamma \ \eta\llbracket x \rrbracket : \tau$.
  And if $v : \tau$ then $\eta[v/x] \llbracket x \rrbracket : \tau$.
  So $\eta[v/x] : \Gamma,x : \tau$, because for all $y : \tau' \in \Gamma,x : \tau$, either $y =
  x$ and the substitution evaluates to the right type, or $y \neq x$ but
  because $\eta : \Gamma$, we have $\eta \llbracket y \rrbracket : \tau'$.
\end{proof}

\newtheorem{theorem}{Theorem}
\begin{theorem}[Semantic Soundness]
  $\Gamma \vdash e : \tau \rightarrow \Gamma \vDash e : \tau$ \\
  If $e$ has type $\tau$, $e$ does actually evaluate to a value in $\tau$.
\end{theorem}
\begin{proof}
  We need to prove $\forall \eta. \eta : \Gamma \rightarrow \mathcal{E} \llbracket e \rrbracket \eta : \tau$. We can do
  this via induction on $e$:

  \begin{description}
  \item[\boxed{x}] This is the base case of the induction. Since
    $\Gamma \vdash x : \tau$ , the rule \textsc{Var} gives us
    $x : \tau \in \Gamma$. The evaluation function produces
    $\mathcal{E} \llbracket x \rrbracket \eta = \eta \llbracket x \rrbracket$, but because
    $\eta : \Gamma$, $x : \tau \in \eta$, so $\eta \llbracket x \rrbracket : \tau$.
  \item[\boxed{e_1 e_2}] The type given is
    $\Gamma \vdash e_1 e_2 : \tau$, and via \textsc{App} we have
    $\Gamma \vdash e_1 : \tau' \rightarrow \tau$ and
    $\Gamma \vdash e_2 : \tau'$.  By applying the induction hypothesis, we get
    $\Gamma \vDash e_1 : \tau' \rightarrow \tau$ and $\Gamma \vDash e_2 : \tau'$, so
    $v_1 \in \tau' \rightarrow \tau$ and therefore $v_1 \in \mathbb{F}$. \\
    This brings us to
    $\mathcal{E} \llbracket e_1 e_2 \rrbracket \eta = (v_1 | \mathbb{F}) v_2$. We can tell
    what the application of the $v_2$ will give us by looking at the
    definition for the ideal of $v_1$:
    ${\tau' \rightarrow \tau = \{ v \in \mathbb{V} | v \in \mathbb{F} \wedge \forall v' \in \tau' (v |
      \mathbb{F}) v' \in \tau \}}$.  So
    $(v_1 | \mathbb{F}) v_2 \in \tau$, as is required to show
    $\Gamma \vDash e_1 e_2 : \tau$.
  \item[\boxed{\lambda x . e}] Our type is
    $\Gamma \vdash \lambda x . e : \tau' \rightarrow \tau$ and our typing rule \textsc{Abs} tells us
    the antecedent is $\Gamma,x:\tau' \vdash e :
    \tau$. Evaluating our expression gives
    $\mathcal{E} \llbracket \lambda x . e \rrbracket \eta = (\lambda v. \mathcal{E} \llbracket e \rrbracket \eta[v / x]) \
    in \ \mathbb{V}$. \\
    If we can prove
    $\mathcal{E} \llbracket e \rrbracket \eta [v/x] : \tau$ where
    $v : \tau'$, then we can prove that
    ${\lambda v . \mathcal{E} \llbracket e \rrbracket \eta [v/x] : \tau' \rightarrow \tau}$. But note that for
    $\Gamma, x : \tau' \vdash e : \tau$, lemma~\ref{lem:1} tells us $\eta[v/x] : \Gamma,x : \tau'$.
    And with this fact, $\Gamma,x : \tau' \vDash e : \tau$ from the
    induction hypothesis gives us $\mathcal{E} \llbracket e \rrbracket \eta [v/x] : \tau$
    where $v : \tau'$, as needed.
  \item[\boxed{\letin{x = e_1}{e_2}}] $\Gamma \vdash \letin{x = e_1}{e_2} : \tau$, and
    \textsc{Let} tells us $\Gamma \vdash e_1 : \tau'$ and $\Gamma,x : \tau' \vdash e_2 : \tau$. The
    evaluation function for let expressions is
    ${\mathcal{E} \llbracket \letin{x = e_1}{e_2} \rrbracket \eta
    = \mathcal{E} \llbracket e_2 \rrbracket (\eta [\mathcal{E} \llbracket e_1 \rrbracket \eta / x ])}$. We know that
  $\mathcal{E} \llbracket e_1 \rrbracket \eta : \tau'$ because of the induction hypothesis on
  $\Gamma \vdash e_1 : \tau'$, and will refer to this value as $v_1 : \tau'$. \\
  From lemma~\ref{lem:1}, in $\Gamma,x : \tau' \vdash e_2 : \tau$ the substituted environment
  respects the type environment $\eta[v_1/x] : \Gamma,x : \tau'$.
  Therefore $\Gamma,x : \tau' \vDash e_2 : \tau$ tells us that
  $\mathcal{E} \llbracket e_2 \rrbracket (\eta [v_1/x]) : \tau$. This is
  identical to our hypothesis, and our proof is done.
  \end{description}
  
\end{proof}

%%% Local Variables:
%%% TeX-master: "report"
%%% TeX-engine: luatex
%%% TeX-command-extra-options: "-shell-escape"
%%% End:

% LocalWords:  instantiation


\chapter{The Resourceful System}\label{chapter:system}

In this chapter we will formalise our resource-tracking, monadic IO
language, based on the Hindley-Damas-Milner type system that we saw
in the previous chapter. Just like before we will define the syntax,
static semantics and dynamic semantics, but to avoid ambiguity we will
also redefine concepts from HDM in more detail.

\section{Syntax}

\setlength{\grammarparsep}{20pt plus 1pt minus 1pt} % increase separation between rules
\setlength{\grammarindent}{12em} % increase separation between LHS/RHS
\renewcommand{\syntleft}{}
\renewcommand{\syntright}{}

%\newcommand{\square}{\ensuremath{\raisebox{-0.35mm}{\square}}}

\def\defaultHypSeparation{\hskip .05in}

We begin by giving the grammar as shown in
figure~\ref{fig:grammar}. The expression language is an extension of
the lambda calculus with let polymorphism. As before we use $x$,
$\lambda x . e$ and $e \ e'$ to refer to variable lookup, abstraction and
application respectively, and $\letin{x = e}{e'}$ for polymorphically
binding $e$ to $x$ inside $e'$.

Our language also contains product types, $\tau \times \tau'$, which are the same
as tuples \texttt{(a, b)} in Haskell. New product types can be
introduced with $e \times e'$, and can similarly be eliminated with the
projection expressions $\pi_1 \ e$ and $\pi_2 \ e$. $\square$ is the unit type
(\mintinline{haskell}{()} in Haskell), and has the eponymous
constructor $\square$.

More interestingly we have the monadic additions. Unlike
Krishnaswami~\cite{krishnaswami2006} we do not separate the language
into expression and computation languages. $\lift{e}$, lifts a regular
expression into the $\IO$ monad (\mintinline{haskell}{return} in
Haskell), and $e \bind e'$ is the standard monadic sequencing or
binding operation.

Resources $r$ represent something that we want to keep track of and
prevent from being accessed concurrently, for example a file system or
database. The resources we chose for the grammar are such examples.

Heaps $\rho$ are then formed from resources. They differ from the notion
of heaps in separation logic, and are constructed from either single
resources or by merging other heaps.  They are used to tag the
resources that a computation in an $\IO$ monad might be
accessing. There is also the notion of a \textsf{World} heap which
encapsulates all possible resources and sub-heaps. It can be thought
of as the ordinary \mintinline{haskell}{IO} monad in Haskell, where
the outside world is one indivisible resource causing everything to be
executed sequentially.

$\use{r}{e}$, pronounced \textit{e using r}, lifts a value into the
$\IO$ monad similarly to $\lift{e}$, but creates a new heap with the
resource $r$ and tags the monad with it. Novel to our language is the
$e \curlyvee e'$ operator. It joins two monadic computations together into one
that uses both their resources, returning a new heap made by merging
$e$ and $e'$s heaps --- provided that they do not overlap and are well
formed. Heap well formedness will be defined later on.

\begin{figure}
\begin{grammar}

  <type $\tau$> ::= $\square$ | $\alpha$ | $\tau \rightarrow \tau'$ | $\tau \times \tau'$ | $\textsf{IO}_\rho \tau$
  
  <type scheme $\sigma$> ::= $\forall \alpha . \sigma$ | $\tau$

  <context $\Gamma$> ::= $\centerdot$ | $\Gamma , x : \sigma$

  <expression $e$> ::= $\square$ | $e \times e'$ | $\pi_1 \ e$ $\pi_2 \ e$
  \alt $x$ | $\lambda x . e$ | $e \ e'$ | $\letin{x = e}{e'}$
  % \alt $\textsf{if} \ e_1 \ \textsf{then} \ e_2 \ \textsf{else} \ e_3$
  \alt $\lift{e}$ | $\use{r}{e}$ | $e \bind e'$ | $e \curlyvee e'$

  <resource $r$> ::= \textsf{File} | \textsf{Network} |
  \textsf{Database} | \ldots

  <heap $\rho$> ::= $r$ | $\rho \cup \rho'$ | \textsf{World}

\end{grammar}
\caption{The grammar for our resourceful language.}\label{fig:grammar}
\end{figure}

\emph{Free type variables} are defined on types, type schemes and contexts as
follows:
\begin{align*}
  \ftv(\square) &= \emptyset \\
  \ftv(\alpha) &= \{ \alpha \} \\
  \ftv(\tau \rightarrow \tau') &= \ftv(\tau) \cup \ftv(\tau') \\
  \ftv(\IO_\rho \tau) &= \IO_\rho \ftv(\tau) \\
  \ftv(\forall \alpha_1 , \ldots , \alpha_n . \tau) &= \ftv(\tau) - \{ \alpha_1 , \ldots, \alpha_n \} \\
  \ftv(\Gamma) &= \bigcup_{x : \sigma \in \Gamma} \ftv(\sigma)
\end{align*}
\begin{samepage}
\emph{Free variables} are defined on terms --- be careful not to mix these up
with free type variables!
\[
\begin{aligned}
  \fv(x) &= \{ x \} & \fv(\lambda x . e) &= \fv(e) \setminus \{ x \} \\
  \fv(e \ e') &= \fv(e) \cup \fv(e') & \fv(\letin{x = e}{e'}) &= \fv(e) \cup
                                    (\fv(e') \setminus \{ x \}) \\
  \fv(\lift{e}) &= \fv(e) & \fv(e \bind e') &= \fv(e) \cup \fv(e') \\
  \fv(\square) &= \emptyset & \fv(\use{r}{e}) &= \fv(e) \\
  \fv(e_1 \times e_2) &= \fv(e_1) \cup \fv(e_2) & \fv(\pi_i e) &= \fv{(e)}_{i = 1,
                                          2} \\
  \fv(e_1 \curlyvee e_2) &= \fv(e_1) \cup \fv(e_2)
\end{aligned}
\]
\end{samepage}

\subsection{Substitution}
Substitution may seem self-explanatory, but it is important we define
it crystal clear. It plays a vital role in proving properties about
our system, and there exists subtle differences with how it is defined
on terms, types and type schemes. A substitution is a map from type
variables to types, written as $\tau'[\tau/\alpha]$ to replace the type the type
$\tau$ for the type variable $\alpha$ in the type $\tau'$.
\begin{align*}
  \square [\tau/\alpha] &= \square \\
  \alpha' [\tau/\alpha] &=
             \begin{cases}
               \tau & \mathsf{if} \ \alpha' = \alpha \\
               \alpha' & \mathsf{otherwise}
             \end{cases} \\
  (e \times e') [\tau/\alpha] &= e[\tau/\alpha] \times e'[\tau/\alpha] \\
  (\tau_1 \rightarrow \tau_2)[\tau/\alpha] &= (\tau_1[\tau/\alpha] \rightarrow \tau_2[\tau/\alpha]) \\
  (\IO_\rho \tau')[\tau/\alpha] &= \IO_\rho \tau'[\tau/\alpha]
\end{align*}
Substitution is associative, and a substitution that substitutes
multiple type variables at once can be written like
$[\tau_1/\alpha_1,\ldots,\tau_m/\alpha_m]$. It is also extended to type schemes, but note
that this only substitutes \emph{free} type variables.
\begin{align*}
  \forall \alpha' . \sigma[\tau/\alpha] &=
            \begin{cases}
              \forall \alpha' . \sigma & \mathsf{if} \ \alpha' = \alpha \\
              \forall \alpha' . \sigma[\tau/\alpha] & \mathsf{otherwise}
            \end{cases} \\
  \tau[\tau/\alpha] &= \tau[\tau/\alpha] \ \textsf{(Substituion on type)}
\end{align*}

\newcommand{\dom}{\operatorname{dom}}

It should not be confused with instantiation, where the \emph{bound}
type variables $\forall \alpha_1 , \ldots , \alpha_n$ are substituted inside the type of a
type scheme. Instantiation is not a function though, it is
a relation $\sigma > \tau$ in which we say $\sigma$ can be instantiated to $\tau$.
\begin{mathpar}
  \boxed{\sigma > \tau} \\
  \infer{\dom(s) = \{\alpha_1 , \ldots, \alpha_n\} \\ \tau'[s] = \tau}{\forall \alpha_1 , \ldots , \alpha_n
    . \tau' > \tau}
\end{mathpar}
The domain of a substitution, $\operatorname{dom}(s)$, is the set of
type variables that it will replace, for example:
\[ \dom([\tau_1/\alpha,\tau_2/\beta]) = \{\alpha, \beta\} \]
Instantiation can then be read as, if
there exists a substitution $s$ that substitutes exactly all the bound
type variables in the type scheme
$\forall \alpha_1, \ldots, \alpha_n . \tau'$ to give the type $\tau$, then
$\forall \alpha_1, \ldots, \alpha_n . \tau' > \tau$.

Furthermore we define the relation $\sigma > \sigma'$ on type schemes as well,
and say that $\sigma$ \emph{is more general than} $\sigma'$ if for all
$\tau$, $\sigma' > \tau \rightarrow \sigma > \tau$.

Substitution (not instantiation) is also defined on
contexts. Substitution on contexts only substitutes \emph{free} type
variables.
\begin{align*}
  \centerdot[\tau/\alpha] = \centerdot \\
  (\Gamma , x : \sigma) [\tau/\alpha] &= \Gamma[\tau/\alpha] , x : \sigma[\tau/\alpha]
\end{align*}

There also exist term-level substitutions which map variables to
other terms, and can be applied as follows:
\begin{align*}
  \square [v/\alpha] &= \square \\
  \alpha' [v/\alpha] &=
             \begin{cases}
               v & \mathsf{if} \ \alpha' = \alpha \\
               \alpha' & \mathsf{otherwise}
             \end{cases} \\
  (e \ e') [v/\alpha] &= e[v/\alpha] \ e'[v/\alpha] \\
  \lambda x . e [v/\alpha] &=
                  \begin{cases}
                    \lambda x . e & \mathsf{if} \ x = \alpha \\
                    \lambda x . (e [v/\alpha]) & \mathsf{otherwise}
                  \end{cases} \\
  \letin{x = e'}{e} [v/\alpha] &=
                               \begin{cases}
                                 \letin{x = (e' [v/\alpha])}{e} & \mathsf{if} \
                                 x = \alpha \\
                                 \letin{x = (e' [v/\alpha])}{(e [v/\alpha])}
                                 & \mathsf{otherwise}
                               \end{cases} \\  
  (e_1 \times e_2) [v/\alpha] &= e_1[v/\alpha] \times e_2[v/\alpha] \\
  (\pi_i \ e) [v/\alpha] &= {\pi_i (e [v/\alpha])}_{i = 1, 2} \\
  \lift{e} [v/\alpha] &= \lift{e[v/\alpha]} \\
  \use{r}{e} [v/\alpha] &= \use{r}{e[v/\alpha]} \\
  (e \bind e') [v/\alpha] &= e[v/\alpha] \bind e'[v/\alpha] \\
  (e_1 \curlyvee e_2) [v/\alpha] &= e_1[v/\alpha] \curlyvee e_2 [v/\alpha]
\end{align*}

\subsection{Barendregt's variable convention}

Sometimes, we will end up in a scenario with two separate expressions
such as $\lambda x . x$ and $(y \ x)$. We know that the $x$ inside the first
expression is distinct from the $x$ in the second expression, but when
working with proofs we will need to show this somehow. We will use the
Barendregt variable convention~\cite{barendregt1984} to deal with
this: If we want to show that the $x$ inside ${\lambda x . x}$ is
different from the $x$ in $(y \ x)$, we can use \emph{alpha
  conversion} to rename $x$ to $z$ and get ${\lambda z . z}$. This new
expression is in fact equivalent to ${\lambda x . x}$, and we can always
choose a new unique name to avoid collisions with any \emph{free
  variables}. The Barendregt variable convention says that whenever we
have a bound variable (an $x$ inside a $\lambda x . e$ or
$\letin{x=e'}{e}$), we can just assume that we have performed alpha
conversion on it so that the variable name is unique.

\section{Static semantics}
Our static semantics begin with the syntax-directed
rules of the Hindley-Damas-Milner system for the typing relation $\Gamma \vdash e : \tau$.

\begin{mathpar}
  \inferrule*[Right=Var]{x : \sigma \in \Gamma \\ \sigma > \tau}{\Gamma \vdash x : \tau} \and
  \inferrule*[Right=App]{\Gamma \vdash e : \tau' \rightarrow \tau \\ \Gamma \vdash e' : \tau'}{\Gamma \vdash e \ e' : \tau} \\
  \inferrule*[Right=Abs]{\Gamma,x:\tau' \vdash e : \tau}{\Gamma \vdash \lambda x . e : \tau' \rightarrow \tau} \and
  \inferrule*[Right=Let]{\Gamma \vdash e' : \tau' \\ \Gamma,x : \overline{\Gamma}(\tau') \vdash e : \tau}
  {\Gamma \vdash \mathsf{let} \ x = e' \ \mathsf{in} \ e : \tau}
\end{mathpar}

Like before, $\overline{\Gamma}$ is defined as
\[ \overline{\Gamma}(\tau) = \forall \alpha_1 \ldots \alpha_n . \tau \
\textsf{where} \ \{ \alpha_1, \ldots, \alpha_n \} = \ftv(\tau) \setminus \ftv(\Gamma) \]
  
The typing rule for $\square$ expressions introduces the monomorphic type,
and the rule for products introduces $e \times e'$.
\begin{mathpar}
  \inferrule*[Right=Unit]{ }{\Gamma \vdash \square : \square} \and
  \inferrule*[Right=Product]{\Gamma \vdash e : \tau \\ \Gamma \vdash e' : \tau'}
    {\Gamma \vdash e \times e' : \tau \times \tau'}
\end{mathpar}
Product types also have two eliminators, which project out the inner
type.
\begin{mathpar}
  \infer*[Right=Proj1]{\Gamma \vdash e : \tau \times \tau'}{\Gamma \vdash \pi_1 e : \tau} \and
  \infer*[Right=Proj2]{\Gamma \vdash e : \tau \times \tau'}{\Gamma \vdash \pi_2 e : \tau'}
\end{mathpar}

Now we introduce the monadic parts of the language. Our language only
has one type of monad, the $\IO$ monad, which is parameterised by both
its \emph{heap} $\rho$ and its encapsulated type. Monadic values can be
introduced into the language with $\lift{e}$, which lifts a pure term into
\textbf{any} \emph{well formed} heap. 
\begin{mathpar}
  \infer*[Right=Lift]{\Gamma \vdash e : \tau \\ \textsf{ok} \ \rho}{\Gamma \vdash \lift{e} : \IO_\rho \tau}
\end{mathpar}
We need to be careful what heaps we allow terms to be lifted into, as
the entire point of this system is to avoid heaps containing duplicate
resources. It would be all too easy to introduce a nonsensical type
like $\IO_{\textsf{File} \cup \textsf{File}} \tau$ with \textsc{Lift}, if it
were not for the premise $\textsf{ok} \ \rho$.

$\textsf{ok} \ \rho$ is a new relation we define in
Figure~\ref{fig:heapwellformedness} to establish what heaps we
consider to be well formed, in a similar vein to
Krishnaswami~\cite{krishnaswami2006}. All heaps consisting of a single
resource are well formed, as well as heaps made by merging two other
well formed heaps \textit{that are distinct}, i.e.\ they do not share
any of the same resources.
\begin{figure}
  \centering
  \begin{mathpar}
    \boxed{\textsf{ok} \ \rho} \and
    \infer{ }{\textsf{ok} \ \textsf{World}} \and
    \infer{ }{\textsf{ok} \ r} \and
    \infer{
      \textsf{ok} \ \rho \\
      \textsf{ok} \ \rho' \\
      \rho \cap \rho' = \emptyset}
    {\textsf{ok} \ \rho \cup \rho'}    
  \end{mathpar}
  \caption{Rules for heap well formedness.}\label{fig:heapwellformedness}
\end{figure}

Whilst \textsc{Lift} lets us construct monads in any heap, we will
eventually want to have constructors for $\IO$ monads that use
specific resources. When first designing the system, we used
placeholder functions that just used fixed resources:
\begin{mathpar}
  \infer{ }{\Gamma \vdash \mathsf{readFile} : \IO_{\mathsf{File}} \square} \and
  \infer{ }{\Gamma \vdash \mathsf{readNetwork} : \IO_{\mathsf{Network}} \square}
\end{mathpar}
$\textsf{readFile}$ and $\textsf{readNet}$ are examples of typical
operations that can consume a specific resource --- their heap consists
of just a single resource. This was then generalised to
$\use{r}{e}$, which lifts any pure term into an $\IO$ monad with a heap
consisting of the resource $r$.
\begin{mathpar}
  \infer*[Right=Use]{\Gamma \vdash e : \tau}{\Gamma \vdash \use{r}{e} : \IO_r e}
\end{mathpar}
Note that it only uses a single resource, not any arbitrary heap. A
heap with just one resource is always well formed, so there is no need
for an $\textsf{ok} \ r$ premise.  \textsc{Use} is meant to be used to
annotate terms with that a computation needs for itself, whilst on the
other hand \textsc{Lift} is for bringing pure expressions into a
monadic computation.

Once we have an $\IO$ value, we can sequence computation by binding
it with a function that returns another $\IO$.
\begin{mathpar}
  \inferrule*[Right=Bind]{\Gamma \vdash e : \IO_\rho \tau' \\ \Gamma \vdash e' : \tau' \rightarrow \IO_\rho
    \tau}{\Gamma \vdash e \bind e' : \IO_\rho \tau}
\end{mathpar}
Note that the types of the two $\IO$s must use the same
resources. However, we want expressions such as
\[ \use{\textsf{File}}{\square} \bind \lambda x . \use{\textsf{Net}}{\square} \]
to also be well typed. In particular, we want the above expression to be of
type $\IO_{\textsf{File} \cup \textsf{Net}} \square$. But if \textsc{Bind}
requires the heaps to be the same, we must first somehow ``cast''
$\textsf{readFile}$ and $\textsf{readNet}$ to the same type.
This is the purpose of the \textsc{Sub} (subsumption) rule:
\begin{mathpar}
  \inferrule*[Right=Sub]{\Gamma \vdash e : \IO_{\rho'} \tau \\ \rho' \subtyp \rho \\
    \textsf{ok} \ \rho}
  {\Gamma \vdash e : \IO_\rho \tau}
\end{mathpar}
It lets something of type $\IO_{\rho'} \tau$ become a $\IO_\rho \tau$, provided
that the heap $\rho$ is a \textit{subheap} of $\rho'$. The subheap
rules for heaps, shown in figure~\ref{fig:subheap}, define the
$\subtyp$ relation.

\begin{figure}

\begin{mathpar}
\boxed{\rho' \subtyp \rho} \\
  
\inferrule*[Right=Top]{ }{\textsf{World} \subtyp \rho} \and
\inferrule*[Right=Refl]{ }{\rho \subtyp \rho} \and
\inferrule*[Right=UnionL]{\rho' \subtyp \rho}{\rho' \cup \rho'' \subtyp \rho} \and
\hskip 1em
\inferrule*[Right=UnionR]{\rho' \subtyp \rho}{\rho'' \cup \rho' \subtyp \rho}
\end{mathpar}

\caption{Subheap rules}\label{fig:subheap}
\end{figure}

Intuitively, a heap $\sigma$ can thought of being a subheap of another heap
$\sigma'$, if $\sigma'$ subsumes $\sigma$, similarly to how subtyping works. For
example, $\textsf{Net} \subtyp \textsf{Net} \cup \textsf{File}$, since
$\textsf{Net} \cup \textsf{File}$ ``overlaps'' with the heap
$\textsf{Net}$.

\begin{figure}
  \centering
  \tikz \graph[layered layout] {
    world/"\textsf{World}"; file/"\textsf{File}"; net/"\textsf{Net}";
    database/"\textsf{Database}"; databasenet/"$\textsf{Database} \cup \textsf{Net}$";
    filenet/"$\textsf{File} \cup \textsf{Net}$";
    world -> filenet;
    world -> databasenet;
    databasenet -> {database, net};
    filenet -> {file, net};
  };
  \caption{An example of some heaps and their subheap orderings.}
\end{figure}

If one views this relation as an ordering, then we have
$\textsf{World}$ defined as the least upper bound --- this represents
using all possible resources, and as mentioned earlier
$\IO_{\textsf{World}}$ can be thought of as the $IO$ monad in Haskell,
where sequencing interacts with the state of the entire world.  In a
sense this could be viewed as a form of subtyping, but by constraining
it to just resources and not actual types, we avoid the extra overhead
and complexity subtyping would normally give us in the presence of
type inference~\cite{dolan2017}.

The subheaping relation is both reflexive (by definition), and
transitive.
\begin{theorem}
  For all $a$, $b$, $c$, if $c \subtyp b$ and $b \subtyp a$ then $c
  \subtyp a$.
\end{theorem}
\begin{proof}
  By induction on $c \subtyp b$. See
  proof~\ref{proof:subheaptransitive} in the appendix.
\end{proof}

The purpose of this type system is to allow programs to be run
concurrently, but reject the ones that concurrently access the same
resource. In this regard, the \textsc{Conc} rule is the heart and soul
of the system. It takes two monadic expressions, merges their heaps
and returns a product of their two inner types, inside $\IO$.
\begin{mathpar}
  \infer*[Right=Conc]{
    \Gamma \vdash e_1 : \IO_{\rho_1} \tau_1 \\
    \Gamma \vdash e_2 : \IO_{\rho_2} \tau_2 \\
    \textsf{ok} \ \rho_1 \cup \rho_2}
  {\Gamma \vdash e_1 \curlyvee e_2 : \IO_{\rho_1 \cup \rho_2} \ \tau_1 \times \tau_2}
\end{mathpar}
The premises ensure that the new merged heap must be well formed, as we do not
want to allow programs that try to use the same resource concurrently,
such as
\[ \use{\textsf{File}}{\square} \curlyvee \use{\textsf{File}}{\square} \]
We do however, want to allow programs that run two expressions that do
not share any resources:
\[ \use{\textsf{File}}{\square} \curlyvee \use{\textsf{Net}}{\square} : \IO_{\textsf{File} \cup \textsf{Net}} \]

All together these rules define the typing relation, and are
displayed in full in Figure~\ref{fig:typingrules}. To give a better
idea of how they are used in practice, we will look at some examples
of various programs and their types.

\begin{figure}
  \begin{mathpar}
    \boxed{\Gamma \vdash e : \tau} \\
    
    \inferrule*[Right=Var]{x : \sigma \in \Gamma \\ \sigma > \tau}{\Gamma \vdash x : \tau} \and
    \inferrule*[Right=App]{\Gamma \vdash e : \tau' \rightarrow \tau \\ \Gamma \vdash e' : \tau'}{\Gamma \vdash e \ e' : \tau} \and
    \inferrule*[Right=Abs]{\Gamma,x : \tau \vdash e : \tau'} {\Gamma \vdash \lambda x . e : \tau \rightarrow
      \tau'} \and
    \inferrule*[Right=Let]{\Gamma \vdash e' : \tau' \\ \Gamma,x : \overline{\Gamma}(\tau') \vdash e : \tau}
    {\Gamma \vdash \mathsf{let} \ x = e' \ \mathsf{in} \ e : \tau} \and
    % \inferrule*[Right=If]{\Gamma \vdash e_1 : \mathbf{Bool} \\ \Gamma \vdash e_2 : \tau \\ \Gamma \vdash e_3 : \tau}
    % {\Gamma \vdash \mathsf{if} \ e_1 \ \mathsf{then} \ e_2 \ \mathsf{else} \
    % e_3 : \tau} \and

    \inferrule*[Right=Unit]{ }{\Gamma \vdash \square : \square} \\
    \inferrule*[Right=Product]{\Gamma \vdash e : \tau \\ \Gamma \vdash e' : \tau'}
    {\Gamma \vdash e \times e' : \tau \times \tau'} \and

    \inferrule*[Right=Proj1]{\Gamma \vdash e : \tau \times \tau'}{\Gamma \vdash \pi_1 : \tau} \and     
    \inferrule*[Right=Proj2]{\Gamma \vdash e : \tau \times \tau'}{\Gamma \vdash \pi_2 : \tau'} \\

    \infer*[Right=Lift]{\Gamma \vdash e : \tau \\ \textsf{ok} \ \rho}{\Gamma \vdash \lift{e} :
      \IO_\rho \tau} \and
    \infer*[Right=Use]{\Gamma \vdash e : \tau}{\Gamma \vdash \use{r}{e} : \IO_r \tau} \\
    \inferrule*[Right=Bind]{\Gamma \vdash e : \IO_\rho \tau' \\ \Gamma \vdash e' : \tau' \rightarrow \IO_\rho
      \tau}{\Gamma \vdash e \bind e' : \IO_\rho \tau} \\
    \infer*[Right=Conc]{
      \Gamma \vdash e_1 : \IO_{\rho_1} \tau_1 \\
      \Gamma \vdash e_2 : \IO_{\rho_2} \tau_2 \\
      \textsf{ok} \ \rho_1 \cup \rho_2}
    {\Gamma \vdash e_1 \curlyvee e_2 : \IO_{\rho_1 \cup \rho_2} \ \tau_1 \times \tau_2} \and

    \inferrule*[Right=Sub]{\Gamma \vdash e : \IO_{\rho'} \tau \\ \rho' \subtyp \rho \\
      \textsf{ok} \ \rho}
    {\Gamma \vdash e : \IO_\rho \tau}

  \end{mathpar}

  \caption{The typing rules for our resourceful language.}\label{fig:typingrules}
\end{figure}

\subsection{Examples}
\subsubsection{Monadic binding}
Here we show a proof tree for the expression
\[\letin{x = \lambda y . \lift{\square}}{(x \ \square) \bind \lambda z . \llbracket z
    \rrbracket_{\textsf{File}}}\]
The lift operator in $\lambda y . \lift{\square}$ can lift into any
heap. But the smallest heap we can use for this overall program is
\textsf{File}, which is needed by $\lambda z . \use{\textsf{File}}{z}$. So
when we choose a heap to lift $\square$ into, we lift it into \textsf{File}.
\begin{mathpar}
  \mprset {sep=1em}
  \infer{
    \infer{
      \infer{}{\centerdot, y : \alpha \vdash \square : \square}
    } {
      \centerdot , y : \alpha \vdash \llbracket \square \rrbracket : \IO_{\textsf{File}} \square
    } \\
    \textsf{ok} \ \textsf{File}
  }
  { \centerdot \vdash \lambda y . \lift{\square} : \alpha \rightarrow \IO_{\textsf{File}} \square \\ \mathbf{(1)} }
  \\
  \infer{
    \infer{
      \infer{}{\centerdot,x :\alpha \rightarrow \IO_{\textsf{File}} \square \vdash x : \alpha \rightarrow
        \IO_{\textsf{File}} \square}
      \\\\
      \infer{}{\centerdot, x : \alpha \rightarrow \IO_{\textsf{File}} \square \vdash \square : \square}
    }{\centerdot, x : \alpha \rightarrow \IO_{\textsf{File}} \square  \vdash x \ \square :
      \IO_{\textsf{File}} \square}
    \\
    \infer{
      \infer{
        \infer{z : \alpha \in \centerdot, x : \alpha \rightarrow \IO_{\textsf{File}} \square, z : \alpha \\ \alpha > \square}
        {\centerdot, x : \alpha \rightarrow \IO_{\textsf{File}} \square, z : \alpha \vdash z : \square}
      }
      {\centerdot, x : \alpha \rightarrow \IO_{\textsf{File}} \square , z : \alpha \vdash \llbracket z
        \rrbracket_{\textsf{File}} : \IO_{\textsf{File}} \square}
    }{
      \centerdot, x : \alpha \rightarrow \IO_{\textsf{File}} \square  \vdash \lambda z . \llbracket z
      \rrbracket_{\textsf{File}} : \IO_{\textsf{File}} \square
    }
  }{
    \centerdot, x : \alpha \rightarrow \IO_{\textsf{File}} \square \vdash (x \ \square) \bind \lambda z . \llbracket z
    \rrbracket_{\textsf{File}} : \IO_{\textsf{File}} \square \\ \mathbf{(2)}
  }
  
  \infer{ \mathbf{(1)} \\ \mathbf{(2)} } {\centerdot \vdash \letin{x = \lambda y
      . \lift{\square}}{(x \ \square) \bind \lambda z . \llbracket z \rrbracket_{\textsf{File}}} :
    \IO_{\textsf{File}} \square}
\end{mathpar}

\subsubsection{Let polymorphism}
This example doesn't contain any resourceful elements, but is just an
example of how let polymorphism allows the same variable to be given
different types at each call site. Note how the premises in
$\mathbf{(1)}$ and $\mathbf{(2)}$ assign $x$ different types.
\begin{mathpar}
  % app
  \infer{
    . , x : \alpha \rightarrow \alpha \vdash x : (\square \rightarrow \square) \rightarrow (\square \rightarrow \square) \\
    . , x : \alpha \rightarrow \alpha \vdash x : \square \rightarrow \square
  }
  {\centerdot, x : \alpha \rightarrow \alpha \vdash x \ x : \square \rightarrow \square \\ \mathbf{(1)}}
  \\
  \infer{
    . , x : \alpha \rightarrow \alpha \vdash x : \square \rightarrow \square \\
    . , x : \alpha \rightarrow \alpha \vdash \square : \square
  }
  {\centerdot, x : \alpha \rightarrow \alpha \vdash x \ \square : \square \\ \mathbf{(2)}}
  \\
  \infer{
    \infer{
      \infer{
        y : \alpha \in \centerdot , y : \alpha \\ \alpha > \alpha
      }{\centerdot , y : \alpha \vdash y : \alpha}
    }
    {\centerdot \vdash \lambda y . y : \alpha \rightarrow \alpha} \\
    % app
    \infer{
      \mathbf{(1)} \\ \mathbf{(2)}
    }{\centerdot, x : \alpha \rightarrow \alpha \vdash (x \ x) \ ( x \ \square) : \square}
  }
  { \centerdot \vdash \letin{x = \lambda y . y}{(x \ x) \ (x \ \square)} : \square}
\end{mathpar}

\subsubsection{Concurrency}
Here is our first example of accessing two resources concurrently ---
with a bit of imagination one can think of this as reading a file
from a disk whilst simultaneously fetching data over the network.
\begin{mathpar}
  \mprset {sep=1em}
  \infer{
    % app
    \infer{
      %abs
      \infer{
        % use
        \infer{
          \centerdot, x : \alpha \vdash x : \alpha \\
          \textsf{ok} \ \textsf{File}
        }{\centerdot, x : \alpha \vdash \llbracket x \rrbracket_{\textsf{File}} : \IO_{\textsf{File}} \alpha}
      }{\centerdot \vdash \lambda x . \llbracket x \rrbracket_{\textsf{File}} : \alpha \rightarrow \IO_{\textsf{File}} \alpha}
      \\
      \centerdot \vdash \square : \square
    }{\centerdot \vdash (\lambda x . \llbracket x \rrbracket_{\textsf{File}}) \ \square : \IO_{\textsf{File}} \square}
    \\
    \infer{ %use
      \centerdot \vdash \square : \square \\ \textsf{ok} \ \textsf{Net}
    }{\centerdot \vdash \llbracket \square \rrbracket_{\textsf{Net}} : \IO_{\textsf{Net}} \square}
    \\
    \textsf{File} \cap \textsf{Net} = \emptyset
  }
  {\centerdot \vdash (\lambda x . \llbracket x \rrbracket_{\textsf{File}}) \ \square) \curlyvee \llbracket \square \rrbracket_{\textsf{Net}} : \IO_{\textsf{File} \cup \textsf{Net}} \square \times \square}
\end{mathpar}

\subsubsection{Subsumption}
% \begin{minipage}{\textwidth}
Assume there is a function inside our context called
\textsf{writeFile} with the type $\square \rightarrow \IO_{\textsf{File}} \square$. We can
subsume its heap to be part of a larger heap, like
$\textsf{File} \cup \textsf{Net}$. We will show this with the expression
\[ \llbracket \square \rrbracket_{\textsf{File}} \curlyvee \llbracket \square \rrbracket_{\textsf{Net}}
\bind
\lambda x . \textsf{writeFile} \ (\pi_1 \ x)
\]
  \begin{mathpar}
    \Gamma = \centerdot, \textsf{writeFile} : \square \rightarrow \IO_{\textsf{File}} \square \\
    \infer{ %conc
      \infer{ %use
        \Gamma \vdash \square : \square
      }{\Gamma \vdash \llbracket \square \rrbracket_{\textsf{File}} : \IO_{\textsf{File}} \square} \\
      \infer{ %use
        \Gamma \vdash \square : \square
      }{\Gamma \vdash \llbracket \square \rrbracket_{\textsf{Net}} : \IO_{\textsf{Net}} \square} \\
      \textsf{File} \cap \textsf{Net} = \emptyset
    }
    {\Gamma \vdash \llbracket \square \rrbracket_{\textsf{File}} \curlyvee \llbracket \square \rrbracket_{\textsf{Net}} : \IO_{\textsf{File}
        \cup \textsf{Net}} \square \\ \mathbf{(1)}}
    \\
    \infer{ %app
      \Gamma , x : \square \times \square \vdash \textsf{writeFile} : \square \rightarrow \IO_{\textsf{File}} \square
      \\
      \infer{
        \Gamma , x : \square \times \square \vdash x : \square \times \square
      }{\Gamma , x : \square \times \square \vdash \pi_1 \ x : \square}
    }{\Gamma , x : \square \times \square \vdash \textsf{writeFile} \ (\pi_1 \ x) :
      \IO_{\textsf{File}} \square \\ \mathbf{(2)}}
    \\
    \infer{ %bind
      \mathbf{(1)} \\
      \infer{ %abs
        \infer*[Right=Sub]{ %sub
          \mathbf{(2)} \\
          \textsf{File} \geq: \textsf{File} \cup \textsf{Net} \\
          \textsf{ok} \ \textsf{File} \cup \textsf{Net}
        }{\Gamma, x : \square \times \square \vdash \textsf{writeFile} \ (\pi_1 \ x) : \IO_{\textsf{File} \cup \textsf{Net}} \square}
      }{\Gamma \vdash \lambda x . \textsf{writeFile} \ (\pi_1 \ x) : \IO_{\textsf{File} \cup \textsf{Net}} \square}
    }{\Gamma \vdash
      \llbracket \square \rrbracket_{\textsf{File}} \curlyvee \llbracket \square \rrbracket_{\textsf{Net}}
      \bind
      \lambda x . \textsf{writeFile} \ (\pi_1 \ x)
      : \IO_{\textsf{File} \cup \textsf{Net}} \square}
  \end{mathpar}

% \end{minipage}

\section{Dynamic semantics}
As shown in Chapter~\ref{chapter:background}, the dynamic semantics in
the original Hindley-Damas-Milner system was based on denotational
semantics. In our type system, we will use operational
semantics. Operational semantics are similar to what we have seen
before in the definition of the static semantics. We define a bunch of
inference rules, and from these build up
proofs. Tofte~\cite{tofte1988} had the idea of using operational
semantics for not just the typing rules, but also for the dynamic
semantics. We choose this approach over denotational semantics as it
unifies our approach to types and evaluation, and as Tofte said,
\textsl{``it seems a bit unfortunate that we should have to understand
  domain theory to be able to investigate whether a type inference
  system admits faulty programs''}.

\subsection{Values}
Before we can talk about how we evaluate a program, we need to define
what constitutes a fully evaluated program. That is, what terms are a
result of a completed computation. We define a unary relation called
\textsf{value}, and give rules describing what terms can be considered
values in Figure~\ref{fig:values}.

For example, we cannot evaluate the unit type any further, therefore
all $\square$s are considered values. The same goes for product types, but
only if both inner components are values themselves:
$\square \times \square$ is a finished value, but
$f \ e \times \square$ might still have evaluation left to do. A lambda on its
own is a value too. Without being applied to an argument it cannot be
evaluated any further: The computation inside of it is suspended. In a
similar fashion a lifted computation cannot be computed any further
\textit{on its own}. We will see later how binding can run this
computation, but by itself it will not evaluate to anything.

\begin{figure}
  \begin{mathpar}
    \boxed{\textsf{value} \ e} \\
    \inferrule{ }{ \textsf{value} \ \square } \and
    \inferrule{\textsf{value} \ e_1 \\ \textsf{value} \ e_2}{ \textsf{value} \ e_1 \times e_2 } \and
    \inferrule{ }{ \textsf{value} \ \lift{e} } \and
    \inferrule{ }{ \textsf{value} \ \use{r}{e} } \and
    \inferrule{ }{ \textsf{value} \ \lambda x. e }
  \end{mathpar}
  \caption{Terminal values}\label{fig:values}
\end{figure}

\subsection{Small-step semantics}
The approach to operational semantics we will be taking is
\emph{small-step operational semantics}. In small-step operational
semantics we define a step relation $a \leadsto b$ which says that in one
``step'', $a$ \textit{reduces to} $b$. $b$ might then go onto reduce
further if it is able to, or it could be a finished value. So
reduction can be thought of as evaluation of a program, bit by
bit. Small-step semantics differs from big-step semantics, where the
relation $a \Downarrow b$ says that at the end of the day, $a$ will reduce to
$b$, and $b$ will not reduce any further.

As an example, if an expression $e_1$ reduces to $e_1'$, i.e.\ $e_1 \leadsto
e_1'$, then we want the application $e_1 \ e_2$ to reduce
as well. We can write this as
\begin{mathpar}
  \inferrule{e_1 \leadsto e_1'}{e_1 \ e_2 \leadsto e_1' \ e_2}
\end{mathpar}

We call this type of reduction rule which takes smaller reductions and
updates it within a bigger structure, \xi-reduction. There are
\xi-reductions for other expressions with structure inside, namely
product types, the concurrent operator and the bind operator.

Another type of reduction rule is \beta-reduction, which comes from the
lambda calculus. When we apply an argument to an abstraction, we
substitute the bound variable inside the abstraction with the
argument. The \beta-reduction rule defines this.
\begin{mathpar}
  \inferrule{\textsf{value} \ e_2}{ (\lambda x . e_1) \ e_2 \leadsto e_1 [ e_2 / x ]}
\end{mathpar}

It is important to note the premise here that states the argument
being applied must be a value. This enforces a strict evaluation
order, since in order for the argument to be a value it must be
completely reduced. A lazily evaluated semantics might forgo this
extra requirement, so that the argument can be reduced after
substitution.

As mentioned earlier, lifted expressions $\lift{e}$ are suspended much
like lambdas, and so are values since they cannot reduce any
further \emph{on their own}. However, with the bind operator, the
value inside them can be extracted out and fed into an abstraction.
\begin{mathpar}
  \inferrule{ }{\lift{e} \bind e' \leadsto e' \ e}
\end{mathpar}
Unlike the semantic rule for \beta-reduction, there is no premise
enforcing that $\lift{e}$ is a value, since all lifted terms are
values anyway.

For an expression lifted into a resourceful $\IO$ monad with
$\use{\rho}{e}$, one might be tempted to just reduce this to a $\lift{e}$.
\begin{mathpar}
  \infer{ }{\use{r}{e} \leadsto \lift{e}}
\end{mathpar}
And we could then also define the reduction for concurrency like so:
\begin{mathpar}
\inferrule{ }{\lift{v} \curlyvee \lift{w} \leadsto \llbracket v \times w \rrbracket}
\end{mathpar}
However, the intermediate $\lift{e}$ can be confusing. The purpose of
the lift operator is to lift a pure value into any possible resource
bound monad. When we see a lift, we think of the typing judgement
\textsc{lift} that allows it to fit any heap, when in fact a use
should restrict what heaps it can go into. For these reasons, we
instead define its reduction identically to lift.
\begin{mathpar}
  \inferrule{ }{\use{r}{e} \bind e' \leadsto e' \ e}
\end{mathpar}
Concurrency is then defined as chaining together two binds, and
returning the lifted product of the two results.
\begin{mathpar}
  \inferrule{ }{v \curlyvee w \leadsto
    v \bind \lambda v . (w \bind \lambda w . \lift{v \times w})}
\end{mathpar}
This might seem like the opposite of concurrency --- executing the
computation in sequence --- but because our monad does not have any
state (see Section~\ref{section:modellingstate}),
\begin{samepage}
\[v \bind \lambda v . (w \bind (\lambda w . \lift{v \times w}))\]
is identical to
\[w \bind \lambda w . (v \bind (\lambda v . \lift{v \times w}))\]
\end{samepage}
In fact the previous
definition for concurrency as
$\lift{v} \curlyvee \lift{w} \leadsto \llbracket v \times w \rrbracket$ will have the same reduction steps at
the end of the day. In a real-world programming language
implementation, the evaluation would actually be implemented concurrently.

Although the $v \curlyvee w$ might evaluate to two separate $\bind$s, the
static semantics of concurrency and binding are different. For example, given
$\Gamma \vdash \lift{v} : \IO_\rho \tau$ we are allowed to bind $lift{v}$ with itself:
\[
\Gamma \vdash \lift{v} \bind \lambda v . (\lift{v} \bind (\lambda v . \lift{v \times v})) : \IO_\rho (\tau \times \tau)
\]
But with the concurrent operator, this is a type error, since the
premise $\textsf{ok} \ \rho \cup \rho$ does not hold: There is no type in any context
that can be given to $v \curlyvee v$.

\begin{figure}
  \begin{mathpar}
    \boxed{e \leadsto e'} \\

    % lambda calculus + HM
    \inferrule{e_1 \leadsto e_1'}{e_1 \ e_2 \leadsto e_1' \ e_2} \and
    \inferrule{e_2 \leadsto e_2'}{e_1 \ e_2 \leadsto e_1 \ e_2'} \and
    \inferrule{\textsf{value} \ e'}{ (\lambda x . e) \ e' \leadsto e [ e' / x ] } \and
    \inferrule{ }{\letin{x = e'}{e} \leadsto e [e' / x]} \\

    % product types
    \inferrule{e_1 \leadsto e_1'}{e_1 \times e_2 \leadsto e_1' \times e_2} \and
    \inferrule{e_2 \leadsto e_2'}{e_1 \times e_2 \leadsto e_1 \times e_2'} \and

    \inferrule{e \leadsto e'}{\pi_1 e \leadsto \pi_1 e'} \and
    \inferrule{e \leadsto e'}{\pi_2 e \leadsto \pi_2 e'} \and

    \inferrule{ }{\pi_1 (e_1 \times e_2) \leadsto e_1} \and
    \inferrule{ }{\pi_2 (e_1 \times e_2) \leadsto e_2} \and

    % monads
    \inferrule{e_1 \leadsto e_1'}{e_1 \bind e_2 \leadsto e_1' \bind e_2} \and
    \inferrule{ }{\lift{v} \bind e_2 \leadsto e_2 \ v} \and
    \inferrule{ }{\use{r}{e} \bind e' \leadsto e' \ e} \\

    % resource stuff
    \inferrule{ }{v \curlyvee w \leadsto v \bind \lambda v . (w \bind \lambda w . \lift{v \times
        w})} \
  \end{mathpar}
  \caption{Dynamic Semantics}\label{fig:dynamicsemantics}
\end{figure}

\subsection{The reflexive and transitive closure}

We can go a \textit{step} further and define $\twoheadrightarrow$ as the reflexive,
transitive closure of $\leadsto$. What does that mean? If $\leadsto$ is a binary
relation of two terms, then the transitive closure
$\twoheadrightarrow$ is a new relation that maintains all the relations of
$\leadsto$ and is transitive, i.e.\ if $a \twoheadrightarrow b$ and
$b \twoheadrightarrow c$, then $a \twoheadrightarrow c$.  A reflexive transitive closure extends this so
that for any $a$, $a \twoheadrightarrow a$.

Intuitively speaking, if the small-step inference rule $a \leadsto b$ says
that $a$ reduces to $b$ in exactly one step, then
$a \twoheadrightarrow b$ says $a$ reduces to $b$ in zero or more steps. Instead of
reducing one step at a time, it can go all the way.

\begin{samepage}
  For example, if we
have the expression ${(\lambda x . x \ \square) \ (\lambda y . y)}$ then have the
following relations:
\[ (\lambda x . x \ \square) \ (\lambda y . y) \leadsto (\lambda y . y) \ \square \leadsto \square \]
\[ (\lambda x . x \ \square) \ (\lambda y . y) \twoheadrightarrow \square\]
\end{samepage}
We do not use this relation to prove any properties about our system,
but it sheds light on how our expression language eventually reduces
to a value in the end.

\begin{figure}
  \begin{center}
    \tikz \graph[layered layout, grow=right, edges={
      decoration={snake,amplitude=0.5mm,segment length=2mm,post
        length=1mm},decorate}] {
      a -> b -> c
    };
    \qquad
    \tikz \graph[layered layout,grow=right,edges={->>}] {
      a ->[orient] b -> c;
      a ->[bend right] c;
      { [same layer] a, b, c };
      a ->[loop above] a;

      b ->[loop above] b;

      c ->[loop above] c;

      % a -> c;
      % b -> c;
      % a -> c;
    };
\end{center}
\caption{Left: the $\leadsto$ relation. Right: The reflexive, transitive
  closure of $\leadsto$, the reduction relation $\twoheadrightarrow$.}\label{fig:reduction}
\end{figure}

%%% Local Variables:
%%% TeX-master: "report"
%%% TeX-engine: luatex
%%% TeX-command-extra-options: "-shell-escape"
%%% End:

% LocalWords:  monadic HDM monomorphic Krishnaswami unary


\chapter{Properties}\label{cha:properties}
A type system is no good unless we can prove it is worth its salt. In
this section we will prove a number of lemmas and theorems which will
eventually show that the type system is sound. We will begin with a
couple of properties about contexts that we will later use on in
various other proofs. The first states that if we have two variables
in the context with the same names but different type schemes, then we
can ignore the first one as it is overshadowed by the second --- any
reference to $x$ will result in $x : \sigma'$.
\begin{lemma}[Drop]\label{lem:drop}
  If $\Gamma , x : \sigma, x : \sigma' \vdash e : \tau$, then $\Gamma , x : \sigma' \vdash e : \tau$
\end{lemma}
The second states that we can sneak in another variable before another
variable of the same name, for the exact same reason.
\begin{lemma}[Sneak]\label{lem:sneak}
  If $\Gamma , x : \sigma' \vdash e : \tau$, then for any $\sigma$, $\Gamma , x : \sigma, x : \sigma' \vdash e : \tau$
\end{lemma}
We can then also say that if two variables $x$ and $y$ are indeed
different, we are free to permute them and swap them about.
\begin{lemma}[Swap]\label{lem:swap}
  If $x \neq y$ and $\Gamma , x : \sigma, y : \sigma' \vdash e : \tau$, then $\Gamma , y : \sigma' , x : \sigma \vdash e
  : \tau$
\end{lemma}
And if we have a typing judgement in an empty context, we can weaken
the judgement to extend this to any other context.
\begin{lemma}[Weaken]\label{lem:weaken}
  If $\centerdot \vdash e : \tau$, then for $\Gamma$, $\Gamma \vdash e : \tau$.
\end{lemma}

We will also need to show this property about instantiation and the
close function.
\begin{lemma}\label{lem:close>}
  If $\sigma > \sigma'$ then $\overline{\Gamma , x : \sigma}(\tau) > \overline{\Gamma, x : \sigma'}(\tau)$.
\end{lemma}
\begin{proof}
  We provide an informal proof as follows.
  \begin{enumerate}
  \item If $\sigma > \sigma'$, then for every $\tau$ that $\sigma > \tau$, $\sigma' > \tau$.
  \item So $\sigma$ must parameterise over at least the same number of type
    variables if not \textit{more} than $\sigma'$.
  \item So $\sigma$ has at least the same number of free type variables if
    not \textit{less} than $\sigma'$.
  \item By definition of the close function,
    $\overline{\Gamma, x : \sigma}(\tau)$ will then have at least the same number
    of bound type variables if not \textit{more} than
    $\overline{\Gamma, x : \sigma'}(\tau)$.
  \item So if $\overline{\Gamma, x : \sigma'}(\tau)$ can be instantiated to some
    $\tau$, then $\overline{\Gamma, x : \sigma}(\tau)$ can also be instantiated to
    that $\tau$, as it has enough bound type variables to handle
    everything the former can --- if it had less type variables than the
    former, then it would not be able to instantiate these types, but
    for the converse excess type variables can be mapped to whatever.
  \end{enumerate}
\end{proof}

Now we prove the generalisation theorem, which states that if a type
scheme inside the context is an instantiation of another type scheme,
then we can use the more general type scheme and preserve how things
are typed. This is an adaptation of a lemma from Wright and
Felleisen~\cite[Lemma 4.6]{wright1994}, which is in turn an
adaptation of a lemma from Damas and Milner.

\begin{theorem}[Generalisation]\label{lem:generalisation}
  If $\Gamma, x : \sigma' \vdash e : \tau$ and $\sigma > \sigma'$, then \\ ${\Gamma, x : \sigma \vdash e : \tau}$.
\end{theorem}
\begin{proof}
  Begin with induction on the proof for $\Gamma , x : \sigma' \vdash e : \tau$.
  \begin{description}
  \item[\rm\textsc{Var}]
    From the premises we have $x : \sigma' \in \Gamma$ and $\sigma' > \tau$. We also have
    $\sigma > \sigma'$ but by definition of $\sigma > \sigma'$, if $\sigma' > \tau$ then $\sigma > \tau$.
    And we also have by definition $x : \sigma \in \Gamma , x : \sigma$. So putting the
    pieces together, we can use \textsc{Var} to get
    \begin{mathpar}
      \infer{x : \sigma \in \Gamma , x : \sigma \\ \sigma > \tau}
      {\Gamma , x : \sigma \vdash x : \tau}
    \end{mathpar}
  \item[\rm\textsc{Abs}]
    We have $\Gamma , x : \sigma' \vdash \lambda y . e : \tau$. If $x = y$, then we can safely
    say the premise $\Gamma , x : \sigma' , y : \sigma'' \vdash e : \tau$ becomes $\Gamma , y :
    \sigma'' \vdash e : \tau$  due to Lemma~\ref{lem:drop}. And from this we can
    then use Lemma~\ref{lem:sneak} to get $\Gamma, x : \sigma, y : \sigma'' \vdash e : \tau$,
    and ultimately $\Gamma , x : \sigma \vdash \lambda y . e : \tau$.

    If $x \neq y$, then we can get $\Gamma , y : \sigma'' , x : \sigma' \vdash e : \tau$ via
    Lemma~\ref{lem:swap}. The induction hypothesis then results in $\Gamma
    , y : \sigma'' , x : \sigma \vdash e : \tau$, which we can then swap back again to
    get $\Gamma , x : \sigma , y : \sigma'' \vdash e : \tau$ and so $\Gamma , x : \sigma \vdash \lambda y . e :
    \tau$.
  \item[\rm\textsc{Let}]
    For $\letin{y = e'}{e}$, we have $\Gamma , x : \sigma' \vdash e' : \tau'$ and $\Gamma , x :
    \sigma' , y : \overline{\Gamma , x : \sigma'}(\tau') \vdash e : \tau$, and we aim to show
    $\Gamma , x : \sigma \vdash \letin{y=e'}{e} : \tau$.
    
    First off, we need to convert the $y : \overline{\Gamma , x : \sigma'}(\tau')$ to a
    $y : \overline{\Gamma, x : \sigma}(\tau')$ somehow. But it can be shown that if $\sigma >
    \sigma'$, then $\overline{\Gamma , x : \sigma}(\tau) > \overline{\Gamma , x : \sigma'}(\tau)$ due
    to Lemma~\ref{lem:close>}. So by the inductive hypothesis,
    we are able to get $\Gamma , x : \sigma' , y : \overline{\Gamma, x : \sigma}(\tau') \vdash e : \tau$.

    If $x = y$ then we can drop $x :
    \sigma'$ from the context and sneak it back in as $\Gamma , x : \sigma, y :
    \overline{\Gamma, x : \sigma}(\tau') \vdash e : \tau$ with Lemma~\ref{lem:drop} and
    Lemma~\ref{lem:sneak}. The induction hypothesis then gives us $\Gamma ,
    x : \sigma \vdash e' : \tau'$ and we construct the proof back together with
    \textsc{Let} to give $\Gamma , x : \sigma \vdash \letin{y = e'}{e} : \tau$.

    If $x \neq y$, then the proof is a bit more complicated. We take the
    following steps:
    \begin{align*}
      \Gamma , x : \sigma' , y : \overline{\Gamma, x : \sigma}(\tau') \vdash e : \tau \\
      \Gamma , y : \overline{\Gamma, x : \sigma}(\tau') , x : \sigma' \vdash e : \tau &&  \text{by
                                                            swapping,
                                                            Lemma~\ref{lem:swap}}
      \\
      \Gamma , y : \overline{\Gamma, x : \sigma}(\tau') , x : \sigma \vdash e : \tau && \text{by
                                                           inductive
                                                           hypothesis} \\
      \Gamma , x : \sigma , y : \overline{\Gamma, x : \sigma}(\tau') \vdash e : \tau && \text{by swapping again}
    \end{align*}
    And proceed to construct the proof for \textsc{Let} as previously.
  \item[\rm\textsc{App}]
    From the premises we have $\Gamma , x : \sigma' \vdash e : \tau' \rightarrow \tau$ and $\Gamma , x : \sigma' \vdash
    e' : \tau'$. We wish to show $\Gamma , x : \sigma \vdash e \ e' : \tau$.
    We apply the induction hypothesis to get $\Gamma , x : \sigma \vdash e : \tau' \rightarrow \tau$
    and  $\Gamma , x : \sigma \vdash e' : \tau$. Then use \textsc{App} to build up a
    proof of $\Gamma , x : \sigma \vdash e \ e' : \tau$.
  \item[The remaining cases] The rest of the possible proofs for
    $\Gamma , x : \sigma' \vdash e : \tau$ can all be proved by applying the
    induction hypothesis on their structure, much like the case for
    \textsc{App}, and so are omitted for brevity.
  \end{description}
\end{proof}

If a context has all the same type schemes for each free variable in
an expression as another context, then we can make the same typing
judgements with that context. In other words, we are free to ignore
extra variables not used by the expression.
\begin{lemma}\label{lem:ignore}
  If for all $x \in \fv(e)$ where $x : \sigma \in \Gamma$, $x : \sigma \in \Delta$, and if $\Gamma \vdash
  e : \tau$ then $\Delta \vdash e : \tau$.
\end{lemma}

We can also show that we can apply a substitution to both the context
and type given in a judgement.
\begin{lemma}\label{lem:subContextTyping}
  If $\Gamma \vdash e : \tau$, then for any substitution $s$, $\Gamma[s] \vdash e : \tau[s]$.
\end{lemma}

An important lemma that we need to show is the substitution
lemma, which relates to substituting a variable for an expression,
whose type we can prove. The following proof is adapted from Wright
and Felleisen~\cite{wright1994}.

\begin{lemma}[Substitution]\label{lem:substitution}
  If $\Gamma \vdash e : \tau$ and ${\Gamma , x : \forall \alpha_1, \ldots, \alpha_n . \tau \vdash e' : \tau'}$, and ${x \notin \dom(\Gamma)}$ and
  ${\alpha_1 , \ldots, \alpha_n \cap \ftv(\Gamma) = \emptyset}$, then ${\Gamma \vdash e' [e/x] : \tau'}$.
\end{lemma}
Note that the domain of a context $\dom(\centerdot , x_1 : \sigma_1 , \ldots , x_n :
\sigma_n)$ is defined as $\{ x_1 , \ldots , x_n \}$.
\begin{proof}
  Begin with induction on the proof of $\Gamma , x : \forall \alpha_1 \ldots \alpha_n . \tau \vdash e'
  : \tau'$. For brevity we will sometimes refer to $\forall \alpha_1,\ldots,\alpha_n . \tau$ as $\sigma$.
  \begin{description}
  \item[\rm\textsc{Var}] We have
    $\Gamma , x : \forall\alpha_1\ldots\alpha_n \vdash y : \tau'$ and want to show $\Gamma \vdash y[e/x] : \tau'$.

    If $y \neq x$, then by definition $y [e/x] = y$, so we just need to
    show $\Gamma \vdash y : \tau'$. From the premises we also have
    $y : \tau' \in \Gamma , x : \forall\alpha_1\ldots\alpha_n$, but since $y \ne x$ we have $y : \tau' \in \Gamma$. And
    from here we can use \textsc{Var} to get $\Gamma \vdash y : \tau'$.

    If $y = x$, then we need to show $\Gamma \vdash x : \tau'$. From the premise of
    ${\Gamma , x : \forall \alpha_1\ldots\alpha_n . \tau \vdash x : \tau'}$ we have
    ${\forall \alpha_1\ldots\alpha_n . \tau > \tau'}$, so there exists a substitution
    $s$ which replaces exactly $\alpha_1,\ldots,\alpha_n$, such that
    $\tau[s] = \tau'$.

    We can use Lemma~\ref{lem:subContextTyping} to get
    $\Gamma[s] \vdash e : \tau[s]$, or $\Gamma[s] \vdash e : \tau'$. And furthermore, because
    ${\alpha_1 , \ldots, \alpha_n \cap \ftv(\Gamma) = \emptyset}$, applying $s$ to $\Gamma$ will do
    nothing since $\Gamma$'s free type variables are distinct, i.e. $\Gamma =
    \Gamma[s]$. Leaving us with $\Gamma \vdash e : \tau'$ as required.

  \item[\rm\textsc{Abs}]
    We have $\Gamma , x : \forall\alpha_1\ldots\alpha_n \vdash \lambda y . e' : \tau_1 \rightarrow \tau_2$ and want to show
    \[\Gamma \vdash (\lambda y . e')[e/x] : \tau_1 \rightarrow \tau_2\]
    Similarly to the case for $\textsc{var}$, we first check to see if
    $x = y$. If $y = x$, then the inner binding of $y$ will shadow $x$. That
    is, we have the premise $\Gamma , x : \sigma, x : \tau_1 \vdash e' : \tau_2$, and we can
    drop the previous $x$ from the context with
    Lemma~\ref{lem:drop} to get $\Gamma , x : \tau_1 \vdash e' : \tau_2$.
    By definition, if $x = y$ then $(\lambda y . e')[e/x] = \lambda y . e'$, so we
    just need to show $\Gamma \vdash \lambda y . e' : \tau_1 \rightarrow \tau_2$, which can construct
    with \textsc{Abs} and $\Gamma , x : \tau_1 \vdash e' : \tau_2$.

    If $y \ne x$, then the proof is more involved. From the premises
    we have
    \[\Gamma , x : \forall \alpha_1\ldots\alpha_n . \tau , y : \tau_1 \vdash e' : \tau_2\]
    We start by choosing a substitution $s$, which maps $\alpha_1 , \ldots, \alpha_n$
    to fresh type variables $\alpha'_1, \ldots,\alpha'_n$. Additionally, they are
    distinct as follows.
    \[\ftv(\Gamma) \cap \alpha_1,\ldots,\alpha_n \cap \alpha'_1,\ldots,\alpha'_n = \emptyset\]

    Now we manipulate the premise in the following order.
    \begin{align*}
      \Gamma , y : \tau_1 , x : \forall\alpha_1\ldots\alpha_n . \tau &\vdash e' : \tau_2 && \text{by swapping,
                                                   Lemma~\ref{lem:swap}}
      \\
      (\Gamma , y : \tau_1 , x : \forall\alpha_1\ldots\alpha_n . \tau)[s] &\vdash e' : \tau_2 && \text{by
                                                     Lemma~\ref{lem:subContextTyping}}
      \\
      \Gamma , y : \tau_1[s] , x : (\forall\alpha_1\ldots\alpha_n . \tau)[s] &\vdash e' : \tau_2 &&
                                                            \text{since
                                                            } \Gamma[s] = \Gamma
    \end{align*}
    And since the range of $s$ is exactly $\alpha_1,\ldots,\alpha_n$, we have
    $(\forall \alpha_1\ldots\alpha_n . \tau)[s] = \forall \alpha_1\ldots\alpha_n . \tau$ --- substitution only
    substitutes free type variables, not bound type variables, and
    here any possible free type variables are being shadowed.
    \begin{equation}
      \Gamma , y : \tau_1[s] , x : \forall \alpha_1\ldots\alpha_n . \tau \vdash e' : \tau_2
      \label{eq:substAbs2}
    \end{equation}

    % $\Gamma(z) = (\Gamma , y : \tau_1[s])(z)$ for all $z \in \fv(e)$
    We can apply Lemma~\ref{lem:ignore} to $\Gamma \vdash e : \tau$, giving us
    \begin{equation}
      \Gamma , y : \tau_1[s] \vdash e : \tau\label{eq:substAbs3}
    \end{equation}
    And when we combine $x \notin \dom(\Gamma)$ with the fact that $x \ne y$, we get
    \begin{equation}
      x \notin \dom(\Gamma , y : \tau_1[s])
      \label{eq:substAbsNotIn}
    \end{equation}
    Because of how we chose $s$, we also have
    \begin{equation}
      \ftv(\Gamma, y : \tau_1[s]) \cap {\alpha_1,\ldots,\alpha_n} = \emptyset\label{eq:substAbs4}
    \end{equation}
    Eventually, we use (\ref{eq:substAbs3}), (\ref{eq:substAbs2}),
    (\ref{eq:substAbsNotIn}) and (\ref{eq:substAbs4}) with the
    induction hypothesis to arrive at
    \[ \Gamma , y : \tau_1[s] \vdash e' [e/x] : \tau_2[s] \]
    But $s$ is a bijection due to how we chose it: That means an
    inverse, $s^{-1}$ exists. We ``apply it to both sides'' with
    Lemma~\ref{lem:subContextTyping}
    \begin{align*}
      (\Gamma , y : \tau_1[s])[s^{-1}] &\vdash e' [e/x] : (\tau_2[s])[s^{-1}] \\
      (\Gamma , y : \tau_1[s])[s^{-1}] &\vdash e' [e/x] : \tau_2 \\
      \Gamma[s^{-1}] , y : \tau_1[s][s^{-1}] &\vdash e' [e/x] : \tau_2 \\
      \Gamma[s^{-1}], y : \tau_1 &\vdash e' [e/x] : \tau_2
    \end{align*}
    And because $\alpha'_1 , \ldots, \alpha'_n \cap \ftv(\Gamma) = \emptyset$, we can get rid of that
    last substitution and arrive at ${\Gamma, y : \tau_1 \vdash e' [e/x] : \tau_2}$.
    From here, we build our way back up with \textsc{Abs}.
    \begin{align*}
      \Gamma \vdash \lambda y . (e' [e/x]) : \tau_1 \rightarrow \tau_2 \\
      \Gamma \vdash (\lambda y . e')[e/x] : \tau_1 \rightarrow \tau_2 && \text{because } x \ne y
    \end{align*}
    
  \item{\rm\textsc{Let}} We have the proof ${\Gamma , x : \sigma \vdash \letin{y = e_1}{e_2} :
      \tau'}$
    and want to show
    \[\Gamma \vdash (\letin{y = e_1}{e_2})[e/x] : \tau'\]
    The first premise can be fed directly into the induction hypothesis
    \begin{align}
      \Gamma , x : \forall \alpha_1\ldots\alpha_n . \tau \vdash e_1 : \tau_1 \nonumber \\
      \Gamma \vdash e_1 [e / x] : \tau_1 \label{eq:substLet7}
    \end{align}

    Now if $x = y$, we take the second premise as follows
    \begin{align*}
      \Gamma , x : \forall \alpha_1\ldots\alpha_n . \tau , x : \overline{\Gamma, x : \forall\alpha_1\ldots\alpha_n . \tau}(\tau_1)
      \vdash e_2 : \tau' \\
      \Gamma , x : \overline{\Gamma, x : \forall\alpha_1\ldots\alpha_n . \tau}(\tau_1) \vdash e_2 : \tau' &&
                                                                \text{by
                                                                Lemma~\ref{lem:drop}}
      \\
      \Gamma \vdash \letin{x = e_1[e / x]}{e_2} : \tau' && \text{by \textsc{Let}}
    \end{align*}
    Since $x = y$, by the definition of substitution
    $\letin{y=e_1[e/x]}{e_2}$ is equivalent to $(\letin{y=e_1}{e_2})[e/x]$.
    And so we reach our goal
    \[\Gamma \vdash (\letin{y = e_1}{e_2})[e/x] : \tau'\]

  If we are unfortunate enough that $y \ne x$, then from the second
  premise we use the swap lemma to get
  \begin{equation}\label{eq:substLet8}
    \Gamma , y : \overline{\Gamma, x : \forall\alpha_1\ldots\alpha_n . \tau}(\tau_1), x : \forall \alpha_1\ldots\alpha_n . \tau  \vdash
    e_2 : \tau'
  \end{equation}
  And since we have $\Gamma \vdash e : \tau$, from Lemma~\ref{lem:ignore}
  there is
  \begin{equation}\label{eq:substLet9}
  \Gamma , y : \overline{\Gamma , x :\forall\alpha_1\ldots\alpha_n . \tau}(\tau_1) \vdash e : \tau    
  \end{equation}
  Now we want to call the inductive hypothesis on (\ref{eq:substLet8})
  and (\ref{eq:substLet9}), but that means we first need to show
  \[\{ \alpha_1,\ldots,\alpha_n\} \cap \ftv(\Gamma , y : \overline{\Gamma, x : \forall\alpha_1\ldots\alpha_n . \tau}(\tau_1))
    = \emptyset\]
  We do this as follows:
  \begin{align*}
    & \{ \alpha_1,\ldots,\alpha_n\} \cap \ftv(\Gamma , y : \overline{\Gamma, x : \forall\alpha_1\ldots\alpha_n . \tau}(\tau_1)) \\
    &\subseteq \{ \alpha_1,\ldots,\alpha_n\} \cap (\ftv(\Gamma) \cup \ftv(\overline{\Gamma, x : \forall\alpha_1\ldots\alpha_n
      . \tau}(\tau_1))) \\
    &= \{ \alpha_1,\ldots,\alpha_n\} \cap (\ftv(\overline{\Gamma, x : \forall\alpha_1\ldots\alpha_n . \tau}(\tau_1)))
    \\
    &= \{ \alpha_1,\ldots,\alpha_n\} \cap (\ftv(\tau_1) \setminus (\ftv(\tau_1) \setminus \ftv(\Gamma,x:\forall
      \alpha_1\ldots\alpha_n.\tau))) \\
    &= \{ \alpha_1,\ldots,\alpha_n\} \cap \ftv(\tau_1) \cap \ftv(\Gamma, x : \forall \alpha_1\ldots\alpha_n . \tau) \\
    &\subseteq \{ \alpha_1,\ldots,\alpha_n\} \cap \ftv(\Gamma, x : \forall \alpha_1\ldots\alpha_n . \tau) \\
    &= \{ \alpha_1,\ldots,\alpha_n\} \cap (\ftv(\Gamma) \cup \ftv(\forall \alpha_1\ldots\alpha_n . \tau)) \\
    &= \{ \alpha_1,\ldots,\alpha_n\} \cap \ftv(\forall \alpha_1\ldots\alpha_n . \tau) \\
    &= \{ \alpha_1,\ldots,\alpha_n\} \cap \ftv(\tau) \setminus \{\alpha_1,\ldots,\alpha_n\} \\
    &= \emptyset
  \end{align*}
  And then we can use it to arrive at
  \[ \Gamma , y : \overline{\Gamma, x : \forall\alpha_1\ldots\alpha_n.\tau}(\tau_1) \vdash e_2 [e/x] : \tau' \]
  The next part involves an observation, that for any $x$, $\tau$ and $\sigma$ in
  any $\Gamma$, ${\overline{\Gamma}(\tau) > \overline{\Gamma, x : \sigma}(\tau)}$, as closing
  over a context with a \textit{smaller} domain will result in the
  type scheme having \textit{more} bound type variables --- i.e.\ it is
  more general. Therefore we can use the generalisation lemma,
  Lemma~\ref{lem:generalisation}, to get
  \begin{align*}
    \Gamma , y : \overline{\Gamma}(\tau_1) \vdash e_2[e/x] : \tau' \\
    \Gamma \vdash \letin{y = e_1[e/x]}{e_2[e/x]} : \tau' && \text{with \textsc{Let}
                                               and
                                               (\ref{eq:substLet7})}
    \\
    \Gamma \vdash (\letin{y = e_1}{e_2})[e/x] : \tau' && \text{by definition of substitution}
  \end{align*}
  
  \item[The remaining cases] The rest of the proofs for $\Gamma \vdash e'[e/x] :
    \tau'$ are proved by applying the induction hypothesis on their structure.
  \end{description}
\end{proof}

\section{Type Soundness}

Now that we have the prerequisite lemmas out of the way, we can move
onto proving that the system is \textbf{sound}. Soundness is a
property of a logical system, stating that every judgement that can be
proved within the system is valid. In the type theory realm, a type
system is considered to be sound if and only if, for every typing
judgement we can produce for an expression in a context, it evaluates
to some member of the set of the type from the typing judgement --- or
as Milner put it, \textit{``well-typed programs can't go
  wrong''}~\cite{milner1978}[§3.7].
\[\Gamma \vdash e : \tau \rightarrow \Gamma \Vdash e : \tau\]
We saw how this was proved with denotational semantics in
Chapter~\ref{chapter:background}, but here we have modelled our
dynamic semantics with operational semantics.  Instead, we prove the
soundness of our type system \textit{syntactically}. This approach was
first introduced by Wright and Felleisen and 1994~\cite{wright1994},
and has since become the de facto method for proving soundness
with operational semantics.

Syntactic type soundness involves proving two properties of a type
system, \textit{type progress} and \textit{type preservation}.
The former states that for any typing judgement we can prove, the
expression is either a terminal value, or can be reduced further. The
latter property ensures that when an expression has a type, reducing
it to another expression does not change the type.
The proofs of both of these properties come from induction on the
proof of the typing judgement:

\begin{theorem}[Progress]\label{thm:progress}
  If $\centerdot \vdash e : \tau$, then either $\textsf{value} \ e$ or there exists a $e'$ such that $e \leadsto e'$.
\end{theorem}
\begin{proof}
  Begin by induction on $\centerdot \vdash e : \tau$.
  \begin{description}
  \item[\rm\textsc{Var}] This case is not possible --- it is impossible
    to construct a proof for $x \in \centerdot$, and thus impossible to arrive at
    $\centerdot \vdash x : \tau$.
  \item[\rm\textsc{Abs}] Any lambda abstraction is a value due to the
    rule
    \[\infer{ }{\textsf{Value} \ \lambda x . e}\]
  \item[\rm\textsc{App}] We have $\centerdot \vdash e_1 \ e_2 : \tau$, so by the premises
    of \textsc{App}, we have $\centerdot \vdash e_1 : \tau' \rightarrow \tau$ and $\centerdot \vdash e_2 :
    \tau'$.

    Consider the induction hypothesis on the first premise. Either
    $\textsf{value} \ e_1$, or $e_1 \leadsto e_1'$. In the latter case where
    $e_1$ reduces, we can then construct a proof that $e_1 \ e_2$
    reduces, and we're done.
    \[\infer{e_1 \leadsto e_1'}{e_1 \ e_2 \leadsto e_1' \ e_2}\]
    In the former case where $e_1$ is a value, we then apply the
    induction hypothesis to $\centerdot \vdash e_2 : \tau'$. Again, if $e_2$ reduces
    further then we can also construct a proof that $e_1 \ e_2$
    reduces.
    
    But if $\textsf{value} \ e_2$, we cannot reduce in the same way
    again. Instead, since we know $e_1$ is a value, and since we also
    know that $\centerdot \vdash e_1 : \tau' \rightarrow \tau$, we can deduce that
    $e_1$ must be an abstraction: it has the form $\lambda x
    . e$. Furthermore, because $e_2$ is a value, we can then apply the
    beta reduction rule.
    \[\infer{\textsf{value} \ e_2}{(\lambda x . e) \ e_2 \leadsto e[e_2/x]}\]
    We are ``calling a function'' in this sense, and now have proof
    that it reduces.
  \item[\rm\textsc{Let}] Our dynamic semantics for \textsc{Let} are
    lazy: ${\letin{x = e'}{e}}$ can reduce without evaluating $e'$
    first. And so for every let statement, we can immediately reduce.
    \[\infer{ }{\letin{x = e'}{e} \leadsto e[e'/x]}\]
  \item[\rm\textsc{Unit}] All unit expressions are values.
    \[\infer{ }{\textsf{value} \ \square}\]
  \item[\rm\textsc{Product}] For $\centerdot \vdash e_1 \times e_2 : \tau_1 \times \tau_2$, we have
    the premises $\centerdot \vdash e_1 : \tau_1$ and $\centerdot \vdash e_2 : \tau_2$. Using the
    induction hypothesis on either, if either of them reduce further,
    then we can also reduce $e_1 \times e_2$.
    \begin{align*}
      \infer{e_1 \leadsto e_1'}{e_1 \times e_2 \leadsto e_1' \times e_2} &&
      \infer{e_2 \leadsto e_2'}{e_1 \times e_2 \leadsto e_1 \times e_2'}
    \end{align*}
    If both of them are values, then the product itself is a value.
    \[\infer{\textsf{value} \ e_1 \\ \textsf{value} \ e_2}{
        \textsf{value} \ e_1 \times e_2}\]
  \item[\rm\textsc{Proj1}] For $\centerdot \vdash \pi_1 \ e : \tau$, we have the premise
    $\centerdot \vdash e : \tau \times \tau'$. Applying the induction hypothesis to the
    premise, if it reduces further, then we can reduce the expression
    as well.
    \[\infer{e \leadsto e'}{\pi_1 \ e \leadsto \pi_1 \ e'}\]
    However, if $e$ is a value then we know it must be of the form
    $e_1 \times e_2$, as that is the only expression that can have a
    product type and be a value. Thus we can use the reduction rule
    \[\infer{ }{\pi_1 \ (e_1 \times e_2) \leadsto e_1}\]
  \item[\rm\textsc{Proj2}] The proof for \textsc{Proj2} is pretty much
    identical to that of \textsc{Proj1}, so we will omit it for
    brevity.
  \item[\rm\textsc{Lift}] All lifted expressions are also values by
    definition.
    \[\infer{ }{\textsf{value} \ \lift{e}}\]
  \item[\rm\textsc{Use}] And likewise, all use expressions are values
    by definition.
    \[\infer{ }{\textsf{value} \ \use{r}{e}}\]
  \item[\rm\textsc{Bind}] For $\centerdot \vdash e_1 \bind e_2 : \tau$, we have two
    premises, $\centerdot \vdash e_1 : \IO_\rho \tau'$ and
    $\centerdot \vdash e_2 : \tau' \rightarrow \IO_\rho \tau$. Applying the induction hypothesis to the
    first premise, if $e_1$ reduces further then we can also say
    $e_1 \bind e_2$ reduces further
    \[\infer{e_1 \leadsto e_1'}{e_1 \bind e_2 \leadsto e_1' \bind e_2} \]
    If $e_1$ turns out to be a value, then because it has the type
    $\IO_\rho \tau'$, it must be either of the form $\lift{e}$ or
    $\use{r}{e}$. These are the only expressions that can both be
    values and have the type $\IO_\rho \tau'$.
    In either case, there exist reduction rules for both of these.
    \begin{align*}
      \infer{ }{\lift{e} \bind e_2 \leadsto e_2 \ e}&&
      \infer{ }{\use{r}{e} \bind e_2 \leadsto e_2 \ e}
    \end{align*}
  \item[\rm\textsc{Conc}] We can immediately reduce any expression
    of the form $v \curlyvee w$.
    \[ \infer{ }{v \curlyvee w \leadsto v \bind \lambda v . (w \bind \lambda w . \lift{v \times
          w})} \]
  \item[\rm\textsc{Sub}] If we have a subsumption judgement resulting
      in $\centerdot \vdash e : \IO_\rho' \tau$, then we also have the premise $\centerdot \vdash e :
      \IO_\rho \tau$ for some other $\rho$. But we can just call the induction
      hypothesis on this premise to show that either $e$ reduces or
      $e$ is a value. In any case, it is the same $e$ --- so we are done.
  \end{description}
\end{proof}

\begin{theorem}[Preservation]\label{thm:preservation}
  If $\centerdot \vdash e : \tau$ and $e \leadsto e'$, then $\centerdot \vdash e' : \tau$.
\end{theorem}
\begin{proof}
  As usual, start with induction on the proof for $\centerdot \vdash e : \tau$.
  \begin{description}
  \item[\rm\textsc{Var}] Like with the proof for progress, it is not
    possible to have a typing judgement for a variable in the empty
    context.
  \item[\rm\textsc{Abs}] Whilst it is possible to have a typing
    judgement of $\centerdot \vdash \lambda x . e$, it is not possible to have a reduction
    of the form $(\lambda x . e) \leadsto e'$. A lambda abstraction cannot reduce
    any further, and so we do not need to consider this case.
  \item[\rm\textsc{App}] We have $\centerdot \vdash e_1 \ e_2 : \tau$. Consider the
    possible cases for $e_1 \ e_2 \leadsto e'$. When it is $e_1$ that has
    reduced
    \[ \infer{e_1 \leadsto e_1'}{e_1 \ e_2 \leadsto e_1' \leadsto e_2} \] We can use the
    induction hypothesis with this alongside the first premise
    ${\centerdot \vdash e_1 : \tau' \rightarrow \tau}$ to get
    \[ \centerdot \vdash e_1' : \tau' \rightarrow \tau \] And then using \textsc{App} and the second
    premise, we build up a proof for
    \[ \centerdot \vdash e_1' \ e_2 : \tau \]
    When it is $e_2$ that is reduced to $e_2'$, the steps are the same: Use the
    induction hypothesis to get $\centerdot \vdash e_2 : \tau'$, and then $\centerdot \vdash e_1 \
    e_2' : \tau$.
    However, there is one other scenario in which an application
    reduce, and that is through beta reduction:
    \[ \infer{\textsf{value} \ e_2}{ (\lambda x . e) \ e_2 \leadsto e [ e_2 / x ]
      } \]
    We need to
    show that $\centerdot \vdash e [ e_2 / x ] : \tau$.
    And in this case $e_1$ must be a lambda abstraction, so the
    typing judgement for the first premise must be a \textsc{Abs}.
    \[
      \infer*[Left=App]{ \infer*[Left=Abs]{\centerdot , x : \tau' \vdash e : \tau}{\centerdot \vdash \lambda x . e : \tau' \rightarrow \tau} \\ \centerdot \vdash e_2 : \tau' }
      {\centerdot \vdash (\lambda x . e) \ e_2  : \tau}
    \]
    Furthermore, we know that $\dom(\centerdot)$ and the bound type variables
    in $\tau'$ are distinct --- the domain of $\centerdot$ is empty, and there are
    no bound type variables in the type scheme $\tau'$.  This allows us
    to use the substitution lemma, Lemma~\ref{lem:substitution}, with $\centerdot \vdash e_2 : \tau'$ and
    $\centerdot , x : \tau' \vdash e : \tau$, giving us
    \[ \centerdot \vdash e [ e_2 / x ] : \tau \]
  \item[\rm\textsc{Let}] There is only way a let expression can
    reduce, and that is by
    \[\letin{x = e'}{e} \leadsto e [e' / x]\]
    The premises are
    \begin{align*}
      \centerdot \vdash e' : \tau' \\
      \centerdot , x : \tau' \vdash e : \tau
    \end{align*}
    And we know that there are no bound type variables in the type
    scheme $\tau'$, and so they are naturally distinct form
    $\dom(\centerdot)$. Therefore we can just use the substitution lemma again
    to show that
    \[ \centerdot \vdash e [e' / x] : \tau \]
  \item[\rm\textsc{Unit}] In this case, the judgement is
    $\centerdot : \square \vdash \square$. But there is no way for a unit
    $\square$ to reduce, i.e. $\square \not\leadsto e$, so we do not need to consider this case.
  \item[\rm\textsc{Product}] An expression of the form
    $\centerdot \vdash e_1 \times e_2: \tau_1 \times \tau_2$ can be reduced in either two
    ways. Either $e_1 \leadsto e_1'$ and
    $e_1 \times e_2 \leadsto e_1' \times e_2$, in which case
    \begin{align*}
      \centerdot \vdash e_1 : \tau_1 && \text{by the premises} \\
      \centerdot \vdash e_1' : \tau_1 && \text{by induction hypothesis} \\
      \centerdot \vdash e_1' \times e_2 : \tau_1 \times \tau_2 && \text{by \textsc{Product}}
    \end{align*}
    Or if $e_2 \leadsto e_2'$, then we repeat the same steps but acting on
    the second premise.
  \item[\rm\textsc{Proj1}] $\centerdot \vdash \pi_1 \ e : \tau$ can reduce in two ways as
    well. If $e \leadsto e'$ and $\pi_1 \ e \leadsto \pi_1 \ e'$ then
    \begin{align*}
      \centerdot \vdash e : \tau \times \tau' && \text{by the premises} \\
      \centerdot \vdash e' : \tau \times \tau' && \text{by induction hypothesis} \\
      \centerdot \vdash \pi_1 \ e' : \tau && \text{by \textsc{Proj1}}
    \end{align*}
    However it can also be reduced if $e$ happens to be a product,
    namely $\pi_1 (e_1 \times e_2) \leadsto e_1$. So we just need to show that
    ${\centerdot \vdash e_1 : \tau}$, which can be done by climbing up the proof tree.
    \[
      \infer*[Left=Proj1]{ \infer*[Left=Product]{\centerdot \vdash e_1 : \tau  \\ \centerdot \vdash
          e_2 : \tau}{\centerdot \vdash e_1
          \times e_2 : \tau \times \tau'} }
      {\centerdot \vdash \pi_1 \ (e_1 \times e_2) : \tau}
    \]
  \item[\rm\textsc{Proj2}] Again, the proof for \textsc{Proj2} is
    pretty much identical to that of \textsc{Proj1} and so is omitted
    for brevity.
  \item[\rm\textsc{Lift}] For
    $\centerdot \vdash \lift{e} : \IO_\rho \tau$, just like $\square$ there is no way
    for $\lift{e}$ to reduce any further --- they are both
    values. There is nothing needed to be done for this case.
  \item[\rm\textsc{Use}] The same also applies for $\use{r}{e}$.
  \item[\rm\textsc{Bind}] The proof we are handling looks like this
    \[ \infer*[Left=Bind]{\centerdot \vdash e_1 : \IO_\rho \tau' \\ \centerdot \vdash e_2 : \tau' \rightarrow \IO_\rho \tau}{\centerdot \vdash e_1 \bind
        e_2 : \IO_\rho \tau} \]
    The first thing to note is that there are
    multiple possible ways in which reduction could have occurred: We
    will look at each of them one by one.
    \begin{description}
    \item[$e_1 \bind e_2 \leadsto e_1' \bind e_2$] The premise for this
      reduction is $e_1 \leadsto e_1'$, so we take the following steps
      \begin{align*}
        \centerdot \vdash e_1 : \IO_\rho \tau' && \text{by the premises} \\
        \centerdot \vdash e_1' : \IO_\rho \tau' && \text{by induction hypothesis} \\
        \centerdot \vdash e_1' \bind e_2 : \IO_\rho \tau && \text{by \textsc{Bind}}
      \end{align*}
    \item[$\lift{e} \bind e_2 \leadsto e_2 \ e$] We know
      $\centerdot \vdash \lift{e} : \IO_\rho \tau'$, but that does not necessarily mean
      that the proof for it came from the \textsc{Lift} rule --- it
      could have also been derived from subsumption, \textsc{Sub}. To
      help when this scenario arises, we use a small helper lemma
      listed in the appendix, Lemma~\ref{lem:unwrapLift}, that says if
      $\Gamma \vdash \lift{e} : \IO_\rho \tau'$, then
      $\Gamma \vdash e : \tau'$. Now we can use \textsc{Abs} to construct
      ${\centerdot \vdash e_2 \ e : \IO_\rho \tau}$ as needed.
    \item[$\use{r}{e} \bind e_2 \leadsto e_2 \ e$] This is identical to the
      case above, except we use Lemma~\ref{lem:unwrapUse} instead.
    \end{description}
  \item[\rm\textsc{Conc}] We have the proof and premises
    \[ \infer*[Left=Conc]{\centerdot \vdash v : \IO_{\rho_1} \tau_1 \\ \centerdot \vdash w : \IO_{\rho_2} \tau_2 \\
      \textsf{ok} \ \rho_1 \cup \rho_2}
      { \centerdot \vdash v \curlyvee w : \IO_{\rho_1 \cup \rho_2} \tau_1 \times \tau_2 } \]
    There is also only one possible reduction that could have
    occurred
    \[ v \curlyvee w \leadsto v \bind \lambda v . (w \bind \lambda w . \lift{v \times w}) \]
    So we need to show ${\centerdot \vdash v \bind \lambda v . (w \bind \lambda w . \lift{v
        \times w}) : \IO_{\rho_1 \cup \rho_2} \tau_1 \times \tau_2}$ --- not exactly the
    prettiest judgement --- which we can construct eventually from a \textsc{Bind}.

    To begin, we will first show that both $v$ and $w$ can be subsumed
    into the larger combined heap, $\rho_1 \cup \rho_2$.
    \begin{align*}
      \centerdot \vdash v : \IO_{\rho_1} \tau_1 && \text{from premises} \\
      \rho_1 \subtyp \rho_1 \cup \rho_2 && \text{by \textsc{UnionL}} \\
      \textsf{ok} \ \rho_1 \cup \rho_2 && \text{from premises} \\
      \centerdot \vdash v : \IO_{\rho_1 \cup \rho_2} \tau_1 && \text{by \textsc{Sub}}
    \end{align*}
    \begin{align*}
      \centerdot \vdash w : \IO_{\rho_2} \tau_2 && \text{from premises} \\
      \centerdot , v : \tau_1 \vdash w : \IO_{\rho_2} \tau_2 && \text{by Lemma~\ref{lem:weaken}} \\
      \rho_2 \subtyp \rho_1 \cup \rho_2 && \text{by \textsc{UnionR}} \\
      \textsf{ok} \ \rho_1 \cup \rho_2 && \text{from premises} \\
      \centerdot , v : \tau_1 \vdash w : \IO_{\rho_1 \cup \rho_2} \tau_2 && \text{by \textsc{Sub}}
    \end{align*}

    Then we show that we can make a lambda, taking a \textit{pure} $\tau_2$, and making
    a $\tau_1\times \tau_2$ inside that new $\IO$ monad.
    \[
      \infer{
        \infer{
          \infer{
            \centerdot, v : \tau_1, w : \tau_2 \vdash v : \tau_1 \\
            \centerdot, v : \tau_1, w : \tau_2 \vdash w : \tau_2}
          {\centerdot, v : \tau_1, w : \tau_2 \vdash v \times w : \tau_1 \times \tau_2}
        }
        {\centerdot, v : \tau_1, w : \tau_2 \vdash \lift{v \times w} : \IO_{\rho_1 \cup \rho_2} \tau_1 \times
          \tau_2}
      }
      {\centerdot, v : \tau_1 \vdash \lambda w . \lift{v \times w} : \tau_2 \rightarrow \IO_{\rho_1 \cup \rho_2} \tau_1 \times \tau_2}
    \]
    We then take this lambda and bind our \textit{monadic} $w$ to it
    \[
      \infer{
        \centerdot , v : \IO_{\rho_1} \tau_1 \vdash w : \IO_{\rho_1 \cup \rho_2} \tau_2 \\
        \centerdot, v : \tau_1 \vdash \lambda w . \lift{v \times w} : \tau_2 \rightarrow \IO_{\rho_1 \cup \rho_2} \tau_1 \times
        \tau_2}
      {\centerdot, v : \tau_1 \vdash w \bind (\lambda w . \lift{v \times w}) : \IO_{\rho_1 \cup
          \rho_2} \tau_1 \times \tau_2}
    \]
    Now we can make another lambda, this time taking in the \textit{pure} $\tau_1$
    \[
      \infer{
        \centerdot, v : \tau_1 \vdash w \bind (\lambda w . \lift{v \times w}) : \IO_{\rho_1 \cup
          \rho_2} \tau_1 \times \tau_2}
      {\centerdot \vdash \lambda v . w \bind (\lambda w . \lift{v \times w}) : \tau_1 \rightarrow \IO_{\rho_1 \cup
          \rho_2} \tau_1 \times \tau_2}
    \]
    And then we finally bind it with our \textit{monadic} $v$
    \[
      \infer{
        \centerdot \vdash v : \IO_{\rho_1 \cup \rho_2} \tau_1 \\
        \centerdot \vdash \lambda v . w \bind \lambda w . \lift{v \times w} : \tau_1 \rightarrow \IO_{\rho_1 \cup
          \rho_2} \tau_1 \times \tau_2}
      {\centerdot \vdash v \bind (\lambda v . w \bind (\lambda w . \lift{v \times w})) : \IO_{\rho_1 \cup
          \rho_2} \tau_1 \times \tau_2}
    \]
    The proof is quite long, but since our typing system is syntax
    directed, it closely matches the expression it reduces to, making
    it easier to construct.
  \item[\rm\textsc{Sub}] Last but not least, we have a subsumption
    typing judgement
    \[
      \infer*[Left=Sub]{\centerdot \vdash e : \IO_\rho \tau \\ \rho \subtyp \rho' \\ \textsf{ok} \ \rho'}
      {\centerdot \vdash e : \IO_{\rho'} \tau}
    \]
    However because we know nothing of the syntax of $e$, $e$ could
    reduce to anything and we are left with some $e'$, such that
    $e \leadsto e'$.  Nevertheless, applying the induction hypothesis with
    $\centerdot \vdash e : \IO_\rho \tau$ and $e \leadsto e'$ gives us
    \[ \centerdot \vdash e' : \IO_\rho \tau \]
    Which we can stick back into \textsc{Sub} to get
    \[ \centerdot \vdash e' : \IO_\rho' \tau \]
  \end{description}
\end{proof}

Together these two properties can ensure that no expression can go
wrong, which within operational semantics means it cannot get
\textit{stuck}. A stuck expression is an expression which cannot be
reduced any further (in its normal form) and is not a
value. The progress theorem has shown that evaluation of any
well-typed expression cannot end up in this state, so it does indeed
evaluate to something in the end. And the preservation theorem
guarantees us that the type of the final value will remain the same.
Therefore we have shown the operational equivalent of the denotational
\[ \Gamma \vdash e : \tau \rightarrow \Gamma \Vdash e : \tau \]



\chapter{Mechanisation}\label{cha:mechanisation}

In the previous chapter, we defined and proved the soundness of our
type system. In this chapter, we will look at how this was
mechanically formalised --- proved within a proof assistant.

For this type system, I decided to mechanise this within
Agda~\cite{norell2009}. Agda is a dependently typed programming
language with an ML syntax, similar to that of Haskell's. It can be
used for general purpose programming, but because it is rooted in
Martin-Löf intuitionistic type theory~\cite{martin-lof1984}, it can
also be used as a proof assistant.

Like many other proof assistants, the way we prove properties within
Agda is by writing programs that satisfy types. It takes advantage of the
Curry-Howard correspondence, which says that propositions are
analogous to types, and proofs are analogous to programs that fulfil
that type.
we wish to prove as types for our programs. We can then prove our
propositions by constructing a value for the program that satisfies the
type, hence intuitionistic logic is also known as constructive logic.

As an example, for the relation that a heap is well typed,
$\textsf{ok} \ \rho$, we can define a new data type parameterised over the
heap.

\setmonofont{CMU Typewriter Text}

\begin{minted}{agda}
Découverte de Tkinter
data Ok : Heap → Set where
  OkZ : ∀ {r}
        --------
      → Ok (` r)
  OkS : ∀ {a b}
       → Ok a
       → Ok b
       → a ∩ b =∅
         ----------
       → Ok (a ∪ b)
  OkWorld : Ok World
\end{minted}

\chapter{Evaluation} \label{chapter:evaluation}


\begin{grammar}

  <type $\tau$> ::= () | $\tau \rightarrow \tau'$ | $\tau \times \tau'$ | $\textsf{IO}_\sigma \tau$
  
  <type scheme $\sigma$> ::= $\forall \alpha . \sigma$ | $\tau$

  <expression $e$> ::= () | $e \times e'$
  \alt $x$ | $\lambda x . e$ | $e \ e'$ | $\letin{x = e}{e'}$
  \alt $\textsf{if} \ e_1 \ \textsf{then} \ e_2 \ \textsf{else} \ e_3$
  \alt $\lift{e}$ | $\lift{e}_\sigma$ | $e \bind e'$ | $e \curlyvee e'$

  <resource $\rho$> ::= $r$ | \textsf{World}

  <heap $\sigma$> ::= $\rho$ | $\sigma \cup \sigma'$

\end{grammar}

\begin{figure}
  \begin{mathpar}
    \inferrule*[Right=Var]{x : \tau' \in \Gamma \\ \tau' > \tau}{\Gamma \vdash x : \tau} \and
    \inferrule*[Right=App]{\Gamma \vdash e : \tau' \rightarrow \tau \\ \Gamma \vdash e' : \tau'}{\Gamma \vdash e \ e' : \tau} \and
    \inferrule*[Right=Abs]{\Gamma,x : \tau \vdash e : \tau'} {\Gamma \vdash \lambda x . \ e : \tau \rightarrow
      \tau'} \and
    \inferrule*[Right=Let]{\Gamma \vdash e : \tau \\ \Gamma,x : \bar{\Gamma}(\tau) \vdash e' : \tau'}
    {\Gamma \vdash \mathsf{let} \ x = e \ \mathsf{in} \ e' : \tau'} \and
    \inferrule*[Right=If]{\Gamma \vdash e_1 : \mathbf{Bool} \\ \Gamma \vdash e_2 : \tau \\ \Gamma \vdash e_3 : \tau}
                         {\Gamma \vdash \mathsf{if} \ e_1 \ \mathsf{then} \ e_2 \ \mathsf{else} \
                           e_3 : \tau} \and


    \inferrule*[Right=Product]{\Gamma \vdash e : \tau \\ \Gamma \vdash e' : \tau'}
    {\Gamma \vdash e \times e' : \tau \times \tau'} \and

    \inferrule*[Right=Lift]{\Gamma \vdash e : \tau}{\Gamma \vdash \llbracket e \rrbracket : \IO_\sigma \tau} \and
    \inferrule*[Right=Bind]{\Gamma \vdash e : \IO_\sigma \tau' \\ \Gamma \vdash e' : \tau' \rightarrow \IO_\sigma
      \tau}{\Gamma \vdash e \bind e' : \IO_\sigma \tau} \and
    \inferrule*[Right=Conc]{\Gamma \vdash e_1 : \IO_\sigma \tau_1 \\ \Gamma \vdash e_2 : \IO_{\sigma'}
      \tau_2 \\
      \sigma \notsubtyp \sigma' \\ \sigma' \notsubtyp \sigma}
    {\Gamma \vdash e_1 \curlyvee e_2 : \IO_{\sigma \cup \sigma'} \ \tau_1 \times \tau_2} \and

    
    \inferrule*[Right=Tag]{ }{\Gamma \vdash \mathsf{readFile} : \IO_{\mathsf{File}} ()} \and
    \inferrule*[Right=ReadNet]{ }{\Gamma \vdash \mathsf{readNet} : \IO_{\mathsf{Net}} ()} \and

    \inferrule*[Right=Subsumption]{\Gamma \vdash e : \IO_\sigma \tau \\ \sigma \subtyp \sigma'}
    {\Gamma \vdash e : \IO_{\sigma'} \tau} \and
    \inferrule*[Right=Unit]{ }{\Gamma \vdash () : ()} \and
    \inferrule*[Right=True]{ }{\Gamma \vdash \textsf{True} : \textsf{Bool}} \and
    \inferrule*[Right=False]{ }{\Gamma \vdash \textsf{False} : \textsf{Bool}}
  \end{mathpar}

  \caption{Typing rules}
\end{figure}
%%% Local Variables:
%%% TeX-master: "report"
%%% TeX-engine: luatex
%%% TeX-command-extra-options: "-shell-escape"
%%% End:


\appendix
\chapter{} % needed to get section numbering?
\section{Definition of a Complete Partial Order}\label{sec:defin-compl-part}
A complete partial order (cpo) is a pair $(D, \sqsubseteq)$ consisting of a set
$D$ and a partial order $\sqsubseteq$ (a function that orders elements in
$D$, but not necessarily all of them, hence the term partial), such
that
\begin{enumerate}
\item there is a least element $\bot$
\item each directed subset $x_0 \sqsubseteq \ldots \sqsubseteq x_n \sqsubseteq \ldots$ has a least upper bound
  (\emph{lub})
\end{enumerate}

\section{Evaluation Function Notation}\label{sec:eval-funct-notat}
If $\mathbb{D} \subset \mathbb{V}$, and $d \in \mathbb{D}$, we will say $d \ \mathsf{in} \
\mathbb{V}$ to represent $d$ but treated as if its in $\mathbb{V}$. \\
We will then define the reverse

\[
  v | \mathbb{D} =
\begin{cases}
  d & \textsf{if} \ v = d \ \textsf{in} \ \mathbb{V} \ \textsf{for
    some} \ d \in \mathbb{D} \\
  \bot_{\mathbb{D}} & \textsf{otherwise}
\end{cases}
\]

\section{Transitivity of the subheap relation}\label{proof:subheaptransitive}
\begin{proof}
  We want to show $a \subtyp c$. Proceed with induction on $b \subtyp c$.
  \begin{description}
  \item[\rm\textsc{Top}]
    $c$ must be \textsf{World}, so from \textsc{Top} we have $a
    \subtyp c$.
  \item[\rm\textsc{Refl}] By definition of \textsc{Refl}, $b =
    c$, and so we get $a \subtyp c$ from $a \subtyp b$.
  \item[\rm\textsc{UnionL}] $c$ is of the form $\rho' \cup \rho''$, and
    from the premise we have $b \subtyp \rho'$. Use the induction
    hypothesis with $a \subtyp b$ and $b \subtyp \rho'$ to get $a \subtyp
    \rho'$, and then \textsc{UnionL} gives us $a \subtyp \rho' \cup \rho''$.
  \item[\rm\textsc{UnionR}]  $c$ is of the form $\rho'' \cup \rho'$, and
    from the premise we have $b \subtyp \rho'$. Use the induction
    hypothesis with $a \subtyp b$ and $b \subtyp \rho'$ to get $a \subtyp
    \rho'$, and then \textsc{UnionR} gives us $a \subtyp \rho'' \cup \rho'$.
  \end{description}
\end{proof}

\section{Helper lemmas}
\begin{lemma}\label{lem:unwrapLift}
  If $\Gamma \vdash \lift{e} : \IO_\rho \tau$, then $\Gamma \vdash e : \tau$.
\end{lemma}

\begin{proof}
  This might seem obvious, but because of subsumption we need to
  unwravel for the proof for it first. Begin with induction on $\Gamma \vdash
  \lift{e} : \IO_\rho \tau$.
  \begin{description}
  \item[\rm\textsc{Lift}] Straight from the premises, $\Gamma \vdash e : \tau$.
  \item[\rm\textsc{Sub}] We have the premise ${\Gamma \vdash \lift{e} : \IO_{\rho'}
    \tau}$. Just put this back into the induction hypothesis to get ${\Gamma \vdash e
    : \tau}$.
  \end{description}
\end{proof}

\begin{lemma}\label{lem:unwrapUse}
  If $\Gamma \vdash \use{r}{e} : \IO_\rho \tau$, then $\Gamma \vdash e : \tau$.
\end{lemma}
\begin{proof}
  Identical to that of Lemma~\ref{lem:unwrapLift}, except we handle
  the case \textsc{Use} instead of \textsc{Lift}.
\end{proof}

\bibliographystyle{plain}
\bibliography{report}

\end{document}

% LocalWords: stdout LocalWords Idris haskell
%%% Local Variables:
%%% TeX-engine: luatex
%%% End: